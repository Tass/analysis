\begin{proof}[Beweis vom Satz \ref{s:1011242}]
  {\bf Konvexität $\implies$ $f'$ ist wachsend.}
  \[f'(x)=\lim_{h\downarrow 0}\frac{f(x+h)-f(x)}{(x+h)-x}\]
  \[f'(y)=\lim_{h\downarrow 0}\frac{f(y+h)-f(y)}{(y+h)-y}\]
  $h$ klein $\implies$ $x<x+h<y<y+h$.
In diesem Fall impliziert Lemma \ref{l:1011293} die Ungleichungen:
\[ 
 \frac{f(x+h)-f(x)}{(x+h)-x}\leq \frac{f(y)-f(x+h)}{y-(x+h)}
\leq \frac{f(y+h)-f(y)}{(y+h)-y}
\]
Deswegen
\[
f' (x) = \lim_{h\downarrow 0}\frac{f(x+h)-f(x)}{(x+h)-x}\leq
\lim_{h\downarrow 0}\frac{f(y+h)-f(y)}{(y+h)-y} = f'(y)
\]
{\bf Konvexität $\Leftarrow$ $f'$ wachsend.} 
Sei $x_1<x<x_2$: Der Satz von Lagrange $\implies$ $\exists \xi_1\in]x_1,x[$ mit
  \[\frac{f(x)-f(x_1)}{x-x_1}=f'(\xi_1)\]
  $\exists \xi_2\in]x_1,x[$ mit
  \[\frac{f(x_2)-f(x)}{x_2-x}=f'(\xi_2)\]
  NB: $\xi_2>\xi_1$. Weil $f'(\xi_2)\geq f'(\xi_2)$, gilt das Lemma 
\ref{l:1011293} und Lemma $\implies$ Konvexität.

Der Beweis der zweiten Behauptung des Satzes ist analog.
\end{proof}
\subsection{Die Lagrange Fehlerabschätzung}
\begin{Def}
  Sei $f$ $n$-mal differenzierbar. Das Taylorpolynom mit Ordnung $n$ an der Stelle $x_0$ ist:
  \[T^n_{x_0}=\sum^n_{T=0}\frac{f^{(i)}(x_0)}{i!}(x-x_0)^i\]
\end{Def}
\begin{Sat}[Lagrange Fehlerabsch\"atzung]
  Sei $f$ $(n+1)$-mal differenzierbar in $I$ und $x_0\in$. $\forall x\in I$ 
$\exists\xi$ zwischen $x_0$ und $x$ so dass
  \begin{equation}\label{e:1011292}
    \underbrace{f(x)-T^n_{x_0}(x)}_{R_{x_0}^n (x)}=\frac{f^{(n+1)}(\xi)}{n+1!}(x-x_0)^{n+1}
  \end{equation}
\end{Sat}
\begin{Bem}
  Für $n=0$ \eqref{e:1011292} ist:
  \[f(x)-\underbrace{f(x_0)}_{T^0_{x_0}(x)}=f'(\xi)(x-x_0)\]
  \[\iff\frac{f(x)-f(x_0)}{x-x_0}=f'(\xi)\]
Deswegen die Lagrange Fehlerabsch\"atzung ist eine Verallgemeinerung
des Satzes von Lagrange.
\end{Bem}
\begin{Bew} Seien
\[
h (x) = R^n_{x_0} (x) \qquad \mbox{und} \qquad g(x) = (x-x_0)^{n+1}\, .
\]
Es ist leicht zu sehen dass
\[
g (x_0) = g' (x_0) = \ldots = g^{(n)} (x_0) = 0
\]
\[
h(x_0) = h'(x_0) = \ldots = h^{(n)} (x_0) = 0
\]
und 
\[
h^{(n+1)} (x) = f^{(n+1)} (x) \quad\forall x\, .
\]
Deswegen, wir wenden $n+1$ Mal den verallgemeinerten Mittelwertsatz (Satz von Cauchy)
und schliessen: 
  \[\frac{h(x)}{g(x)} = \frac{h(x)-h(x_0)}{g(x)-g(x_0)}
\stackrel{\text{Cauchy}}{=}\frac{h'(\xi_1)}{g'(\xi_1)}=
\frac{h'(\xi_1)-h'(x_0)}{g'(\xi_1)-g'(x_0)}\stackrel{\text{Cauchy}}{=}
\frac{h''(\xi_2)}{g''(\xi_2)}\]
\[=\cdots\stackrel{\text{Cauchy}}{=}
\frac{h^{(n+1)}(\xi_{n+1})}{g^{(n+1)}(\xi_{n+1})}=\frac{f^{(n+1)}(\xi_{n+1})}{(n+1)!}\, ,\]
wobei: $\xi_1$ eine Stelle zwischen $x$ und $x_0$ ist; $\xi_2$ eine Stelle zwischen $\xi_1$
und $x_0$ ist; \ldots $\xi_{n+1}$ eine Stelle zwischen $\xi_n$ und $x_0$ ist.

Wenn wir $\xi:=\xi_{n+1}$ setzen, dann
\[f(x)-T^n_{x_0}(x) = R^n_{x_0} (x) = h (x)=
\frac{f^{(n+1)} (\xi)}{(n+1)!} g (x) =
\frac{f^{(n+1)}(\xi)}{(n+1)!} (x-x_0)^{n+1}\, .\]
\end{Bew}
\begin{Bsp} Wir wissen schon dass
  \[e^x=\sum^\infty_{j=0}\frac{x^j}{j!}\qquad\forall x\in\mb{R}\, .\]
Diese Identit\"at kann man auch aus der Lagrange Fehlerabsch\"atzung
schliessen. Das Taylor Polynom mit Ordnung $n$ in $0$ ist 
\[T_0^n(x)=\sum^n_{j=0}\frac{x^j}{j!}\, .\]
Sei $x\in\mb{R}$ fixiert. 
  \[\Bigg|\overbrace{e^x-\sum_{j=0}^{n}\frac{x^j}{j!}}^{R^{n+1}_0 (x)}\Bigg|=
\left|\frac{e^{\xi_n}x^{n+1}}{(n+1)!}\right|,\]
wobei $\xi_n$ eine Stelle zwischen $x$ und $0$ ist. 
Deswegen, $\abs{\xi_n}\leq \abs{x}$ und
\begin{equation}\label{e:101129-1}
\left|e^x-\sum_{j=0}^{n}\frac{x^0}{j!}\right|\leq e^{|x|}\frac{\abs{x}^{n+1}}{(n+1)!}
\end{equation}
Wir wissen schon dass
\begin{equation}\label{e:101129-2}
\Limi{n}\frac{\abs{x}^{n+1}}{(n+1)!}=0\, .
\end{equation}
(In der Tat, sei $N$ so dass $N\geq 2\abs{x}$. Dann
 \[\left.\frac{\abs{x}^{n+1}}{(n+1)!} = \frac{\abs{x}^N}{N!}
\frac{\abs{x}}{N+1}\frac{\abs{x}}{N+2}\cdots\frac{\abs{x}}{n+1}
\leq \frac{\abs{x}^N}{N!}\left(\frac{1}{2} \right)^{n+1-N}\, .\right)\]
\eqref{e:101129-1} und \eqref{e:101129-2} implizieren dass
\[
0 \leq \left|f(x) - \sum_{j=0}^\infty \frac{x^j}{j!}\right|
= \left|f (x) - \lim_{n\to\infty} T^n_0 (x)\right| 
\]
\[
= \left|\lim_{n\to\infty} R^n_0 (x)\right|
= \lim_{n\to \infty} |R^n_0 (x)| \leq \lim_{n\to\infty}
 e^{|x|}\frac{\abs{x}^{n+1}}{(n+1)!} = 0\, .
\]
\end{Bsp}
\begin{Bsp} Sei
  \[f(x)=\ln(x+1)\]
  (Bem: das Taylorpolynom (bzw. die Taylorreihe) in 0 von $f$ ist 
das Taylorpolynom (bzw. die Taylorreihe) von $\ln x$ an der Stelle 1.)
Dann
  \[T^n_0(x)=\sum^n_{j=0}\frac{f^{(j)}(0)}{j!}x^j=
\sum^n_{j=1}\frac{(-1)^{j-1}}{j}x^j=x-\frac{x^2}{2}+\frac{x^3}{3}-\frac{x^4}{4}+\cdots
+ (-1)^{n-1} \frac{x^n}{n}\]
Wir wollen zeigen dass
\begin{equation}\label{e:logreihe}
\ln(x+1)= \sum_{j=1}^n (-1)^{n-1} \frac{x^j}{j}
\quad \left( = x-\frac{x^2}{2}+\frac{x^3}{3}-\frac{x^4}{4}+\cdots\right)
\end{equation}
in einer Umgebung von $0$.

Sei $x> -1$. Die Lagrange Fehlerabsch\"atzung impliziert: 
\[\big|\overbrace{f (x) -T_0^n(x)}^{R^n_0 (x)}\big|=
\frac{\abs{f^{(n+1)}(\xi_n)}}{(n+1)!}\abs{x}^{n+1}=
\frac{\frac{n!}{\left( 1+\xi_n \right)^{n+1}}}{(n+1)!}\abs{x}^{n+1}\]
wobei $\xi_n$ zwischen 0 und $x$ liegt. Deswegen $\xi_n> -1$
und $1+\xi_n \geq 1-|\xi_n|\geq 1-|\xi|$.
  % TODO unsichere Reihenfolge
Wir schliessen
\[|R^n_0 (x)|\leq \frac{1}{(n+1)}\frac{\abs{x}^{n+1}}{(1-\abs{x})^{n+1}}\, .\]
Aber $\abs{x}\leq \frac{1}{2}$ $\implies \frac{\abs{x}}{1-\abs{x}}\leq 1$
und
\[\Lim_{n\to\infty} \abs{R^n_0 (x)}\leq\lim_{n\to\infty} \frac{1}{n+1} = 0
\qquad \forall x\in \left[\frac{1}{2}, \frac{1}{2} \right]\]

Falls $x\in ]0,1]$, dann $\xi_n>0$ und
\[|R^n_0 (x)| \leq \frac{|x|^{n+1}}{n+1} \leq \frac{1}{n+1}\, .\]
Deswegen gilt die Gleichung \eqref{e:logreihe} auch f\"ur $x\in ]1/2, 1]$.
In der Tat, gilt diese Gleichung auch f\"ur $x\in ]-1, -1/2[$, aber diesen
Fall ist keine einfache Folgerung der Lagrange Fehlerabsc\"atzung.

Ausserdem, die Reihe $\sum_{j=1}^\infty\frac{x^j}{j}(-1)^j$ 
hat Konvergenzradius $R=1$. Deswegen ist die Gleichung \eqref{e:logreihe}
falsch wenn $x>1$.
\end{Bsp}
\section{Integralrechnung}
Sei $f: [x_0, x_1]= I \to \mb{R}$ eine stetig nichtnegative Funktion.
Das Ziel der Integralrechnung ist den Inhalt der folgenden Fl\"ache zu finden: 
\[G=\left\{ (x,y): x\in I \;\;\mbox{und}\;\; 0\leq y \leq f(x) \right\}\]
\subsection{Treppenfunktion}
\begin{Def}
  Eine $\phi:[a,b]\to\mb{R}$ heisst Treppenfunktion wenn $\exists a=x_0<x_1<\cdots<x_n=b$ so dass $\phi$ in jedem Intervall $]x_{k-1},x_k[$ konstant ist.
\end{Def}
\begin{Def} Sei $\phi$ eine Treppenfunktion und $x_0<x_1< \ldots < x_n$ wie oben.
Falls $c_k$ der Wert von $\phi$ in $]x_{k-1}, x_k[$ ist, dann
\[\int^b_af(x)\md x:=\sum^n_{k=1} (x_k-x_{k-1})c_k\, .\]
\end{Def}
\begin{Bem} Es ist leicht zu sehen dass
die Zahl $\int^b_af(x)\md x$ unabhängig von der Verteilung ist.
\end{Bem}
