\section{Die reellen Zahlen}
$\mb{Q}$ ist nicht genug!

\begin{Sat}
  Es gibt kein $q\in\mb{Q}$ so dass $q^2=2$
\end{Sat}
\begin{proof}[Beweis]
  Falls $q^2=2$, dann $(-q)^2=2$ OBdA $q\geq 0$ Deswegen $q>0$. Sei $q>0$ und $q\in\mb{Q}$ so dass $q^2=2$. 
$q=\frac{m}{n}$ mit $m,n\in\mb{N}\setminus \{0\}$ und $\text{GGT}(m,n)=1$ (d.h. falls $r\in\mb{N}$ $m$ und $n$ dividiert, dann $r=1$!).
  \begin{align*}
    m^2=2n^2&\implies m \text{ ist gerade}&\implies m=2k \text{ für } k\in\mb{N}\\\\
    4k^2=2n^2&\implies n \text{ ist gerade}&\implies 2| n \text{(2 dividiert $n$)}
  \end{align*}
  $\implies$ Widerspruch! Weil $2$ dividiert $m$ und $n$! (d.h. es gibt \underline{keine} Zahl $q\in\mb{Q}$ mit $q^2=2$).
\end{proof}
\begin{Bsp}
  \begin{align*}
    \sqrt{2}=1,414\cdots
  \end{align*}
  Intuitiv:
  \begin{align*}
    1,4^2 & < & 2 & < & 1,5^2 & & 1,4 & < & \sqrt{2} & < & 1,5 \\
    1,41^2 & < & 2 & < & 1,42^2 & \qquad\implies & 1,41 & < & \sqrt{2} & < & 1,42 \\
    1,414^2 & < & 2 & < & 1,415^2 & & 1,414 & < & \sqrt{2} & < & 1,415 \\
  \end{align*}
\end{Bsp}
\paragraph{Intuitiv}
\begin{itemize}
  \item $\mb{Q}$ hat ``Lücke''
  \item $\mb{R}$ $= \{$ die reellen Zahlen $\}$ haben ``kein Loch''.
\end{itemize}
\paragraph{Konstruktion}
Die reellen Zahlen kann man ``konstruieren''. (Dedekindsche Schnitte, 
siehe Kapitel I.10 in H. Amann, J. Escher {\em Analysis I}, oder
Kapitel 1.8 in W. Rudin {\em Principle of Mathematical Analysis};
Cantorsche ``Vervollst\"andigung'', siehe I. Stewart {\em Introduction to metic
and topological spaces}). 
Wir werden ``operativ'' sein, d.h. wir beschreiben einfach die wichtigsten Eigenschaften von $\mb{R}$
durch:
\begin{itemize}
 \item die K\"operaxiomen (K1) -- (K4);
\item die Anordnugsaxiomen (A1)-- (A3);
\item das Vollst\"andigkeitsaxiom (V).
\end{itemize}

\subsection{K\"orperstrukturen}
\begin{itemize}
  \item[K1] Kommutativgesetz
  \begin{align*}
    % TODO align that on =
    a+b &= b + a &\\
    a\cdot b &= b\cdot a &\\
  \end{align*}
  \item[K2] Assoziativgesetz
    \begin{align*}
      (a+b)+c &=a+(b+c)\\
      (a\cdot b)\cdot c&=a\cdot(b\cdot c)\\
    \end{align*}
  \item[K3] Distributivgesetz
    \begin{align*}
      (a+b)\cdot c&= a\cdot c + b\cdot c
    \end{align*}
  \item[K4] Die L\"osungen $x$ folgender Gleichungen existieren: 
    \begin{align*}
      a+x&=b &\qquad \forall a,b\in \mb{R} \\
      a\cdot x&=b &\qquad \forall a,b\in \mb{R}, a\neq 0.\\
    \end{align*}
NB: $0$ ist das ``Annallierungselement'', d.h. das einzige Element $0$ so dass $a 0=0$
f\"ur jede $a\in \mb{R}$.
\end{itemize}
\subsection{Die Anordnung von $\mb{R}$}
\begin{itemize}
  \item[A1] $\forall a\in\mb{R}$ gilt genau eine der drei Relationen:
    \begin{itemize}
      \item $a<0$
      \item $a=0$
      \item $a>0$
    \end{itemize}
  \item[A2] Falls $a>0$, $b>0$, dann $a+b>0$, $a\cdot b>0$
  \item[A3] Archimedisches Axiom: $\forall a\in\mb{R}\; \exists n\in\mb{N}$ mit $n>a$
\end{itemize}
\begin{Ueb}
  Beweisen Sie dass $a\cdot b>0$ falls $a<0$, $b<0$
\end{Ueb}
\begin{Sat}[Bernoullische Ungleichung]
  $\forall x>-1$, $x\neq 0$ und $\forall n\in\mb{N}\\\{0,1\}$ gilt $(1+x)^n > (1+nx)$
\end{Sat}
\begin{proof}[Beweis]Vollst\"andige Induktion.\\
{\bf Schritt 1}
  $$(1+x)^2 = 1+2x+\underbrace{x^2}_{>0}>1+2x$$
  weil $x\neq0$.\\
  Nehmen wir an dass
  \begin{align*}
    (1+x)^n&>& 1+nx &\qquad (x>-1, x\neq 0)\\
   \end{align*}
Dann
$$
   \underbrace{(1+x)}_a \underbrace{(1+x)^n}_c > \underbrace{(1+nx)}_d(1+x) \qquad (\mbox{weil}\quad (1+x)>0)\\
$$
(In der Tat,  
$$c>d \iff c-d>0 \stackrel{\text{A2}}{\implies} a(c-d) > 0 \stackrel{\text{K4}}{\implies} ac-ad > 0 
\stackrel{\text{A2}}{\implies} ac>ad)$$
  \begin{align*}
    (1+x)^{n+1} > (1+nx)(1+x) = 1+nx+x+nx^2=\\
    1+(n+1)x+\underbrace{nx^2}_{>0}>1+(n+1)x\\
    \implies (1+x)^{n+1} > 1+(n+1)x
  \end{align*}

 \end{proof}
\begin{Def}
  Für $a\in\mb{R}$ setzt man
  \begin{align*}
  \abs{a}=
    \begin{cases}
      a &\qquad\mbox{falls}\quad a\geq0\\
      -a &\qquad\mbox{falls}\quad a < 0\\
    \end{cases}
  \end{align*}
\end{Def}
\begin{Bem}
  $$\abs{x}=max\left\{ -x,x \right\}$$
\end{Bem}
\begin{Sat}
  Es gilt :
  \begin{eqnarray}
    \abs{ab}&=&\abs{a}\abs{b}\label{e:1}\\
    \abs{a+b}&\leq&\abs{a}+\abs{b}\qquad \mbox{(Dreiecksungleichung)}\label{e:2}\\
    \abs{\abs{a}-\abs{b}}&\leq&\abs{a-b}\label{e:3}
  \end{eqnarray}
\end{Sat}
\begin{proof}[Beweis]
  \begin{itemize}
    \item \eqref{e:1} ist trivial.
    \item Zu \eqref{e:2}:
      \begin{align*}
         a+b\leq \abs{a}+\abs{b} 
      \end{align*}
      (weil $x\leq\abs{x}$ $\forall x\in\mb{R}$ und die Gleichung gilt genau, dann wenn $x\geq 0$).
      \begin{align*}
        -(a+b)=-a-b\leq \abs{-a}+\abs{-b} = \abs{a}+\abs{b}
      \end{align*}
      Aber
      \begin{align*}
        \abs{a+b}=max\left\{ a+b, -(a+b) \right\}\leq \abs{a}+\abs{b}
      \end{align*}
    \item Zu \eqref{e:3}.
      \begin{eqnarray}
        &\abs{a}=\abs{(a-b)+b}\leq \abs{a-b} + \abs{b} \nonumber\\
        \implies &\abs{a}-\abs{b}\leq \abs{a-b} \label{e:+}\\
        &\abs{b}=\abs{a+(b-a)}\leq \abs{a}+\abs{b-a}\nonumber\\
        \implies &\abs{b}-\abs{a}\leq \abs{b-a} = \abs{a-b} \label{e:-}
      \end{eqnarray}
      \begin{align*}
        \abs{\abs{a}-\abs{b}}=max\left\{ \abs{a}-\abs{b}, -\left( \abs{a}-\abs{b} \right) \right\}\stackrel
{\eqref{e:+}, \eqref{e:-}}{\leq}\abs{a-b}
      \end{align*}
  \end{itemize}
\end{proof}
\subsection{Die Vollständigkeit der reellen Zahlen}
Für $a<b$, $a,b\in\mb{R}$, heisst:
\begin{itemize}
  \item abgeschlossenes Intervall: $\left[ a,b \right]=\left\{ x\in\mb{R}: a\leq x\leq b \right\}$ 
  \item offenes Intervall: $\left] a,b \right[=\left\{ x\in\mb{R}: a< x< b \right\}$
  \item (nach rechts) halboffenes Intervall: $\left[ a,b \right[=\left\{ x\in\mb{R}: a\leq x< b \right\}$
  \item (nach links) halboffenes Intervall: $\left] a,b \right]=\left\{ x\in\mb{R}: a< x\leq b \right\}$
\end{itemize}
Sei $I=[a,b]$ (bzw. $]a,b[$ \ldots). Dann $a,b$ sind die \underline{Randpunkte} von $I$. Die Zahl $\abs{I}=b-a$ ist die Länge von $I$. ($b-a>0$)
\begin{Def}
  Eine Intervallschachtelung ist eine Folge $I_1, I_2,\cdots$ geschlossener Intervalle (kurz $(I_n)_{n\in\mb{N}}$ oder $(I_n)$) mit diesen Eigenschaften:
  \begin{itemize}
    \item[I1] $I_{n+1}\subset I_n$
    \item[I2] Zu jedem $\epsilon >0$ gibt es ein Intervall $I_n$ so dass $\abs{I_n} < \epsilon$
  \end{itemize}
\end{Def}
\begin{Bsp}
  $\sqrt{2}$
  \begin{align*}
    1,4^2 & < & 2 & < & 1,5^2 & & I_1 = \left[ 1,4 / 1,5 \right] &\quad \abs{I_1} = 0.1\\
    1,41^2 & < & 2 & < & 1,42^2 & \implies & I_2 = \left[ 1,41 / 1,42 \right] &\quad \abs{I_2} = 0.01\\
    1,414^2 & < & 2 & < & 1,415^2 & & I_3 = \left[ 1,414, 1,415 \right] &\quad \abs{I_2} = 0.001\\
\ldots\\
\ldots &&&&&& I_n = \ldots
  \end{align*}
I1 und I2 sind beide erf\"ullt.
\end{Bsp}
\begin{Axi}
  Zu jeder Intervallschachtelung $\exists x\in\mb{R}$ die allen ihren Intervallen angeh\"ort.
\end{Axi}
\begin{Sat}\label{s:Int_Ein}
  Die Zahl ist eindeutig.
\end{Sat}
\begin{proof}[Beweis]
  Sei $(I_n)$ eine Intervallschachtelung. Nehmen wir an dass $\exists \alpha < \beta$ so dass $\alpha, \beta\in I_n$
f\"ur alle $n$. Dann $\abs{I_n}\geq\abs{\beta-\alpha}> a$. Widerspruch!
\end{proof}
\begin{Sat}\label{s:Wurz}
$\forall a\in\mb{R}$ mit $a\geq 0$ und $\forall k\in\mb{N}\setminus\left\{ 0 \right\}$, 
$\exists ! x\in\mb{R}$ mit 
$x\geq 0$ und $x^k=a$ ($\exists ! x$ bedeutet ``es gibt genau ein $x$''). 
Wir nennen $x=\sqrt[k]{a}=a^\frac{1}{k}$.\\
\end{Sat}

Sei $a>0$ und $m,n\in\mb{N}$. Dann $a^{m+n}=a^ma^n$. Wir definieren $a^{-m}:=\frac{1}{a^m}$ f\"ur $m\in\mb{N}$.
(so dass die Gleichung $a^{m-m}=a^0=1$ stimmt). Wir haben dann die Eigenschaft: $a^{j+k}= a^j\cdot a^k$ $\forall
j,k\in \mb{Z}$. Wir haben aber auch, f\"ur $m,n\in \mb{N}$, 
  \begin{align*}
    (a^m)^n=\underbrace{a^m\cdot a^m \cdots a^m}_{\text{$n$ Mal}} = a^{\overbrace{m+\cdots+m}^{\text{$n$ Mal}}} = a^{nm}
  \end{align*}
(Und mit $a^{-m}=\frac{1}{a^m}$ stimmt die Regel $(a^m)^n=a^{mn}$ auch $\forall m,n\in\mb{Z}$!).
Diese Gleichung motiviert die Notation
$a^\frac{1}{k}$ f\"ur $\sqrt[k]{a}$.

\begin{Def}
  $\forall q=\frac{m}{n}\in\mb{Q}$, $\forall a>0$, wir setzen $a^q :=\left(\sqrt[n]{a}\right)^m$
\end{Def}

Es ist leicht zu sehen dass die Gleichungen
$$
a^{q+r} = a^q\cdot a^r \qquad \mbox{und} \qquad a^{qr}=(a^q)^r
$$
f\"ur alle $q,r\in \mb{Q}$ gelten.
