
\begin{proof}[Beweis vom Satz \ref{s:Wurz}]
  OBdA $x>1$ (sonst würden wir $\frac{1}{x}$ betrachten). 
wir konstruieren eine Intervallschachtelung $(I_n)$, $I_n = [a_n, b_n]$ so dass $a_n^k \geq x \geq b_n^k$
$\forall n\in \mb{N}$  Wie setzten
$$
I_1:=[1,x]
$$
$$
I_{n+1} \;=\; \left\{\begin{array}{ll}
      \left[ a_n, \frac{a_n+b_n}{2} \right] & \qquad\mbox{falls } x \leq \left( \frac{a_n+b_n}{2} \right)^k\\
      \left[ \frac{a_n+b_n}{2}, b_n \right] & \qquad\mbox{falls } x > \left( \frac{a_n+b_n}{2} \right)^k
    \end{array}\right.
$$
$\abs{I_n} = \frac{1}{2^{n-1}}\abs{I_1}$ und $I_{n+1}\subset I_n$.
Intervallschachtelungsprinzip $\implies$ $\exists y\in\mb{R}$ s.d. $y\in I_n \forall n\in\mb{N}$ 

Wir behaupten dass $y^k=x$.

Man definiert $J_n=[a_n^k, b_n^k]$. Wir wollen zeigen, dass $J_n$ eine Intervallschachtelung ist.
  \begin{itemize}
    \item $J_{n+1}\subset J_n$ weil $I_{n+1}\subset I_n$
    \item \[\abs{J_n} = b_n^k-a_n^k = \underbrace{\left( b_n - a_n \right)}_{\abs{I_n}} \underbrace{(b_n^{k-1}+b_n^{k-2}a_n + \cdots + a_n^{k-1})}_{\leq k b_1^{k-1}}\]
  \end{itemize}
  $\implies$ $\abs{J_n}\leq \abs{I_n}k b_1^{k-1}$.\\
  Sei $\varepsilon$ gegeben. Man wähle $N$ gross genug, so dass
  \[\abs{I_n}\leq\varepsilon'=\frac{\varepsilon}{kb_1^{k-1}}\implies\abs{J_n}\leq \varepsilon' kb_1^{k-1}=\varepsilon\]
  Einerseits
  \[y\in\left[ a_n,b_n \right]\quad\implies\quad y^k\in\left[ a_n^k, b_n^k \right]=J_n\]
  Andererseits
  \[x\in J_n\qquad\forall n\in\mb{N}\]
Satz \ref{s:Int_Ein} $\implies$ $x=y^k$
\end{proof}
\subsection{Supremumseigenschaft, Vollständigkeit}
\begin{Def}
  $s\in\mb{R}$ heisst obere (untere) Schranke der Menge $M\subset \mb{R}$ falls $s\geq x$ ($s\leq x$) $\forall x\in M$.
\end{Def}
\begin{Def}
  $s\in\mb{R}$ ist das Supremum der Menge $M\subset\mb{R}$ ($s=\sup M$)
falls es die kleinste obere Schranke ist. D.h.
  \begin{itemize}
    \item $s$ ist die obere Schranke
    \item falls $s'<s$, dann ist $s'$ keine obere Schranke.
  \end{itemize}
\end{Def}
\begin{Bsp}
  $M=]0,1[$. In diesem Fall $\sup M=1 \not\in M$
\end{Bsp}
\begin{Bsp}
  $M=[0,1]$. $\sup M=1\in M$
\end{Bsp}
\begin{Def}
  $s\in\mb{R}$ heisst Infimum einer Menge $M$ ($s=\inf M$) falls $s$ die grösste obere Schranke ist.
\end{Def}
\begin{Def}
  Falls $s=\sup M\in M$, nennt man $s$ das Maximum von $M$. Kurz: $s=\max M$. Analog Minimum.
\end{Def}
\begin{Sat}
  Falls $M\subset \mb{R}$ nach oben (unten) beschränkt ist, dann existiert $\sup M$ ($\inf M$).
\end{Sat}
\begin{proof}[Beweis]
  Wir konstruieren eine Intervallschachtelung $I_n$, so dass $b_n$ eine obere Schranke ist, und $a_n$ keine obere Schranke ist.
  \begin{itemize}
    \item $I_1=[a_1, b_1]$, wobei $b_1$ eine obere Schranke
    \item $a_1$ ist keine obere Schranke
  \end{itemize}
  Sei $I_n$ gegeben. 
  \begin{align*}
    I_{n+1} = \begin{cases}
      \left[ a_n,\frac{a_n+b_n}{2} \right]&\text{falls $\frac{a_n+b_n}{2}$ eine obere Schranke ist;}\\ \\
      \left[ \frac{a_n+b_n}{2}, b_n \right]&\text{sonst\, .}
    \end{cases}
  \end{align*}
  Also, $\exists s$ s.d. $s\in I_n\quad \forall n$.

Wir behaupten dass $s$ das Supremum von $M$ ist.
  \begin{itemize}
    \item Warum ist $s$ eine obere Schranke? \\
    Angenommen $\exists x\in M$ so dass $x>s$. Man wähle $\abs{I_n}<x-s$. Daraus folgt
    \begin{align*}
      x-s>b_n-a_n \geq b_n-s \implies x>b_n
    \end{align*}
    Widerspruch.
  \item Warum ist $s$ die kleinste obere Schranke?\\
    Angenommen $\exists s'<s$. Dann wähle $n'$ so dass $I_{n'} <s-s'$.
    \begin{align*}
      s-s'>b_{n'}-a_{n'}\geq s-a_{n'} \implies a_{n'}>s'
    \end{align*}
    Widerspruch.
  \end{itemize}
\end{proof}
\begin{Lem}\label{l:Voll}
  Jede nach oben (unten) beschränkte Menge $M\subseteq \mb{Z}$ mit $M\neq \emptyset$
besitzt das grösste (kleinste) Element.
\end{Lem}
\begin{proof}[Beweis]
OBdA betrachte nur nach unten beschränkte Mengen $M\subset N$. Angenommen $M$ hat kein kleinstes Element.
Mit der Vollst\"andigen Induktion beweisen wir dass $M=\emptyset$. 
\begin{itemize}
 \item $0\not\in M$, sonst ist $0$ das kleinste Element;
\item Angenommen dass $\{0, 1, \ldots, k\}\cap M=\emptyset$, wir schliessen auch
$\{0, 1, \ldots, k+1\}\cap M =\emptyset$, sonst ist $k+1$ das kleinste Element von $M$.
\end{itemize}
Vollst\"andige Induktion $\implies$ $\{0, \ldots, n\}\cap M=\emptyset$ $\forall n\in \mb{N}$.
D.h. $M\cap \mb{N}=\emptyset$.
\end{proof}
\begin{Sat}
  $\mb{Q}$ ist dich in $\mb{R}$, bzw. für beliebige zwei $x,y\in\mb{R}$, $y>x$, gibt es eine rationelle Zahl $q\in\mb{Q}$, so dass $x<q<y$.
\end{Sat}
\begin{proof}[Beweis]
  Man wähle $n\in\mb{N}$ so dass $\frac{1}{n}<y-x$. Betrachte die Menge $A\subseteq\mb{Z}$, so dass $M\in A$ $\implies$ $M>nx$. Lemma \ref{l:Voll} $\implies$ $\exists m=\min A$.
  \begin{align*}
    x<\frac{m}{n}=\frac{m-1}{n}+\frac{1}{n}<x+y-x=y
  \end{align*}
  Also setze $q=\frac{m}{n}$
\end{proof}
\subsection{Abzählbarkeit}
\begin{Def}
  Die Mengen $A$ \& $B$ sind \underline{gleichmächtig}, wenn es eine Bijektion $f:A\to B$ gibt. D.h. es gibt eine Vorschrift $f$ s.d.
\begin{itemize}
\item $f$ zuordnet ein Element $b\in B$ jedem $a \in A$; dieses Element wird mit $f(a)$ bezichnet;
\item $f(a)\neq f(b)$ falls $a\neq b$; 
\item $\forall b\in B$ $\exists a\in A$ mit $b= f(a)$.
\end{itemize}
($f$ ist eine {\em bijektive Abbildung}; siehe Kapitel \ref{s:F}).
$A$ hat gr\"ossere Mächtigkeit als $B$, falls $B$ gleichmächtig wie eine Teilmenge von $A$ ist, aber $A$ zu keiner Teilmenge von $B$ gleichmächtig ist.
\end{Def}
\begin{Bsp}
  \begin{itemize}
    \item $\{1,2\}$ \& $\{3,4\}$ sind gleichmächtig.
    \item $\{1,2,\cdots,n\}$ hat kleinere Mächtigkeit als $\{1,2,\cdots,m\}$, wenn $n<m$ ist.
  \end{itemize}
\end{Bsp}
\begin{Def}
  Eine Menge $A$ ist abzählbar, wenn es eine Bijektion zwischen $\mb{N}$ und $A$ gibt
D.h. $A=\left\{ a_1,a_2,\cdots,a_n,\cdots \right\}$.
\end{Def}
\begin{Lem}
  $\mb{Z}$ ist abzählbar
\end{Lem}
\begin{proof}[Beweis]
  \begin{tabular}{c|cccccc}
    $\mb{N}$ & 1 & 2 & 3 & 4 & 5 & \ldots \\
    $\mb{Z}$ & 0 & 1 & -1 & 2 & -2 & \ldots
  \end{tabular}

\medskip

Formal, definiere
  \begin{align*}
    f:\mb{N}\to \mb{Z}\\
    f(n):=\begin{cases}
      \frac{n}{2} & \text{wenn $n$ gerade}\\
      \frac{1-n}{2} & \text{wenn $n$ ungerade}
    \end{cases}
  \end{align*}
\end{proof}
\begin{Sat}
  $\mb{Q}$ ist abzählbar
\end{Sat}

\begin{Sat}
  $\mb{R}$ ist nicht abzählbar.
\end{Sat}
(F\"ur die Beweise siehe Kapitel 2.4 von K. K\"onigsberger {\em Analysis I}).
