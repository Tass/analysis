\section{Komplexe Zahlen}
\begin{Bem}
  $\forall a\in \mb{R}$, $a^2>0$. Deswegen ist $x^2=-1$ unlösbar.
  Die Erfindung der imagin\"are Einheit $i$ (die imaginäre Zahl mit $i^2=-1$) 
hat sehr interessante Konsequenzen auch für die üblichen reellen Zahlen.
\end{Bem}
\subsection{Definition}[Erste Definition der Komplexen Zahlen]\label{d:C1}
\begin{Def} Sei $a,b\in\mb{R}$, dann $a+bi\in\mb{C}$. Wir definieren die Summe:
  \begin{equation*}
    (a+bi)+(\alpha+\beta i) = (a+\alpha)+(b+\beta)i
\end{equation*}
und das Produkt
\begin{equation*}
    (a+bi)(\alpha+\beta i) = (a\alpha-b\beta)+ \underbrace{(a\beta+b\alpha)}_A i
  \end{equation*}
\end{Def}
\begin{Def}
  Seien $A$ und $B$ zwei Mengen. Dann ist $A\times B$ die Menge der Paare $(a,b)$ mit $a\in A$ und $b\in B$.
\end{Def}
\begin{Def}[Zweite Definition der Komplezen Zahlen]\label{d:C2}
  $\mb{C}=\mb{R}\times\mb{R}$ mit $+$ und $\cdot$ , die wir so definieren:
  \begin{eqnarray*}
    (a,b)+(\alpha,\beta)&=&(a+\alpha,b+\beta)\\
    (a,b)(\alpha,\beta)&=&(a\alpha-b\beta, \underbrace{a\beta+b\alpha}_{A})
  \end{eqnarray*}
\end{Def}
\begin{Bem}
  \begin{equation*}
    \mb{R}\simeq \left\{ (a,0), a\in\mb{R} \right\}\subset\mb{C}\\
\end{equation*}
In der Sprache der abstrakte Algebra $\mb{R}$ ist isomorph zu 
$\mb{R}' := \{(a,0):a\in \mb{R}\}$: d.h. die Summe und 
das Produkt in $\mb{R}$ und $\mb{R}'$ sind ``gleich'': 
\begin{eqnarray*} 
    (a,0)+(\alpha,0)=(a+\alpha,0)\\
    (a,0)(\alpha,0)=(a\alpha,0)
  \end{eqnarray*}
Deswegen wir schreiben $a$ statt $(a,0)$.
\end{Bem}
\begin{Bem}
  \begin{equation*}
    (0,a)(0,b)=(-ab, 0)
  \end{equation*}
  Deswegen:
  \begin{equation*}
    \underbrace{(0,1)}_{\text{Wurzel von -1}}(0,1)=(-1,0)\\
    \underbrace{(0,-1)}_{\text{auch eine Wurzel von -1}}(0,-1)=(-1,0)
  \end{equation*}
\end{Bem}
\begin{Bem}
  $i=(0,1)$ und wir schreiben $(a,b)$ für $a+bi$. D.h. die zwei Definitionen der
komplezen Zahlen sind equivalent!
\end{Bem}
\begin{Bem}
  $0=(0,0)=0+0i$. $\xi\in\mb{C}$
  \begin{align*}
    0\xi=0\\
    0+\xi=\xi
  \end{align*}
\end{Bem}
\begin{Sat}\label{s:CK}
  Alle Körperaxiome (K1-K4) gelten.
\end{Sat}
\begin{proof}[Beweis]
  \begin{itemize}
    \item[K1] Kommultativität: {\em trivial}
    \item[K2] Assoziativität: {\em trivial}
    \item[K3] Distributivität: {\em trivial}.
    \item[K4] Seien $\xi, \zeta \in\mb{C}$.
      \begin{align}
        \exists \omega\in\mb{C} :&\qquad \xi+\omega=\zeta\label{e:minus}\\
        \xi\neq 0\exists \omega: &\qquad \xi\omega=\zeta\label{e:/}
      \end{align}
  \end{itemize}
{\em Zu \eqref{e:minus}}. Wir setzen
  \begin{eqnarray*}
    \xi=a+bi\\
    \zeta=c+di\\
    \omega=x+yi
  \end{eqnarray*}
  \begin{equation*}
    \xi+\omega = (a+x)+(b+y)i = \xi = c+di
  \end{equation*}
  Sei $x:=c-a$, $y:=d-b$. Dann $\xi+\omega=\zeta$.

\medskip

{\em Zu \eqref{e:/}} $1$ ( $= 1+0i)$) das neutrale Element.
  \begin{equation*}
    (a+bi)(1+0i)=\underbrace{(a1-b0)}_{a}+\underbrace{(b1+a0)}_{b}=(a+bi)
  \end{equation*}
  Sei $\xi\neq 0$ und suchen wir $\alpha$ so dass $\xi\alpha=1$. Dann ist $\omega=\alpha\zeta$ eine Lösung von 
\eqref{e:/} (eigentlich DIE Lösung). Falls $\xi=a+bi$, dann
  \begin{equation*}
    \alpha=\frac{a}{a^2+b^2}-\frac{b}{a^2+b^2}\, .
\end{equation*}
In der Tat:
\begin{equation*}
    \xi\alpha=\overbrace{\left( \frac{aa}{a^2+b^2}-\frac{b(-b)}{a^2+b^2} \right)}^1 + 
    \overbrace{\left( \frac{a(-b)}{a^2+b^2}-\frac{ab}{a^2+b^2} \right)}^0 i=1\, .
\end{equation*}
\end{proof}
\begin{Def}
  Sei $\xi=(x+yi)\in\mb{C}$. Dann:
  \begin{itemize}
    \item $x$ ist der reelle Teil von $\xi$ $(\Re\xi=x)$
    \item $y$ ist der imaginäre Teil von $\xi$ $(\Im\xi=y)$
    \item $x-yi$ ist die konjugierte Zahl $\left( \ol\xi =x-yi \right)$
  \end{itemize}
\end{Def}
\begin{Bem}
  \[\sqrt{\ol\xi\ol\xi}=\sqrt{\left( \Re\xi \right)^2+\left( \Im\xi \right)^2}=:\abs{\xi}\]
\end{Bem}
\begin{Def}
  $\abs{\xi}$ ist der Betrag von $\xi$.
\end{Def}
\begin{Sat}
  Es gilt: $(\forall a,b\in\mb{C})$:
  \begin{itemize}
    \item 
      \begin{itemize}
        \item \[\ol{a+b}=\ol{a}+\ol{b}\]
        \item \[\ol{ab}=\ol{a}\ol{b}\]
      \end{itemize}
    \item 
      \begin{itemize}
        \item \[\Re a =\frac{a+\ol{a}}{2}\]
        \item \[(\Im a) i=\frac{a-\ol{a}}{2}\]
      \end{itemize}
    \item $a=\ol{a}$ genau dann wenn $a\in\mb{R}$.
    \item \[a\ol a=\abs{a}^2=\sqrt{\left( \Re a \right)^2+\left( \Im a \right)^2}\geq 0\]
      (die Gleicheit gilt genau dann wenn $a=0$)
  \end{itemize}
\end{Sat}
\begin{Bem}
  Sei $\omega$ so dass $\xi\omega=1$ $(\xi\neq 0)$. Man schreibt $\omega\frac{1}{\xi}$.
Der Beweis vom Satz \ref{s:CK} impliziert $\omega=\frac{\ol\xi}{\abs{\xi}^2}$
\end{Bem}
\begin{Sat}
  $\forall a,b\in\mb{C}$
  \begin{itemize}
    \item $\abs{a}>0$ für $a\neq 0$ (trivial)
    \item $\abs{\ol a}=\abs{a}$ (trivial)
    \item $\abs{\Re a}\leq\abs{a}$, $\abs{\Im a}\leq\abs{a}$ (trivial)
    \item $\abs{ab}=\abs{a}\abs{b}$
    \item $\abs{a+b}\leq\abs{a}+\abs{b}$
  \end{itemize}
\end{Sat}
\begin{proof}[Beweis]
  \begin{equation*}
    \abs{ab}^2=(ab)\ol{(ab)}=ab\ol a\ol b=a\ol a\ol b=\abs{a}^2\abs{b}^2\\
    \implies \abs{ab}=\abs{a}\abs{b}
  \end{equation*}
\begin{eqnarray}
\underbrace{|a+b|^2}{\in \mb{R}} &=& (a+b)\ol{(a+b)} = 
    (a+b)(\ol a+\ol b)=a\ol a+b\ol b+a\ol b+b\ol a \nonumber\\
&=& \underbrace{\abs{a}^2+\abs{b}^2}_{\in\mb{R}}+\left( a\ol b+b\ol a \right)\, .\label{e:10}
\end{eqnarray}
Bemerkung: die Identit\"at implizert dass $a\ol b+b\ol a$. In der Tat
$a\ol b + b\ol a= a\ol b+ \ol{a\ol b} = 2 \re a \ol b$. Deswegen
\begin{eqnarray}
|a+b|^2 &=& |a|^2 +|b|^2 + 2 \re a\ol b \;\leq\; |a|^2+|b|^2+ 2|a \ol b|\nonumber\\
&=& |a|^2 + |b|^2 + 2 |a||b| = (|a|+|b|)^2\, .
\end{eqnarray}
\end{proof}

