\subsection{Kettenregel}
\begin{Sat} Seien
\[f:\underbrace{U}_{\subset \mb{R}^n}\to\underbrace{V}_{\subset \mb{R}^n}
\quad \mbox{und}\quad g:V\to\mb{R}^k\, .\]
  Falls $f$ in $a$ differenzierbar ist und $g$ in $b=f(a)$ differenzierbar ist, dann ist $g\circ f$ in $a$ differenzierbar und
  \begin{equation}
    \label{e:1104111}
    \md(g\circ f)|_{a} = \md g|_b\circ\md f|_a
  \end{equation}
\end{Sat}
\begin{Bew}
  Die Differenzierbarkeit von $f$ in $a$ bedeutet
  \[f(a+h)=f(a)+\md f|_a(h) +\overbrace{R(h)}^{o(\Norm{h})}\]
  Die Differential von $g$ in $b$ bedeutet
  \[g(b+k)=g(b)+\md g|_b(k) +\underbrace{\bar R(k)}_{o(\Norm{k})}\]
  \[g(f(a+h))=g(\underbrace{f(a)}_b+k)=g(b)+\md g|_b (k)+\bar R(k)\]
  \[=g(b)+\md g|_b\left( \md f|_a(h)+R(h) \right)+\bar R(k)\]
Linearität von $\md g|_b$
  \[=\underbrace{g(b)}_{g\circ f(a)}+\underbrace{\md g|_b(\md f|_a(h))}_{\text{ist linear in}\s h}+\underbrace{\md g|_b(R(h))+\bar R(k)}_{:=\rho(h)}\]
Wir werden zeigen dass  
\[\rho(h)=o(\Norm{h})\, .\]

\medskip

{\bf Schritt 1.  Linearität von $h\mapsto \md g|_b(\md f|_a(h))$.}
  \[\md g|_b\circ\md f|_a(\lambda_1h_1+\lambda_2h_2)\s\lambda_1,\lambda_2\in\mb{R}, h_1,h_2\in\mb{R}^n\]
  \[=\md g|_b\left( \md f|_a(\lambda_1h_+\lambda_2h_2 \right))\]
  \[=\md g|_b\left( \lambda_1\overbrace{\md f|_a(h_1)}^{\in\mb{R}^m}+\lambda_2\overbrace{\md f|_a(h_2)}^{\in\mb{R}^m} \right)\]
  \[=\lambda_1\md g|_b\left( \md f|_a(h_1) \right)+\lambda_2\md g|_b\left( \md f(h_2) \right)\left( \md f(h_2) \right)\]
  \[=\lambda_1\md g|_b\circ \md f|_a(h_1)+\lambda_2\md g|_b\circ \md f|_a(h_2)\]

\medskip
{\bf Schritt 2: $\rho (h) = o (\|h\|)$}.
 \[\frac{\rho(h)}{\Norm{h}}\leq\frac{\abs{\md g|_b (R(h))}}{\Norm{h}}+\frac{\abs{\bar R(k)}}{\Norm{h}}\]
 \begin{equation}\label{e:absch}
\leq\frac{\Norm{\md g|_b}_0\Norm{R(h)}}{\Norm{h}}+\frac{\Norm{\bar R(k)}}{\Norm{h}}
\end{equation}
Wir wissen dass   
\[\frac{\Norm{R(h)}}{\Norm{h}}\to 0\]
und deswegen konvergiert der erste Teil von \eqref{e:absch} zu null. Ausserdem
  \[\frac{\Norm{\bar R(k)}}{\Norm{h}} = \begin{cases}
    0 & \text{falls}\s k=0\\
    \frac{\Norm{\bar R(k)}}{\Norm{k}}\frac{\Norm{k}}{\Norm{h}}
  \end{cases}\]
  \[\Norm{k}=\Norm{\md f|_a(h)+R(h)}\leq \Norm{\md f|_a(h)}+\Norm{R(h)}\]
  \begin{equation}
    \label{e:1104110}
    \leq \Norm{\md f|_a}_0\Norm{h}+\Norm{R(h)}
  \end{equation}
Da
  \[\frac{\Norm{R(h)}}{\Norm{h}}\to 0\]
  $\exists \delta>0$ so dass
  \[\Norm{h}<\delta\implies \frac{\Norm{R(h)}}{\Norm{h}}\]
  Falls $\Norm{h}<\delta$
  \[\|k\|\leq (\Norm{\md f|_a}_0+1)\Norm{h}\]
  Deswegen: wenn $\Norm{h}\to 0$, dann $\Norm{k}\to 0$ und für $\Norm{h}<\delta$
  \[\frac{\Norm{\bar R(k)}}{\Norm{h}}\leq \underbrace{\frac{\Norm{\bar R(k)}}{\Norm{k}}}_{\to 0\s\text{für}\s\Norm{h}\to 0}(\Norm{\md f|_a}_0+1)\]
  Deswegen:
  \[0\leq \limsup_{\Norm{h}\to 0}\frac{\Norm{\rho(h)}}{\Norm{h}}\]
  \[\leq\Limo{\Norm{h}}\frac{\Norm{\bar R(k)}}{\Norm{h}}+\Limo{\Norm{h}}\frac{\Norm{\md g|_b(R(h))}}{\Norm{h}}=0+0=0\]
  \[\implies \Limo{\Norm{h}}\frac{\Norm{\rho(h)}}{\Norm{h}}=0\]
\end{Bew}
\begin{Bem} Sei $n=m=k=1$. $b=f(a)$
  \[\md f|_a(h)=f'(a)h\]
  \[\md g|_b(k)=g'(b)k\]
  \[\md g|_b\circ \md f|_a( h)=\md g|_b(\md f|_a(h))=\md g|_b(f'(a)h)\]
  \begin{equation}
    \label{e:1104112}
    =g'(b)f'(a)h=g'(f(a))f'(a)h
  \end{equation}
  $\phi=g\circ f$
  \[\md \phi|_a(h)=\phi'(a)h=(g\circ f)'(a)h\]
  \eqref{e:1104111} (d.h. die allgemeine Kettenregel) impliziert
  \begin{eqnarray*}
    \md \phi|_a(h)&=&\md (g\circ f)|_a(h)
    =\md g|_b\circ \md f|_a(h)\\
&\stackrel{\eqref{e:1104112}}{=}&g'(f(a))f'(a)h
  \end{eqnarray*}
  \begin{eqnarray*}
    \implies(g\circ f)'(a)\not h=g'(f(a))f'(a)\not h\\
    \implies \underbrace{(g\circ f)'(a)=g'(f(a))f'(a)}_{\text{alte Kettenregel}}
  \end{eqnarray*}
\end{Bem}
\begin{Bem}
  Kettenregel für die Jacobi-Matrizen. Sei $M$ die Jacobi-Matrix für $\md g|_{b(=f(a)}$ und $N$ düe für $\md f_a$.
Die Jacobi für $\md (g\circ f)|_a$ ist $MN$ \\
  $\implies$ $g=(g_1,\cdots,g_k)$ $f=(f_1,\cdots,f_m)$ Es gibt eine Formel für $\Part{(g\circ f)_i}{x_j}$
  \[\md g_b\circ \md g|_a(w)=\md g|_b(\underbrace{\md f|_a(w)}_v)\]
  \begin{eqnarray*}
    \md g|_b\circ \md f|_a(w)=\md g|_b(v)\\
    =\left( \sum^m_{i=1}M_{1i}v_i,\sum^m_{i=1}M_{2i}v_i,\cdots,\sum^m_{i=1}M_{ki}v_i \right)\\
    =\left( \sum^m_{i=1}M_{1i}\sum^n_{j=1}N_{ij}w_j,\cdots,\sum^m_{i=1}M_{ki}\sum^n_{j=1}N_{ij}w_j \right)\\
  \end{eqnarray*}
  \begin{eqnarray*}
    v=\md f|_a(w)=\left( \sum^n_{j=1}N_{1j}w_j,\cdots,\sum^n_{j=1}N_{mj}v_j \right)\\
    \iff v_i=\sum^n_{j=1}N_{ij}w_j
  \end{eqnarray*}
  \[\md g|_b\circ \md f|_a(v)=\left( \sum^m_{i=1}\sum^n_{j=1}M_{1i}N_{ij}v_j,\cdots,\sum^m_{i=1}\sum^n_{j=1}M_{ki}N_{ij}v_j \right)\]
  (Sei $A$ die Matrix 
  \[A_{lj}=\sum_{i=1}^mM_{li}N_{ij} \iff A=M\cdot N\]
  \[=\left( \sum^n_{j=1}A_{1j}v_j,\cdots,\sum_{j=1}^nA_{kj}v_j \right)\]
  Deswegen ist $A$ die Matrixdarstellung von
  \[\md g|_b\circ \md f|_a=\md (g\circ f)|_a\]
  $\iff$ $A$ ist die Jacobi-Matrix für $\md (g\circ f)|_a$
\end{Bem}
\begin{Bem}
  $f:U\to V\subset\mb{R}^m$ $f=(f_1,\cdots,f_m)$, $f_i(x)=f(x_1,\cdots,x_n)$\\
  $g:V\to \mb{R}^k$ $g=(g_1,\cdots,g_k)$, $g_j(x)=g(y_1,\cdots,y_m)$\\
  \[g\circ f(x)=\left( g_1(f(x)),\cdots,g_k(f(x)) \right)\]
  \[g_j(x)=g_j(f_1(x),\cdots,f_m(x))\]
  \[g_j(y)=g_j(f(x_1,\cdots,x_n),\cdots,f_m(x_1,\cdots,x_n))\]
  \[A_{lj}=\Part{}{x_j}(g_l\circ f)(a)\]
  \[M_{li}=\Part{g_l}{y_i}(b)=\Part{g_l}{y_i}(f(a))\]
  \[N_{ij}=\Part{f_i}{x_j}(a)\]
  \begin{eqnarray*}
    \Part{}{x_j}(g_l\circ f)(a)=A_{lj}=\sum_{i=1}M_{li}N_{ij}\\
    =\sum_{i=1}^m\Part{g_l}{y_i}(f(a))\Part{f_i}{x_j}(a)
  \end{eqnarray*}
\end{Bem}
\begin{Kor}
  Sei $f:U\to V(\subset\mb{R}^m)$ und $\phi:V\to\mb{R}$ mit:
  \begin{itemize}
    \item $a\in U$ und $U$ offen
    \item $b\in V$, $V$ offen und $b=f(a)$
    \item $f$ differenzierbar in $a$ und $\phi$ differenzierbar in $b$
  \end{itemize}
  Dann ist $\phi\circ f$ differenzierbar in $a$ und
  \[\Part{(\phi\circ f)}{x_j}(a)=\sum_{i=1}^m\Part{\phi}{y_i}(f(a))\Part{f_i}{x_j}(a)\]
  Das ist die ``konkrete'' allgemeine Kettenregel.
\end{Kor}
