\section{Folgen}
\begin{Def}
  Eine Folge komplexer (reeller) Zahlen ist eine Abbildung: $f:\mb{N}\to\mb{C}$(bzw. $\mb{R})$. Das heisst:
  \[\forall n\in\mb{N}\qquad f(n)\in\mb{C}\;\;(\mbox{bzw. }\mb{R})\, .\]
Wir schreiben $a_n$ f\"ur $f(n)$
 \end{Def}
N.B.: $\mb{N}$ ist auch eine Folge: $a_n=f(n)=n$.
\begin{Def}
  Eine Folge $(a_n)$ heisst konvergent, falls $\exists a\in\mb{C}$ so dass:
\begin{equation}\label{e:konvergiert}
 \forall\varepsilon>0\s\exists N\in\mb{N}:\quad\abs{a_n-a}<\varepsilon\s\forall n\geq N\, .
\end{equation}
\end{Def}
\begin{Bsp}
  $a_n=\frac{1}{n}$ ist eine konvergente Folge. Sei $a=0$. Wählen wir $\varepsilon>0$. 
Sei dann $N\in \mb{N}$ mit $N>\frac{1}{\varepsilon}$ (diese Zahl existiert wegen des
Axioms von Archimedes!). F\"ur $n\geq N$:
  \[\abs{a_n}=\left( \frac{1}{n}-0 \right)=\frac{1}{n}\leq\frac{1}{N} < \varepsilon\]
\end{Bsp}
\begin{Bem}
  Die Zahl $a$ im Konvergenzkriterium ist eindeutig. Sie heisst \ul{der Limes} der Folge $(a_n)$.
\end{Bem}
\begin{proof}[Beweis]
  Seien $a\neq a'$ zwei relle Zahlen, die das Konvergenzkriterium \eqref{e:konvergiert}
erf\"ullen. Sei $\varepsilon:=\frac{\abs{a-a'}}{2}$
  \[\exists N:\abs{a_n-a}<\varepsilon\s\forall n\geq N\]
  \[\exists N':\abs{a_n-a'}<\varepsilon\s\forall n\geq N'\]
  Für $n\geq \max \left\{ N,N' \right\}$
  \[\abs{a'-a}\leq\abs{a'-a_n}+\abs{a-a_n}<2\varepsilon=\abs{a'-a}\]
  \[\implies \abs{a'-a}<\abs{a'-a}\quad \mbox{Widerspruch!}\]
\end{proof}
Wenn eine Folge kovergiert und die Zahl $a$ \eqref{e:konvergiert} erf\"ullt, wir schreiben
$$ 
a=\lim_{n\to+\infty}(a_n)\, 
$$
oder
$$
a_n\to a\, .
$$
\begin{Bem} Sei $\alpha = A+ 0,b_0 b_1 b_2\ldots $ eine reelle Zahl, wobei $A\in \mb{N}$ und $b_i$ die
Ziffern der Dezimaldarstellung von $\alpha-A$ sind. F\"ur jede $n\in \mb{N}$ sei
$$
a_n := A + 0,b_0\ldots b_n \quad\in \mb{Q}\, .
$$
Die Folge $(a_n)$ konvergiert gegen $\alpha$. In der Tat, sei $\varepsilon$ eine beliebige
positive reelle Zahl. Sei $N$ s.d. $10^N > \frac{1}{\varepsilon}$. F\"ur $n\geq N$ gilt
$|a_n-\alpha|\leq 10^{-N} < \varepsilon$.
\end{Bem}
\begin{Def}
  Sei $(a_n)$ eine Folge und $A(n)$ eine ``Folge von Aussagen über $a_n$''. 
Wir sagen dass $A(n)$ wahr für ``fast alle'' $a_n$ ist, wenn $\exists N$ so dass $A(n)$ 
stimmt $\forall n\geq N$. Eine alternative Formulierung von \eqref{e:konvergiert} ist also:
  \[\abs{a_n-a}<\varepsilon\s\text{f\"ur fast alle}\s a_n\]
\end{Def}
\begin{Bsp}
  Sei $s\in\mb{Q}$ $s>0$. Sei $a_n=\frac{1}{n^s}$. Dann
  \[\lim_{n\to+\infty}\left( \frac{1}{n^s} \right)=0\]
  Sei $N\in \mb{N}$ mit $N>\varepsilon^{\frac{1}{s}}$ (Axiom von Archimedes!). Dann
  \[\abs{0-a_n}=\frac{1}{n^s}<\varepsilon\qquad =\mbox{falls $n\geq N$}\]
(NB: $\frac{1}{s}$ ist wohldefiniert weil $s\neq 0$. Ausserdem
\[\frac{1}{n^s}<\varepsilon \iff n^s>\frac{1}{\varepsilon}\iff n>
\frac{1}{\varepsilon^{\frac{1}{s}}}\quad\s\text{weil $s> 0$.)}\]
\end{Bsp}
\begin{Bsp}
  $a>0$
  \[\lim_{n\to+\infty}\sqrt[n]{a}=1\]
  {\bf Fall $a>1$}. Zu beweisen:
  \[\forall \varepsilon>0 \s\exists N:\quad \sqrt[n]{a}-1<\varepsilon\s\forall n\geq N\in\mb{N}\]
  Sei $x_n=\sqrt[n]{a}-1>0$ und $n\geq 1$
  \[a=(1+x_n)^n=1+nx_n+\binom{n}{2}x^2_n+\binom{n}{3}x_n^3+\dots+x_n^n\, .\]
  Deswegen\[a\geq 1+nx_n\s x_n\leq\frac{a-1}{n}\s\text{für}\s\]
  Sei $\varepsilon>0$. Wähle $N\geq \frac{a-1}{\varepsilon}$
  \[\implies\sqrt[n]{a}-1=x_n\leq\frac{a-1}{n}\leq\frac{a-1}{N}<\frac{a-1}{\frac{a-1}{\varepsilon}}=
  \varepsilon\]
  {\bf Fall $0<a<1$} Wir haben $\frac{1}{a}>1$ und nuzten die Rechenregeln (siehe Satz 
\ref{s:Regeln_Folgen}(iii), unten!):
\[\sqrt[n]{a}=\frac{1}{\sqrt[n]{\frac{1}{a}} \to \frac{1}{1} = 1}\]
 {\bf Fall $a=1$} Trivial! Die Folge ist ``konstant'': $a_n=1 \forall n$.
\end{Bsp}
\begin{Bsp}
  $\lim_{n\to+\infty}\sqrt[n]{2} =1$. Wie oben
  \[x_n=\sqrt[n]{n}-1\]
  \[n=(1+x_n)^n=1+nx_n+\binom{n}{2}x_n^2+\dots+x_n^n\]
Hier wir nuzte die st\"arkere Ungleichung: ($n\geq 2$)
\[n\geq1+\binom{n}{2} x_n^2=1+\frac{n(n-1)}{2}x_n^2 \]
\[x_n^2\leq\frac{2}{n}\implies x_n\leq \sqrt[2]{\frac{2}{n}}\]
Sei $\varepsilon>0$, wähle $N$ so dass
  \[\sqrt{\frac{N}{2}}>\varepsilon^{-1}\qquad (\iff N> 2\varepsilon^{-2})\]
Dann, f\"ur $n\geq N$,
  \[0\geq\sqrt[n]{n}-1<\sqrt{\frac{2}{n}}\leq\sqrt{\frac{2}{N}}<\sqrt{\frac{2}{\frac{2}{\varepsilon^2}}}=\varepsilon\]
  \[\implies \abs{\sqrt[n]{n}-1}<\varepsilon\]
\end{Bsp}
\begin{Ueb} Sei $k\in \mb{N}$. Dann $\lim_n \sqrt[n]{n^k} =1$.
\end{Ueb}
\begin{Bsp}
  Sei $q\in\mb{C}$ mit $\abs{q}<1$. Dann,
  \[\lim_{n\to+\infty}q^n=0\]
In der Tat \[\abs{q^n-0}=\abs{q^n}-\abs{0}\leq\abs{q}^n\] 
Sei nun $\varepsilon>0$. Da $\sqrt[n]{\varepsilon} \to 1$ und $|q|<1$, $\exists N$ s.d.
\[|\sqrt[n]{\varepsilon} - 1| < 1- |q| \qquad\forall n\geq N\]
Deswegen, f\"ur $n\geq N$,
$$
\sqrt[n]{\varepsilon}> 1-(1-|q|) = |q| \quad \implies \quad
\varepsilon > |q|^n\, .
$$
\end{Bsp}
\begin{Ueb}
Sei $k\in \mb{N}$ und $q\in \mb{C}$ mit $|q|<1$. Dann
$$
\lim_{n\to\infty} n^k q^n\;=\; 0\, .
$$
\end{Ueb}

\subsection{Rechenregeln}
\begin{Sat}\label{s:Regeln_Folgen}
  Seien $(a_n)$ und $(b_n)$ zwei konvergente Folgen, mit $a_n\to a$ und $b_n\to b$, dann:
  \begin{itemize}
     \item[(ii)] $a_n+b_n\to a+b$
     \item[(i)] $a_n b_n\to ab$
     \item[(iii)] $\frac{a_n}{b_n}\to\frac{a}{b}$ falls $b\neq 0$
  \end{itemize}
\end{Sat}
\begin{proof}[Beweis vom Satz \ref{s:Regeln_Folgen}(i)]
  \begin{equation}\label{e:(4.1)}
\abs{\left( a_n+b_n \right)-\left( a-b \right)}=\abs{\left( a_n-a \right)+\left( b_n-b 
  \right)}\leq\abs{a_n-a}+\abs{b_n-b}
\end{equation}
  Sei $\varepsilon>0$:
  \begin{equation}\label{e:(4.2)}
\exists N:\qquad\abs{a_n-a}<\frac{\varepsilon}{2}\quad\forall n\geq N
  \end{equation}
\begin{equation}\label{e:(4.3)}
\exists N':\qquad\abs{a_n-a}<\frac{\varepsilon}{2}\quad\forall n\geq N'
\end{equation}
F\"ur $n\geq \max \{N, N'\}$:
$$
|(a_n+b_n)-(a+b)| \stackrel{\eqref{e:(4.1)},\eqref{e:(4.2)} \& \eqref{e:(4.3)}}{<}
\frac{\varepsilon}{2}+\frac{\varepsilon}{2} = \varepsilon\, .
$$
\end{proof}
\begin{Def}
  Eine Folge heisst beschränkt, falls
\begin{equation}\label{e:besch}
\exists M>0:\qquad \abs{a_n}\leq M\quad \forall N\, .
\end{equation}
\end{Def}
\begin{Lem}\label{l:kon->bes}
  Eine konvergente Folge ist immer beschränkt.
\end{Lem}
\begin{proof}[Beweis]
Sei $a_n\to a$. Dann $\exists N$ s.d. $|a_n-a|<1$ $\forall n\geq N$.
Deswegen, $|a_n|< |a|+1$ $\forall n\geq N$. W\"ahlen wir
$$
M := \max \{|a_0|, \ldots, |a_{N-1}|, |a|+1\}\, .
$$
Dann $|a_n|\leq M$ $\forall n$.
\end{proof}
\begin{proof}[Beweis vom Satz \ref{s:Regeln_Folgen}(ii)\&(iii)]
{\bf (ii)} Wegen des Lemmas \ref{l:kon->bes} $\exists M>0$ die \eqref{e:besch}
erf\"ullt
  \begin{eqnarray}
\abs{a_nb_n-ab} &=& \abs{a_nb_n-a_nb+a_nb-ab}
    = \abs{a_n(b_n-b)+b(a_n-a)}\nonumber\\
    &\leq& \abs{a_n}\abs{b_n-b}+\abs{b}\abs{a_n-a}
\leq M\abs{b_n-b}+\abs{b}\abs{a_n-a}\label{e:(4.4)}
\end{eqnarray}
W\"ahle
\begin{eqnarray*}
&&N\in \mb{N} : \qquad |b_n-b|\leq \frac{\eps}{2M} \quad \forall n\geq N\label{e:(4.5)}\\
&&N\in \mb{N'} : \qquad |a_n-a|\leq \frac{\eps}{2|b|} \quad \forall n\geq N'\label{e:(4.6)}
\end{eqnarray*}
F\"ur $n\geq \max \{N, N'\}$ gilt 
$$
|a_nb_n-ab| \stackrel{\eqref{e:(4.4)},\eqref{e:(4.5)} \& \eqref{e:(4.6)}}{<}
\frac{\varepsilon}{2}+\frac{\varepsilon}{2} = \varepsilon
$$ 

\medskip
{\bf (iii)} Folgt aus (ii) und
\begin{equation}\label{e:inv}
 \frac{1}{b_n}\to\frac{1}{b} \qquad \mbox{falls $b_n\to b\neq 0$}
\end{equation}
Um \eqref{e:inv} zu beweisen:
\begin{equation}\label{e:(4.7)}
\left|\frac{1}{b_n}-\frac{1}{b}\right|
=\abs{\frac{b-b_n}{b_nb}}
=\frac{1}{\abs{b}}\frac{\abs{b-b_n}}{\abs{b_n}}
\end{equation}
Da $|b|>0$ und $b_n\to b$,
  \[\exists N:\qquad\abs{b_n-b}<\frac{\abs{b}}{2}\quad\forall n\geq N\]
Deswegen, f\"ur $n\geq N$,
\begin{equation}\label{e:(4.8)} 
\abs{b_n}\geq \abs{b}-\abs{b-b_n}\geq \frac{\abs{b}}{2}>0
\end{equation}
und 
$$
\left|\frac{1}{b_n}-\frac{1}{b}\right| \leq \frac{2}{|b|^2} |b_n-b|\, .
$$
Sei $\varepsilon > 0$ und w\"ahle $N'$ s.d. $|b_n-b| < \eps |b|^2/2$ forall
$n\geq N$. F\"ur $n\geq \max\{N, N'\}$ schliessen wir 
$$
\left|\frac{1}{b_n} - \frac{1}{b}\right| < \eps\, .
$$ 
\end{proof}

\begin{Bem} Falls $a_n\to a$ und $\lambda\in \mb{C}$, folgt aus dem Satz \ref{s:Regeln_Folgen}(ii)
dass $\lambda a_n \to \lambda a$: wir setzen eifach $b_n := \lambda$ $\forall n$!
\end{Bem}


\begin{Sat}
  Sei $a_n\to a$ ($a_n\in\mb{C}$), dann:
  \begin{itemize}
    \item $\abs{a_n}\to\abs{a}$
    \item $\bar{a_n}\to\bar{a}$
    \item $\Re a_n\to\Re a$
    \item $\Im a_n\to\Im a$
  \end{itemize}
\end{Sat}
\begin{proof}[Beweis] Die Behauptungen sind triviale Konsequenzen des
Konvergenzkriteriums \eqref{e:konvergiert} und der folgenden Ungleichungen:
  \begin{itemize}
    \item $\abs{\abs{a_n}-\abs{a}}\leq \abs{a_n-a}$
    \item $\abs{\bar{a_n}-\bar{a}}= \abs{a_n-a}$
    \item $\abs{\Im a_n-\Im a}\leq \abs{a_n-a}$
    \item $\abs{\Re a_n-\Re a}\leq \abs{a_n-a}$
  \end{itemize}
\end{proof}
\begin{Sat}
  Seien $a_n\to a$, $b_n\to b$ mit $a_n\leq b_n$. Dann $a\geq b$.
\end{Sat}
\begin{proof}[Beweis] Sei $\varepsilon>0$. Dann
\begin{eqnarray*}
&&\exists N\in \mb{N}: \qquad |a_n-a|< \varepsilon \quad \forall n\geq N\\
&&\exists N'\in \mb{N}: \qquad |a_n-a|< \varepsilon \quad \forall n\geq N'
\end{eqnarray*}
F\"ur $n=\max \{N, N'\}$:
$$
b-a\geq b_n - |b_n-b| - a_n - |a-a_n| \geq (b_n-a_n) + 2\varepsilon \geq 2\varepsilon\, . 
$$
Da $\varepsilon$ eine beliebige positive Zahl ist, gilt $b-a\geq 0$. 
\end{proof}

\begin{Sat}
  Seien $a_n\to a$, $b_n\to a$. Sei $(c_n)$ mit $a_n\geq c_n\geq b_n$. 
Dann ist $(c_n)$ eine konvergente Folge mit $c_n\to a$
\end{Sat}
\begin{proof}[Beweis]
Sei $\eps>0$ und w\"ahle
\begin{eqnarray*}
&& N\in \mb{N}: \qquad |a_n-a|< \varepsilon \quad \forall n\geq N\\
&& N'\in \mb{N}: \qquad |a_n-a|< \varepsilon \quad \forall n\geq N'
\end{eqnarray*}
F\"ur $n\geq \max \{N, N'\}$:
$$
a- \varepsilon < a - |a -a_n| \leq a_n = a_n \leq c_n
\leq b_n \leq a + |b_n-a| < a+\varepsilon\, .
$$
$$
\implies |c_n-a|<\varepsilon\, .
$$
\end{proof}

\begin{Bsp} Sei $s\geq 0$ und w\"ahle $k\in \mb{N}$ mit $k\leq s\leq k+1$.
Sei $q\in \mb{C}$ mit $|q|<1$. Dann 
\begin{eqnarray*}
&&\sqrt[n]{n^k}\leq \sqrt[n]{n^s}\leq \sqrt[n]{n^{k+1}}\\
&& 0\leq n^s|q|^n \leq n^k |q|^n\, .
\end{eqnarray*}
Deswegen $\sqrt[n]{n^s} \to 1$ und $n^k q^n \to 0$.
\end{Bsp}

