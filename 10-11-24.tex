
\begin{Bew}
  Sei $x\in I$. Die Differenzierbarkeit an dieser Stelle bedeutet:
  \[\Limo{h}\left|\frac{f(x+h)-f(x)}{h}-f'(x)\right|=0\, .\]
Deswegen m\"ussen wir beweisen dass
  \[\Limo{h}\underbrace{\left|\sum^\infty_{h=0}\left( \frac{f_n(x+h)-f_n(x)}{h}-f_n'(x) \right)
\right|}=0\, .\]
F\"ur jede $N\in \mb{N}$ und jede $h$ mit $x+h\in I$ gilt:
  
\[D\leq \underbrace{\left|\sum_{n=0}^N\left(\frac{f_n(x+h)-f_n(x)}{h}-f'_n(x)\right)\right|}_{A} + 
\underbrace{\left|\sum_{n=N+1}^\infty\left(\frac{f_n(x+h)-f_n(x)}{h}-f_n(x)\right)\right|}_{B}\]
Sei $\varepsilon>0$ gegeben. Wir zeigen dass $\exists N\in \mb{N}$ und 
$\exists \bar h > 0$ s.d.
\[
A<\frac{\varepsilon}{2} \qquad \mbox{und}\qquad  
B<\frac{\varepsilon}{2} \qquad \forall h \mbox{ mit $|h|< \bar{h}$}.
\] 
Zuerst w\"ahlen wir $N$. 
\[B\leq \sum^\infty_{n=N+1}\left\{ \left|\frac{f_n(x+h)-f_n(x)}{h}\right|+\abs{f_n'(x)} \right\}\]
\[\stackrel{Schrankensatz}{\leq} 
\sum^\infty_{n=N+1}\left\{ \Norm{f_n'}_{C^0 (I)}+\Norm{f_n'}_{C^0} \right\}\]
\[= \sum^\infty_{n=N+1}2\Norm{f_n'}_{C^\circ(I)}=
2\left\{\overbrace{\sum^\infty_{n=0} \Norm{f_n'}_{C^\circ(I)}}^{b} - 
\overbrace{\sum^N_{n=0}\Norm{f_n'}_{C^\circ(I)}}^{b_N\to b}\right\}\]
  \[<\frac{\varepsilon}{2}\s\text{für $N$ gross genug\, .}\]
Eigentlich, diese Wahl von $N$ garantiert dass $A<\eps/2$ f\"ur {\em jede} $h$.

Nun w\"ahlen wir $\bar{h}$.
\[A=\left|\sum_{n=0}^N\underbrace{\left( \frac{f_n(x+h)-f_n(x)}{h}-f_n'(x) \right)}_{\rightarrow 0}\right|
\to 0 \qquad \mbox{f\"ur $h\to 0$}\]
Deswegen, $\exists \bar{h}>0$ s.d. $A< \eps/2$ wenn $|h|< \bar{h}$.
\end{Bew}
\subsection{Ableitungen h\"oherer Ordnung und Taylorreihe}

\begin{Def}
  Eine Funktion $f$ ist 2 mal differenzierbar an einer Stelle $x\in I$ wenn:
  \begin{itemize}
    \item $f'$ existiert $\forall y\in J$, wobei $x\in ]a,b[$ $\leftrightarrow$ $x \in$ Innern von
J, $J=[x,\tilde b[$ falls $I=[x,b[$ und $J= ]\tilde a, x]$ falls $I=]a, x]$)
    \item $f'$ ifferenzierbar in $x$ ist.
  \end{itemize}
  \[(f')' (x)=:f''(x)\qquad s\text{ist die Ableitung zweiter Ordnung}\]
  Induktiv: $f$ $n$-mal differenzierbar in $x$ falls:
  \begin{itemize}
    \item $f^{(n-1)}$ (d.h. die Ableitung $n-1$-ter Ordnung von $f$) 
in einer Umgebung von $x$ existiert
    \item $f^{(n-1)}$ differenzierbar in $x$ ist. 
  \end{itemize}
\[f^{(n)}(x):=\left( f^{(n-1)} \right)'(x) \qquad \mbox{ist die Ableitung $n$-ter Ordnung}.
 \]
Eine Funktion heisst belieb mal differenzierbar auf $I$ falls die Ableitung
aller Ordnungen auf jeder Stelle existieren.
\end{Def}
\begin{Bem}\label{b:Taylor}
  $f(x)=\sum^\infty_{n=0}a_nx^n$ mit Konvergenzradius $R$.
Dann ist $f$ beliebig mal differezierbar auf $]-R,R[$. Ausserdem, k\"onnten wir
$f^{(k)} (x)$ wie folgt bestimmen:
  \[f^{(k)}(x)=\sum^\infty_{n=k}n(n-1)(n-2)\cdots(n-k+1)a_nx^{n-k}\]
Es folgt dass 
\[f(0)=a_0\]
\[f'(0)=a_1\]
\[f^{(k)}(0)=k!a_k\]
\end{Bem}
\begin{Def}
  Eine Funktion $f$ heisst analytisch an einer Stelle $x_0$, falls auf einem Intervall 
$]x_0-\rho, x_0+\rho[$ gilt
\[f(x)=\sum a_n(x-x_0)^n\]
\end{Def}
Die Bemerkung \ref{b:Taylor} hat deswegen die folgende Konsequenz:
\begin{Kor}\label{k:Taylor}
  Sei $f$ analytisch in $x_0$. Dann $\exists \rho>0$ s.d.:
\begin{itemize}
 \item $f$ beliebig mal differenzierbar auf $I=]x-\rho, x+\rho[$ ist
  \item $f(x)=\sum\infty_{n=0}\frac{f^{(n)}(x_0)}{n!}(x-x_0)^n$ $\forall x\in I$.
\end{itemize}
\end{Kor}

Aber Vorsicht:  {\em Beliebig mal differenzierbar $\not\implies$ analytisch!}

\begin{Bsp}
  \[x_0=0\]
  \[f(x)=e^x=f'(x)=f''(x)=\cdots\implies f^{(k)}(x)=e^x\]
  \[\implies f^n(0)=1\implies e^x=\sum\frac{x^k}{k!}\]
\end{Bsp}
\subsection{Konvexität}
\url{http://de.wikipedia.org/wiki/Konvexe_und_konkave_Funktionen}
\begin{Def}
  Eine $f:I\to R$ heisst konvex, wenn:
  $\forall x_1 < x_2\in I$
  \begin{equation}\label{e:kk}
    f(x)\leq\frac{x-x_1}{x_2-x_1}f(x_2)+\frac{x_2-x}{x_2-x_1}f(x_1)=g(x)\s\forall x\in ]x_1,x_2[
  \end{equation}
  \begin{tabular}{l|c|l}
    streng konvex & $<$\\
    konkav & $\geq$& in \eqref{e:kk}\\
    streng konkav & $>$\\
  \end{tabular}
\end{Def}
\begin{Bem}
Allgemein, die Konvexit|'at impliziert nicht die Differenzierbarkeit.
Nehmen Sie z.B. $f(x)= |x|$ auf $\mb{R}$. 
\end{Bem}
\begin{Sat}\label{s:1011242}
  Sei $f:I\to\mb{R}$ stetig und differenzierbar im Inneren
  \[f\s\text{konvex} \iff f'(x_1)\leq f'(x_2)\s\forall x_1<x_2\]
  \[f\s\text{streng konvex} \iff f'(x_1) < f'(x_2)\s\forall x_1<x_2\]
\end{Sat}
\begin{Kor}
  Sei $f$ wie im Satz \ref{s:1011242} aber 2-mal differenzierbar im Inneren
  \[f\s\text{konvex}\iff f''\geq 0\]
  \[f\s\text{streng konvex}\Leftarrow f''> 0\]
\end{Kor}
\begin{Bsp}
Sei  $f(x)=x^4$. $f$ ist streng konvex und $f''(x)=12x^2$. Deswegen $f''(0)=0$
\end{Bsp}
\begin{Bem}
  Sei $f$ differenzierbar überall und 2 mal differenzierbar ain einer Stelle $x_0$ mit $f'(x_0)=0$. 
Falls 
  \begin{itemize}
    \item $f''(x_0)>0$ $\implies$ $x_0$ ist ein lokales Minimum
    \item $f''(x_0)<0$ $\implies$ $x_0$ ist ein lokales Maximum
  \end{itemize}
Nehmen z.B. dass $f'(x_0)=0$, $f''(x_0)>0$. Dann $\exists \varepsilon$ so dass
  \[f'(x)>0\s\forall x\in ]x_0-\varepsilon,x_0[\]
und
  \[f'(x)<0\s\forall x\in ]x_0,x_0+\varepsilon[\]
In der Tat,
  \[\lim_{x\to x_0}\frac{f'(x)-f'(x_0)}{x-x_0}=f''(x_0)\implies\lim_{x\to x_0}
\frac{f`(x)}{x-x_0} =f''(x_0)>0\]
  \[\implies\exists\varepsilon:\frac{f'(x)}{x-x_0}>0\s\forall x
\in]x_0-\varepsilon,x_0+\varepsilon[\setminus\left\{ x_0 \right\}\]
  \[\implies f'(x)>0\s\forall x\in ]x_0,x_0+\varepsilon[\]
  \[\implies f'(x)<0\s\forall x\in ]x_0-\varepsilon,x_0[ .\]
\end{Bem}
\begin{Lem}
  \[\eqref{e:kk}\iff f(\lambda x_1+(1-\lambda)x_2)\leq\lambda f(x_1)+(1-\lambda)f(x_2)\s\forall x_1 < x_2\s\forall \lambda\in ]0,1[\]
\end{Lem}
\begin{Bew}
  $x_1<x_2$
\begin{equation}\label{e:X3}
f(x)=\frac{x_2-x}{x_2-x_1}f(x_1)+\frac{x-x_1}{x_2-x_1}f(x_2)\s
\forall x\in ]x_1,x_2[
\end{equation}
Wir setzen $\lambda = \frac{x_2-x}{x_2-x_1}$
  \[\forall x\in ]x_1,x_2[\implies \lambda=\frac{x_2-x}{x_2-x_1}\in ]0,1[\]
  \[\forall \lambda\in ]0,1[\implies x=\lambda x_1+(1-\lambda)x_2\in ]x_1,x_2[\]
  \[\lambda=\frac{x_2-x}{x_2-x_1}\iff \lambda(x_2-x_1)=x_2-x\iff x=\lambda x_1+(1-\lambda)x_2\]
Wir schliessen dass die Abbildung
  \[ ]0,1[\ni\lambda\mapsto \lambda x_1+(1-\lambda x_2)\in  ]x_1,x_2[\]
bijektiv ist. Deswegen wir k\"onnen $\lambda$ statt $x$ in der Identit\"at \eqref{e:X3}
nutzen.
Aber 
\[\lambda=\frac{x_2-x}{x_2-x_1}\iff 1-\lambda=1-\frac{x_2-x}{x_2-x_1}
=\frac{\not x_2-x_1-\not x_2+x}{x_2-x_1}=\frac{x-x_1}{x_2-x_1}\]
Deswegen ist \eqref{e:X3} equivalent zu
  \[f(\lambda x_1+(1-\lambda)x_2)\leq \lambda f(x_1)+(1-\lambda)f(x_2)\]
\end{Bew}
\begin{Lem}\label{l:1011293}
  $f:I\to \mb{R}$ ist genau dann konvex wenn für jedes Tripel $x_1<x<x_2\in I$ die folgende Ungleichung gilt:
\begin{equation}\label{e:konv}
\frac{f(x)-f(x_1)}{x-x_1}\leq \frac{f(x_2)-f(x)}{x_2-x}
\end{equation}
$f$ ist genau dann streng konvex wenn f\"ur jedes Tripel $x_1<x<x_2$ die echte Ungleichung
in \eqref{e:konv} gilt.
\end{Lem}
\begin{Bew}
  \[\frac{f(x)-f(x_1)}{x-x_1}\leq \frac{f(x_2)-f(x)}{x_2-x}\]
  \[\iff f(x)\left( \frac{1}{x-x_1}+\frac{1}{x_2-x} \right)\leq \frac{f(x_1)}{x-x_1}+\frac{f(x_2)}{x_2-x}\]
  \[\iff f(x)\frac{x_2-x+x-x_1}{(x-x_1)(x_2-x)}\left( \frac{(x_2-x)(x-x_1)}{x_2-x_1} \right)\]
  \[ \leq f(x_1)\frac{x_2-x}{x_2-x_1}+f(x_2)\frac{x-x_1}{x_2-x_1}\]
  \[\iff f(x)\leq f(x_1)\frac{x_2-x}{x_2-x_1}+f(x_2)\frac{x-x_1}{x_2-x_1}\]
\end{Bew}
