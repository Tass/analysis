
\begin{Def}
  Sei $f:D\to\mb{R}(\mb{C})$, $D\subset \mb{R}(\mb{C})$. 
Sei $x_0\in D$ ein Häufungspunkt. 
Der Grenzwert von $f$ (falls er existiert) an der Stelle 
$x_0$ ist die einzige Zahl $a\in\mb{R}(\mb{C})$ so dass
  \[F(x)=\begin{cases}
    f(x)&x\in D\setminus \left\{ x_0 \right\}\\
    a&x=x_0
  \end{cases}\]
  stetig in $x_0$ ist.
\end{Def}
\begin{Bem}
  $f(x_0)=a$ und $x_0\in D$ $\implies$ $f$ ist stetig an der Stelle $x_0$.
Aber nicht unbedingt $f(x_0)=a$!
\end{Bem}
\begin{Sat}\label{s:Ch}
  Die folgenden Aussagen sind äquivalent:
  \begin{itemize}
    \item $\lim_{x\to x_0}f(x)=a$
    \item $\forall \left\{ x_n \right\}\subset D\setminus\left\{ x_0 \right\}$ mit $x_n\to x_0$ gilt $\lim_{n\to +\infty} f(x_n)=0$
    \item $\forall \varepsilon >0$ $\exists\delta>0$ so dass $\abs{x-x_0}<\delta$ und $x\in D\setminus\left\{ x_0 \right\}$ $\implies$ $\abs{f(x)-a}=0$
  \end{itemize}
\end{Sat}
\begin{proof}[Beweis] Die sind triviale Folgerungen der 
Definitionen und des Folgenkriteriums f\"ur die Stetigkeit von $f$.
\end{proof}

\begin{Sat}
  (Rechenregeln) $f,g:D\to\mb{R}(\mb{C})$, $x_0$ Häufungspunkt von $D$
  \[\lim_{x\to x_0}(f+g)(x)=\left( \lim_{x\to x_0} f(x) \right) + \left( \lim_{x\to x_0} g(x_0) \right)\]
  falls die Grenzwerte existieren!
  \[\lim_{x\to x_0}(fg)(x)=\left( \lim_{x\to x_0} f(x) \right) \left( \lim_{x\to x_0} g(x_0) \right)\]
  \[\lim_{x\to x_0}\frac{f}{g}(x)=
\frac{\lim_{x\to x_0} f (x)}{\lim_{x\to x_0} f(x)}\s
\text{falls}\s\lim_{x\to x_0}g(x)\neq 0\]
\end{Sat}
\begin{Sat}
  Seien $f:D\to E$, $g:E\to \mb{R} (\mb{C})$ mit
  \begin{itemize}
    \item $x_0$ Häufungspunkt von $D$ und $y_0=\lim_{x\to x_0}f(x)$
    \item $y_0\in E$ und $g$ ist stetig and der Stelle $y_0$
  \end{itemize}
  Dann:
  \[\lim_{x_\to x_0} g\circ f(x)=g(y_0)=g(\lim_{x\to x_0}f(x))\]
\end{Sat}
\begin{proof}[Beweis]
  Wenden Sie die entsprechenden Rechenregeln für Folgen.
Als Beispiel:
Teil 1 von Satz 3.Für $f,g$ wir haben: $\forall \left\{ x_n \right\}\subset D\setminus \left\{ x_0 \right\}$ mit $x_n\to x_0$
  \[\lim_{n\to \infty} f(x_n)=\lim_{x\to x_0}f(x_0)\wedge\lim_{n\to \infty} g(x_n)=\lim_{x\to x_0}g(x_0)\]
  \[\implies \lim_{n\to\infty}(f+g) (x_n)=\overbrace{\lim_{n\to\infty}f(x_n)}^{\lim_{x\to x_0}f(x)}+\overbrace{\lim_{n\to \infty}g(x_n)}^{\lim_{x\to x_0}g(x)}\]
  \[\stackrel{\text{Satz \ref{s:Ch}}}{\implies}\lim_{x\to x_0}(f+g)(x)=\lim_{x\to x_0} f(x)+\lim_{x\to x_0} g(x)\]
\end{proof}
\begin{Def}
  Falls $f:D\to\mb{R}$ und $x_0$ ein Häufungspunkt von $D$ ist, dann:
  \begin{itemize}
    \item $\lim_{x\to x_0}f(x)=+\infty(-\infty)$ falls $\forall \left\{ x_n \right\}\subset D\setminus \left\{ x_0 \right\}$ mit $x_n\to x_0$ gilt $\lim_{n\to +\infty}f(x_n)=+\infty$ (bzw. $-\infty$)
  \end{itemize}
  Ähnlich $f:D\to\mb{C}$ und:
  \begin{itemize}
    \item $D$ ist nicht nach oben beschränkt. Wir schreiben $\lim_{x\to+\infty}f(x)=a$ genau dann, wenn $\forall\left\{ x_n \right\}\subset D$ mit $x_n\to\infty$ gilt $\lim_{n\to+\infty}f(x_n)=a$
  \end{itemize}
  gleich wenn $D$ nicht nach unten beschränkt ist. $\lim_{x\to-\infty}f(x)=a$
\"Ahnlicherweise handelt man die F\"alle
  \[\lim_{x\to+\infty}f(x)=\pm\infty\]
  \[\lim_{x\to-\infty}f(x)=\pm\infty\]
\end{Def}
\begin{Def}
  Seien $D\subset\mb{R}$, $f:D\to\mb{R}(\mb{C})$ und $x_0$ ein Häufungspunkt 
von $\left] -\infty,x_0 \right[\cap D$. Dann $\lim_{x\uparrow x_0}f(x)=a$ falls 
\[\forall \left\{ x_n \right\}\subset D\cap \left] -\infty, x_0 \right[ 
\;\;\mbox{mit } x_n\to x \quad\mbox{gilt}\quad 
\lim_{n\to+\infty}f(x_n)=a\] 
Man schreibt auch
\[
\lim_{x\to x_0^-} f(x)=a
\]
Falls $x_0$ ein Häufingspunkt von $D\cap \left] x_0+\infty \right[$ ist
  \[\lim_{x\downarrow x_0}f(x)=a \qquad \left(\lim_{x\to x_0^+} f(x)\right)\]
  falls $\forall \left\{ x_n \right\}\subset D\cap \left] x_0, +\infty \right[$ 
mit $x_n\to x_0$ gilt $f(x_n)\to a$. Ähnlich definiert
man $\lim_{x\to x_0^\pm}f(x)=\pm\infty$.
\end{Def}
\begin{Bsp}
  Stetigkeit:
  \[\lim_{x\to x_0}f(x)=f(x_0)\]
  \begin{itemize}
    \item $\exists \lim_{x\to x_0}$ $f(x)\neq f(x_0)$ $\implies$ 
$f$ in $x_0$ nicht stetig ist
\item $\lim_{x\to x_0}f(x)=+\infty$: die Funktion $f$ in $x_0$ hat eine Asymptote.
  \end{itemize}
\end{Bsp}
\section{Exponentialfunktion}
Sei $a\in\mb{R}$ $a>0$. Dann $a^q=a^\frac{m}{n}=\sqrt[n]{m}$, f\"ur jede
$q=\frac{m}{n}\in\mb{Q}$. Ziel dieses Kapitel ist die Funktion $a^z$
auf der ganzen komplexen Ebene zu definieren.

\subsection{Existenz und Eindeutigkeit}
\begin{Sat}\label{s:Exp}
  $\exists ! \Exp:\mb{C}\to\mb{C}$ mit folgenden Eigenschaften:
  \begin{itemize}
    \item[(AT)] Additionstheorem $\Exp(z+w)=\Exp(z)\Exp(w)$
    \item[(WT)] Wachstum $\lim_{z\to 0}\frac{\Exp(z)-1}{z}=1$
  \end{itemize}
  Für $\Exp$ wissen wir:
  \begin{itemize}
    \item $\Exp(z)=\sum_{n=0}^\infty\frac{z^n}{n!}$ $\forall z\in\mb{C}$
    \item $\Exp(z)=\lim_{n\to+\infty}\left( 1+\frac{z}{n} \right)^n$ $\forall z\in \mb{C}$
    \item $\Exp$ ist stetig und falls $e=\Exp(1)$ dann $e^q=\Exp(1)$ $\forall q\in \mb{R}$,
wobei 
  \[e=\sum\frac{1}{n!}=\lim_{n\to+\infty}\left( 1+\frac{1}{n} \right)^n\]
 \end{itemize}
\end{Sat}
\begin{Bem}\label{b:Kern}
  Kernidee: Wir suchen eine Funktion $\Exp(z)=f(z)$ mit den Eigenschaften
(AT) und (WT)
\begin{equation}\label{e:100} f(z) = f\left(\frac{nz}{n}\right)=
f\left(\frac{z}{n}+\frac{z}{n}+\cdots+\frac{z}{n}\right)\stackrel{\text{(AT)}}{=}
f\left(\frac{z}{n}\right)^n
\end{equation}
Wir definieren
  \[f\left( \frac{z}{n} \right)=1+\frac{z_n}{n}\qquad
\mbox{d.h.} \qquad z_n = n \left(f\left(\frac{z}{n}\right) -1\right)\]
F\"ur  $n\to+\infty$ $\frac{z}{n}\to 0$ und aus (WT) schliessen wir
\begin{equation}\label{e:Subs}
\lim_{n\to+\infty}z_n=z\lim_{n\to+\infty}
\frac{f\left( \frac{z}{n} \right)-1}{\frac{z}{n}}=z
\end{equation}
Aus \eqref{e:100} folgt
\begin{equation}\label{e:101}
f(z)=\left( 1+\frac{z_n}{n}\right)^n\implies f(z)
=\lim_{n\to+\infty}\left( 1+\frac{z_n}{n}\right)^n
\end{equation}
D\"urfen wir, wegen \eqref{e:Subs}, $z_n$ durch $z$ in \eqref{e:101}
ersetzen? Die Antwort ist im n\"achsten Lemma enthalten
\end{Bem}
\begin{Lem}
  Fundamentallemma: $\forall \left\{ z_n \right\}\subset\mb{C}$ mit $z_n\to z$ gilt:
  \[\Limi{n}\left( 1+\frac{z_n}{n}\right)^n=\Limi{n}\left( 1+\frac{z}{n}\right)^n=\sum \frac{z^n}{n!}\]
\end{Lem}
\begin{Bem}
  $\sum \frac{z^n}{n!}$ konvergiert auf $\mb{C}$ (und konvergiert deswegen absolut)
\end{Bem}
\begin{proof}[Beweis]
  Das Kriterium von Hadamand:
  \[R:=\frac{1}{\limsup_{n\to+\infty}\sqrt{\frac{1}{n!}}}=+\infty\]
  Das bedeutet:
  \[\lim_{n\to +\infty}\sqrt[n]{\frac{1}{n!}}=0\]
  \[\begin{cases}
    n\s\text{gerade}& n!\geq \underbrace{n(n-1)\cdots \left(\frac{n}{2}+1\right)}_{\frac{n}{2}}
\frac{n}{2}\cdot 1 \geq \left(\frac{n}{2}\right)^{\frac{n}{2}}\\
    n\s\text{ungerade}& n!\geq \underbrace{n(n-1)\cdots \frac{n+1}{2}}_{\frac{n+1}{2}}\frac{n-1}{2}\cdot 1
\geq \left(\frac{n+1}{2}\right)^{\frac{n+1}{2}}\geq \left(\frac{n}{2}\right)^{\frac{n}{2}}\\
  \end{cases}\]
Deswegen
  \[\sqrt[n]{n!}\geq\left(\left(\frac{n}{2}\right)^{\frac{n}{2}}\right)^{1/n}
= \frac{\sqrt[2]{n}}{\sqrt[n]{2}} \to \infty\]
\end{proof}
