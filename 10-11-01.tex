\subsection{Grenzwerte}
\begin{Def}
  Sei $f:D\to\mb{R}(\mb{C})$, $D\subset \mb{R}(\mb{C})$. Sei $x_0\in D$ ein Häufungspunkt. Der Grenzwert von $f$ (falls er existiert) an der Stelle $x_0$ ist die einzige Zahl $a\in\mb{R}(\mb{C})$ so dass
  \[F(x)=\begin{cases}
    f(x)&x\in D\setminus \left\{ x_0 \right\}\\
    a&x=x_0
  \end{cases}\]
  stetig in $x_0$ ist. $\left( f(x_0)\neq a \right)$ falls $x_0\in D$
\end{Def}
\begin{Bem}
  $f(x_0)=a$ und $x_0\in D$ $\implies$ $f$ ist stetig an der Stelle $x_0$
\end{Bem}
\begin{Sat}
  Die folgenden Aussagen sind äquivalent:
  \begin{itemize}
    \item $\lim_{x\to x_0}f(x)=a$
    \item $\forall \left\{ x_n \right\}\subset D\setminus\left\{ x_0 \right\}$ mit $x_n\to x_0$ gilt $\lim_{n\to +\infty} f(x_n)=0$
    \item $\forall \varepsilon >0$ $\exists\delta>0$ so dass $\abs{x-x_0}<\delta$ und $x\in D\setminus\left\{ x_0 \right\}$ $\implies$ $\abs{f(x)-a}<0$
  \end{itemize}
\end{Sat}
\begin{Sat}
  (Rechenregeln) $f,g:D\to\mb{R}(\mb{C})$, $x_0$ Häufungspunkt von $D$
  \[\lim_{x\to x_0}(f+g)(x)=\left( \lim_{x\to x_0} f(x) \right) + \left( \lim_{x\to x_0} g(x_0) \right)\]
  falls die Grenzwerte existieren!
  \[\lim_{x\to x_0}(fg)(x)=\left( \lim_{x\to x_0} f(x) \right) \left( \lim_{x\to x_0} g(x_0) \right)\]
  \[\lim_{x\to x_0}\frac{f}{g}(x)\s\text{falls}\s\lim_{x\to x_0}g(x)\neq 0\]
\end{Sat}
\begin{Sat}
  Seien $f:D\to E$, $g:E\to \mb{R}\mb{C}$ mit
  \begin{itemize}
    \item $x_0$ Häufungspunkt von $D$ und $y_0=\lim_{x\to x_0}f(x)$
    \item $y_0\in E$ und $g$ ist stetig and der Stelle $y_0$
  \end{itemize}
  Dann:
  \[\lim_{x_\to x_0} g\circ f(x)=g(y_0)=g(\lim_{x\to x_0}f(x))\]
\end{Sat}
\begin{Bew}
  Wenden Sie die entsprechenden Rechenregeln für Folgen $x_\to x_0$ $\left( \left\{ x_n \right\}\subset D\setminus \left\{ x_0 \right\} \right)$
\end{Bew}
\begin{Bsp}
  Teil 1 von Satz 3. A $\implies$ B: für $f,g\implies\forall \left\{ x_n \right\}\subset D\setminus \left\{ x_0 \right\}$ mit $x_n\to x_0$
  \[\lim_{n\to \infty} f(x_n)=\lim_{x\to x_0}f(x_0)\wedge\lim_{n\to \infty} g(x_n)=\lim_{x\to x_0}g(x_0)\]
  \[\implies \lim_{n\to\infty}(f+g)=\overbrace{\lim_{n\to\infty}f(x_n)}^{\lim_{x\to x_0}f(x)}+\overbrace{\lim_{n\to \infty}g(x_n)}^{\lim_{x\to x_0}g(x)}\]
  \[\stackrel{\text{Satz 2}}{\implies}\lim_{x\to x_0}(f+g)(x)=\lim_{x\to x_0} f(x)+\lim_{x\to x_0} g(x)\]
\end{Bsp}
\begin{Def}
  Falls $f:D\to\mb{R}$ und $x_0$ ein Häufungspunkt von $D$ ist, dann:
  \begin{itemize}
    \item $\lim_{x\to x_0}f(x)=+\infty(-\infty)$ falls $\forall \left\{ x_n \right\}\subset D\setminus \left\{ x_0 \right\}$ mit $x_n\to x_0$ gilt $\lim_{n\to +\infty}f(x_n)=+\infty$ (bzw. $-\infty$)
  \end{itemize}
  Ähnlich $f:D\to\mb{C}$ und:
  \begin{itemize}
    \item $D$ ist nicht nach oben beschränkt. Wir schreiben $\lim_{x\to+\infty}f(x)=a$ genau dann, wenn $\forall\left\{ x_n \right\}\subset D$ mit $x_n\to\infty$ gilt $\lim_{n\to+\infty}f(x_n)=a$
  \end{itemize}
  gleich wenn $D$ nicht nach unten beschränkt ist. $\lim_{x\to\infty}f(x)=a$
  \[\lim_{x\to+\infty}f(x)=\pm\infty\]
  \[\lim_{x\to-\infty}f(x)=\pm\infty\]
\end{Def}
\begin{Def}
  Sei $D\subset\mb{R}$, $f:D\to\mb{R}(\mb{C})$. Falls $x_0$ ein Häufungspunkt von $\left] -\infty,x_0 \right[\cap D$ ist, dann $\lim{x\uparrow x_0}f(x)=a$ falls $\forall \left\{ x_n \right\}\subset D\cap \left] -\infty, x_0 \right[$ mit $x_n\in D$ $x_n<x_n$ \left( $\lim_{x\to x_0^-}f(x)$ \right) gilt $\lim_{n\to+\infty}f(x_n)=a$. Falls $x_0$ ein Häufingspunkt von $D\cap \left] x_0+\infty \right[$ ist
  \[\lim_{x\downarrow x_0}f(x)=a\wedge(\lim_{\to x_0^+} f(x))\]
  falls $\forall \left\{ x_n \right\}\subset D\cap \left] x_0, +\infty \right[$ mit $x_n\to x_0$ gilt $f(x_n)\to a$. Ähnlich $\lim_{x\to x_0^\pm}f(x)=\pm\infty$.
\end{Def}
\begin{Bsp}
  Stetigkeit:
  \[\lim_{x\to x_0}f(x)=f(x_0)\]
  \begin{itemize}
    \item $\exists \lim_{x\to x_0}$ $f(x)\neq f(x_0)$ wenn die Funktion $f$ in $x_0$ nicht stetig ist
    \item $\lim_{x\to x_0}f(x)=+\infty$ wenn die Funktion $f$ in $x_0$ eine Asymptote hat.
  \end{itemize}
\end{Bsp}
\section{Exponentialfunktion}
$a\in\mb{R} a>0$ $a^a=a^\frac{m}{n}=\sqrt[n]{m}$, $q=\frac{m}{n}\in\mb{Q}$
\subsection{Existenz und Eindeutigkeit}
\begin{Sat}
  $\exists ! \Exp:\mb{C}\to\mb{C}$ mit folgenden Eigenschaften:
  \begin{itemize}
    \item Additionstheorem $\Exp(z+w)=\Exp(z)\Exp(w)$
    \item $\lim_{z\to 0}\frac{\Exp(z)-1}{z}=1$
  \end{itemize}
  Für $\Exp$ wissen wir:
  \begin{itemize}
    \item $\Exp(z)=\sum_{n=0}^\infty\frac{z^n}{n!}$ $\forall z\in\mb{C}$
    \item $\Exp(z)=\lim_{n\to+\infty}\left( 1+\frac{z}{n} \right)^n$ $\forall z\in \mb{C}$
    \item $\Exp$ ist stetig und falls $e=\Exp(1)$ dann $e^q=\Exp(1)$ $forall q\in \mb{R}$
  \end{itemize}
\end{Sat}
\begin{Bem}
  \[e=\sum\frac{1}{n!}=\lim_{n\to+\infty}\left( 1+\frac{1}{n} \right)^n\]
\end{Bem}
\begin{Bem}
  Kernidee:
  \[\Exp(z)=f(z)\]
  \[f(z) = f(\frac{nz}{n})=f(\frac{z}{n}+\frac{z}{n}+\cdots+\frac{z}{n})\stackrel{\text{Add}}{=}f(z)^n\]
  \[f\left( \frac{z}{n} \right)=1+\frac{z_n}{n}=\left( 1+\frac{z_n}{n} \right)^n\]
  \[z_n=n\left( f\left( \frac{z}{n} \right) -1 \right)=\left( \frac{f\left( \frac{z}{n} \right)-1}{\frac{z}{n}} \right)z\]
  $n\to+\infty$ $\frac{z}{n}\to 0$
  \[\lim_{n\to+\infty}z_n=z\lim_n\to{n\to+\infty}\frac{f\left( \frac{z}{n} \right)-1}{\frac{z}{n}}=z\]
  \[f(z)=\left( 1+\frac{z_n}{n}\right)^n\implies f(z)=\lim_{n\to+\infty}\left( 1+\frac{\{overbrace{z_n}^z}{n}\right)^n\stackrel{?}{=}\lim_{n\to+\infty}\left( 1+\frac{z}{n}\right)^n\]
\end{Bem}
\begin{Bem}
  Zusammenfassung: Falls $f$ die Bedingungen Additionstheorem und Wachstum erfüllt, dann:
  \[f(z)=\lim_{n\to +\infty}\left( 1+\frac{z}{n}\right)^n\]
  wobei $\lim_{n\to +\infty}z_n=z$
\end{Bem}
\begin{Lem}
  Fundamentallemma: $\forall \left\{ z_n \right\}\subset\mb{C}$ mit $z_n\to z$ gilt:
  \[\Limi{n}\left( 1+\frac{z_n}{n}\right)^n=\Limi{n}\left( 1+\frac{z}{n}\right)^n=\sum \frac{z^n}{n!}\]
\end{Lem}
\begin{Bem}
  (1)+(2) liefern Eindeutigkeit und zwei Darstellungen: Die Darstellungen definieren eine stetige Funktion mit den ganzen Eigenschaften des Theorems.
\end{Bem}
\begin{Bem}
  $\sum \frac{z^n}{n!}$ konvergiert auf $\mb{C}$ (und konvergiert deswegen absolut)
\end{Bem}
\begin{Bew}
  Das Kriterium von Hadamand:
  \[R:=\frac{1}{\limsup_{n\to+\infty}\sqrt{\frac{1}{n!}}}=+\infty\]
  Das bedeutet:
  \[\lim_{n\to +\infty}\sqrt[n]{\frac{1}{n}}=0\]
  \[\begin{cases}
    n\s\text{gerade}& n!\geq \underbrace{n(n-1)\cdots}_{\frac{n}{2}}\\
    n\s\text{ungerade}& n!\geq \underbrace{n(n-1)\cdots}_{\frac{n-1}{2}}\\
  \end{cases}
  \[n!\geq\frac{n}{2}^{\frac{n}{2}}\]
  $n$ gerade:
  \[0\geq\sqrt[n]{\frac{1}{n!}=\frac{1}{\left( n! \right)^{\frac{1}{n}}}\leq\frac{1}{\left( \left( \frac{n}{2} \right)^{\frac{n}{2} \right)^{\frac{1}{n}}}}=\frac{1}{\sqrt{\frac{1}{2}}}\to 0\]
  $n$ ungerade
  \[0\geq\sqrt[n]{\frac{1}{n!}=\frac{1}{\left( n! \right)^{\frac{1}{n}}}\leq\frac{1}{\left( \left( \frac{n}{2} \right)^{\frac{n-1}{2} \right)^{\frac{1}{n}}}}=\frac{1}{\sqrt{\frac{1}{2}} \left( \frac{1}{2} \right)^{-\frac{1}{2n}}\to 0\]
  % TODO missing
\end{Bew}
