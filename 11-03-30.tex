\begin{Def}
  $X\subset \mb{R}^n$, $\exists:X\to \mb{R}$. $f$ hat in $a\in X$ ein lokales Minimum/Maximum 
  \[\iff\exists a\in V\s\text{(Umgebung)}\s. f(a)\leq f(x) (\text{bzw.}\s \geq f(x)) \forall x\in V\]
  Man sagt das Minimum/Maximum ist \ul{strikt} (oder \ul{isoliert})
  \[\iff f(a)<f(x)(\text{bzw.}\s>f(x))\forall x\in V\setminus \left\{ a \right\}\]
\end{Def}
\begin{Sat}
  (Notwendiges Kriteroium für lokale Extrema). Sei $U\subset\mb{R}^n$ offen, $f:U\to\mb{R}$ haben ein lokales Extremum in $a\in U$ und sei partiell differenzierbar. Dann gilt
  \[\partial_1f(a)=\cdots=\partial_nf(a)=0\]
  D.h. wenn $f$ differenzierbar ist, dann gilt $\md f|_a=0$
\end{Sat}
\begin{Bew}
  $F(t)=f(a+t_{e_i})$ (für $t$ sehr klein, so dass $a+t_{e_i}\in U$ $F$ hat lokale Extrema in 0, d.h. $F'(0)=\partial_1 f(a)=0$
\end{Bew}
\begin{Def}
  $f$ differenzierbar, dann heisst $a$ mit $\md f|_a=0$ \ul{kritischer Punkt}. Man sagt auch $f$ ist \ul{stationär} in $a$. 
\end{Def}
\begin{Bem}
  lokale Extremum $\implies$ $\not\Leftarrow$ kritischer Punkt.
\end{Bem}
\begin{Sat}
  (Hinreichendes Kriterium für lokale Extrema) $U\subset\mb{R}^n$ offen, $f\in C^2(U,\mb{R})$ it $\md f|_a=0$. Dann
  \begin{eqnarray*}
    H_f(a)>0\implies a\s\text{lokales Minimum}\\
    H_f(a)<0\implies a\s\text{lokales Maximum}\\
    H_f(a)\s\text{indefinit}\implies a\s\text{kein Extremum}
  \end{eqnarray*}
  Im indefiniten Fall gilt: $\exists$ Geraden $G_1$, $G_2$ durch $a$ so dass $f|_{G_1\cap U}$ in $a$ ein lokales Minimum und $f|_{G_2\cap U}$ in $a$ ein lokales Maximum hat, d.h. $a$ ist ein Sattelpunkt.
\end{Sat}
\begin{Bem}
  \begin{itemize}
    \item 
      $H_f(a)>0$ bedeutet $H_f(a)$ \ul{positiv definit}, d.h. 
      \[v^TH_f(a)v>0\s\forall v\in\mb{R}\setminus\left\{ 0 \right\}\]
    \item
      $H_f(a)$ indefinit, $\exists v,w\in\mb{R}^n\setminus\left\{ 0 \right\}$ mit
      \[v^tH_f(a)v>0\]
      \[w^tH_f(a)w<0\]
  \end{itemize}
\end{Bem}
\begin{Bew}
  \begin{itemize}
    \item[$H_f(a)>0$]
      \[\md f|_a=0\xRightarrow{\text{Taylor}}f(a+h)=f(a)+\frac{1}{2}h^TH_f(a)h+R(h)\\\]
      mit
      \[\frac{R(h)}{\Norm{h}^2}\to 0 \s(\Norm{H}\to 0)\]
      $f\in C^2$
      \begin{itemize}
        \item $\implies$ $h\mapsto h^TH_f(a)h$ ist stetig
        \item $\implies$ \ldots hat ein Minimum auf $\left\{ \Norm{h}=1 \right\}$ (kompakt), $m > 0$ (da $H_f(a)>0$).
        \item $\implies$ $h^TH_f hh\geq m\Norm{h}^2$ (da $h=\Norm{h}\frac{h}{\Norm{h}}$, $h\neq 0$
      \end{itemize}
      Wähle $\varepsilon>0$ so klein, dass $B_\varepsilon(a)\subset U$
      \[\Abs{R(h)}\leq \frac{m}{4}\Norm{h}^2\s\forall h\in B_\varepsilon(a)\]
      \[\implies f(a+h)\geq f(a)+\frac{m}{4}\Norm{h}^2>f(a)\s\forall h\in B_\varepsilon(a)\]
      d.h. $f$ hat in $a$ ein lokales Minimum
    \item[$H_f(a)<0$]
       Betrachte $-f$ wie oben.
    \item[$H_f(a)<>0$] % TODO
      \[\exists v,w :\s v^T H_f(a) v>0, w^T H_f(a) w<0\]
      \[F_v(t):=f(a+tv), F_w(t)=f(a+tw)\]
      \[\implies F_v''(0)>0 \implies \text{lokales Maximum}\]
      \[\implies F_v''(0)<0 \implies \text{lokales Minimum}\]
      $\implies$ Beh
  \end{itemize}
\end{Bew}
\begin{Bem}
  Mit diesem Satz lässt sich keine Aussage machen, falls $H_f(a)$ semidefinitiv ist, d.h. $H_f(a)\geq 0$, $H_f(a)\leq 0$.
\end{Bem}
\begin{Bsp}
  $f(x,y)=y^2(x-1)+x^2(x+1)$
  \[\md f|_{(x,y)} = (y^2+3x^2+2x, 2(x-1)y)\]
  $\implies$ $\md f|_{(x,y)}=(0,0)$ $\implies$ kritische Punkte:
  \[P_1=(0,0), P_2(-\frac{2}{3}, 0)\]
  \[\implies H_f(x,y)= \begin{pmatrix}
    6x+2&2y\\
    2y&2(x-1)
  \end{pmatrix}\]
  d.h.
  \[\implies H_f(P_1)= \begin{pmatrix}
    2&0\\
    0&-2
  \end{pmatrix}\]
  indefinit, d.h. Sattelpunkt.
  \[\implies H_f(P_2)= \begin{pmatrix}
    -2&0\\
    0&-\frac{10}{3}
  \end{pmatrix}<0\]
  d.h. lokales Maximum
\end{Bsp}
\begin{Bsp}
  $f(x,y)=x^2+y^3$, $g(x,y)=x^2+y^4$
  % TODO insert graph here
  Beim Punkt 0 ist die Hesse-Matrix in beiden Fällen $\begin{pmatrix} 2&0\\ 0&0 \end{pmatrix}$. Daraus kann man nichts schliessen (sehe Graphen (Freiwilliger gesucht))
\end{Bsp}
\subsection{Konvexität}
\begin{Def}
  $U\subset\mb{R}^n$ heisst \ul{konvex}
  \[\iff \forall x,y\in U: \s \left[ x,y \right]\subset U\]
\end{Def}
\begin{Def}
  $f:U\to\mb{R}$ heisst \ul{konvex}
  \[\iff \forall x,y\in U:\s f(tx+(1-t)y)\leq tf(x)+(1-t)f|_y\]
  \begin{itemize}
    \item Falls $\forall x,y\in U$ $\forall t\in (0,1)$ ``$<$'', heisst die Funktion \ul{strikt} konvex.
    \item $f$ heisst (streng) \ul{konkav}, falls $-f$ (streng) \ul{konvex}
  \end{itemize}
\end{Def}
\begin{Bem}
  $f$ ist konvex
  \[\iff \forall x\neq y\in U:\s F_{x,y}(t)=f(x+t(y-x))\s\text{konvex (auf $\left[ x,y \right]$)}\]
\end{Bem}
\begin{Sat}
  (Konvexitätskriterium)
  Sei $f:U\to\mb{R}, C^2$ $U\subset\mb{R}^n$ offen, konvex. Es gilt:
  \begin{enumerate}
    \item $f$ konvex $\iff$ $H_f(x)\geq 0$ $\forall x\in U$
      \label{i:1103301}
    \item $H_f(x)>0$ $\forall x\in U$ $\implies$ $f$ streng konvex
      \label{i:1103302}
  \end{enumerate}
\end{Sat}
\begin{Bem}
  Umkehrung von \ref{i:1103301} gilt nicht, z.B. $f(x,y)=x^4+y^4$
\end{Bem}
\begin{Bew}
  \begin{enumerate}
    \item $f$ konvex: $\forall x\in U$ wähle $r>0$: $B_r(x)\subset U$
      \[\implies F_{x,x+h}(t)\s\text{konvex}\s\forall h\in B_r(0)\]
      \[\implies h^TH_f(x)h=F_{x,x+h}''(0)\underbrace{\geq}_{\text{Konvexität in 1-Dim}} 0\forall h\in B_r(0)\]
      \[\xRightarrow{\text{homogenität}}h^TH_f(x)h\geq 0\s\forall h\in \mb{R}^n, \text{d.h.}\s H_f(x)\s\text{positiv semidefinit}\]
      $H_f(x)\geq 0$ $\forall x\in U$:
      \[a,b\in U\implies F_{a,b}''(t)=(b-a)^TH_f(a+t(b-a))(b-a)\geq 0\]
      \[\implies F_{a,b} \s\text{konvex}\s\forall a,b\in U\implies\text{Behauptung}\]
    \item Analog wie die zweite Richtung im Ersten.
  \end{enumerate}
\end{Bew}
\subsection{Differentation parameterabhängiger Integrale}
$f:U\times [a,b]\to\mb{R}$, $U\subset\mb{R}^n$ offen. Sei $t\to f(x,t)$ stetig. $\forall x\in U$. Definiere
\[F(X):=\int_a^bf(x)\md t\s x\in U\]
\begin{Sat}
  Sei $f$ wie oben und es gelte:
  \begin{enumerate}
    \item $\forall t\in [a,b]$: $x\mapsto f(x,t)$ nach $x_i$ partiell differenzierbar
    \item $(x,t)\mapsto \partial_if(x,t)$ ist stetig auf $U\times [a,b]$
  \end{enumerate}
  $\implies$ $F$ ist nach $x_i$ stetig partiell differenzierbar, und es gilt:
  \[\Part{F}{x_i}(x)=\int_a^b\Part{f}{x_i}(x,t)\md t\]
\end{Sat}
