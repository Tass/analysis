% headers by Alexander Berthold van der Bourg / Pirmin Weigele 

%= Document-Class ==================================================================================
\documentclass[10pt,a4paper]{article}

%= Packages ========================================================================================
\usepackage[utf8]{inputenc}
\usepackage{ngerman,amsmath,amssymb,amsfonts,mathrsfs}
\usepackage{amsthm}
\usepackage{bbm}
\usepackage{ulsy}
\usepackage{epic,eepic,pstricks,pst-node,pst-plot}
\usepackage{pictexwd,dcpic}
\usepackage{pstricks}
\usepackage{colortbl}
\usepackage{graphicx}
\usepackage{makeidx}
\usepackage{fancyhdr}
\usepackage{latexsym}
\usepackage{psfrag}
\usepackage{enumerate}
\usepackage{float}
%\usepackage{mathtext}
\usepackage{dsfont}
\pagestyle{fancy}
\usepackage{multirow, bigdelim, bigstrut}
\usepackage{rotating}
\usepackage{ifthen}
\usepackage{boxedminipage}
\usepackage{mathtools}
\usepackage{ulsy}
\usepackage{trfsigns}
\usepackage{url}
%\usepackage{savetrees}

%= Seiten-Layout =========================================================================
\voffset-22mm \textheight715pt 

%Seitenbreite==============================================================

%\oddsidemargin=-0.2in
%\evensidemargin=-0.4in
%\textwidth=5.2in
%\headwidth=5.2in

%= Index-Befehle ========================================================================
\renewcommand{\indexname}{Stichwortverzeichnis}
\makeindex

%= Befehl-Overwriting =======================================================================
\makeatletter
\makeatother

%= Strings ================================================================
\newcommand{\mainfold}{.}
\newcommand{\prefix}{A1-}

%= Eigene Befehle ==========================================================================
\DeclareMathOperator{\id}{Id}
\DeclareMathOperator{\arccot}{arccot}
\DeclareMathOperator{\arcsinh}{arcsinh}
\DeclareMathOperator{\arccosh}{arccosh}
\DeclareMathOperator{\arctanh}{arctanh}
\DeclareMathOperator{\md}{d}
\DeclareMathOperator{\Grad}{grad}
\DeclareMathOperator{\Spur}{Spur}
\DeclareMathOperator{\Graph}{Graph}
\DeclareMathOperator{\sign}{sign}
\DeclareMathOperator{\Hom}{Hom}
\DeclareMathOperator{\rot}{rot}
\DeclareMathOperator{\Ker}{Ker}
\DeclareMathOperator{\Exp}{Exp}

\newcommand{\Diff}[2]{\displaystyle\frac{\mathrm{d}#1}{\mathrm{d}#2}}
\newcommand{\End}{\hfill{\hbox{$\Box$}}\par\vspace{2mm}}
\newcommand{\eps}{\varepsilon}
\newcommand{\ePic}[1]{\input{\mainfold/graphics/\prefix#1.eepic}}
\newcommand{\pst}[1]{\input{\mainfold/graphics/\prefix#1.pst}}
\newcommand{\pic}[1]{\input{\mainfold/graphics/\prefix#1.pic}}
\newcommand{\Mx}[1]{\begin{pmatrix}#1\end{pmatrix}}
%\newcommand{\im}[1]{\operatorname{Im}(#1)}
%\newcommand{\Include}[4]{\rhead{#2.#3.20#4}\input{\mainfold/lectures/#1-#4-#3-#2.tex}}
\newcommand{\Index}[1]{\emph{#1}\index{#1}}
\newcommand{\Int}[4]{\displaystyle\int\limits_{#1}^{#2}#3\,\mathrm{d}#4}
\newcommand{\diff}[1]{\operatorname{d}\!#1}
\newcommand{\Limi}[1]{\displaystyle\lim_{#1\to+\infty}}
\newcommand{\Limo}[1]{\displaystyle\lim_{#1\rightarrow0}}
\newcommand{\mb}[1]{\mathbb{#1}}
\newcommand{\ds}{\displaystyle}
\newcommand{\ol}[1]{\overline{#1}}
\newcommand{\Part}[2]{\dfrac{\partial #1}{\partial #2}}
\newcommand{\QED}{\hfill{\hbox{(QED)}}\par\vspace{2mm}}
\newcommand{\re}[1]{\operatorname{Re}(#1)}
\newcommand{\s}{\hspace{2mm}}
\newcommand{\vsa}{\vspace{1mm} \\}
\newcommand{\vsb}{\vspace{2mm} \\}
\newcommand{\vsc}{\vspace{3mm} \\}
% \newcommand{\tr}[1]{\textrm{#1}}
\newcommand{\tr}[1]{\text{#1}}
\newcommand{\ra}{\rightarrow}
\newcommand{\Ra}{\Rightarrow}
\newcommand{\Lra}{\Leftrightarrow}
\newcommand{\La}{\Leftarrow}
\newcommand{\ul}[1]{\underline{#1}}
\newcommand{\rsa}{\rightsquigarrow}
\newcommand{\ara}[2]{\autorightarrow{\ensuremath{#1}}{\ensuremath{#2}}}
\newcommand{\dcp}[2]{\begindc{\commdiag}[#1] #2 \enddc}

%\newcommand{\detmx}{\left| \begin{array} #1 \end{array} \right|}

\newcommand{\grad}[1]{\Grad(#1)}
\newcommand{\fr}[2]{\displaystyle\frac{#1}{#2}} % fertiger bullshit, daf�r gibts \dfrac{}{}
\renewcommand{\Re}{\operatorname{Re}}
\renewcommand{\Im}{\operatorname{Im}}

% ---- DELIMITER PAIRS ----
\def\floor#1{\lfloor #1 \rfloor}
\def\ceil#1{\lceil #1 \rceil}
\def\seq#1{\langle #1 \rangle}
\def\set#1{\{ #1 \}}
\def\abs#1{\mathopen| #1 \mathclose|}	% use instead of $|x|$ 
\def\norm#1{\mathopen\| #1 \mathclose\|}% use instead of $\|x\|$ 

% --- Self-scaling delmiter pairs ---
\def\Floor#1{\left\lfloor #1 \right\rfloor}
\def\Ceil#1{\left\lceil #1 \right\rceil}
\def\Seq#1{\left\langle #1 \right\rangle}
\def\Set#1{\left\{ #1 \right\}}
\def\Abs#1{\left| #1 \right|}
\def\Norm#1{\left\| #1 \right\|}

%Adrians Abbildungs-Environment ==============================================

\newcommand{\Sidein}{\begin{rotate}{90}\small$\in$\end{rotate}}

\newcommand{\Abb}[5][]{\ensuremath{
    \begin{array}{lc}
      \ifthenelse{\equal{#1}{}}{}{#1:}\;\; & 
      \begin{xy}
        \xymatrixrowsep{1em}\xymatrixcolsep{2em}%
        \xymatrix{ #2 \ar[r] \ar@{}[d]^<<<<{\hspace{0.001em} \Sidein}
          & #3  \ar@{}[d]^<<<<{\hspace{0.001em} \Sidein} \\
          #4 \ar@{|->}[r] & #5} \end{xy}
    \end{array}
  }%
}

%= Environments ========================================================================
\def\thechapter{\Roman{chapter}}
\def\thesection{\arabic{section}}
\newtheorem{Bew}{Beweis}
\newtheorem{Axi}{Axiom}
\newtheorem{Lem}{Lemma}
\newtheorem{Kor}{Korollar}
\newtheorem{Sat}{Satz}
\newtheorem{Prop}{Proposition}
\theoremstyle{definition}
\newtheorem{Bsp}{Beispiel}
\newtheorem{Def}{Definition}
\newtheorem{Ueb}{Übung}
\theoremstyle{remark}
\newtheorem{Bem}{Bemerkung}
\newtheorem{Eig}{Eigenschaften}
\newtheorem{Not}{Notation}

\def\pstexInput#1{%
  \begin{center}
    \begin{picture}(0,0)%
      \special{psfile=\mainfold/graphics/A2-#1.pstex}%
    \end{picture}%
    \input{\mainfold/graphics/A2-#1.pstex_t}%
  \end{center}
}

%= Titelseite ===========================================================================
\begin{document}
\headheight15pt
\begin{titlepage}
\hfill
\vspace{20mm}
\pagenumbering{roman}
\begin{center}
{\LARGE Analysis I - Vorlesungs-Script} \vskip 3em {\large Prof. Dr. Camillo De Lellis} \vskip 1.5em
{\large Basisjahr 10 Semester II}\vspace{30mm}\\
{\large {\bf Mitschrift:} \vspace{2mm}\\
Simon Hafner}\vspace{5mm}\\ %30mm
%{\large {\bf Graphics:} \vspace{2mm}\\
%Pirmin Weigele }\vspace{30mm}\\ %30mm
\author{Simon Hafner}

\end{center}
\vfill

\end{titlepage}


%= Inhaltsverzeichnis ==========================================================================
\lhead{}
\rhead{}
\tableofcontents
\newpage
\pagenumbering{arabic}
\setcounter{page}{1}

%= Vorlesung-Skripts ==========================================================================
\cfoot{\thepage}
\fancyhead[L]{\nouppercase{\leftmark}}
\newpage

%= Analysis I & & II ==========================================================================

%Analysis I
\section{Die reellen Zahlen}
\begin{Bsp}
  $\mb{R}$ ist nicht genug
\end{Bsp}
\begin{Sat}
  Es gibt kein $q\in\mb{Q}$ so dass $q^2=2$
\end{Sat}
\begin{Bew}
  Falls $q^2=2$, dann $(-q)^2=2$ OBdA $q\geq 0$ Deswegen $q>0$. Sei $q>0$ und $q\in\mb{Q}$ so dass $q^2=2$. $q=\frac{m}{n}$ mit $m>0$, $>0$. $\text{GGT}(m,n)=1$ (d.h. falls $r\in\mb{N}$ $m$ und $n$ dividiert, dann $r=1$!).
  \begin{align*}
    m^2=2n^2&\implies m \text{ ist gerade}&\implies m=2k \text{ für } k\in\mb{N}\\\{0\}\\
    4k^2=2n^2&\implies n \text{ ist gerade}&\implies 2| n \text{(2 dividiert $n$)}
  \end{align*}
  $\implies$ Widerspruch! Weil $2$ dividiert $m$ und $n$! (d.h. es gibt \underline{keine} Zahl $q\in\mb{Q}$ mit $q^2=2$
\end{Bew}
\begin{Bsp}
  \begin{align*}
    \sqrt{2}=1,414\cdots
  \end{align*}
  Intuitiv:
  \begin{align*}
    1,4^2 & < & 2 & < & 1,5^2 & & 1,4 & < & \sqrt{2} & < & 1,5 \\
    1,41^2 & < & 2 & < & 1,42^2 & \implies & 1,41 & < & \sqrt{2} & < & 1,42 \\
    1,414^2 & < & 2 & < & 1,415^2 & & 1,414 & < & \sqrt{2} & < & 1,415 \\
  \end{align*}
\end{Bsp}
\paragraph{Intuitiv}
\begin{itemize}
  \item $\mb{Q}$ hat ``Lücke''
  \item $\mb{R}$ $= \{$ die reellen Zahlen $\}$ haben ``kein Loch''.
\end{itemize}
\paragraph{Konstruktion}
Die reellen Zahlen kann man ``konstruieren''. (Dedekindsche Schritte, Cantor ``Vervollständigung''). Google knows more. Wir werden ``operativ'' sein, d.h. wir beschreiben einfach die wichtigsten Eigenschaften von $\mb{R}$
\subsection{Körperstrukturen}
\begin{itemize}
  \item[K1] Kommutativgesetz
  \begin{align*}
    % TODO align that on =
    a+b &=& b + a &\\
    a\cdot b &=& b\cdot a &\\
  \end{align*}
  \item[K2] Assoziativgesetz
    \begin{align*}
      (a+b)+c &=&a+(b+c)\\
      (a\cdot b)\cdot c&=&a\cdot(b\cdot c)\\
    \end{align*}
  \item[K3] Distributivgesetz
    \begin{align*}
      (a+b)\cdot c&=& a\cdot c + b\cdot c
    \end{align*}
  \item[K4] 
    \begin{align*}
      a+x&=&b\\
      a\cdot x&=&b \text{falls $a\neq 0$}\\
    \end{align*}
\end{itemize}
\subsection{Die Anordnung von $\mb{R}$}
\begin{itemize}
  \item[A1] $\forall a\in\mb{R}$ gilt genau eine der drei Relationen:
    \begin{itemize}
      \item $a<0$
      \item $a=0$
      \item $a>0$
    \end{itemize}
  \item[A2] Falls $a>0$, $b>0$, dann $a+b>0$, $a\cdot b>0$
  \item[A3] Archimedisches Axiom: $\forall a\in\mb{R} \exists n\in\mb{N}$ mit $n>a$
\end{itemize}
\begin{Ueb}
  Beweisen Sie dass $a\cdot b>0$ falls $a<0$, $b<0$
\end{Ueb}
\begin{Sat}
  $\forall x>-1$, $x\neq 0$ und $\forall n\in\mb{N}\\\{0,1\}$ gilt $(1+x)^n > (1+nx)$
\end{Sat}
\begin{Bew}
  $$(1+x)^2 = 1+2x+\underbrace{x^2}_{>0}>1+2x$$
  weil $x\neq0$.\\
  Nehmen wir an dass
  \begin{align*}
    (1+x)^n&>& 1+nx & (x>-1, x\neq 0)\\
    \underbrace{(1+x)}_a \underbrace{(1+x)^n}_c&>&\underbrace{(1+nx)}_d(1+x) & (\text{weil} (1+x)>0)\\
  \end{align*}
  $$c>d \iff c-d>0 \stackrel{\text{A2}}{\implies} a(c-d) > 0 \stackrel{\text{K4}}{\implies} ac-ad > 0 \stackrel{\text{A2}}{\implies} ac>ad$$
  \begin{align*}
    (1+x)^{n+1} > (1+nx)(1+x) = 1+nx+x+nx^2=\\
    1+(n+1)x+\underbrace{nx^2}_{>0}>1+(n+1)x\\
    \implies (1+x)^{n+1} > 1+(n+1)x
  \end{align*}
  Vollständige Induktion.
\end{Bew}
\begin{Def}
  Für $a\in\mb{R}$ setzt man
  \begin{align*}
  \abs{a}=
    \begin{cases}
      a &\text{falls} a\geq0\\
      -a &\text{falls} a < 0\\
    \end{cases}
  \end{align*}
\end{Def}
\begin{Sat}
  Es gilt (Dreiecksungleichung):
  \begin{align*}
    \abs{ab}&=&\abs{a}\abs{b}\\
    \abs{a+b}&\leq&\abs{a}+\abs{b}\\
    \abs{\abs{a}-\abs{b}}&\leq&\abs{a-b}
  \end{align*}
\end{Sat}
\begin{Bew}
  \begin{itemize}
    \item $\abs{ab} = \abs{a}\abs{b}$ trivial
    \item 
      \begin{align*}
         a+b\leq \abs{a}+\abs{b} 
      \end{align*}
      ($a>0$ und $b>0$ $\implies$ $a+b=\abs{a}+\abs{b}$ sonst $a+b<\abs{a}+\abs{b}$ weil $x\leq\abs{x}$ $\forall x\in\mb{R}$ und die Gleichung gilt).
      \begin{align*}
        -(a+b)=-a-b\leq \abs{-a}+\abs{-b} = \abs{a}+\abs{b}
      \end{align*}
      Aber
      \begin{align*}
        \abs{a+b}=max\left\{ a+b, -(a+b) \right\}\leq \abs{a}+\abs{b}
      \end{align*}
    \item
      \begin{align*}
        \abs{\abs{a}-\abs{b}} \leq \abs{a-b}
      \end{align*}
      Zuerst:
      \begin{align*}
        \abs{a}=\abs{(a-b)+b}\leq \abs{a-b} + \abs{b} \\
        \implies \abs{a}-\abs{b}\leq \abs{a-b}\\
        \abs{b}=\abs{a+(b-a)}\leq \abs{a}+\abs{b-a}\\
        \implies \abs{b}-\abs{a}\leq \abs{b-a} = \abs{a-b} \\
        \implies \left( \abs{a}-\abs{b} \right)\leq \abs{a-b}\\
      \end{align*}
      \begin{align*}
        \abs{\abs{a}-\abs{b}}=max\left\{ \abs{a}-\abs{b}, -\left( \abs{a}-\abs{b} \right) \right\}\leq\abs{a-b}
      \end{align*}
  \end{itemize}
\end{Bew}
\begin{Bem}
  $$\abs{x}=max\left\{ -x,x \right\}$$
\end{Bem}
\subsection{Die Vollständigkeit der reellen Zahlen}
Für $a<b$, $a\in\mb{R}$, heisst:
\begin{itemize}
  \item abgeschlossenes Intervall: $\left[ a,b \right]=\left\{ x\in\mb{R}: a\leq x\leq b \right\}$ 
  \item offenes Intervall: $\left] a,b \right[=\left\{ x\in\mb{R}: a< x< b \right\}$
  \item (nach rechts) halboffenes Intervall: $\left[ a,b \right[=\left\{ x\in\mb{R}: a\leq x< b \right\}$
  \item (nach links) halboffenes Intervall: $\left] a,b \right]=\left\{ x\in\mb{R}: a< x\leq b \right\}$
\end{itemize}
Sei $I=[a,b]$ (bzw. $]a,b[$ \ldots). Dann $a,b$ sind die \underline{Randpunkte} von $I$. Die Zahl $\abs{I}=b-a$ ist die Länge von $I$. ($b-a>0$)
\begin{Def}
  Eine Intervallschachtelung ist eine Folge $I_1, I_2,\cdots$ geschlossener Intervalle (kurz $(I_n)_{n\in\mb{N}}$ oder $(I_n)$) mit diesen Eigenschaften:
  \begin{itemize}
    \item[I1] $I_{n+1}\subset I_n$
    \item[I2] Zu jedem $\epsilon >0$ gibt es ein Intervall $I_n$ so dass $\abs{I_n} < \epsilon$
  \end{itemize}
\end{Def}
\begin{Bsp}
  $\sqrt{2}$
  \begin{align*}
    1,4^2 & < & 2 & < & 1,5^2 & & I_1 = \left[ 1,4 / 1,5 \right] & \abs{I_1} = 0.1\\
    1,41^2 & < & 2 & < & 1,42^2 & \implies & I_2 = \left[ 1,41 / 1,42 \right] & \abs{I_2} = 0.01\\
    1,414^2 & < & 2 & < & 1,415^2 & & I_3 = \left[ 1,414, 1,415 \right] & \abs{I_2} = 0.001 
  \end{align*}
\end{Bsp}
\begin{Bew}
  I1 und I2 sind beide erfüllt.
\end{Bew}
\begin{Axi}
  Zu jeder Intervallschachtelung $\exists x\in\mb{R}$ die allen ihren Intervallen angehört.
\end{Axi}
\begin{Sat}
  Die Zahl ist eindeutig.
\end{Sat}
\begin{Bew}
  Sei $(I_n)$ eine Intervallschachtelung. Nehmen wir an dass $\exists \alpha < \beta$ so dass $\alpha, \beta\in I_n\forall n$. Dann $\abs{I_n}\geq\abs{\beta-\alpha}> a$. Widerspruch!
\end{Bew}
\begin{Sat}
  $\forall a\geq 0, a\in\mb{R}$ und $\forall x\in\mb{N}\\\left\{ 0 \right\}$, $\exists$ eine einziges $x\geq 0$, $x\in \mb{R}$ s.d. $x^k=a$. Wir nennen $x=\sqrt[k]{a}=a^\frac{1}{k}$.\\
  Sei $m,n\in\mb{N}$, $a^{m+n}=a^ma^n$ und deswegen $a^{-m}=\frac{1}{a^m}$ für $m\in\mb{N}$ (so dass die Regel $a^{m-m}=a^0=1$.\\
  $n,m\in\mb{N}\\\left\{ 0 \right\}$ $n$ Mal.
  \begin{align*}
    (a^m)^n=\underbrace{a^m\cdot a^m \cdots a^m}_{\text{$n$ Mal}} = a^{\overbrace{m+\cdots+m}^{\text{$n$ Mal}}} = a^{nm}
  \end{align*}
  Und mit $a^{-m}=\frac{1}{a^m}$ stimmt die Regel $(a^m)^n=a^{mn}$ auch $\forall m,n\in\mb{Z}$!
\end{Sat}
\begin{Bem}
    $x^k=\left( a^\frac{1}{k} \right)^k=a\left( =a^{\frac{1}{k}k} = a^1\right)$
\end{Bem}
\begin{Def}
  $\forall q=\frac{m}{n}\in\mb{Q}$, $\forall a>0$ mit definiertem $a^q=\left(\sqrt[n]{a}\right)^m$
\end{Def}
\begin{Bew}
  Mit dieser Definition gilt $a^{q+q_2} = a^qa^{q_2}$ $\forall a>0$ und $\forall q,q_2\in\mb{Q}$.
\end{Bew}

\begin{Sat}
  Zu jedem $x>0$ $(x\in\mb{R})$ und zu jedem $k\in\mb{N}$ gibt es eine reelle Zahl $y>0$ so dass $y^k=x$. In Zeichen:
  \[y=x^\frac{1}{k}, y=\sqrt[k]{x}\]
\end{Sat}
\begin{Bew}
  oBdA $x>1$ (sonst würden wir $\frac{1}{x}$ betrachten). wir konstruieren eine Intervallschachtelung $(I_n)$ so dass $\forall n a_n^k \geq x \geq b_n^k$
  \begin{equation*}
    I_1:=[1,x]\\
    I_{n+1} = \begin{cases}
      \left[ a_n, \frac{a_n+b_n}{2} \right] & \text{falls } x \leq \left( \frac{a_n+b_n}{2} \right)^k
      \left[ \frac{a_n+b_n}{2}, _n \right]
    \end{cases}\\
    \abs{I_n} = \frac{1}{2^{n-1}}\abs{I_1}
  \end{equation*}
  Intervallschachtelungsprinzip $\implies$ $\exists y\in\mb{R}$ s.d. $y\in I_n \forall n\in\mb{N}$ 
\end{Bew}
\begin{Beh}
  $y^k=x$
\end{Beh}
\begin{Bew}
   Man definiert $J_n=[a_n^k, b_n^k$. Wir wollen zeigen, dass $J_n$ eine Intervallschachtelung ist.
  \begin{itemize}
    \item $J_{n+1}\subset J_n$ weil $I_{n+1}\subset I_n$
    \item \[\abs{J_n} = b_n^k-a_n^k = \underbrace{\left( b_n - a_n \right)}_{\abs{I_n}} \underbrace{(b_n^{k-1}+b_n^{k-2}a_n + \cdots + a_n^{k-1})}_{\leq k b_1^{k-1}}\]
  \end{itemize}
  $\implies$ $\abs{J_n}\leq \abs{I_n}k k_1^{k-1}$.\\
  Sei $\varepsilon$ gegeben. Man wähle $N$ gross genug, so dass
  \[\abs{I_n}\leq\varepsilon'=\frac{\varepsilon}{kb_1^{k-1}}\implies\abs{J_n}\leq \varepsilon kb_1^{k-1}=\varepsilon\]
  Einerseits
  \[y\in\left[ a_n,b_n \right]\implies y^k\in\left[ a_n^k, b_n^k \right]=J_n\]
  Andererseits
  \[x\in J_n\forall n\in\mb{N}\]
  Intervallschachtelungsprinzip $\implies$ $x=y^k$
\end{Bew}
\subsection{Supremumseigenschaft, Vollständigkeit}
\begin{Def}
  $s\in\mb{R}$ heisst obere (untere) Schranke der Menge $M\subset \mb{R}$ falls $s\geq x$ ($s\leq x$) $\forall x\in M$.
\end{Def}
\begin{Def}
  $s\in\mb{R}$ ist das Supremum der Menge $M\subset\mb{R}$ falls es die kleinste obere Schranke ist. D.h.
  \begin{itemize}
    \item $s$ ist die obere Schranke
    \item falls $s'<s$, dass ist $s'$ keine obere Schranke.
  \end{itemize}
\end{Def}
\begin{Bsp}
  $M=]0,1[$. In diesem Fall $s=\sup M\not\in M$
\end{Bsp}
\begin{Bsp}
  $M=[0,1]$. $\sup M=1\in M$
\end{Bsp}
\begin{Def}
  $s\in\mb{R}$ heisst Infimum einer Menge $M$ ($s=\inf M$) falls $s$ die grösste obere Schranke ist.
\end{Def}
\begin{Def}
  Falls $s=\sup M\in M$, nennt man $s$ das Maximum von $M$. Kurz: $s=\max M$. Analog Minimum.
\end{Def}
\begin{Sat}
  Falls $M\subset \mb{R}$ nach oben (unten) beschränkst ist, dann existiert $\sup M$ ($\inf M$).
\end{Sat}
\begin{Bew}
  Wir konstruieren eine Intervallschachtelung $I_n$, so dass $b_n$ eine obere Schranke ist, und $a_n$ keine obere Schranke ist.
  \begin{itemize}
    \item $I_1=[a_1, b_1]$, wobei $b_1$ eine obere Schranke
    \item $a_1$ ist keine obere Schranke
  \end{itemize}
  Sei $I_n$ gegeben. 
  \begin{align*}
    I_{n+1} = \begin{cases}
      \left[ a_n,\frac{a_n+b_n}{2} \right]&\text{Falls $\frac{a_n+b_n}{2}$ eine obere Schranke ist-}\\
      \left[ \frac{a_n+b_n}{2}, b_n \right]&\text{sonst}
    \end{cases}
  \end{align*}
  Also, $\exists s\in I_n\forall n$
\end{Bew}
\begin{Beh}
  $s$ ist das Supremum von $M$
  \begin{itemize}
    \item Warum ist $s$ eine obere Schranke? \\
    Angenommen $\exists x\in M$ so dass $x>s$. Man wähle $\abs{I_n}<x-s$. Daraus folgt
    \begin{align*}
      x-s>b_n-a_n \geq b_n-s \implies x>b_n
    \end{align*}
    Widerspruch.
  \item Warum ist $s$ die kleinste obere Schranke?\\
    Angenommen $\exists s'<s$. Dann wähle $n'$ so dass $I_{n'} <s-s'$.
    \begin{align*}
      s-s'>b_{n'}-a_{n'}\geq s-a_{n'} \implies a_{n'}>s'
    \end{align*}
    Widerspruch.
  \end{itemize}
\end{Beh}
\begin{Lem}
  Jede nach oben (unten) beschränkte Menge $M\subsetequal \mb{Z}$ besitzt das grösste (kleinste) Element.
\end{Lem}
\begin{Bew}
  oBdA betrachte nur nach unten beschränkte Mengen $M\subset N$. Angenommen $M$ hat kein kleinstes Element.
\end{Bew}
\begin{Beh}
  \begin{align*}
    \forall n M\cut \left\{ 1,\cdots,n \right\} = \null\\
    n=1\\
    M\cut\left\{ 1 \right\}\\
  \end{align*}
  Angenommen
  \begin{align*}
    M\cut\left\{ 1,\cdots,n \right\} = \null\\
    M\cut\left\{ 1,2,\cdots,n+1 \right\} = M\cut \left\{ 1,\cdots,n \right\}\union M\cut\left\{ n+1 \right\}=\null\\
    \implies M\cut\mb{N}=\null
  \end{align*}
\end{Beh}
\begin{Sat}
  $\mb{Q}$ ist dich in $\mb{R}$, bzw. für beliebige zwei $x,y\in\mb{R}$, $y>x$, gibt es eine rationelle Zahl $q\in\mb{Q}$, so dass $x<q<y$.
\end{Sat}
\begin{Bew}
  Man wähle $n\in\mb{N}$ so dass $\frac{1}{n}<y-x$. Betrachte die Menge $A\subsetequal\mb{Z}$, so dass $M\in A$ $\implies$ $M>nx$. Lemma $\implies$ $\exist m=\min A$.
  \begin{align*}
    x<\frac{m}{n}=\frac{m-1}{n}+\frac{1}{n}<x+y-x=y
  \end{align*}
  Also setze $q=\frac{m}{n}$
\end{Bew}
\subsection{Abzählbarkeit}
\begin{Def}
  Die Mengen $A$ \& $B$ sind \underline{gleichmächtig}, wenn es eine Bijektion $f:A\to B$ gibt. $A$ hat grässere Mächtigkeit als $B$, falls $B$ gleichmächtig wie eine Teilmenge von $A$ ist, aber $A$ zu keiner Teilmenge von $B$ gleichmächtig ist.
\end{Def}
\begin{Bsp}
  \begin{itemize}
    \item ${1,2}$ \& ${3,4}$ sind gleichächtig.
    \item ${1,2,\cdots,n}$ hat kleinere Mächtigkeit als ${1,2,\cdots,m}$, wenn $n<m$ ist.
  \end{itemize}
\end{Bsp}
\begin{Def}
  Eine Menge $A$ ist abzählbar, wenn es eine Bijektion zwischen $\mb{N}$ und $A$ gibt. D.h. $A=\left\{ a_1,a_2,\cdots,a_n,\cdots \right\}$.
\end{Def}
\begin{Lem}
  $\mb{Z}$ ist abzählbar
\end{Lem}
\begin{Bew}
  \begin{tabular}{c|ccccc}
    \mb{N} & 1 & 2 & 3 & 4 & 5 & \ldots \\
    \mb{Z} & 0 & 1 & -1 & 2 & -2 & \ldots
  \end{tabular}
  Formal:
  \begin{align*}
    f=\mb{N}\to \mb{Z}\\
    f(n)=\begin{cases}
      \frac{n}{2} & \text{wenn $n$ gerade}\\
      \frac{1-n}{2} & \text{wenn $n$ ungerade}
    \end{cases}
  \end{align*}
\end{Bew}
\begin{Sat}
  $\mb{Q}$ ist abzählbar
\end{Sat}
\begin{Bew}
  Sucht euch die Graphik auf Wikipedia oder sonstwo.
\end{Bew}
\begin{Sat}
  $\mb{R}$ ist nicht abzählbar.
\end{Sat}

\section{Komplexe Zahlen}
\begin{Bem}
  $\forall a\in \mb{R}$, $a^2>0$. Deswegen ist $x^2=-1$ unlösbar.
  Die Erfindung der imagin\"are Einheit $i$ (die imaginäre Zahl mit $i^2=-1$) 
hat sehr interessante Konsequenzen auch für die üblichen reellen Zahlen.
\end{Bem}
\subsection{Definition}[Erste Definition der Komplexen Zahlen]\label{d:C1}
\begin{Def} Sei $a,b\in\mb{R}$, dann $a+bi\in\mb{C}$. Wir definieren die Summe:
  \begin{equation*}
    (a+bi)+(\alpha+\beta i) = (a+\alpha)+(b+\beta)i
\end{equation*}
und das Produkt
\begin{equation*}
    (a+bi)(\alpha+\beta i) = (a\alpha-b\beta)+ \underbrace{(a\beta+b\alpha)}_A i
  \end{equation*}
\end{Def}
\begin{Def}
  Seien $A$ und $B$ zwei Mengen. Dann ist $A\times B$ die Menge der Paare $(a,b)$ mit $a\in A$ und $b\in B$.
\end{Def}
\begin{Def}[Zweite Definition der Komplezen Zahlen]\label{d:C2}
  $\mb{C}=\mb{R}\times\mb{R}$ mit $+$ und $\cdot$ , die wir so definieren:
  \begin{eqnarray*}
    (a,b)+(\alpha,\beta)&=&(a+\alpha,b+\beta)\\
    (a,b)(\alpha,\beta)&=&(a\alpha-b\beta, \underbrace{a\beta+b\alpha}_{A})
  \end{eqnarray*}
\end{Def}
\begin{Bem}
  \begin{equation*}
    \mb{R}\simeq \left\{ (a,0), a\in\mb{R} \right\}\subset\mb{C}\\
\end{equation*}
In der Sprache der abstrakte Algebra $\mb{R}$ ist isomorph zu 
$\mb{R}' := \{(a,0):a\in \mb{R}\}$: d.h. die Summe und 
das Produkt in $\mb{R}$ und $\mb{R}'$ sind ``gleich'': 
\begin{eqnarray*} 
    (a,0)+(\alpha,0)=(a+\alpha,0)\\
    (a,0)(\alpha,0)=(a\alpha,0)
  \end{eqnarray*}
Deswegen wir schreiben $a$ statt $(a,0)$.
\end{Bem}
\begin{Bem}
  \begin{equation*}
    (0,a)(0,b)=(-ab, 0)
  \end{equation*}
  Deswegen:
  \begin{equation*}
    \underbrace{(0,1)}_{\text{Wurzel von -1}}(0,1)=(-1,0)\\
    \underbrace{(0,-1)}_{\text{auch eine Wurzel von -1}}(0,-1)=(-1,0)
  \end{equation*}
\end{Bem}
\begin{Bem}
  $i=(0,1)$ und wir schreiben $(a,b)$ für $a+bi$. D.h. die zwei Definitionen der
komplezen Zahlen sind equivalent!
\end{Bem}
\begin{Bem}
  $0=(0,0)=0+0i$. $\xi\in\mb{C}$
  \begin{align*}
    0\xi=0\\
    0+\xi=\xi
  \end{align*}
\end{Bem}
\begin{Sat}\label{s:CK}
  Alle Körperaxiome (K1-K4) gelten.
\end{Sat}
\begin{proof}[Beweis]
  \begin{itemize}
    \item[K1] Kommultativität: {\em trivial}
    \item[K2] Assoziativität: {\em trivial}
    \item[K3] Distributivität: {\em trivial}.
    \item[K4] Seien $\xi, \zeta \in\mb{C}$.
      \begin{align}
        \exists \omega\in\mb{C} :&\qquad \xi+\omega=\zeta\label{e:minus}\\
        \xi\neq 0\exists \omega: &\qquad \xi\omega=\zeta\label{e:/}
      \end{align}
  \end{itemize}
{\em Zu \eqref{e:minus}}. Wir setzen
  \begin{eqnarray*}
    \xi=a+bi\\
    \zeta=c+di\\
    \omega=x+yi
  \end{eqnarray*}
  \begin{equation*}
    \xi+\omega = (a+x)+(b+y)i = \xi = c+di
  \end{equation*}
  Sei $x:=c-a$, $y:=d-b$. Dann $\xi+\omega=\zeta$.

\medskip

{\em Zu \eqref{e:/}} $1$ ( $= 1+0i)$) das neutrale Element.
  \begin{equation*}
    (a+bi)(1+0i)=\underbrace{(a1-b0)}_{a}+\underbrace{(b1+a0)}_{b}=(a+bi)
  \end{equation*}
  Sei $\xi\neq 0$ und suchen wir $\alpha$ so dass $\xi\alpha=1$. Dann ist $\omega=\alpha\zeta$ eine Lösung von 
\eqref{e:/} (eigentlich DIE Lösung). Falls $\xi=a+bi$, dann
  \begin{equation*}
    \alpha=\frac{a}{a^2+b^2}-\frac{b}{a^2+b^2}\, .
\end{equation*}
In der Tat:
\begin{equation*}
    \xi\alpha=\overbrace{\left( \frac{aa}{a^2+b^2}-\frac{b(-b)}{a^2+b^2} \right)}^1 + 
    \overbrace{\left( \frac{a(-b)}{a^2+b^2}-\frac{ab}{a^2+b^2} \right)}^0 i=1\, .
\end{equation*}
\end{proof}
\begin{Def}
  Sei $\xi=(x+yi)\in\mb{C}$. Dann:
  \begin{itemize}
    \item $x$ ist der reelle Teil von $\xi$ $(\Re\xi=x)$
    \item $y$ ist der imaginäre Teil von $\xi$ $(\Im\xi=y)$
    \item $x-yi$ ist die konjugierte Zahl $\left( \ol\xi =x-yi \right)$
  \end{itemize}
\end{Def}
\begin{Bem}
  \[\sqrt{\ol\xi\ol\xi}=\sqrt{\left( \Re\xi \right)^2+\left( \Im\xi \right)^2}=:\abs{\xi}\]
\end{Bem}
\begin{Def}
  $\abs{\xi}$ ist der Betrag von $\xi$.
\end{Def}
\begin{Sat}
  Es gilt: $(\forall a,b\in\mb{C})$:
  \begin{itemize}
    \item 
      \begin{itemize}
        \item \[\ol{a+b}=\ol{a}+\ol{b}\]
        \item \[\ol{ab}=\ol{a}\ol{b}\]
      \end{itemize}
    \item 
      \begin{itemize}
        \item \[\Re a =\frac{a+\ol{a}}{2}\]
        \item \[(\Im a) i=\frac{a-\ol{a}}{2}\]
      \end{itemize}
    \item $a=\ol{a}$ genau dann wenn $a\in\mb{R}$.
    \item \[a\ol a=\abs{a}^2=\sqrt{\left( \Re a \right)^2+\left( \Im a \right)^2}\geq 0\]
      (die Gleicheit gilt genau dann wenn $a=0$)
  \end{itemize}
\end{Sat}
\begin{Bem}
  Sei $\omega$ so dass $\xi\omega=1$ $(\xi\neq 0)$. Man schreibt $\omega\frac{1}{\xi}$.
Der Beweis vom Satz \ref{s:CK} impliziert $\omega=\frac{\ol\xi}{\abs{\xi}^2}$
\end{Bem}
\begin{Sat}
  $\forall a,b\in\mb{C}$
  \begin{itemize}
    \item $\abs{a}>0$ für $a\neq 0$ (trivial)
    \item $\abs{\ol a}=\abs{a}$ (trivial)
    \item $\abs{\Re a}\leq\abs{a}$, $\abs{\Im a}\leq\abs{a}$ (trivial)
    \item $\abs{ab}=\abs{a}\abs{b}$
    \item $\abs{a+b}\leq\abs{a}+\abs{b}$
  \end{itemize}
\end{Sat}
\begin{proof}[Beweis]
  \begin{equation*}
    \abs{ab}^2=(ab)\ol{(ab)}=ab\ol a\ol b=a\ol a\ol b=\abs{a}^2\abs{b}^2\\
    \implies \abs{ab}=\abs{a}\abs{b}
  \end{equation*}
\begin{eqnarray}
\underbrace{|a+b|^2}{\in \mb{R}} &=& (a+b)\ol{(a+b)} = 
    (a+b)(\ol a+\ol b)=a\ol a+b\ol b+a\ol b+b\ol a \nonumber\\
&=& \underbrace{\abs{a}^2+\abs{b}^2}_{\in\mb{R}}+\left( a\ol b+b\ol a \right)\, .\label{e:10}
\end{eqnarray}
Bemerkung: die Identit\"at implizert dass $a\ol b+b\ol a$. In der Tat
$a\ol b + b\ol a= a\ol b+ \ol{a\ol b} = 2 \re a \ol b$. Deswegen
\begin{eqnarray}
|a+b|^2 &=& |a|^2 +|b|^2 + 2 \re a\ol b \;\leq\; |a|^2+|b|^2+ 2|a \ol b|\nonumber\\
&=& |a|^2 + |b|^2 + 2 |a||b| = (|a|+|b|)^2\, .
\end{eqnarray}
\end{proof}


\section{Funktionen}\label{s:F}
\subsection{Definition}
\begin{Def}
  Seien $A$ und $B$ zwei Mengen. Eine Funktion $f:A\to B$ ist eine Vorschrift die jedem Element $a\in A$ ein eindeutiges Element $f(a)\in B$ zuordnet.
\end{Def}
\begin{Bsp}
  $A\subset \mb{R}$, $B=\mb{R}$ (oder $\mb{C}$)
  \[f(x)=x^2\]
\end{Bsp}
\begin{Def}
  $A$ ist der \underline{Definitionsbereich}.
  \[f(A)=\left\{ f(x): x\in A \right\}\]
  ist der Wertbereich
\end{Def}
\begin{Bem}
  Wertbereich von $x^2$:
  \[\left\{ y\in\mb{R}: y\geq 0 \right\}\]
\end{Bem}
\begin{Def}
  Der Graph einer Funktion $f:A\to B$ ist
  \[G(f)=\left\{ (x, f(x))\in A \times B:x\in A \right\}\]
\end{Def}
\begin{Bsp}
  % TODO scannen
  Verboten: zwei Werte für die Stelle $x$.
\end{Bsp}
\begin{Bsp}
  $f:\mb{R}\to\mb{R}$ $f(x)=\abs{x}$
\end{Bsp}
\subsection{Algebraische Operationen}
Wenn $B=\mb{R}$ oder $\mb{C}$. Seien $f,g$ zwei Funktionen mit gleichem Definitionsbereich.
\begin{itemize}
  \item $f+g$ ist die Funktion $h$ so dass $h:A\to B$ \[h(x)=f(x) + g(x)\]
  \item Die Funktion $fg$ ist $k:A\to B$ \[k(x)=f(x)g(x)\]
  \item $\frac{f}{g}$ ist wohldefiniert falls der Wertebereich von $g$ in 
$B\setminus \left\{ 0 \right\}$ enthalten ist: \[\frac{f}{g}(x)=\frac{f(x)}{g(x)}\]
\end{itemize}
Falls $B=\mb{C}$, kann man auch $\Re f$, $\Im f$, $\ol f$.
\begin{Def}
  Sei $f:A\to B$, $g:B\to C$. Die Komposition $g\circ f: A\to C$.
  \[g\circ f(x)=g(f(x))\]
\end{Def}
\begin{Bem} Sei
  $f:A\to\mb{R}$, $g:A\to \mb{R}$. Wir definieren $\Xi:A\to\mb{R}\times\mb{R}$:
  \[\Xi(a)=(f(a),g(a))\]
und $\Phi:\mb{R}\times\mb{R}\to\mb{R}$:
$$
\Phi(x,y)=xy
$$
Dann
$$
\Phi\circ\Xi(a)=\Phi(\Xi(a))=\Phi\left( \left( f(a) \right),g(a) \right)=f(a)g(a)
$$
Also: die ``algebraischen Operationen'' sind ``Kompositionen''.
\end{Bem}
\begin{Def}
  \begin{itemize}
    \item Wenn $f:A\to B$ und $f(A)=B$ dann ist $f$ \ul{surjektiv}.
    \item Wenn $f:A\to B$ und die folgende Eigenschaft hat:
      \[f(x)\neq f(y)\forall x\neq y\in A\]
      dann ist $f$ \ul{injektiv}.
    \item Falls $f$ surjektiv und injektiv ist, dann sagen wir, dass $f$ bijektiv ist.
  \end{itemize}
\end{Def}
\begin{Bem}
  Die bijektiven Funktionen sind umkehrbar. Sei $f:A\to B$ bijektiv. $\forall b\;\exists a:f(a)=b$ (surjektiv), $a$ ist eindeutig (injektiv) (die Notation f\"ur die Eindeutigkeit ist $\exists! a:f(a)=b$). 
Dann $g(b)=a$ ist eine ``wohldefinierte Funktion'', $g:B\to A$.
\end{Bem}
\begin{Def}
  $g$ wird Umkehrfunktion genannt. $f:A\to B$, $g:B\to A$, $f\circ g: B\to B$,
$g\circ f:A\to A$ und
\begin{equation}\label{e:Um}
f\circ g(b)=b \quad \forall b\in B \qquad  g\circ f(a)=a\quad\forall a\in A\, .
\end{equation}
\end{Def}
\begin{Def}
  Die ``dumme Funktion'' $h:A\to A$ mit $h(a)=a\;\forall a\in A$ heisst Identitätsfunktion ($\id$). 
Deswegen, \eqref{e:Um} $\iff$ $f\circ g=\id$ und $g\circ f = \id$.
\end{Def}
\subsection{Zoo}
\subsubsection{Exponentialfunktion}
$a\in\mb{R}, a>0$. Defintionsbereich $\mb{Q}$ (momentan!):
\[\Exp_a:\mb{Q}\to\mb{R}\]
\[\Exp_a(n)=a^n \qquad(=1 \mbox{ falls }n=0)\]
\[\Exp_a(-n)=\frac{1}{a^n}\]
\[\Exp_a\left( \frac{m}{n} \right) = \left( \sqrt[n]{a} \right)^m\]
$\Exp_a$ ist die \ul{einzige} Funktion $\Phi:\mb{Q}\to\mb{R}$ mit den folgenden Eigenschaften:
\begin{itemize}
  \item $\Phi(1)=a$
  \item $\Phi(q+r)=\Phi(q)\Phi(r)$ $\forall q,r\in\mb{Q}$
\end{itemize}
\begin{Bem}
  Später werden wir $\Exp_a$ auf \ul{$\mb{R}$ fortsetzen}.
\end{Bem}
\subsubsection{Polynome}
\[f(x)=a_nx^n+a_{n-1}x^{n-1}+\cdots+a_1x+a_0\]
\[f:\mb{R} (\mb{C}) \ni x\mapsto f(x)\in\mb{R} (\mb{C})\]
Produkt von Polynomen $x\mapsto f(x)g(x)$:
\begin{eqnarray*}
  f(x)g(x) &=& \left( a_nx^n+\cdots+a_0 \right)\left( b_mx^m+\cdots+b_0 \right)\\
&=&  b_ma_nx^{n+m}+b_na_{n-1}x^{n-1+m}+\cdots\\
&=&  b_ma_nx^{n+m}+\left( b_ma_{n-1}+b_{m-1}a_n \right)x^{n+m-1}+\cdots + a_0b_0\\ 
&=&
  c_{m+n}x^{m+n} +\cdots +c_0\, ,
\end{eqnarray*}
wobei
\[c_k=\sum_{i+j=k}a_ib_j=\sum_{i=0}^ka_ib_{k-i}\]
\begin{Def}
  Der Grad von $a_nx^n+\cdots+a_0$ ist $n$ wenn $a_n\neq 0$
\end{Def}
\begin{Sat}\label{s:PDiv}
  Sei $g\neq 0$ ein Polynom. Dann gibt es zu jedem Polynom $f$ zwei Polynome $q$ und $r$ so dass
  \[g=qf+r\]
  \[\Grad r < \Grad f\]
\end{Sat}
\begin{proof}[Beweis]
  \url{http://de.wikipedia.org/wiki/Polynomdivision}
\end{proof}
\begin{Bem}
  Sei $g=x-x_0$. Sei $f$ mit Grad $\geq 1$, Satz \ref{s:PDiv} 
$\implies f=gq+r=gq+c_0$ und Grad von $r < 1$. $r$ ist eine Konstante $r=c_0$. Deswegen
  \[f(x)=q(x)(x-x_0)+c_0\]
  \[f(x_0)=q(x_0)0+c_0 = c_0\]
\end{Bem}
\begin{Kor}
  Falls $f$ ein Polynom ist und $f(x_0)=0$, dann $\exists q$ Polynom so dass $f=q(x-x_0)$
\end{Kor}

Das Polynom $a_n x^n + \ldots + a_0$ mit $a_n= \ldots = 0$ ist das {\em Trivialpolynom}.

\begin{Kor}
  Ein Polynom $P$ hat höchstens $\Grad f$ Nullstellen falls $P$ ist nicht das Trivialpolynom.
\end{Kor}



\begin{Kor}
  Falls $f(x)=0$ $\forall x\in\mb{R}$, dann ist $f$ das Trivialpolynom.
\end{Kor}
\begin{Kor}
  Falls $f,g$ Polynome sind und $f(x)=g(x)$ $\forall x\in\mb{R}$ dann sind die Koeffizienten von $f$ und $g$ gleich.
\end{Kor}
\begin{proof}[Beweis]
  $f-g$ ist ein Polynom mit $(f-g)(x)=0$ $\forall x$. 
\end{proof}
\begin{Def}
  Seien $f,g$ Polynome. Dann ist $\frac{f}{g}$ eine rationale Funktion.
\end{Def}


\newpage

%= Stichwortverzeichnis ======================================================================
\rhead{}
\addcontentsline{toc}{section}{Stichwortverzeichnis}
\printindex

\end{document}
