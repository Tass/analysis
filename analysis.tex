% headers by Alexander Berthold van der Bourg / Pirmin Weigele 

%= Document-Class ==================================================================================
\documentclass[10pt,a4paper]{article}

%= Packages ========================================================================================
\usepackage[utf8x]{inputenc}
\usepackage{ngerman,amsmath,amssymb,amsfonts,mathrsfs}
\usepackage{amsthm}
\usepackage{bbm}
\usepackage{ulsy}
\usepackage{epic,eepic,pstricks,pst-node,pst-plot}
\usepackage{pstricks}
\usepackage{colortbl}
\usepackage{graphicx}
\usepackage{makeidx}
\usepackage{fancyhdr}
\usepackage{latexsym}
\usepackage{psfrag}
\usepackage{enumerate}
\usepackage{float}
\usepackage{dsfont}
\pagestyle{fancy}
\usepackage{multirow, bigdelim, bigstrut}
\usepackage{rotating}
\usepackage{ifthen}
\usepackage{boxedminipage}
\usepackage{mathtools}
\usepackage{ulsy}
\usepackage{trfsigns}
\usepackage{url}
%\usepackage{savetrees}

%= Seiten-Layout =========================================================================
\voffset-22mm \textheight715pt 

%Seitenbreite==============================================================

%\oddsidemargin=-0.2in
%\evensidemargin=-0.4in
%\textwidth=5.2in
%\headwidth=5.2in

%= Index-Befehle ========================================================================
\renewcommand{\indexname}{Stichwortverzeichnis}
\makeindex

%= Befehl-Overwriting =======================================================================
\makeatletter
\makeatother

%= Strings ================================================================
\newcommand{\mainfold}{.}
\newcommand{\prefix}{A1-}

%= Eigene Befehle ==========================================================================
\DeclareMathOperator{\id}{Id}
\DeclareMathOperator{\arccot}{arccot}
\DeclareMathOperator{\arcsinh}{arcsinh}
\DeclareMathOperator{\arccosh}{arccosh}
\DeclareMathOperator{\arctanh}{arctanh}
\DeclareMathOperator{\md}{d}
\DeclareMathOperator{\Grad}{grad}
\DeclareMathOperator{\Spur}{Spur}
\DeclareMathOperator{\Graph}{Graph}
\DeclareMathOperator{\sign}{sign}
\DeclareMathOperator{\Hom}{Hom}
\DeclareMathOperator{\rot}{rot}
\DeclareMathOperator{\Ker}{Ker}
\DeclareMathOperator{\Exp}{Exp}

\newcommand{\Diff}[2]{\displaystyle\frac{\mathrm{d}#1}{\mathrm{d}#2}}
\newcommand{\End}{\hfill{\hbox{$\Box$}}\par\vspace{2mm}}
\newcommand{\eps}{\varepsilon}
\newcommand{\ePic}[1]{\input{\mainfold/graphics/\prefix#1.eepic}}
\newcommand{\pst}[1]{\input{\mainfold/graphics/\prefix#1.pst}}
\newcommand{\pic}[1]{\input{\mainfold/graphics/\prefix#1.pic}}
\newcommand{\Mx}[1]{\begin{pmatrix}#1\end{pmatrix}}
%\newcommand{\im}[1]{\operatorname{Im}(#1)}
%\newcommand{\Include}[4]{\rhead{#2.#3.20#4}\input{\mainfold/lectures/#1-#4-#3-#2.tex}}
\newcommand{\Index}[1]{\emph{#1}\index{#1}}
\newcommand{\Int}[4]{\displaystyle\int\limits_{#1}^{#2}#3\,\mathrm{d}#4}
\newcommand{\diff}[1]{\operatorname{d}\!#1}
\newcommand{\Limi}[1]{\displaystyle\lim_{#1\rightarrow\infty}}
\newcommand{\Limo}[1]{\displaystyle\lim_{#1\rightarrow0}}
\newcommand{\Lim}[2]{\displaystyle\lim_{#1\rightarrow#2}}
\newcommand{\mb}[1]{\mathbb{#1}}
\newcommand{\ds}{\displaystyle}
\newcommand{\ol}[1]{\overline{#1}}
\newcommand{\Part}[2]{\dfrac{\partial #1}{\partial #2}}
\newcommand{\QED}{\hfill{\hbox{(QED)}}\par\vspace{2mm}}
\newcommand{\re}[1]{\operatorname{Re}(#1)}
\newcommand{\s}{\hspace{2mm}}
\newcommand{\vsa}{\vspace{1mm} \\}
\newcommand{\vsb}{\vspace{2mm} \\}
\newcommand{\vsc}{\vspace{3mm} \\}
% \newcommand{\tr}[1]{\textrm{#1}}
\newcommand{\tr}[1]{\text{#1}}
\newcommand{\ra}{\rightarrow}
\newcommand{\Ra}{\Rightarrow}
\newcommand{\Lra}{\Leftrightarrow}
\newcommand{\La}{\Leftarrow}
\newcommand{\ul}[1]{\underline{#1}}
\newcommand{\rsa}{\rightsquigarrow}
\newcommand{\ara}[2]{\autorightarrow{\ensuremath{#1}}{\ensuremath{#2}}}
\newcommand{\dcp}[2]{\begindc{\commdiag}[#1] #2 \enddc}

%\newcommand{\detmx}{\left| \begin{array} #1 \end{array} \right|}

\newcommand{\grad}[1]{\Grad(#1)}
\newcommand{\fr}[2]{\displaystyle\frac{#1}{#2}} % fertiger bullshit, daf�r gibts \dfrac{}{}
\renewcommand{\Re}{\operatorname{Re}}
\renewcommand{\Im}{\operatorname{Im}}

% ---- DELIMITER PAIRS ----
\def\floor#1{\lfloor #1 \rfloor}
\def\ceil#1{\lceil #1 \rceil}
\def\seq#1{\langle #1 \rangle}
\def\set#1{\{ #1 \}}
\def\abs#1{\mathopen| #1 \mathclose|}	% use instead of $|x|$ 
\def\norm#1{\mathopen\| #1 \mathclose\|}% use instead of $\|x\|$ 

% --- Self-scaling delmiter pairs ---
\def\Floor#1{\left\lfloor #1 \right\rfloor}
\def\Ceil#1{\left\lceil #1 \right\rceil}
\def\Seq#1{\left\langle #1 \right\rangle}
\def\Set#1{\left\{ #1 \right\}}
\def\Abs#1{\left| #1 \right|}
\def\Norm#1{\left\| #1 \right\|}

%Adrians Abbildungs-Environment ==============================================

\newcommand{\Sidein}{\begin{rotate}{90}\small$\in$\end{rotate}}

\newcommand{\Abb}[5][]{\ensuremath{
    \begin{array}{lc}
      \ifthenelse{\equal{#1}{}}{}{#1:}\;\; & 
      \begin{xy}
        \xymatrixrowsep{1em}\xymatrixcolsep{2em}%
        \xymatrix{ #2 \ar[r] \ar@{}[d]^<<<<{\hspace{0.001em} \Sidein}
          & #3  \ar@{}[d]^<<<<{\hspace{0.001em} \Sidein} \\
          #4 \ar@{|->}[r] & #5} \end{xy}
    \end{array}
  }%
}

%= Environments ========================================================================
\def\thechapter{\Roman{chapter}}
\def\thesection{\arabic{section}}
\newtheorem{theorem}{Theorem}[section]
\newenvironment{Bew}{\begin{proof}[Beweis]}{\end{proof}}
\newtheorem{Axi}[theorem]{Axiom}
\newtheorem{Lem}[theorem]{Lemma}
\newtheorem{Kor}[theorem]{Korollar}
\newtheorem{Sat}[theorem]{Satz}
\newtheorem{Prop}[theorem]{Proposition}
\newtheorem{Beh}[theorem]{Behauptung}
\theoremstyle{definition}
\newtheorem{Bsp}[theorem]{Beispiel}
\newtheorem{Def}[theorem]{Definition}
\newtheorem{Ueb}[theorem]{\"Ubung}
\theoremstyle{remark}
\newtheorem{Bem}[theorem]{Bemerkung}
\newtheorem{Eig}[theorem]{Eigenschaften}
\newtheorem{Not}[theorem]{Notation}

\def\pstexInput#1{%
  \begin{center}
    \begin{picture}(0,0)%
      \special{psfile=\mainfold/graphics/A2-#1.pstex}%
    \end{picture}%
    \input{\mainfold/graphics/A2-#1.pstex_t}%
  \end{center}
}

%= Titelseite ===========================================================================
\begin{document}
\headheight15pt
\begin{titlepage}
\hfill
\vspace{20mm}
\pagenumbering{roman}
\begin{center}
{\LARGE Analysis I - Vorlesungs-Script} \vskip 3em {\large Prof. Dr. Camillo De Lellis} \vskip 1.5em
{\large Basisjahr 10 Semester II}\vspace{30mm}\\
{\large {\bf Mitschrift:} \vspace{2mm}\\
Simon Hafner}\vspace{5mm}\\ %30mm
%{\large {\bf Graphics:} \vspace{2mm}\\
%Pirmin Weigele }\vspace{30mm}\\ %30mm
\author{Simon Hafner}

\end{center}
\vfill

\end{titlepage}


%= Inhaltsverzeichnis ==========================================================================
\lhead{}
\rhead{}
\tableofcontents
\newpage
\pagenumbering{arabic}
\setcounter{page}{1}

%= Vorlesung-Skripts ==========================================================================
\cfoot{\thepage}
\fancyhead[L]{\nouppercase{\leftmark}}
\newpage

%= Analysis I & & II ==========================================================================

%Analysis I
\section{Die reellen Zahlen}
\begin{Bsp}
  $\mb{R}$ ist nicht genug
\end{Bsp}
\begin{Sat}
  Es gibt kein $q\in\mb{Q}$ so dass $q^2=2$
\end{Sat}
\begin{Bew}
  Falls $q^2=2$, dann $(-q)^2=2$ OBdA $q\geq 0$ Deswegen $q>0$. Sei $q>0$ und $q\in\mb{Q}$ so dass $q^2=2$. $q=\frac{m}{n}$ mit $m>0$, $>0$. $\text{GGT}(m,n)=1$ (d.h. falls $r\in\mb{N}$ $m$ und $n$ dividiert, dann $r=1$!).
  \begin{align*}
    m^2=2n^2&\implies m \text{ ist gerade}&\implies m=2k \text{ für } k\in\mb{N}\\\{0\}\\
    4k^2=2n^2&\implies n \text{ ist gerade}&\implies 2| n \text{(2 dividiert $n$)}
  \end{align*}
  $\implies$ Widerspruch! Weil $2$ dividiert $m$ und $n$! (d.h. es gibt \underline{keine} Zahl $q\in\mb{Q}$ mit $q^2=2$
\end{Bew}
\begin{Bsp}
  \begin{align*}
    \sqrt{2}=1,414\cdots
  \end{align*}
  Intuitiv:
  \begin{align*}
    1,4^2 & < & 2 & < & 1,5^2 & & 1,4 & < & \sqrt{2} & < & 1,5 \\
    1,41^2 & < & 2 & < & 1,42^2 & \implies & 1,41 & < & \sqrt{2} & < & 1,42 \\
    1,414^2 & < & 2 & < & 1,415^2 & & 1,414 & < & \sqrt{2} & < & 1,415 \\
  \end{align*}
\end{Bsp}
\paragraph{Intuitiv}
\begin{itemize}
  \item $\mb{Q}$ hat ``Lücke''
  \item $\mb{R}$ $= \{$ die reellen Zahlen $\}$ haben ``kein Loch''.
\end{itemize}
\paragraph{Konstruktion}
Die reellen Zahlen kann man ``konstruieren''. (Dedekindsche Schritte, Cantor ``Vervollständigung''). Google knows more. Wir werden ``operativ'' sein, d.h. wir beschreiben einfach die wichtigsten Eigenschaften von $\mb{R}$
\subsection{Körperstrukturen}
\begin{itemize}
  \item[K1] Kommutativgesetz
  \begin{align*}
    % TODO align that on =
    a+b &=& b + a &\\
    a\cdot b &=& b\cdot a &\\
  \end{align*}
  \item[K2] Assoziativgesetz
    \begin{align*}
      (a+b)+c &=&a+(b+c)\\
      (a\cdot b)\cdot c&=&a\cdot(b\cdot c)\\
    \end{align*}
  \item[K3] Distributivgesetz
    \begin{align*}
      (a+b)\cdot c&=& a\cdot c + b\cdot c
    \end{align*}
  \item[K4] 
    \begin{align*}
      a+x&=&b\\
      a\cdot x&=&b \text{falls $a\neq 0$}\\
    \end{align*}
\end{itemize}
\subsection{Die Anordnung von $\mb{R}$}
\begin{itemize}
  \item[A1] $\forall a\in\mb{R}$ gilt genau eine der drei Relationen:
    \begin{itemize}
      \item $a<0$
      \item $a=0$
      \item $a>0$
    \end{itemize}
  \item[A2] Falls $a>0$, $b>0$, dann $a+b>0$, $a\cdot b>0$
  \item[A3] Archimedisches Axiom: $\forall a\in\mb{R} \exists n\in\mb{N}$ mit $n>a$
\end{itemize}
\begin{Ueb}
  Beweisen Sie dass $a\cdot b>0$ falls $a<0$, $b<0$
\end{Ueb}
\begin{Sat}
  $\forall x>-1$, $x\neq 0$ und $\forall n\in\mb{N}\\\{0,1\}$ gilt $(1+x)^n > (1+nx)$
\end{Sat}
\begin{Bew}
  $$(1+x)^2 = 1+2x+\underbrace{x^2}_{>0}>1+2x$$
  weil $x\neq0$.\\
  Nehmen wir an dass
  \begin{align*}
    (1+x)^n&>& 1+nx & (x>-1, x\neq 0)\\
    \underbrace{(1+x)}_a \underbrace{(1+x)^n}_c&>&\underbrace{(1+nx)}_d(1+x) & (\text{weil} (1+x)>0)\\
  \end{align*}
  $$c>d \iff c-d>0 \stackrel{\text{A2}}{\implies} a(c-d) > 0 \stackrel{\text{K4}}{\implies} ac-ad > 0 \stackrel{\text{A2}}{\implies} ac>ad$$
  \begin{align*}
    (1+x)^{n+1} > (1+nx)(1+x) = 1+nx+x+nx^2=\\
    1+(n+1)x+\underbrace{nx^2}_{>0}>1+(n+1)x\\
    \implies (1+x)^{n+1} > 1+(n+1)x
  \end{align*}
  Vollständige Induktion.
\end{Bew}
\begin{Def}
  Für $a\in\mb{R}$ setzt man
  \begin{align*}
  \abs{a}=
    \begin{cases}
      a &\text{falls} a\geq0\\
      -a &\text{falls} a < 0\\
    \end{cases}
  \end{align*}
\end{Def}
\begin{Sat}
  Es gilt (Dreiecksungleichung):
  \begin{align*}
    \abs{ab}&=&\abs{a}\abs{b}\\
    \abs{a+b}&\leq&\abs{a}+\abs{b}\\
    \abs{\abs{a}-\abs{b}}&\leq&\abs{a-b}
  \end{align*}
\end{Sat}
\begin{Bew}
  \begin{itemize}
    \item $\abs{ab} = \abs{a}\abs{b}$ trivial
    \item 
      \begin{align*}
         a+b\leq \abs{a}+\abs{b} 
      \end{align*}
      ($a>0$ und $b>0$ $\implies$ $a+b=\abs{a}+\abs{b}$ sonst $a+b<\abs{a}+\abs{b}$ weil $x\leq\abs{x}$ $\forall x\in\mb{R}$ und die Gleichung gilt).
      \begin{align*}
        -(a+b)=-a-b\leq \abs{-a}+\abs{-b} = \abs{a}+\abs{b}
      \end{align*}
      Aber
      \begin{align*}
        \abs{a+b}=max\left\{ a+b, -(a+b) \right\}\leq \abs{a}+\abs{b}
      \end{align*}
    \item
      \begin{align*}
        \abs{\abs{a}-\abs{b}} \leq \abs{a-b}
      \end{align*}
      Zuerst:
      \begin{align*}
        \abs{a}=\abs{(a-b)+b}\leq \abs{a-b} + \abs{b} \\
        \implies \abs{a}-\abs{b}\leq \abs{a-b}\\
        \abs{b}=\abs{a+(b-a)}\leq \abs{a}+\abs{b-a}\\
        \implies \abs{b}-\abs{a}\leq \abs{b-a} = \abs{a-b} \\
        \implies \left( \abs{a}-\abs{b} \right)\leq \abs{a-b}\\
      \end{align*}
      \begin{align*}
        \abs{\abs{a}-\abs{b}}=max\left\{ \abs{a}-\abs{b}, -\left( \abs{a}-\abs{b} \right) \right\}\leq\abs{a-b}
      \end{align*}
  \end{itemize}
\end{Bew}
\begin{Bem}
  $$\abs{x}=max\left\{ -x,x \right\}$$
\end{Bem}
\subsection{Die Vollständigkeit der reellen Zahlen}
Für $a<b$, $a\in\mb{R}$, heisst:
\begin{itemize}
  \item abgeschlossenes Intervall: $\left[ a,b \right]=\left\{ x\in\mb{R}: a\leq x\leq b \right\}$ 
  \item offenes Intervall: $\left] a,b \right[=\left\{ x\in\mb{R}: a< x< b \right\}$
  \item (nach rechts) halboffenes Intervall: $\left[ a,b \right[=\left\{ x\in\mb{R}: a\leq x< b \right\}$
  \item (nach links) halboffenes Intervall: $\left] a,b \right]=\left\{ x\in\mb{R}: a< x\leq b \right\}$
\end{itemize}
Sei $I=[a,b]$ (bzw. $]a,b[$ \ldots). Dann $a,b$ sind die \underline{Randpunkte} von $I$. Die Zahl $\abs{I}=b-a$ ist die Länge von $I$. ($b-a>0$)
\begin{Def}
  Eine Intervallschachtelung ist eine Folge $I_1, I_2,\cdots$ geschlossener Intervalle (kurz $(I_n)_{n\in\mb{N}}$ oder $(I_n)$) mit diesen Eigenschaften:
  \begin{itemize}
    \item[I1] $I_{n+1}\subset I_n$
    \item[I2] Zu jedem $\epsilon >0$ gibt es ein Intervall $I_n$ so dass $\abs{I_n} < \epsilon$
  \end{itemize}
\end{Def}
\begin{Bsp}
  $\sqrt{2}$
  \begin{align*}
    1,4^2 & < & 2 & < & 1,5^2 & & I_1 = \left[ 1,4 / 1,5 \right] & \abs{I_1} = 0.1\\
    1,41^2 & < & 2 & < & 1,42^2 & \implies & I_2 = \left[ 1,41 / 1,42 \right] & \abs{I_2} = 0.01\\
    1,414^2 & < & 2 & < & 1,415^2 & & I_3 = \left[ 1,414, 1,415 \right] & \abs{I_2} = 0.001 
  \end{align*}
\end{Bsp}
\begin{Bew}
  I1 und I2 sind beide erfüllt.
\end{Bew}
\begin{Axi}
  Zu jeder Intervallschachtelung $\exists x\in\mb{R}$ die allen ihren Intervallen angehört.
\end{Axi}
\begin{Sat}
  Die Zahl ist eindeutig.
\end{Sat}
\begin{Bew}
  Sei $(I_n)$ eine Intervallschachtelung. Nehmen wir an dass $\exists \alpha < \beta$ so dass $\alpha, \beta\in I_n\forall n$. Dann $\abs{I_n}\geq\abs{\beta-\alpha}> a$. Widerspruch!
\end{Bew}
\begin{Sat}
  $\forall a\geq 0, a\in\mb{R}$ und $\forall x\in\mb{N}\\\left\{ 0 \right\}$, $\exists$ eine einziges $x\geq 0$, $x\in \mb{R}$ s.d. $x^k=a$. Wir nennen $x=\sqrt[k]{a}=a^\frac{1}{k}$.\\
  Sei $m,n\in\mb{N}$, $a^{m+n}=a^ma^n$ und deswegen $a^{-m}=\frac{1}{a^m}$ für $m\in\mb{N}$ (so dass die Regel $a^{m-m}=a^0=1$.\\
  $n,m\in\mb{N}\\\left\{ 0 \right\}$ $n$ Mal.
  \begin{align*}
    (a^m)^n=\underbrace{a^m\cdot a^m \cdots a^m}_{\text{$n$ Mal}} = a^{\overbrace{m+\cdots+m}^{\text{$n$ Mal}}} = a^{nm}
  \end{align*}
  Und mit $a^{-m}=\frac{1}{a^m}$ stimmt die Regel $(a^m)^n=a^{mn}$ auch $\forall m,n\in\mb{Z}$!
\end{Sat}
\begin{Bem}
    $x^k=\left( a^\frac{1}{k} \right)^k=a\left( =a^{\frac{1}{k}k} = a^1\right)$
\end{Bem}
\begin{Def}
  $\forall q=\frac{m}{n}\in\mb{Q}$, $\forall a>0$ mit definiertem $a^q=\left(\sqrt[n]{a}\right)^m$
\end{Def}
\begin{Bew}
  Mit dieser Definition gilt $a^{q+q_2} = a^qa^{q_2}$ $\forall a>0$ und $\forall q,q_2\in\mb{Q}$.
\end{Bew}

\begin{Sat}
  Zu jedem $x>0$ $(x\in\mb{R})$ und zu jedem $k\in\mb{N}$ gibt es eine reelle Zahl $y>0$ so dass $y^k=x$. In Zeichen:
  \[y=x^\frac{1}{k}, y=\sqrt[k]{x}\]
\end{Sat}
\begin{Bew}
  oBdA $x>1$ (sonst würden wir $\frac{1}{x}$ betrachten). wir konstruieren eine Intervallschachtelung $(I_n)$ so dass $\forall n a_n^k \geq x \geq b_n^k$
  \begin{equation*}
    I_1:=[1,x]\\
    I_{n+1} = \begin{cases}
      \left[ a_n, \frac{a_n+b_n}{2} \right] & \text{falls } x \leq \left( \frac{a_n+b_n}{2} \right)^k
      \left[ \frac{a_n+b_n}{2}, _n \right]
    \end{cases}\\
    \abs{I_n} = \frac{1}{2^{n-1}}\abs{I_1}
  \end{equation*}
  Intervallschachtelungsprinzip $\implies$ $\exists y\in\mb{R}$ s.d. $y\in I_n \forall n\in\mb{N}$ 
\end{Bew}
\begin{Beh}
  $y^k=x$
\end{Beh}
\begin{Bew}
   Man definiert $J_n=[a_n^k, b_n^k$. Wir wollen zeigen, dass $J_n$ eine Intervallschachtelung ist.
  \begin{itemize}
    \item $J_{n+1}\subset J_n$ weil $I_{n+1}\subset I_n$
    \item \[\abs{J_n} = b_n^k-a_n^k = \underbrace{\left( b_n - a_n \right)}_{\abs{I_n}} \underbrace{(b_n^{k-1}+b_n^{k-2}a_n + \cdots + a_n^{k-1})}_{\leq k b_1^{k-1}}\]
  \end{itemize}
  $\implies$ $\abs{J_n}\leq \abs{I_n}k k_1^{k-1}$.\\
  Sei $\varepsilon$ gegeben. Man wähle $N$ gross genug, so dass
  \[\abs{I_n}\leq\varepsilon'=\frac{\varepsilon}{kb_1^{k-1}}\implies\abs{J_n}\leq \varepsilon kb_1^{k-1}=\varepsilon\]
  Einerseits
  \[y\in\left[ a_n,b_n \right]\implies y^k\in\left[ a_n^k, b_n^k \right]=J_n\]
  Andererseits
  \[x\in J_n\forall n\in\mb{N}\]
  Intervallschachtelungsprinzip $\implies$ $x=y^k$
\end{Bew}
\subsection{Supremumseigenschaft, Vollständigkeit}
\begin{Def}
  $s\in\mb{R}$ heisst obere (untere) Schranke der Menge $M\subset \mb{R}$ falls $s\geq x$ ($s\leq x$) $\forall x\in M$.
\end{Def}
\begin{Def}
  $s\in\mb{R}$ ist das Supremum der Menge $M\subset\mb{R}$ falls es die kleinste obere Schranke ist. D.h.
  \begin{itemize}
    \item $s$ ist die obere Schranke
    \item falls $s'<s$, dass ist $s'$ keine obere Schranke.
  \end{itemize}
\end{Def}
\begin{Bsp}
  $M=]0,1[$. In diesem Fall $s=\sup M\not\in M$
\end{Bsp}
\begin{Bsp}
  $M=[0,1]$. $\sup M=1\in M$
\end{Bsp}
\begin{Def}
  $s\in\mb{R}$ heisst Infimum einer Menge $M$ ($s=\inf M$) falls $s$ die grösste obere Schranke ist.
\end{Def}
\begin{Def}
  Falls $s=\sup M\in M$, nennt man $s$ das Maximum von $M$. Kurz: $s=\max M$. Analog Minimum.
\end{Def}
\begin{Sat}
  Falls $M\subset \mb{R}$ nach oben (unten) beschränkst ist, dann existiert $\sup M$ ($\inf M$).
\end{Sat}
\begin{Bew}
  Wir konstruieren eine Intervallschachtelung $I_n$, so dass $b_n$ eine obere Schranke ist, und $a_n$ keine obere Schranke ist.
  \begin{itemize}
    \item $I_1=[a_1, b_1]$, wobei $b_1$ eine obere Schranke
    \item $a_1$ ist keine obere Schranke
  \end{itemize}
  Sei $I_n$ gegeben. 
  \begin{align*}
    I_{n+1} = \begin{cases}
      \left[ a_n,\frac{a_n+b_n}{2} \right]&\text{Falls $\frac{a_n+b_n}{2}$ eine obere Schranke ist-}\\
      \left[ \frac{a_n+b_n}{2}, b_n \right]&\text{sonst}
    \end{cases}
  \end{align*}
  Also, $\exists s\in I_n\forall n$
\end{Bew}
\begin{Beh}
  $s$ ist das Supremum von $M$
  \begin{itemize}
    \item Warum ist $s$ eine obere Schranke? \\
    Angenommen $\exists x\in M$ so dass $x>s$. Man wähle $\abs{I_n}<x-s$. Daraus folgt
    \begin{align*}
      x-s>b_n-a_n \geq b_n-s \implies x>b_n
    \end{align*}
    Widerspruch.
  \item Warum ist $s$ die kleinste obere Schranke?\\
    Angenommen $\exists s'<s$. Dann wähle $n'$ so dass $I_{n'} <s-s'$.
    \begin{align*}
      s-s'>b_{n'}-a_{n'}\geq s-a_{n'} \implies a_{n'}>s'
    \end{align*}
    Widerspruch.
  \end{itemize}
\end{Beh}
\begin{Lem}
  Jede nach oben (unten) beschränkte Menge $M\subsetequal \mb{Z}$ besitzt das grösste (kleinste) Element.
\end{Lem}
\begin{Bew}
  oBdA betrachte nur nach unten beschränkte Mengen $M\subset N$. Angenommen $M$ hat kein kleinstes Element.
\end{Bew}
\begin{Beh}
  \begin{align*}
    \forall n M\cut \left\{ 1,\cdots,n \right\} = \null\\
    n=1\\
    M\cut\left\{ 1 \right\}\\
  \end{align*}
  Angenommen
  \begin{align*}
    M\cut\left\{ 1,\cdots,n \right\} = \null\\
    M\cut\left\{ 1,2,\cdots,n+1 \right\} = M\cut \left\{ 1,\cdots,n \right\}\union M\cut\left\{ n+1 \right\}=\null\\
    \implies M\cut\mb{N}=\null
  \end{align*}
\end{Beh}
\begin{Sat}
  $\mb{Q}$ ist dich in $\mb{R}$, bzw. für beliebige zwei $x,y\in\mb{R}$, $y>x$, gibt es eine rationelle Zahl $q\in\mb{Q}$, so dass $x<q<y$.
\end{Sat}
\begin{Bew}
  Man wähle $n\in\mb{N}$ so dass $\frac{1}{n}<y-x$. Betrachte die Menge $A\subsetequal\mb{Z}$, so dass $M\in A$ $\implies$ $M>nx$. Lemma $\implies$ $\exist m=\min A$.
  \begin{align*}
    x<\frac{m}{n}=\frac{m-1}{n}+\frac{1}{n}<x+y-x=y
  \end{align*}
  Also setze $q=\frac{m}{n}$
\end{Bew}
\subsection{Abzählbarkeit}
\begin{Def}
  Die Mengen $A$ \& $B$ sind \underline{gleichmächtig}, wenn es eine Bijektion $f:A\to B$ gibt. $A$ hat grässere Mächtigkeit als $B$, falls $B$ gleichmächtig wie eine Teilmenge von $A$ ist, aber $A$ zu keiner Teilmenge von $B$ gleichmächtig ist.
\end{Def}
\begin{Bsp}
  \begin{itemize}
    \item ${1,2}$ \& ${3,4}$ sind gleichächtig.
    \item ${1,2,\cdots,n}$ hat kleinere Mächtigkeit als ${1,2,\cdots,m}$, wenn $n<m$ ist.
  \end{itemize}
\end{Bsp}
\begin{Def}
  Eine Menge $A$ ist abzählbar, wenn es eine Bijektion zwischen $\mb{N}$ und $A$ gibt. D.h. $A=\left\{ a_1,a_2,\cdots,a_n,\cdots \right\}$.
\end{Def}
\begin{Lem}
  $\mb{Z}$ ist abzählbar
\end{Lem}
\begin{Bew}
  \begin{tabular}{c|ccccc}
    \mb{N} & 1 & 2 & 3 & 4 & 5 & \ldots \\
    \mb{Z} & 0 & 1 & -1 & 2 & -2 & \ldots
  \end{tabular}
  Formal:
  \begin{align*}
    f=\mb{N}\to \mb{Z}\\
    f(n)=\begin{cases}
      \frac{n}{2} & \text{wenn $n$ gerade}\\
      \frac{1-n}{2} & \text{wenn $n$ ungerade}
    \end{cases}
  \end{align*}
\end{Bew}
\begin{Sat}
  $\mb{Q}$ ist abzählbar
\end{Sat}
\begin{Bew}
  Sucht euch die Graphik auf Wikipedia oder sonstwo.
\end{Bew}
\begin{Sat}
  $\mb{R}$ ist nicht abzählbar.
\end{Sat}

\section{Komplexe Zahlen}
\begin{Bem}
  $\forall a\in \mb{R}$, $a^2>0$. Deswegen ist $x^2=-1$ unlösbar.
  Die Erfindung der imagin\"are Einheit $i$ (die imaginäre Zahl mit $i^2=-1$) 
hat sehr interessante Konsequenzen auch für die üblichen reellen Zahlen.
\end{Bem}
\subsection{Definition}[Erste Definition der Komplexen Zahlen]\label{d:C1}
\begin{Def} Sei $a,b\in\mb{R}$, dann $a+bi\in\mb{C}$. Wir definieren die Summe:
  \begin{equation*}
    (a+bi)+(\alpha+\beta i) = (a+\alpha)+(b+\beta)i
\end{equation*}
und das Produkt
\begin{equation*}
    (a+bi)(\alpha+\beta i) = (a\alpha-b\beta)+ \underbrace{(a\beta+b\alpha)}_A i
  \end{equation*}
\end{Def}
\begin{Def}
  Seien $A$ und $B$ zwei Mengen. Dann ist $A\times B$ die Menge der Paare $(a,b)$ mit $a\in A$ und $b\in B$.
\end{Def}
\begin{Def}[Zweite Definition der Komplezen Zahlen]\label{d:C2}
  $\mb{C}=\mb{R}\times\mb{R}$ mit $+$ und $\cdot$ , die wir so definieren:
  \begin{eqnarray*}
    (a,b)+(\alpha,\beta)&=&(a+\alpha,b+\beta)\\
    (a,b)(\alpha,\beta)&=&(a\alpha-b\beta, \underbrace{a\beta+b\alpha}_{A})
  \end{eqnarray*}
\end{Def}
\begin{Bem}
  \begin{equation*}
    \mb{R}\simeq \left\{ (a,0), a\in\mb{R} \right\}\subset\mb{C}\\
\end{equation*}
In der Sprache der abstrakte Algebra $\mb{R}$ ist isomorph zu 
$\mb{R}' := \{(a,0):a\in \mb{R}\}$: d.h. die Summe und 
das Produkt in $\mb{R}$ und $\mb{R}'$ sind ``gleich'': 
\begin{eqnarray*} 
    (a,0)+(\alpha,0)=(a+\alpha,0)\\
    (a,0)(\alpha,0)=(a\alpha,0)
  \end{eqnarray*}
Deswegen wir schreiben $a$ statt $(a,0)$.
\end{Bem}
\begin{Bem}
  \begin{equation*}
    (0,a)(0,b)=(-ab, 0)
  \end{equation*}
  Deswegen:
  \begin{equation*}
    \underbrace{(0,1)}_{\text{Wurzel von -1}}(0,1)=(-1,0)\\
    \underbrace{(0,-1)}_{\text{auch eine Wurzel von -1}}(0,-1)=(-1,0)
  \end{equation*}
\end{Bem}
\begin{Bem}
  $i=(0,1)$ und wir schreiben $(a,b)$ für $a+bi$. D.h. die zwei Definitionen der
komplezen Zahlen sind equivalent!
\end{Bem}
\begin{Bem}
  $0=(0,0)=0+0i$. $\xi\in\mb{C}$
  \begin{align*}
    0\xi=0\\
    0+\xi=\xi
  \end{align*}
\end{Bem}
\begin{Sat}\label{s:CK}
  Alle Körperaxiome (K1-K4) gelten.
\end{Sat}
\begin{proof}[Beweis]
  \begin{itemize}
    \item[K1] Kommultativität: {\em trivial}
    \item[K2] Assoziativität: {\em trivial}
    \item[K3] Distributivität: {\em trivial}.
    \item[K4] Seien $\xi, \zeta \in\mb{C}$.
      \begin{align}
        \exists \omega\in\mb{C} :&\qquad \xi+\omega=\zeta\label{e:minus}\\
        \xi\neq 0\exists \omega: &\qquad \xi\omega=\zeta\label{e:/}
      \end{align}
  \end{itemize}
{\em Zu \eqref{e:minus}}. Wir setzen
  \begin{eqnarray*}
    \xi=a+bi\\
    \zeta=c+di\\
    \omega=x+yi
  \end{eqnarray*}
  \begin{equation*}
    \xi+\omega = (a+x)+(b+y)i = \xi = c+di
  \end{equation*}
  Sei $x:=c-a$, $y:=d-b$. Dann $\xi+\omega=\zeta$.

\medskip

{\em Zu \eqref{e:/}} $1$ ( $= 1+0i)$) das neutrale Element.
  \begin{equation*}
    (a+bi)(1+0i)=\underbrace{(a1-b0)}_{a}+\underbrace{(b1+a0)}_{b}=(a+bi)
  \end{equation*}
  Sei $\xi\neq 0$ und suchen wir $\alpha$ so dass $\xi\alpha=1$. Dann ist $\omega=\alpha\zeta$ eine Lösung von 
\eqref{e:/} (eigentlich DIE Lösung). Falls $\xi=a+bi$, dann
  \begin{equation*}
    \alpha=\frac{a}{a^2+b^2}-\frac{b}{a^2+b^2}\, .
\end{equation*}
In der Tat:
\begin{equation*}
    \xi\alpha=\overbrace{\left( \frac{aa}{a^2+b^2}-\frac{b(-b)}{a^2+b^2} \right)}^1 + 
    \overbrace{\left( \frac{a(-b)}{a^2+b^2}-\frac{ab}{a^2+b^2} \right)}^0 i=1\, .
\end{equation*}
\end{proof}
\begin{Def}
  Sei $\xi=(x+yi)\in\mb{C}$. Dann:
  \begin{itemize}
    \item $x$ ist der reelle Teil von $\xi$ $(\Re\xi=x)$
    \item $y$ ist der imaginäre Teil von $\xi$ $(\Im\xi=y)$
    \item $x-yi$ ist die konjugierte Zahl $\left( \ol\xi =x-yi \right)$
  \end{itemize}
\end{Def}
\begin{Bem}
  \[\sqrt{\ol\xi\ol\xi}=\sqrt{\left( \Re\xi \right)^2+\left( \Im\xi \right)^2}=:\abs{\xi}\]
\end{Bem}
\begin{Def}
  $\abs{\xi}$ ist der Betrag von $\xi$.
\end{Def}
\begin{Sat}
  Es gilt: $(\forall a,b\in\mb{C})$:
  \begin{itemize}
    \item 
      \begin{itemize}
        \item \[\ol{a+b}=\ol{a}+\ol{b}\]
        \item \[\ol{ab}=\ol{a}\ol{b}\]
      \end{itemize}
    \item 
      \begin{itemize}
        \item \[\Re a =\frac{a+\ol{a}}{2}\]
        \item \[(\Im a) i=\frac{a-\ol{a}}{2}\]
      \end{itemize}
    \item $a=\ol{a}$ genau dann wenn $a\in\mb{R}$.
    \item \[a\ol a=\abs{a}^2=\sqrt{\left( \Re a \right)^2+\left( \Im a \right)^2}\geq 0\]
      (die Gleicheit gilt genau dann wenn $a=0$)
  \end{itemize}
\end{Sat}
\begin{Bem}
  Sei $\omega$ so dass $\xi\omega=1$ $(\xi\neq 0)$. Man schreibt $\omega\frac{1}{\xi}$.
Der Beweis vom Satz \ref{s:CK} impliziert $\omega=\frac{\ol\xi}{\abs{\xi}^2}$
\end{Bem}
\begin{Sat}
  $\forall a,b\in\mb{C}$
  \begin{itemize}
    \item $\abs{a}>0$ für $a\neq 0$ (trivial)
    \item $\abs{\ol a}=\abs{a}$ (trivial)
    \item $\abs{\Re a}\leq\abs{a}$, $\abs{\Im a}\leq\abs{a}$ (trivial)
    \item $\abs{ab}=\abs{a}\abs{b}$
    \item $\abs{a+b}\leq\abs{a}+\abs{b}$
  \end{itemize}
\end{Sat}
\begin{proof}[Beweis]
  \begin{equation*}
    \abs{ab}^2=(ab)\ol{(ab)}=ab\ol a\ol b=a\ol a\ol b=\abs{a}^2\abs{b}^2\\
    \implies \abs{ab}=\abs{a}\abs{b}
  \end{equation*}
\begin{eqnarray}
\underbrace{|a+b|^2}{\in \mb{R}} &=& (a+b)\ol{(a+b)} = 
    (a+b)(\ol a+\ol b)=a\ol a+b\ol b+a\ol b+b\ol a \nonumber\\
&=& \underbrace{\abs{a}^2+\abs{b}^2}_{\in\mb{R}}+\left( a\ol b+b\ol a \right)\, .\label{e:10}
\end{eqnarray}
Bemerkung: die Identit\"at implizert dass $a\ol b+b\ol a$. In der Tat
$a\ol b + b\ol a= a\ol b+ \ol{a\ol b} = 2 \re a \ol b$. Deswegen
\begin{eqnarray}
|a+b|^2 &=& |a|^2 +|b|^2 + 2 \re a\ol b \;\leq\; |a|^2+|b|^2+ 2|a \ol b|\nonumber\\
&=& |a|^2 + |b|^2 + 2 |a||b| = (|a|+|b|)^2\, .
\end{eqnarray}
\end{proof}


\section{Funktionen}\label{s:F}
\subsection{Definition}
\begin{Def}
  Seien $A$ und $B$ zwei Mengen. Eine Funktion $f:A\to B$ ist eine Vorschrift die jedem Element $a\in A$ ein eindeutiges Element $f(a)\in B$ zuordnet.
\end{Def}
\begin{Bsp}
  $A\subset \mb{R}$, $B=\mb{R}$ (oder $\mb{C}$)
  \[f(x)=x^2\]
\end{Bsp}
\begin{Def}
  $A$ ist der \underline{Definitionsbereich}.
  \[f(A)=\left\{ f(x): x\in A \right\}\]
  ist der Wertbereich
\end{Def}
\begin{Bem}
  Wertbereich von $x^2$:
  \[\left\{ y\in\mb{R}: y\geq 0 \right\}\]
\end{Bem}
\begin{Def}
  Der Graph einer Funktion $f:A\to B$ ist
  \[G(f)=\left\{ (x, f(x))\in A \times B:x\in A \right\}\]
\end{Def}
\begin{Bsp}
  % TODO scannen
  Verboten: zwei Werte für die Stelle $x$.
\end{Bsp}
\begin{Bsp}
  $f:\mb{R}\to\mb{R}$ $f(x)=\abs{x}$
\end{Bsp}
\subsection{Algebraische Operationen}
Wenn $B=\mb{R}$ oder $\mb{C}$. Seien $f,g$ zwei Funktionen mit gleichem Definitionsbereich.
\begin{itemize}
  \item $f+g$ ist die Funktion $h$ so dass $h:A\to B$ \[h(x)=f(x) + g(x)\]
  \item Die Funktion $fg$ ist $k:A\to B$ \[k(x)=f(x)g(x)\]
  \item $\frac{f}{g}$ ist wohldefiniert falls der Wertebereich von $g$ in 
$B\setminus \left\{ 0 \right\}$ enthalten ist: \[\frac{f}{g}(x)=\frac{f(x)}{g(x)}\]
\end{itemize}
Falls $B=\mb{C}$, kann man auch $\Re f$, $\Im f$, $\ol f$.
\begin{Def}
  Sei $f:A\to B$, $g:B\to C$. Die Komposition $g\circ f: A\to C$.
  \[g\circ f(x)=g(f(x))\]
\end{Def}
\begin{Bem} Sei
  $f:A\to\mb{R}$, $g:A\to \mb{R}$. Wir definieren $\Xi:A\to\mb{R}\times\mb{R}$:
  \[\Xi(a)=(f(a),g(a))\]
und $\Phi:\mb{R}\times\mb{R}\to\mb{R}$:
$$
\Phi(x,y)=xy
$$
Dann
$$
\Phi\circ\Xi(a)=\Phi(\Xi(a))=\Phi\left( \left( f(a) \right),g(a) \right)=f(a)g(a)
$$
Also: die ``algebraischen Operationen'' sind ``Kompositionen''.
\end{Bem}
\begin{Def}
  \begin{itemize}
    \item Wenn $f:A\to B$ und $f(A)=B$ dann ist $f$ \ul{surjektiv}.
    \item Wenn $f:A\to B$ und die folgende Eigenschaft hat:
      \[f(x)\neq f(y)\forall x\neq y\in A\]
      dann ist $f$ \ul{injektiv}.
    \item Falls $f$ surjektiv und injektiv ist, dann sagen wir, dass $f$ bijektiv ist.
  \end{itemize}
\end{Def}
\begin{Bem}
  Die bijektiven Funktionen sind umkehrbar. Sei $f:A\to B$ bijektiv. $\forall b\;\exists a:f(a)=b$ (surjektiv), $a$ ist eindeutig (injektiv) (die Notation f\"ur die Eindeutigkeit ist $\exists! a:f(a)=b$). 
Dann $g(b)=a$ ist eine ``wohldefinierte Funktion'', $g:B\to A$.
\end{Bem}
\begin{Def}
  $g$ wird Umkehrfunktion genannt. $f:A\to B$, $g:B\to A$, $f\circ g: B\to B$,
$g\circ f:A\to A$ und
\begin{equation}\label{e:Um}
f\circ g(b)=b \quad \forall b\in B \qquad  g\circ f(a)=a\quad\forall a\in A\, .
\end{equation}
\end{Def}
\begin{Def}
  Die ``dumme Funktion'' $h:A\to A$ mit $h(a)=a\;\forall a\in A$ heisst Identitätsfunktion ($\id$). 
Deswegen, \eqref{e:Um} $\iff$ $f\circ g=\id$ und $g\circ f = \id$.
\end{Def}
\subsection{Zoo}
\subsubsection{Exponentialfunktion}
$a\in\mb{R}, a>0$. Defintionsbereich $\mb{Q}$ (momentan!):
\[\Exp_a:\mb{Q}\to\mb{R}\]
\[\Exp_a(n)=a^n \qquad(=1 \mbox{ falls }n=0)\]
\[\Exp_a(-n)=\frac{1}{a^n}\]
\[\Exp_a\left( \frac{m}{n} \right) = \left( \sqrt[n]{a} \right)^m\]
$\Exp_a$ ist die \ul{einzige} Funktion $\Phi:\mb{Q}\to\mb{R}$ mit den folgenden Eigenschaften:
\begin{itemize}
  \item $\Phi(1)=a$
  \item $\Phi(q+r)=\Phi(q)\Phi(r)$ $\forall q,r\in\mb{Q}$
\end{itemize}
\begin{Bem}
  Später werden wir $\Exp_a$ auf \ul{$\mb{R}$ fortsetzen}.
\end{Bem}
\subsubsection{Polynome}
\[f(x)=a_nx^n+a_{n-1}x^{n-1}+\cdots+a_1x+a_0\]
\[f:\mb{R} (\mb{C}) \ni x\mapsto f(x)\in\mb{R} (\mb{C})\]
Produkt von Polynomen $x\mapsto f(x)g(x)$:
\begin{eqnarray*}
  f(x)g(x) &=& \left( a_nx^n+\cdots+a_0 \right)\left( b_mx^m+\cdots+b_0 \right)\\
&=&  b_ma_nx^{n+m}+b_na_{n-1}x^{n-1+m}+\cdots\\
&=&  b_ma_nx^{n+m}+\left( b_ma_{n-1}+b_{m-1}a_n \right)x^{n+m-1}+\cdots + a_0b_0\\ 
&=&
  c_{m+n}x^{m+n} +\cdots +c_0\, ,
\end{eqnarray*}
wobei
\[c_k=\sum_{i+j=k}a_ib_j=\sum_{i=0}^ka_ib_{k-i}\]
\begin{Def}
  Der Grad von $a_nx^n+\cdots+a_0$ ist $n$ wenn $a_n\neq 0$
\end{Def}
\begin{Sat}\label{s:PDiv}
  Sei $g\neq 0$ ein Polynom. Dann gibt es zu jedem Polynom $f$ zwei Polynome $q$ und $r$ so dass
  \[g=qf+r\]
  \[\Grad r < \Grad f\]
\end{Sat}
\begin{proof}[Beweis]
  \url{http://de.wikipedia.org/wiki/Polynomdivision}
\end{proof}
\begin{Bem}
  Sei $g=x-x_0$. Sei $f$ mit Grad $\geq 1$, Satz \ref{s:PDiv} 
$\implies f=gq+r=gq+c_0$ und Grad von $r < 1$. $r$ ist eine Konstante $r=c_0$. Deswegen
  \[f(x)=q(x)(x-x_0)+c_0\]
  \[f(x_0)=q(x_0)0+c_0 = c_0\]
\end{Bem}
\begin{Kor}
  Falls $f$ ein Polynom ist und $f(x_0)=0$, dann $\exists q$ Polynom so dass $f=q(x-x_0)$
\end{Kor}

Das Polynom $a_n x^n + \ldots + a_0$ mit $a_n= \ldots = 0$ ist das {\em Trivialpolynom}.

\begin{Kor}
  Ein Polynom $P$ hat höchstens $\Grad f$ Nullstellen falls $P$ ist nicht das Trivialpolynom.
\end{Kor}



\begin{Kor}
  Falls $f(x)=0$ $\forall x\in\mb{R}$, dann ist $f$ das Trivialpolynom.
\end{Kor}
\begin{Kor}
  Falls $f,g$ Polynome sind und $f(x)=g(x)$ $\forall x\in\mb{R}$ dann sind die Koeffizienten von $f$ und $g$ gleich.
\end{Kor}
\begin{proof}[Beweis]
  $f-g$ ist ein Polynom mit $(f-g)(x)=0$ $\forall x$. 
\end{proof}
\begin{Def}
  Seien $f,g$ Polynome. Dann ist $\frac{f}{g}$ eine rationale Funktion.
\end{Def}

\section{Folgen}
\begin{Def}
  Eine Folge komplexer (reeller) Zahlen ist eine Abbildung: $f:\mb{N}\to\mb{C}(\mb{R})$. Das heisst:
  \[\forall n\in\mb{N}, f(n)\in\mb{C}(\mb{R}), a_n:=f(n)\]
  $\mb{N}$ ist auch eine Folge! $a_n=f(n)=n$.
\end{Def}
\begin{Def}
  Eine Folge $(a_n)$ heisst konvergent, falls $\exists a\in\mb{C}$ so dass:
  \[\forall\varepsilon>0\s\exists N\in\mb{N}:\abs{a_n-a}<\varepsilon\s\forall n\geq N\]
\end{Def}
\begin{Bsp}
  $a_n=\frac{1}{n}$ ist eine konvergente Folge. Sei $a=0$. Wählen wir $\varepsilon>0$. Sei dann $N>\frac{1}{\varepsilon}$. Für $n\geq N$:
  \[\abs{a_n}=\left( \frac{1}{n}-0 \right)=\frac{1}{n}\leq\frac{1}{N}\]
\end{Bsp}
\begin{Bem}
  Die Zahl $a$ im Konvergenzkriterium ist eindeutig. Sie heisst \ul{der Limes} der Folge $(a_n)$.
\end{Bem}
\begin{Bew}
  Seien $a\neq a'$ zwei relle Zahlen, die das Konvergenzkriterium erfüllen. $\varepsilon:=\frac{\abs{a-a'}}{2}$
  \[\exists N:\abs{a_n-a}<\varepsilon\s\forall n\geq N\]
  \[\exists N:\abs{a_n-a'}<\varepsilon\s\forall n\geq N'\]
  Für $n\geq \max \left\{ N,N' \right\}$
  \[\abs{a'-a}\leq\abs{a'-a_n}+\abs{a-a_n}<2\varepsilon=\abs{a'-a}\]
  \[\abs{a'-a}<\abs{a'-a}\]
  $\implies$ Widerspruch
\end{Bew}
\begin{Bem}
  $a=\lim_{n\to+\infty}(a_n)$
\end{Bem}
\begin{Bem}
  $\exists M$ so dass $M>\frac{1}{\varepsilon}$, $M$ hat $N$ Ziffern: $10^{N+1}>M>\frac{1}{\varepsilon}$
\end{Bem}
\begin{Def}
  Sei $(a_n)$ eine Folge und $A(n)$ eine ``Folge von Aussagen über $a_n$''. Wir sagen dass $A(n)$ wahr für ``fast alle'' $a_n$ ist, wenn $\exists N$ so dass $A(n)$ stimmt $\forall n\geq N$. Ein alternatives Konvergenzkriterium ist also:
  \[\abs{a_n-a}<\varepsilon\s\text{für fast alle}\s a_n\]
\end{Def}
\begin{Bsp}
  Sei $s\in\mb{Q}$ $s>0$. Sei $a_n=\frac{1}{n^s}$. Dann
  \[\lim_{n\to+\infty}\left( \frac{1}{n^s} \right)=0\]
  Sei $n>\varepsilon^{\frac{1}{s}}$.
  \[\abs{0-a_n}=\frac{1}{n^s}<\varepsilon\]
  falls $n\geq N$. $\frac{1}{s}$ ist wohldefiniert $(s\neq 0)$.
  \[\frac{1}{n^s}<\varepsilon \iff n^s>\frac{1}{\varepsilon}\iff n>\frac{1}{\varepsilon^{\frac{1}{s}}}\s\text{falls}\s s>0\]
\end{Bsp}
\begin{Bsp}
  $a>0$
  \[\lim_{n\to+\infty}\sqrt[n]{a}=1\]
  $a>1$ zu beweisen:
  \[\forall \varepsilon>0 \s\exists N:\sqrt[n]{a}-1<\varepsilon\s\forall n\geq N\in\mb{N}\]
  Sei $x_n=\sqrt[n]{a}-1>0$
  \[a=(1+x)^n=1+nx_1+\binom{n}{2}x^2_2+\binom{n}{3}x^3+\dots+x^n\]
  \[a\geq +nx_n\s x_n\leq\frac{a-1}{n}\s\text{für}\s n\geq N\]
  Sei $\varepsilon>0$. Wähle $N\geq \frac{a-1}{\varepsilon}$
  \[\implies\sqrt[n]{a}-1=x_n\leq\frac{a-1}{n}\leq\frac{a-1}{N}<\frac{a-1}{\frac{a-1}{\varepsilon}}=\varepsilon\]
  $0<a<1$
  \[\frac{1}{a}>1\s\sqrt[n]{a}=\frac{1}{\sqrt[n]{\frac{1}{a}}}\s\text{ist fast 1}\]
\end{Bsp}
\begin{Bsp}
  $\lim_{n\to+\infty}\sqrt[n]{n}$
  \[x_n=\sqrt[n]{n}-1\]
  \[n=(1+x_n)^n=1+nx_n+\binom{n}{2}x_2^2+\dots+x_n^n\]
  $a_n=\sqrt[n]{n}$
  \[n\geq 1\]
  $a_0=$ Pietro % TODO WTF?
  \[n\geq 1+nx_n\]
  \[0\leq x_n\leq\frac{n-1}{n}=1-\frac{1}{n}\]
  \[n\geq1+\binom{n}{2} x_n^2=1+\frac{n(n-1)}{2}x_n^2 \]
  \[x_n^2\leq\frac{2}{n-1}\implies x_n\leq \sqrt[2]{\frac{2}{n-1}}\]
  Sei $\varepsilon>0$, wähle $N$ so dass
  \[\sqrt{\frac{N-1}{2}}>\varepsilon^{-1}\iff N>\left( 2\varepsilon^{-2}+1 \right)\]
  \[0\geq\sqrt[n]{n}-1<\sqrt{\frac{2}{n-1}}\leq\sqrt{\frac{2}{N-1}}<\sqrt{\frac{2}{\frac{2}{\varepsilon^2}}}=\varepsilon\]
  \[\implies \abs{\sqrt[n]{n}-1}<\varepsilon\]
\end{Bsp}
\begin{Ueb}
  $\sqrt[n]{n^2}$, $\sqrt[n]{n^k}$ für $k$ konstant. Antwort $\lim = 1$
\end{Ueb}
\begin{Bsp}
  Sei $q\in\mb{C}$ mit $\abs{q}<1$.
  \[\lim_{n\to+\infty}q^n=0\]
  \[\abs{q^n-0}=\abs{q^n}-\abs{0}\leq\abs{q}^n\implies\forall \varepsilon>0\s \exists N:\abs{q^n}<\varepsilon\s\forall n\geq N\]
\end{Bsp}
\subsection{Rechenregeln}
\begin{Sat}
  Seien $(a_n)$ und $(b_n)$ zwei konvergente Folgen, mit $a_n\to a$ und $b_n\to b$, dann:
  \begin{itemize}
    \item $a_n b_n\to ab$
    \item $a_n+b_n\to a+b$
    \item $\frac{a_n}{b_n}\to\frac{a}{b}$ falls $b\neq 0$
  \end{itemize}
\end{Sat}
\begin{Bew}
  \[\abs{\left( a_n+b_n \right)-\left( a-b \right)}=\abs{\left( a_n-a \right)+\left( b_n-b \right)}\leq\abs{a_n-a}+\abs{b_n-b}\]
  $\varepsilon>0$:
  \[exists N:\abs{a_n-a}<\frac{\varepsilon}{2}\s\forall n\in\mb{N}\forall n\geq N\]
  \[exists N':\abs{a_n-a}<\frac{\varepsilon}{2}\s\forall n\in\mb{N}\forall n\geq N'\]
\end{Bew}
\begin{Def}
  Eine Folge heisst beschränkt, falls
  \[\exists M>0:\abs{a_n}\leq M\]
\end{Def}
\begin{Lem}
  Eine konvergente Reihe ist immer beschränkt.
\end{Lem}
\begin{Bew}
  \begin{align*}
    = \abs{a_nb_n-ab}\\
    = \abs{a_nb_n-a_nb+a_nb-ab}\\
    = \abs{a_n(b_n-b)+b(a_n-a)}\\
    \leq \abs{a_n}\abs{b_n-b}+\abs{b}\abs{a_n-a}\\
    \leq M\abs{b_n-b}+\abs{b}\abs{a_n-a}\\
    < M\left( \frac{\varepsilon}{2M}\right)+\abs{b}\left( \frac{\varepsilon}{2\abs{b}} \right)
  \end{align*}
\end{Bew}
\begin{Bew}
  Folgt aus dem oberen $+\frac{1}{b_n}\to\frac{1}{b}$ falls $b\neq 0$
  \begin{align*}
    \frac{1}{b_n}-\frac{1}{b}\\
    =\abs{\frac{b-b_n}{b_nb}}\\
    \leq\frac{1}{\abs{b}}\frac{\abs{b-b_n}}{\abs{b_n}}\\
    \leq\frac{2}{\abs{b}^2}\abs{b_n-b}\\
    < \cdots
  \end{align*}
  Sei $\varepsilon=\frac{\abs{b}}{2}$ dann
  \[\exists N:\abs{b_n-b}<\frac{\abs{b}}{2}\s\forall n\geq N\]
  \[\abs{b_n}\geq \abs{b}-\abs{b-b_n}\geq \frac{\abs{b}}{2}>0\]
\end{Bew}
\begin{Sat}
  Sei $a_n\to a$ ($a_n\in\mb{C}$), dann:
  \begin{itemize}
    \item $\abs{a_n}\to\abs{a}$
    \item $\bar{a_n}\to\bar{a}$
    \item $\Re a_n\to\Re a$
    \item $\Im a_n\to\Im a$
  \end{itemize}
\end{Sat}
\begin{Bew}
  \begin{itemize}
    \item $\abs{\abs{a_n}-\abs{a}}\leq \abs{a_n-a}$
    \item $\abs{\bar{a_n}-\bar{a}}= \abs{a_n-a}$
    \item $\abs{\Im a_n-\Im a}\leq \abs{a_n-a}$
    \item $\abs{\Re a_n-\Re a}\leq \abs{a_n-a}$
  \end{itemize}
\end{Bew}
\begin{Sat}
  Seien $a_n\to a$, $b_n\to b$ mit $a_n\leq b_n$. Dann $a\geq b$.
\end{Sat}
\begin{Kor}
  Seien $a_n\to a$, $b_n\to a$. Sei $a_n$ mit $a_n\geq c_n\geq b_n$. Dann ist $c_n$ eine konvergente Folge mit $c_n\to a$
\end{Kor}

%\begin{Sat}
%  Seien $a_n\to a$ und $b_n\to b$ reelle Folgen. Falls $a_n\leq b_n$, dann $a\leq b$.
%\end{Sat}
%\begin{Bew}
%  Sei $\varepsilon>0$. Dann:
%  \begin{itemize}
%    \item 
%      \[\exists N:\abs{a_n-a}<\varepsilon\s\forall n\geq N\]
%    \item
%      \[\exists N:\abs{b_n-b}<\varepsilon\s\forall n\geq N'\]
% \end{itemize}
%  Sei $n\geq \max\left\{ N',N \right\}$.
%  \[b-a=b_n+(b-b_n)-a_n+(a_n-a)\geq (b-b_n)+(a_n-a)\]
%  \[\geq -\abs{a_n-a}-\abs{b_n-b}\geq -2\varepsilon\]
%  \[b-a\geq -2\varepsilon\stackrel{\forall \varepsilon>0}{\implies} b-a\geq 0\]
%  (wäre $b-a<0$: Sei
%  \[Sei \varepsilon=\frac{\abs{b-a}}{3}=\frac{-(b-a)}{3}\]
%  Widerspruch!)
%\end{Bew}
%\begin{Sat}
%  (Einschliessungsregel). Sei $c_n$ eine Folge reeller Zahlen. Seien $a_n\to a$ und $b_n\to a$ so dass $a_n\leq c_n\leq b_n$. Dann $c_n\to a$.
%\end{Sat}
%\begin{Bew}
%  Sei $\varepsilon>0$.
%  \begin{itemize}
%    \item 
%      \[\exists N:\abs{a_n-a}<\varepsilon\s\forall n\geq N\]
%    \item
%      \[\exists N:\abs{b_n-a}<\varepsilon\s\forall n\geq N'\]
%  \end{itemize}
%  Sei $n\geq \max\left\{ N,N' \right\}$.
%  \[a\varepsilon\leq a_n\leq c_n\leq b_n<a+\varepsilon\]
%  \[\implies \abs{c_n-a}<\varepsilon\]
%\end{Bew}
%\begin{Bsp}
%  \[\Limi{n}\sqrt[n]{n^k}=1\s k\in\mb{N}\]
%  \[\sqrt[n]{n^s}\s s\in\mb{Q}, s>0\]
%  \[\underbrace{1}_{a_n}\leq\underbrace{\sqrt[n]{n^s}}_{c_n}\leq\underbrace{\sqrt[n]{n^k}}_{b_n}\]
%  Einschliessungsregel: $\sqrt[n]{n^s}\to 1$.
%\end{Bsp}
\subsection{Monotone Folgen}
\begin{Def}
  Eine Folge $(a_n)$ reeller Zahlen heisst fallend (bzw. wachsend) 
falls $a_n\leq a_{n-1}$ $\forall n\in\mb{N}$ (bzw. $a_n\geq a_{n-1}\forall n\in\mb{N}$). 
Monoton bedeuted fallend oder wachsend.
\end{Def}
\begin{Sat}
  Eine monotone beschränkte Folge konvergiert.
\end{Sat}
\begin{proof}[Beweis]
  OBdA k\"onnen wir $(a_n)$ wachsend annehmen. (Sei $a_n$ fallend, 
dann $-a_n$ ist wachsend. Falls $-a_n \to L$, dann 
\[
a_n=(-1)(-a_n) \to \lim(-1)\lim(-a_n) = -L)\, .
\] 
Sei 
  \[s=\sup \underbrace{\left\{ a_n:n\in\mb{N} \right\}}_{=:M}\]
  Behauptung:
  \[s=\Limi{n} a_n\]
Da $a_n\geq a$, wir sollen beweisen dass:
\begin{equation}\label{e:zuBeweisen}
 \forall \varepsilon>0\s\exists N: \qquad
a_n> s-\varepsilon\quad\forall n\geq N\, .
\end{equation}
Sei $\varepsilon > 0$. Dann
  \[\exists a_j\in M:\qquad a_j>s-\varepsilon\]
(sonst w\"are $s-\varepsilon$ eine obere Schranke kleinere als
$s$).
  Die Folge wächst $\implies$ $a_n\geq a_j>s-\varepsilon$ $\forall n\geq j$.
\end{proof}
\begin{Bsp} Die Beschr\"akheit impliziert nicht die Konvergenz:
 \[a_n=(-1)^n\]
\end{Bsp}
\subsection{Der Satz von Bolzano-Weierstrass}
\begin{Def}
  Sei $(a_n)$ eine Folge. Eine Teilfolge von $(a_n)$ ist eine neue Folge 
$b_k:= a_{n_k}$, wobei $n_k\in\mb{N}$ mit $n_k>n_{k-1}$ (zum Beispiel:
\[\underbrace{a_0}_{b_0} \s a_1\s\underbrace{a_2}_{b_1}
\s a_3\s a_4 \s a_5 \s\underbrace{a_6}_{b_2} \quad \ldots )\, .\]
\end{Def}
\begin{Sat}[Bolzano-Weierstrass]
  Jede berschränkte Folge $(a_n)$ $(\subset\mb{R}, \mb{C})$ 
besitzt eine konvergente Teilfolge.
\end{Sat}
\begin{proof}[Beweis]
  {\bf Schritt 1}: Sei $(a_n)$ eine Folge reeller Zahlen. 
Sei $I$ und $M\in\mb{R}$ so dass $I\leq a_n\leq M$ 
$\forall n\in\mb{N}$. OBdA $I<M$, sonst ist $(a_n)$ eine konstante
Folge. Wir definieren $J_0 = [I, M]$ und teilen es in zwei Intervallen: 
 \[ J_0 = \left[ I,A_0 \right]\cup 
\left[ A_0, M \right]\s A_0=\frac{M-I}{2}+I=\frac{M+I}{2}\]
  mindestens ein Intervall enthält unendlich viele $a_n$. 
Nennen wir dieses Intervall $J_1$.

Rekursiv definieren wir eine Folge von Intervallen $J_k$ s.d.
  \begin{itemize}
    \item $J_{k+1}\subset J_k$;
    \item Die Länge $\ell_k$ von $J_k$ ist $(M-I)2^{-k}$;
    \item jedes Intervall ent\"alt unendlich viele gliedern der Folge
$(a_n)$.
  \end{itemize}
Diese Folge ist eine Intervallschachtelung und deswegen $\exists ! L$
mit $L\in J_i \;\forall i$.

Wir w\"ahlen $n_0\in \mb{N}$ s.d. $a_{n_0}\in J_0$. Da $J_1$ ent\"alt 
unendlich viele $a_n$, $\exists n_1>n_0$ mit $a_{n_1}\in J_1$.
Rekursiv definieren wir eine Folge nat\"urlicher Zahlen $(n_k)$
mit $n_{k+1}>n_k$ und $a_{n_k}\in J_k$. Die Folge $b_k:= a_{n_k}$
ist eine Teilfoge von $(a_n)$. Ausserdem
\[
|b_k-L|\leq \ell_k = 2^{-k} (M-I)\, ,
\]
weil $b_k, s\in J_k$. Deswegen $b_k\to L$.

\[\exists a\in J_k\s\forall k\in\mb{N}\]
  \[\exists n_0: a_{n_0}\in J_0\]
  $J_1$ enthält unendlich viele $a_n\implies \exists n_1>n_0$ mit $a_{n_1}\in J$. 

\medskip

{\bf Schritt 2.} Sei nun $a_k=\xi_k+i\zeta_k$ eine beschr\"ankte 
komplexe Folge. 
$(\xi_k)$ ist eine beschränkte Folge reeller Zahlen. 
Aus dem Schritt 1 $\exists (\xi_{k_j})$ Teilfolge die konvergiert.
$(\zeta_{k_j})$ ist auch eine beschränkte Folge reeller Zahlen 
und deswegen besitzt eine konvergente Teilfolge $(\zeta_{k_{j_n}})$.
Dann
\[b_n := a_{k_{j_n}}=\xi_{k_{j_n}}+i\zeta_{k_{j_n}}\]
ist eine konvergente Teilfolge!
\end{proof}
\begin{Def}
  Falls $(a_k)$ eine Folge ist und $a$ der Limes einer Teilfolge, dann heisst $a$ \ul{Häufungswert}.
\end{Def}
\begin{Lem}\label{l:char_hauf}
  Sei $(a_k)$ eine Folge. $a$ Häufungswert $\iff$ 
$\forall$ offenes Invervall mit $a\in I$ $\exists$ unendlich viele $a_k\in I$.
\end{Lem}
\begin{proof}[Beweis]
Trivial \end{proof}
\begin{Def}
Wenn die Menge der Häufungswerte von $(a_n)$ (relle Folge) ein Supremum 
(bzw. ein Infimum) besitzen, heisst dieses Supremum 
``Limes Superior'' (bzw. ``Limes Inferior'') und wir nuzten die Notation
\[\limsup_{n\to\infty} a_n 
\qquad (\mbox{bzw. } \liminf_{n\to\infty} a_n).\]
Wenn die Folge keine obere (bzw. untere) Schranke besitzt,
wir schreiben
\[ \limsup_{n\to \infty} a_n = \infty \quad
 (\mbox{bzw. } \liminf_{n\to\infty} a_n = -\infty.)
\]
\end{Def}

\begin{Bem}
Eine konvergente Folge hat genau einen H\"aufungswert, d.h.
der Limes der Folge! 
\end{Bem}

\begin{Lem}
Der Limes Superior (bzw. Inferior) ist das Maximum (bzw. Minimum) 
der Häufungswerte, falls er enldich ist. 
Ausserdem eine reelle Folge konvergiert genau
dann, wenn der Limes superior und der Limes inferior gleich und endlich
sind.
\end{Lem}
\begin{proof}[Beweis]
{\bf Teil 1} Sei $\limsup_n a_n= S\in \mb{R}$. Zu beweisen
ist dass $S$ ein H\"aufungswert ist. Sei $I=]a,b[$ ein Intervall mit
$S\in I$. Wir behaupten dass $I$
unendlich viele Gliedern von $(a_n)$ besitzt: es folgt dann
aus Lemma \ref{l:char_hauf} dass $S$ ein H\"aufungswert ist. Da $S$ das Supremum
der H\"aufungswerte ist, $\exists$ ein H\"aufungswert $h >a$. 
Aber dann $h\in I$, und aus Lemma \ref{l:char_hauf}
folgt dass $I$ unendlich viele $a_n$ enth\"alt. 

\medskip

{\bf Teil 2}. Sei $(a_n)$ eine Folge mit
\[
 \liminf_{n\to\infty} a_n = \limsup_{n\to \infty} a_n = L \in \mb{R}\, .
\]
Es folgt dass $(a_n)$ eine beschr\"ankte Folge ist. Falls
$a_n$ nicht nach $L$ konvergiert, dann $\exists \varepsilon >0$
und unendlich viele $a_n$ mit $|a_n-L|>\varepsilon$, d.h. eine
Teilfolge $b_k=a_{n_k}$ von $(a_n)$ mit $|b_k-L|>\varepsilon$.
Aus dem Satz von Bolzano-Weiestrass schliessen wir die Existenz
einer konvergenten Teilfoge von $(b_n)$ mit Limes $\ell\neq L$.
$\ell$ ist ein H\"aufungswert von $(a_n)$. Das ist ein Widerspruch
weil, nach der Definition von Liminf und Limsup, $L\leq \ell \leq L$.
\end{proof}



\subsection{Konvergenzkriterium von Cauchy}
\begin{Sat}
  Eine Folge komplexer Zahlen konvergiert genau dann, wenn:
\begin{equation}\label{e:Cauchy}
\forall \varepsilon>0\s\exists N:\abs{a_n-a_m}<\varepsilon\s\forall n,m\geq N\, .
\end{equation}
\end{Sat}

\begin{proof}[Beweis]
  {\bf Konvergenz $\implies$ Cauchy.} Sei $(a_n)$ s.d. 
$a_n\to a$. Sei $\varepsilon>0$. Dann 
  \[\exists N:\abs{a_n-a}<\frac{\varepsilon}{2}\s\forall n\geq N\]
Deswegen:
  \[\abs{a_n-a_m}=\abs{a_n-a+a-a_m}
\leq \abs{a_n-a}+\abs{a-a_n}<\frac{\varepsilon}{2}
+\frac{\varepsilon}{2}\s\forall n,m\geq N\]

\medskip

{\bf Cauchy $\implies$ Konvergenz}. Sei $(a_n)$ eine ``Cauchy-Folge''
(d.h. \eqref{e:Cauchy} gilt).   
\begin{Bem}\label{b:h_gen}
Falls $a$ ein Häufungswert ist, dann konvergiert die Ganze Folge nach $a$! 
\end{Bem}
In der Tat, sei $a_{n_k}$ eine Teilfoge die nach $a$ konvergiert.
\begin{equation}\label{e:est1}
\forall \varepsilon>0\s\exists K:\qquad 
k>K \quad \implies\quad \abs{a_{n_k}-a}<\frac{\varepsilon}{2}
\end{equation}
\begin{equation}\label{e:est2}
\exists N: \qquad \abs{a_{n_k}-a}<\frac{\varepsilon}{2}\quad \forall m,n>N\, .
\end{equation}
  Sei nun $n\geq N$. 
Sicher $\exists n_k>N$ mit $k\geq K$. Deswegen, f\"ur $n>N$,
  \[\abs{a-a_n}=\abs{a-a_{n_k}+a_{n_k}-a_n}
\leq\abs{a-a_{n_k}}+\abs{a_{n_k}-a_n}\stackrel{\eqref{e:est1} \&
\eqref{e:est2}}{<} \varepsilon\, .\]
Das beweist die Bemerkung \ref{b:h_gen}.

Deswegen, um den Satz zu beweisen, es gen\"ugt die Existenz
eines H\"aufungspunkts zu zeigen. Nach Bolzano-Weiestrass,
die beschr\"anktheit der Folge impliziert die Existenz eines
H\"aufungspunts. Wähle nun $\varepsilon=1$. 
  \[\exists \bar N:\abs{a_n-a_m}<1\s\forall n,m\geq \bar N\, .\]
Deswegen
\[\abs{a_n}\leq \abs{a_n-a_{\bar N}}+\abs{a_{\bar N}}<\abs{a_{\bar N}}+1\s
\qquad\forall n
\geq \bar N\]
Sei nun
\[M:=\max\left( \left\{ \abs{a_k}:k<\bar N \right\}\cup\left\{ \abs{a_{\bar N}+1} 
\right\} \right)\]
Dann $\abs{a_n}\leq M$ und die Folge ist beschr\"ankt.
\end{proof}

\begin{Def}
  Sei $a_n$ eine Folge von reellen Zahlen. Dann sagen wir:
  \begin{itemize}
    \item $a_n\to +\infty$ (oder $\lim_{n\to +\infty} a_n=+\infty$) falls $\forall M\in\mb{R}$ $\exists N\in\mb{R}:a_n\geq M$ $\forall n\geq N$ (oder  $a_n\geq$ für fast alle $n\in\mb{R}$)
    \item $a_n\to-\infty$ ($\lim_{n\to -\infty} a_n=-\infty$) falls $\forall M\in\mb{R}$, $a_n\leq M$ für fast alle $n$.
  \end{itemize}
  Wenn die Folge $a_n$ keine obere Schranke besitzt: $\ol{\lim_{n\to+\infty}} a_n=+\infty$. Dasselbe gilt equivalent auch für untere Schranken.
\end{Def}
\begin{Ueb}
  $\ol{\lim_{n\to+\infty}}a_n=+\infty$ $\iff$ $\exists$ Teilfolge $\left\{ a_{n_k} \right\}_{k\in\mb{N}}$ mit $a_{n_k}\stackrel{k\to+\infty}{\to}+\infty$
\end{Ueb}
\begin{Bem}
  Sei $a_n$ eine wachsende (bzw. fallende) Folge. Dann:
  \begin{itemize}
    \item entweder konvergiert $a_n$
    \item oder $\lim_{n\to+\infty}a_n=+\infty$ (bzw. $\lim_{n\to+\infty}a_n=-\infty$)
  \end{itemize}
\end{Bem}
\section{Reihen}
\subsection{Konvergenz der Reihen}
\begin{Def}
  Sei $(a_n)_{n\in\mb{N}}$ eine Folge komplexer Zahlen. Wir setzen:
  \begin{align*}
    s_0=a_0\\
    s_1=a_0+a_1\\
    s_2=a_0+a_1+a_2\\
    \cdots\\
  \end{align*}
  \begin{equation*}
    s_k:=\sum^k_{i=0}a_i
  \end{equation*}
\end{Def}
\begin{Def}
  Die $(s_k)_{k\in\mb{N}}$ ist die Folge der Partialsummen. Die Reihe ist die Folge $(s_k)_{k\in\mb{N}}$ falls der Limes von $s_k$ existiert, dann ist $\lim_{n\to+\infty}s_n$ ist der \ul{Wert der Reihe}. Und wir sagen dass $(s_k)$ eine \ul{konvergente Reihe} ist.
\end{Def}
\begin{Not}
  Die Notation der Reihe ist $\sum^\infty_{i=0}a_i$ bezeichnet \ul{die Reihe} und \ul{den Wert der Reihe}.
\end{Not}
\begin{Bsp}
  Sei $z$ eine komplexe Zahl. Dann ist die Reihe $\sum^\infty_{n=0}z^n$ die \ul{geometrische Reihe}.
  \begin{itemize}
    \item $\abs{z}<1$ dann konvergiert $\implies$ die geometrische Reihe.
  \end{itemize}
  Falls $z=0$ ist der Wert der Reihe $1$.
  \begin{align*}
    0\neq z, \abs{z}<1\\
    (1-z)(1+z+\cdots z^n)=1-z^{n+1}\\
    s_n=\frac{1-z^{n+1}}{1-z}\\
    \lim_{n\to+\infty}\frac{1-z^n}{1-z}=\lim_{n\to+\infty}\left( \frac{1}{1-z} \right)-\frac{1}{1-z}\underbrace{\left( \lim_{n\to+\infty}z^n \right)}_{=0 \text{ weil } \abs{z}<1}=\frac{1}{1-z}
  \end{align*}
  Für $\abs{z}>1$ ist $s_n=\frac{(1-z)^{n+1}}{1-z}$ falls $\lim_{n\to+\infty}s_n$ existiert, dann konvergiert die Folge $z^{n+1}$ $\implies$ die Folge $\underbrace{\abs{z}^n+1}_{\text{falsch weil $\abs{z^n}$ divergiert}}$ konvergiert. Sei $a\in\mb{R}$, $a>1$
  \[a^n=(1+(a-1))^n=1+n(a-1)\]
  \begin{itemize}
    \item $z=1$ $s_n=1+1+\cdots+1=n+1$ $\implies$ $(s_n)$ konvergiert nicht!
    \item $s\neq 1$ $s_n$ konvergiert nicht weil $z^{n+1}$ nicht konvergiert!
    \item $\abs{z}=1$ $\implies$ 
      \[z=\cos\theta+i\sin\theta\implies z^{n+1}=\cos((n+1)\theta)+i\sin((n+1)\theta)\]
      (Übung 4, Blatt 3)
  \end{itemize}
  \begin{Bem}
    Falls $z\in\mb{R}$, $z\geq 1$. Dann ist $s_n$ eine Folge reeller Zahlen, $s_n\geq 0$, $s_n$ ist monoton wachsend ($s_{n+1}=s_n+z^{n+1}\geq s_n$). $\implies$ in diesem Fall $\sum^\infty_{n=0}z^n=+\infty$
  \end{Bem}
  $z\in\mb{R}$, $z=-1$
  \[s_n=\begin{cases}
    1&\text{für gerade $n$}\\
    0&\text{für ungerade $n$}
  \end{cases}\]
  $\implies$ $s_n$ ist beschränkt und $s_n$ konvergiert nicht (Häufungspunkte ${0,1}$.\\
  $z\in\mb{R}$, $z<-1$.
  \[s_n=\frac{1-z^{n+1}}{1+z}\]
  $\implies$ $(s_n)$ ist nicht beschränkt
\end{Bsp}
\begin{Bem}
  Wenn die Partialsumme eine Folge reeller Zahlen ist und $s_n\to+\infty$ (bzw. $-\infty$), dann $\sum a_n=+\infty$ (bzw. $-\infty$).
\end{Bem}
\begin{Bsp}
  Harmonische Reihe: $\sum^\infty_{n=1}\frac{1}{n}$ $s_{n+1}\geq s_n$ $\implies$ entweder $\lim_{n\to+\infty}s_n$ existiert oder $\lim{n\to+\infty}=+\infty$
  \begin{align*}
    s_{2^n-1}=1+\underbrace{\frac{1}{2}+\frac{1}{3}}+\underbrace{\cdots}_{2^{k-1}\leq j\leq 2^k-1}+\cdots+\underbrace{\cdots}_{2^{n-1}\leq j\leq 2^n-1}\\
    \geq 1+\frac{1}{4}+\cdots+\underbrace{\frac{1}{2^k}+\cdots+\frac{1}{2^k}}_{2^{k-1}}+\cdots\\
    \geq 1+\fac{1}{2}+\frac{1}{2}+ \cdots\\
    =1+\frac{n-1}{2}
    \sigma_n=s_{2^n-1}\geq +1\frac{n-1}{2}\implies\lim_{n\to+\infty}\sigma_n=+\infty
  \end{align*}
  $\implies$ die ursprüngliche Folge $(s_n)$ konvergiert nicht! 
  \[\implies \lim_{n\to+\infty}s_n=+\infty\implies+\infty \implies \sun\frac{1}{n}=+\infty\]
\end{Bsp}
\subsection{Konvergenzkriterien für reelle Reihen}
\begin{Bem}
  (gilt auch für komplexe Reihen!)
  \[\sum^\infty_{n=0}a_n \text{konvergiert}\implies a_n\to 0\]
\end{Bem}
\begin{Ueb}
  ganz schnell: die geometrische Reihe konvergiert nicht falls $\abs{z}\geq 1$
\end{Ueb}
\begin{Bem}
  $a_\to 0$ $\not\implies$ $\sum^\infty_{n=0}a_n$ konvergiert! Bsp: $a_n=\frac{1}{n}$
\end{Bem}
\begin{Sat}
  Sei $\sum a_n$ eine Reihe mit reellen Zahlen $a_n\geq 0$. Dann:
  \begin{itemize}
    \item entweder ist die Folge $(s_n)$ beschränkt (und die Reihe konvergiert deswegen)
    \item oder $\sum^\infty_{n=0}s_n=+\infty$
  \end{itemize}
\end{Sat}
\begin{Sat}
  (Konvergenzkriterium Leibnitz). Sei $(a_n)$ eine fallende Nullfolge. Dann konvergiert $\sum^\infty_{n=0}(-1)^na_n$ (eine alternierende Reihe).
\end{Sat}
\begin{Bew}
  Betrachten wir 
  \[s_k-s_{k-2}=(-1)^{k-1}a_{k-1}+(-1)^ka_k)(-1)^k\overbrace{(a_k-a_{k-1})}^{\leq 0}\]
  \begin{itemize}
    \item $s_k-s_{k-2}\geq 0$ falls $k$ ungerade ist
    \item $s_k-s_{k-2}\leq 0$ falls $k$ gerade ist
  \end{itemize}
  Für $k$ ungerade:
  \[s_1\leq s_3\leq s_5\leq \cdots\]
  \[\underbrace{s_k}_{\text{gerade}}=\underbrace{s_{k+1}}_{\text{ungerade}}+\underbrace{(-1)^{k+1}}_{\geq 0}\underbrace{a_{n+1}}_{\geq 0}\leq s_{k+1}\leq s_n\]
  Für $k$ gerade:
  \[s_1\leq s_3\leq s_5\leq \cdots\]
  (Beweis gleich wie für ungerade)\\
  $\implies$ die Folge $s_0,s_2,s_4,\cdots$ ist monoton fallend und von unten beschränkt $\implies$ $\lim_{k\to+\infty} 2k=S_g\in\mb{R}$
  \[S_u-S_g=\lim_{n\to+\infty}s_{2n+1}-\lim_{n\to+\infty}s_{2n}=\lim_{n\to+\infty}(s_{2n+1}-s_{2n})\lim_{n\to+\infty}a_{2n+1}=0\]
  \[\implies S_u=S_g\implies \lim_{n\to+\infty}s_n=S_u(=S_g)\]
\end{Bew}
\begin{Kor}
  \[1-\frac{1}{2}+\frac{1}{3}-\frac{1}{4}+\cdots\]
  konvergiert
\end{Kor}
\subsection{Konvergenzkriterien für allgemeine (komplexe) Reihen}
\begin{Bem}
  $\sum a_n$ konvergiert $\iff$ $(s_n)$ konvergiert $\iff$ $(s_n)$ ist eine Cauchyfolge. $\iff$ $\forall\varepsilon>0$ $\exists N: \abs{s_n-s_m}<\varepsilon$ $\forall n\geq m\geq N$. $\varepsilon>\abs{s_n-s_m}=\abs{a_{m+1}+\cdots+a_n}$.
\end{Bem}
\begin{Kor}
  Majorantenkriterium: Sei $\sum a_n$ eine Reihe komplexer Zahlen und $\sum b_n$ eine konvergente Reihe nichtnegativer reeller Zahlen. Falls $\abs{a_n}\leq b_n$ (d.h. $\sum b_n$ majorisiert $\sum a_n$, dann ist $\sum a_n$ konvergent.
\end{Kor}
\begin{Bew}
  $\sum b_n$ konvergiert $\iff$ $\sigma_n=\sum^n_{k=0} b_n$ ist eine Cauchyfolge.
  \[\iff \forall \varepsilon>0 \exists N:\underbrace{\abs{\sigma_n-\sigma_m}}_<\varepsilon\forall n\geq m\geq N\]
  \[b_n+\cdots+b_{m+1}\geq\abs{a_n}+\cdots+\abs{a_{m+1}}\geq \abs{a_n+\cdots+a_{m+1}}=\abs{s_n-s_m}\]
  Wobei $\sum s_n=\sum^n_{k=0}a_n$.
  \[\iff \forall \varepsilon>0 \abs{s_n-s_m}\leq \abs{\sigma_n-\sigma_m}<\varepsilon\forall n\geq m\geq N\]
  $\iff$ $(s_n)$ ist eine Cauchyfolge $\iff$ $\sum a_n$ konvergiert
\end{Bew}

\subsection{Wurzel- und Quotientenkriterium}
\begin{Def}
  Eine Reihe $\sum^\infty_{n=0} a_n$ heisst \ul{absolut konvergent}, falls $\sum_{a=0}^\infty \abs{a_n}$ eine konvergente Reihe ist.
\end{Def}
\begin{Bem}
  Majorantenkriterium $\iff$ die absolute Konvergent impliziert die Konvergent.
\end{Bem}
\begin{Sat} (Quotientenkriterium)
  Sei $\sum a_n$ eine Reihe mit $a_n\neq 0$ für fast alle $n$ und s.d. $\lim_{n\to+\infty}\abs{\frac{a_{n+1}}{a_n}}=q$ existiert. Falls
  \begin{itemize}
    \item $q<1$ konvergiert die Reihe absolut.
    \item $q>1$ divergiert die Reihe.
    \item $q=1$ unentschieden.
  \end{itemize}
\end{Sat}
\begin{Bew}
  \begin{itemize}
    \item $q>1$ $\implies \exists N$ so dass $\abs{a_{n+1}}\geq \tilde q\abs{a_n}$ falls $n\geq N$. $1<\tilde q=\frac{1}{2}+\frac{q}{2}<q$.
      \[\abs{a_n}\geq \tilde q\abs{a_{n-1}}\geq \tilde{q}^2\abs{a_{n-2}}\cdots>\tilde{q}^{n-N}\abs{a_N}\]
      oBdA $\abs{a_N}\neq 0$
      \[\implies \lim_{n\to+\infty}\abs{a_n}=+\infty\implies\sum a_n\text{ divergiert}\]
    \item $q<1$ $1<\tilde q=\frac{1}{2}+\frac{q}{2}<q$ $\exists N$ so dass $\abs{a_n}\leq \tilde{q}^{n-N}\abs{a_N}$ (das gleiche Argument wie vorher).
      \[b_n=\tilde{q}^{n-N}\abs{a_N}=C\tilde{q}^n\]
      \[b_n=\abs{a_n}\]
      $\sum b_n$ majorisiert $\sum a_n$
      \[\sum b_n \text{ konv }\stackrel{\text{Maj.}}{\implies}\sum\abs{a_n} \text{ konvergiert}\]
  \end{itemize}
\end{Bew}
\begin{Sat} (Wurzelkriterium)
  Sei $\sum a_n$ eine Reihe und $L:=\limsup_{n\to+\infty}\sqrt[n]{\abs{a_n}}$ (``$L=+\infty$'' falls $\abs{a_n}$ unbeschränkt ist!) Dann:
  \begin{itemize}
    \item $L<1$ konvergiert die Reihe absolut
    \item $L>1$ divergiert die Reihe
    \item $L=1$ unentschieden
  \end{itemize}
\end{Sat}
\begin{Bew}
  \begin{itemize}
    \item $L<1$ 
      \[L<\tilde L=\frac{L}{2}+\frac{1}{2}<1 \implies \exists N: \sqrt[n]{\abs{a_n}}\leq\tilde L\implies\abs{a_n}\leq\tilde{L}^n\]
      für $n\geq N$ haben wir wie oben die absolute Konvergenz.
    \item $L>1$
      \[\exists k_n: \sqrt[k_n]{\abs{a_{k_n}}}\to L\]
      \[1<\tilde{L}=\frac{L}{2}+\frac{1}{2}<L\]
      \[\exists N: k_n\geq N: \sqrt[k_n]{\abs{a_{k_n}}}\geq \tilde L\]
      \[\implies \abs{a_{k_n}}\geq \tilde{L}^{k_n}\to+\infty\text{ für } n\to+\infty\]
      \[\implies a_n\not\to 0\implies \sum a_n\text{ divergiert}\]
  \end{itemize}
\end{Bew}
\begin{Bsp}
  Sei $s\geq 1$ $\sum\frac{1}{n^s}$
  \begin{itemize}
    \item $s=1$ harmonische Reihe divergiert
    \item $s>1$ konvergiert! $\sum\frac{1}{n^2}$ Bernoulli?? $=\frac{\pi^2}{6}$
      \[\sum\frac{1}{n^{2k}}\sim \underbrace{c_k}_{\in\mb{Q}}\pi^{2k}\]
    \item $a_n=\frac{1}{n^s}$
      \[\lim_{n\to+\infty}\frac{a_{n+1}}{a_n}=1\forall s\geq 1\]
      \[\lim_{n\to+\infty}\sqrt[n]{a_n}=1\forall s\geq 1\]
      $s=1$ $\implies$ Divergenz, $s>1$ $\implies$ Konvergenz.
      \[\implies \frac{1}{1^s}+\frac{1}{2^s}+\frac{1}{3^s}+\frac{1}{4^s}+\cdots+\frac{1}{7^s}+\cdots\]
      \[\sum^\infty_{n=0} b_n= \frac{1}{1^s}+\frac{1}{2^s}+\frac{1}{4^s}+\frac{1}{4^s}+\cdots+\frac{1}{4^s}+\cdots\]
      $b_n\geq 0$
      \[s_n=b_0+\cdots+b_n\]
      $\left\{ s_n \right\}$ ist beschränkt. Wir setzen 
      \[s_{2k-1}=\frac{1}{1^s}+\frac{2}{2^s}+\cdots+\frac{2^{k-1}}{2^{(k-1)s}}=\frac{1}{1^{s-1}}+\frac{1}{2^{s-1}}+\cdots+\frac{1}{2^{(s-1)(k-1)}}\]
      \[=\frac{1}{\alpha}+\frac{1}{\alpha^1}+\cdots+\frac{1}{a^{k-1}}\leq\sum^\infty_{k=0}\frac{1}{\alpha^s}<+\infty\]
      \[\alpha:=2^{s-1}>2^0=1\]
      \[\stackrel{\text{Majo.}}{\implies}\sum \frac{1}{n^s}\text{ konvergiert}\]
  \end{itemize}
\end{Bsp}
\subsection{Das Cauchyprodukt}
\begin{Def}
  $\sum a_n$ und $\sum b_n$. Das CP ist die Reihe $\sum c_n$
  \[c_n=a_0b_n+a_1b_{n-1}+\cdots+a_nb_0=\sum^n_{j=0}a_jb_{n-j}=\sum_{j+k=n}a_jb_k\]
\end{Def}
\begin{Sat}
  Falls $\sum a_n$ und $\sum b_n$ absolut konvergieren, dann konvergiert das CP absolut.
  \[\sum c_n=\left( \sum a_n \right)\left( \sum b_n \right)\]
\end{Sat}
\begin{Bew}
  \[s-k=\sum^k_{j=0}a_j, \sigma_k=\sum_{i=0}^k b_i\]
  \[s_k\sigma_k=\sum^n_{j=0}\sum^n_{i=0}b_ia_j\]
  \[c_n=\sum_{j+i=n}a_ib_j, \beta_k=\sum^k_{n=0}c_k\]
  \[\sum^k_{n=0}\sum_{i+j=n}a_ib_j=\sum_{i+j\leq n}a_ib_j\]
  \begin{align*}
    c_0=a_0b_0\\
    c_1=a_0b_1+a_1b_0\\
    c_1=a_0b_2+a_1b_1+a_2b_0\\
  \end{align*}
  \[\beta_k-\sigma_ks_k\]
  Absolute Konvergenz:
  \begin{itemize}
    \item $\sum\abs{c_k}<+\infty$
    \item $B_n=\sum^\infty_{k=0}\abs{c_k}$
  \end{itemize}
  $\left( B_n \right)$ ist eine beschränkte Folge
  \[B_n=\sum^N_{k=0}\abs{sum_{i+j\geq k}a_ib_j}\leq\sum^N_{k=0}\sum_{i+j=k}\abs{a_i}\abs{b_j}\]
  \[=\sum_{i+j\leq N}\abs{a_i}\abs{b_j}\leq\sum^N_{i=0}\sum^N_{j=0}\abs{a_i}\abs{b_j}\]
  \[=\left( \sum^N_{i00}\abs{a_i} \right)\left( \sum_{j=0}^N\abs{b_j} \right)\leq\left( \sum\abs{a_i} \right)\left( \sum\abs{b_j} \right)\]
  \[=LM\]
  Wobei $L=\sum\abs{a_i}$ und $M=\sum\abs{b_j}$. $\implies$ $\left( B_n \right)$ konvergiert $\implies$ $\sum c_n$ konvergiert absolut.
  \[\abs{ \sum^N_{i=0}a_i \sum^N_{j=0}b_j-\sum^N_{k=0}}\]
  \[=\abs{\sum^N_{i=0, j=0}a_ib_j-\sum_{i+j\leq N}a_ib_j}\]
  \[=\abs{\sum_{i+j>N, i\leq , j\leq N}a_ib_j}\leq\sum_{i+j>N, i\leq , j\leq N}\abs{a_i}\abs{b_j}\]
  \[\leq \sum_{i\leq N, j\leq N, i\geq \frac{N}{2}, j\geq\frac{N}{2}}\abs{a_i}\abs{b_j}=\sum_{i,j\leq N}\abs{a_i}\abs{b_j}=\sum_{i,j<\frac{N}{2}}\abs{a_i}\abs{b_j}\]
  \[=\underbrace{\left( \sum^N_{i=0}\abs{a_i} \right)\left( \sum^N_{j=0}\abs{b_j} \right)-\left( \sum_{i=0}^{\floor\frac{N}{2}}\abs{a_i} \right)\left( \sum^{\floor\frac{N}{2}}_{j=0}\abs{b_j} \right)}_{\Gamma_N}\]
  \[0\leq\abs{\sum_{i=0}^Na_i\sum^N_{j=0}b_j-\sum^N_{k=0}c_k}\leq \Gamma_N\]
  \[\lim_{N\to+\infty}\Gamma_N=\sum^\infty_{i=0}\abs{a_i}\sum^\infty_{j=0}\abs{b_j}-\sum^\infty_{i=0}\abs{a_i}\sum^\infty_{j=0}\abs{b_j}\]
  \[\implies \sum c_k=\sum a_i\sum b_j\]
\end{Bew}
\subsection{Potenzreihen}
\begin{Def}
  Die Potenzreihen: $\sum a_nz^n$, $z\in\mb{C}$
\end{Def}
\begin{Lem}
  Falls $a_nz_0^n$ eine konvergente Reihe ist, dann $\forall z$ mit $\abs{z}<\abs{z_0}$ konvergiert $\sum a_nz^n$ absolut.
\end{Lem}
\begin{Bew}
  $a_nz_0^n$ ist eine Nullfolge.
  \[\implies \exists C:\abs{a_nz_0^n}\leq C\forall n\]
  \[\abs{a_nz^n}\leq\abs{a_nz^n_0}\underbrace{\frac{\abs{z}^n}{\abs{z_0}^n}}_\alpha\leq C\alpha^n\]
  \[\abs{z}<\abs{z_0}\implies \alpha<1\]
  $\implies$ $\sum C\alpha^n$ eine konvergente Majorante.
\end{Bew}
\begin{Sat}
  Sei $(a_n)$ eine Folge von Koeffizienten $a_n\in\mb{C}$. Sei $K:=\left\{ z\in\mb{C}:\sum a_nz^n\text{ konvergiert} \right\}$ $K\ni z\to f(z)=\sum^\infty_{n=0}a_nz^n$. Wenn
  \[f(z)=\sum a_nz^n, g(z)=\sum b_nz^n\]
  \[\implies f(z)+g(z)=\sum(a_n+b_n)z^n\]
  \[\implies f(z)g(z)=\underbrace{\sum c_nz^n}_{\text{falls $z$ absolute Konvergenz garantiert}}\]
\end{Sat}
\begin{Bew}
  Sei $\sum \gamma_n$ das CP von $\sum a_nz^n$ und $b_nz^n$.
  \[\sum\gamma_n=\sum a_nz^n\sum b_nz^n\]
  \[=\sum\sum_{i+j=n}\left( a_iz^i \right)\left( b_jz^j \right)=\sum^\infty_{n=0}\sum_{i+j=n}a_ib_jz^{i+j}\]
  \[=\sum {n=0}z^n\underbrace{\sum_{i+j=n}a_ib_j}_{=c_n}\]
\end{Bew}
\begin{Sat}
  (Cauchy-Hadamard) $\sum a_nz^n$. Sei $L:=\limsup \sqrt[n]{\abs{a_n}}$. Dann (Wurzerlkriterium)
  \begin{itemize}
    \item $\abs{z}<\frac{1}{L}\implies$ $\sum a_nz^n$ konvergiert absolut
    \item $\abs{z}>\frac{1}{L}\implies$ $\sum a_nz^n$ divergiert
    \item $\abs{z}=1$ unentschieden
  \end{itemize}
\end{Sat}

\section{Stetige Funktionen und Grenzwerte}
\subsection{Stetigkeit}
In $D\subset\mb{R}$, $D\subset\mb{C}$.
\begin{Def}
  Eine Funktion $f:D\mapsto \mb{R}(\mb{C})$. Sei $x_0\in D$. $f$ heisst stetig in $x_0$ falls $\forall \varepsilon >0$, $\exists\delta>0$ mit
  \[\abs{x-x_0}<\delta, x\in D\implies \abs{f(x)-f(x_0)}<\varepsilon\]
  (Bedingung S). Gegenüber:
  \[\forall\delta>0 \exists x\in\left] x_0-\delta,x_0+\delta  \right[\text{ mit } \abs{f(x)-f(x_0)}\geq \varepsilon\]
\end{Def}
\begin{Bsp}
  Die Polynome sind stetige Funktionen.
\end{Bsp}
\begin{Bsp}
  (Später), Summe und Produkte stetiger Funktionen sind auch stetig.
\end{Bsp}
\begin{Bem}
  \begin{itemize}
    \item Die Bedingung (S) ist trivial für die Funktion $f=const$
    \item Die Bedingung (S) ist trivial für die Funktion $f(x)=x$
      \[\abs{x-x_0}<\delta=\varepsilon\implies\abs{f(x)-f(x_0)}=\abs{x-x_0}<\varepsilon\]
  \end{itemize}
\end{Bem}
\begin{Def}
  Eine Funktion $f:D\to \mb{R}(\mb{C})$ heisst Lipschitz(-stetig) falls $\exists L\geq 0$ mit
  \[\abs{f(x)-f(y)}\leq L\abs{x-y}, \forall x,y\in D\]
  (L) $\implies$ (S): wähle $\delta=\frac{\varepsilon}{L}$
\end{Def}
\begin{Kor}
  $g(x):=\abs{x}$ ist stetig.
  \[\abs{g(x)-g(y)}=\abs{\abs{x}-\abs{y}}\leq\abs{x-y}\]
  d.h. (L) mit $L=1$
\end{Kor}
\begin{Bsp}
  (Später): $\frac{f}{g}$ ist stetig falls $f$,$g$ stetig und $g(x)\neq 0$ $\forall x\in D$. $\implies$ Rationale Funktionen $\frac{P(x)}{Q(x)}$ sind stetig auf $d=\mb{C}\setminus\left\{ x:Q(x)=0 \right\}$
\end{Bsp}
\begin{Bsp}
  $f(x)=x^k$, $k\in\mb{N}$ ist ein Polynom $\implies$ $f$ ist stetig. Sei $g(x):=x^\frac{1}{k}=\sqrt[k]{x}$, $k\in\mb{N}\setminus\left\{ 0 \right\}$ ( $g(x)$ ist die einzige relle Zahl $y\in\mb{R}$ mit $y\geq 0$ und $y^k=x$ ). $x_0\in\mb{R}$, $\varepsilon>0$
  \[\abs{\underbrace{\sqrt[k]{x}}_y-\underbrace{\sqrt[k]{x_0}}_{y_0}} \leq \sqrt[k]{abs{x-x_0}}\]
  \[\iff \abs{y-y_0}^k\leq\abs{y^k-y_0^k}\]
  oBdA $y\geq y_0$
  \[\underbrace{\left( y-y_0 \right)^k}_a\leq \underbrace{y^k}_c-\underbrace{y_0^k}_b\]
  \[\iff a^k+b^k\leq c^k=(a+b)^k\]
  \[a^k+b^k\leq (a+b)^k=a^k+\overbrace{\binom{k}{1}a^{k-1}b+\cdots}^{\geq 0}+b^k\]
  Deswegen: $\delta=\varepsilon^{k}$. $\abs{x-x_0}<\delta$ $x>x_0$, $x<x_0+\delta$
  \[\abs{\sqrt[k]{x}-\sqrt[k]{x_0}}=\left( \sqrt[k]{x}-\sqrt[k]{x_0} \right)<\left( \sqrt[k]{x_0+\delta}+\sqrt[k]{x_0} \right)\]
  \[=\abs{\sqrt[k]{x_0+\delta}-\sqrt[k]{x_0}}\leq\sqrt[k]{\delta}=\sqrt[k]{\varepsilon^k}=\varepsilon\]
  Oder wähle $\delta=\left( \frac{\varepsilon}{2} \right)^k$
  \[\abs{x-x_0}< \delta \implies \abs{\sqrt[k]{x}-\sqrt[k]{x_0}}\leq\sqrt[k]{\abs{x-x_0}}\leq\sqrt[k]{\left( \frac{\varepsilon}{2} \right)^k}=\frac{\varepsilon}{2}<\varepsilon\]
\end{Bsp}
\begin{Bsp}
  Sei $a>0$ und $f(x)=a^x$ $\forall x\in\mb{Q}$ $f:\mb{Q}\to\mb{R}$ ist stetig!
\end{Bsp}
\begin{Sat}
  Sei $f:D\to\mb{R}(\mb{C})$. Sei $x_0\in D$. Diese zwei Aussagen sind equivalent:
  \begin{itemize}
    \item $f$ ist stetig an der Stelle $x_0$.
    \item $\forall(x_n)\subset D$ mit $x_n\to x_0$ haben wir $f(x_n)\to f(x_0)$
  \end{itemize}
\end{Sat}
\begin{Bew}
  Sei $\varepsilon >0$. $f$ stetig in $x_0$ $\implies$ $\exists\delta>0$ mit $\abs{f(x)-f(x_0)}<\varepsilon$ falls $\abs{x-x_0}<\delta$. $x_n\to x_0$ $\implies$ $\exists N$:
  \[\abs{x_n-x_0}<\delta\forall n\geq N\implies \abs{f(x_n)-f(x_0)}<\varepsilon\]
  Andere Richtung: Nehmen wir an dass $f$ stetig falsch ist.
  \[\implies \exists \varepsilon>0: \forall \delta>0 \exists x: \abs{x-x_0}<\delta\wedge\abs{f(x)-f(x_0)}\geq \varepsilon\]
  $forall n\in\mb{N}\setminus\left\{ 0 \right\}$. Ich setze $\delta=\frac{1}{n}\implies \exists x_n$ mit $\abs{x_n-x_0}<\frac{1}{n}$ und $\abs{f(x_n)-f(x_0)}\geq \varepsilon$ $\implies$ $x_n\to x_0$ und $f(x_n)\not\to f(x_0)$.
\end{Bew}
\begin{Sat}
  Seien $f,g:D\to\mb{R}(\mb{C})$ zwei stetige Funktionen. Dann:
  \begin{itemize}
    \item $f+g$, $fg$ sind stetig
    \item $\frac{f}{g}$ ist stetig auf $D\setminus\left\{ x:g(x)=0 \right\}$
  \end{itemize}
\end{Sat}
\begin{Bew}
  Sei $x_0\in D$, $(x_n)\subset D$ $x_n\to x_0$ (für $\frac{f}{g}$ $g(x_n)\neq 0, q(x_0)\neq 0$ weil $(x_n), {x_0}\subset D\setminus\left\{ x:g/x(=0 \right\})$
  \begin{align*}
    f(x_n)+g(x_n)&\to& f(x_0)+g(x_0)\\
    f(x_n)g(x_n)&\to& f(x_0)g(x_0)\\
    \frac{f(x_n)}{g(x_n)}&\to& \frac{f(x_0)}{g(x_0)}\\
  \end{align*}
\end{Bew}
\begin{Sat}
  $f:D\to A$, $g:A\to B$ stetig $\implies g\circ f:D\to B$ stetig.
\end{Sat}
\begin{Bew}
  $x_0, (x_n)\subset D$ mit $x_n\to x_0$ $\implies \underbrace{f(x_n)}_{y_n}\to f(x_0)_{y_0}$ $(y_n), y_0\in A$ $\implies$ 
  \begin{itemize}
    \item $g(y_n)\to g(y_0)$
    \item $g(f(x_n))\to g(f(x_0))$
  \end{itemize}
  $\implies$ $g\circ f(x_n)\to g\circ f(x_0)$ $\implies$ Stetigkeit von $g\circ f$
\end{Bew}
\begin{Sat}
  Sei $f:\left[ a,b \right]\to\mb{R}(\mb{C})$ injektiv. Sei
  \[B:=f\left( \left[ a,b \right] \right)\left( =\left\{ z:\exists x\in\left[ a,b \right]\text{ mit } f(x)=z \right\} \right)\]
  \begin{Bem}
    $f:\left[ a,b \right]\to B$ ist bijektiv und deswegen umkehrbar.
  \end{Bem}
  Sei $f^{-1}:B\to\left[ a,b \right]$ die Umkehrfunktion. Dann ist $f^{-1}B\to \left[ a,b \right]$ stetig, falls $f$ stetig ist.
\end{Sat}
\begin{Bew}
  Sei $x_0\in B$, $(x_n)$ mit $(x_n)\subset B$ und $x_n\to x_0$. Die Folge
  \[\underbrace{f^{-1}(x_n)}_{=y_n}\stackrel{?}{\to}\underbrace{f^{-1}(x_0)}_{=y_0}\]
  $(y_n)\subset\left[ a,b \right]$, $y_0\in\left[ a,b \right]$. Falls $y_n\not\to y_0$, dann:
  \[\exists\varepsilon>0: \forall N\in\mb{N}\exists \underbrace{n}_{n_k}\geq \underbrace{N}_{k}: \abs{y_n-y_0}\geq\varepsilon\]
  \[n_k\geq n_{k-1}\implies\text{Teilfolge } (y_{n_k}): \abs{y_{n_k}-y_0}\geq \varepsilon\forall k\in \mb{N}\]
  \[\text{Bolzano-Weiterstrass}\implies \exists y_{n_k}\to\bar{y}\implies \bar{y}\neq y_0\]
  \[f(y_{n_{k_j}})=x_{n_{k_j}}\]
  Stetigkeit von $f$:$f(y_{n_{k_j}})\to f(\bar y)$ Und da $x_{n_{k_j}}\to x_0$ sowie  $x_{n_{k_j}} = f(y_{n_{k_j}}$, heisst dass das $f(\bar y)=x_0$, aber $f(y_0)=x_0$ $\implies f(\bar y)=f(y_0)$, mit $\bar y\neq y_0$. Widerspruch mit der Injektivität von $f$. Deswegen $f^{-1}(x_n)=y_n\to y_0=f^{-1}(x_0)$ $\implies$ $f^{-1}$ ist stetig.
\end{Bew}
\begin{Bem}
  Aus diesem Satz schliessen Sie die Stetigkeit von $x\mapsto x^{\frac{1}{k}}$ von der Stetigkeit $x\mapsto x^k$.
\end{Bem}
\begin{Def}
  Wenn eine Funktion $f:D\to\mb{R}(D)$ stetig ist für $x\in D$, dann ist $f$ stetig auf $D$.
\end{Def}
\begin{Bem}
  Für Satz 1 genügt die Stetigkeit der beiden Funktionen ander der Stelle $x_0$. Für Satz 2 ähnlich. Für Satz 3 ist die Stetigkeit auf dem ganzen $D$ wichtig.
\end{Bem}
\subsection{Zwischenwertsatz}
\begin{Sat}
  Sei $f:\left[ a,b \right]\to\mb{R}$ stetig, mit $f(b)\geq f(a)$ (bzw. $f(b)\leq f(a)$). Dann $\forall y\in\left[ f(a),f(b) \right]$ (bzw. $\forall y\in\left[ f(b),f(a) \right]$) $exists x\in\left[ a,b \right]$ mit $f(x)=y$.
\end{Sat}

\subsection{Zwischenwertsatz}
\begin{Sat}
  Eine stetige Abbildung $f:[a,b]\to\mb{R}$ nimmt jeden Wert $\gamma$ zwischen $f(a)$ und $f(b)$ an.
\end{Sat}
\begin{Bew}
  oBdA $f(a)\leq f(b)$ und $f(a)\leq \gamma\leq f(b)$
  \[I_0=\left[ a,b \right]=\left[ a_0,b_0 \right]\]
  \[\left[ a,\frac{a+b}{2} \right],\left[ \frac{a+b}{2},b \right]\]
  \[f\left( \frac{a+b}{2} \right)\geq\gamma\implies I_1=\left[ a,\frac{a+b}{2} \right]=\left[ a_1,b_1 \right]\]
  \[f\left( \frac{a+b}{2} \right)<\gamma\implies I_1=\left[ \frac{a+b}{2},b \right]=\left[ a_1,b_1 \right]\]
  Rekursiv $I_k=\left[ a_k,b_k \right]$ mit $f(a_k)\leq\gamma\leq f(b_k)$, $I_{k+1}=\left[ a_{k+1},b_{k+1} \right]$
  \[I_{k+1}=\begin{cases}
    \left[ a_k,\frac{a_k+b_k}{2} \right]&f\left( \frac{a_k+b_k}{2} \right)\geq\gamma\\
    \left[ \frac{a_k+b_k}{2},b_k \right]&\text{sonst}\\
  \end{cases}\]
  \[\abs{I_k}=2^{-k}(b-a)\stackrel{k\to+\infty}{\to}0\]
  Intervallschachtelung $\implies$ $\exists ! x_0$ mit $x_0\in I_k$ $\forall k$.
  \[b_k\downarrow x_0\implies f(x_0)=\lim_{k\to+\infty}f(b_k)\geq \gamma\]
  \[b_k\downarrow x_0\implies f(x_0)=\lim_{k\to+\infty}f(a_k)\geq \gamma\]
  \[\implies f(x_0)=\gamma\]
\end{Bew}
\begin{Kor}
  \ul{Fixpunktsatz}: Sei $f:\left[ a,b \right]\to\left[ a,b \right]$ eine stetige Abbildung. Dann besitzt $f$ einen Fixpunkt, d.h.
  \[\exists x_0\in\left[ a,b \right]:f(x_0)=x_0\]
\end{Kor}
\begin{Bew}
  $g(x):=f(x)-x$
  \[g(a)=f(a)-a\geq 0\]
  \[g(b)=f(b)-b\geq 0\]
  Mithilfe des oberen Satzes $\implies$ $\exists x_0$ mit 
  \[g(x_0)=0\iff f(x_0)-x_0=0\iff f(x_0)=x_0\]
\end{Bew}
\subsection{Maxima und Minima}
\begin{Sat}
  Sei $f:[a,b]\to\mb{R}$ stetig. Dann $\exists x_M, x_m\in [a,b]$ mit
  \[f(x_m)\geq f(x)\geq f(x_M)\s\forall x\in [a,b]\]
\end{Sat}
\begin{Bew}
  oBdA suche ich die Maximumstelle
  \[S=\sup \left\{ f(x):x\in\left[ a,b \right] \right\}\]
  \[(=+\infty\s\text{falls}\s\left\{ f(x):x\in\left[ a,b \right] \right\}\s\text{keine obere Schranke}\]
  $S\in\mb{R}$, sei $S_n=S-\frac{1}{n}$ $\implies$ $\exists x_n$ mit $f(x_n)\geq S-\frac{1}{n}$
  \[(x_n)\subset\left[ a,b \right]\implies\exists\left( x_{n_k} \right)\s\text{mit}\s x_{n_k}\to \bar x\]
  \[\stackrel{S\in \mb{R}}{\implies}f(\bar x)=\lim_{k\to+\infty}f(x_{n_k})=S\stackrel{!}{=}\max_{x\in\left[ a,b \right]}f(x)=\max_{\left[ a,b \right]} f\]
  \[\stackrel{S=+\infty}{\implies}f(\bar x)=\lim_{k\to +\infty}f(x_{n_k})=+\infty\implies \text{Widerspruch}\]
\end{Bew}
\begin{Bem}
  Sei $E\subset\mb{R}$ eine Menge mit der Eigenschaft $\forall (x_n)\subset E$ $\exists$eine Teilfolge $(x_{n_k})$ $x\in E$ mit
  \[x_{n_k}\to x\]
  Ist $E$ immer ein abgeschlossenes Intervall? Nein
  \[E:=\left[ 0,1 \right]\cup \left[ 2,3 \right]\]
  Sei $(x_n)\subset\left[ 0,1 \right]\cup\left[ 2,3 \right]$. Dann $\exists\left( x_{n_k} \right)$ die entweder in $\left[ 0,1 \right]$ oder in $\left[ 2,3 \right]$ enthalten ist $\implies$ $\exists$ eine konvergente Teilfolge.
\end{Bem}
\begin{Def}
  Die Mengen $E(\subset\mb{R},\subset\mb{C})$ mit der Eigenschaft in der Bemerkung oben heissen \ul{kompakte Mengen}.
\end{Def}
\begin{Sat}
  Eine reellwertige stetige Funktion auf einem kompakten Definitionbereich besitzt mindestens eine Maximumstelle (und eine Minimumstelle).
\end{Sat}
\begin{Def}
  \ul{Stetigkeit} an einer Stelle $x$:
  \[\forall \varepsilon>0\s\exists\delta>0\s\text{mit}\s\underbrace{\abs{x-y}<\delta\s\text{und}\s y\in D}\implies \abs{f(x)-f(y)}<\varepsilon\]
  Stetigkeit auf $D$ bedeutet Stetigkeit an jeder Stelle $x\in D$.
\end{Def}
\begin{Def}
  Eine Funktion $f:D\to\mb{R}(\mb{C})$ heisst \ul{gleichmässig stetig} falls
  \[\forall \varepsilon>0\s\exists\delta>0\s\text{mit}\s\abs{x-y}<\delta\s\text{mit}\s x,y\in D\implies \abs{f(x)-f(y)}<\varepsilon\]
\end{Def}
\begin{Bsp}
  $f$ Lipschitz
  \[\abs{f(x)-f(y)}\leq L\abs{x-y}\s\forall x,y\in D\]
  Dann ist $f$ gleichmässig stetig $\delta=\frac{\varepsilon}{L}$
  \[\abs{x-y}<\frac{\varepsilon}{L}\implies \abs{f(x)-f(y)}\leq L\abs{x-y}<L \delta=L\frac{\varepsilon}{L}=\varepsilon\]
\end{Bsp}
\begin{Sat}
  Falls $D$ eine kompakte Menge ist, ist jede stetige Funktion $f:D\to \mb{R}(\mb{C})$ gleichmässig stetig!
\end{Sat}
\begin{Bew}
  (Widerspruchsbeweis) $f$ stetig aber nicht gleichmässig. Dann $\exists \varepsilon>0:\forall \delta$ die ich wählen kann
  \[\exists x,y\in D\s\text{mit}\s\abs{x-y}<\delta\s\text{und}\s\abs{f(x)-f(y)}\geq \varepsilon\]
  \[\delta=\frac{1}{n}>0\implies\exists x_n,y_n\s\text{mit}\s\abs{x_n-y_n}<\frac{1}{n}\s\text{und}\s\abs{f(x_n)-f(y_n)}\geq \varepsilon\]
  \[\text{Kompaktheit}\implies\exists x_{n_k}\s\text{Teilfolge mit}\s x_{n_k}\to x\in D\]
  \[\implies y_{n_k}\to x\in D\]
  \[\implies \stackrel{f(x_{n_k})\to f(x)}{f(y_{n_k})\to f(x)}\implies \abs{f(x_{n_k})-f(y_{n_k})}\to 0\]
\end{Bew}
\subsection{Stetige Fortsetzung, Grenzwerte}
\begin{Def}
  Sei $f:D\to\mb{R}(\mb{C})$ stetig. Sei $E>D$. Eine stetige Fortsetzung von $f$ ist eine $\tilde f:E\to\mb{R}(\mb{C})$ stetig mit $f(x)=\tilde f(x)\s\forall x\in D$
\end{Def}
\begin{Def}
  $g:E\to A$, $D\subset E$, 
  \[g\|_D\to A\s\text{mit}\s g\|_D(x)=g(x)\s\forall x\in D\]
\end{Def}
\begin{Bem}
  Sei $f:D\to\mb{R}(\mb{C})$ stetig. Sei $x_0\not\in D$. Die Fragen:
  \begin{itemize}
    \item gibt es eine stetige Fortsetzung von $f$ auf $D\cup \left\{ x_0 \right\}$
    \item ist diese Fortsetzung eindeutig?
  \end{itemize}
\end{Bem}
\begin{Def}
  $x_0$ ist ein Häufungspunkt von einer Menge $E$ wenn $\forall \varepsilon>0$ $\exists$ unendlich viele Punkte $x\in E$ mit
  \[\abs{x-x_0}<\varepsilon\]
\end{Def}
\begin{Bem}
  $x_0$ ist ein Häufungspunkt von $E\iff \exists(x_n)\subset\setminus\left\{ x_0 \right\}$ mit $x_n\to x_0$
\end{Bem}
\begin{Bem}
  In Bem 10, 1. Frage: Falls $x_0$ kein Häufungspunkt von $D$ ist: $\exists$ stetige Fortsetzungen, $\exists$ unendlich viele!
\end{Bem}
\begin{Bem}
  Wenn $x_0$ ein Häufungspunkt von $D$ ist, die Antwort zur 2. Frage ist ja. Die Antwort zur 1. ist undefiniert.
\end{Bem}
\begin{Def}
  $x_0$ Häufungspunkt von $D$, $x_0\not\in D$, falls $\exists$ stetige Fortsetzung $\tilde f$ von $f$ auf $D\cup\left\{ x_0 \right\}$ existiert. Dann $\tilde f(x_0)=\lim_{x\to x_0}f(x)$
\end{Def}


\begin{Def}
  Sei $f:D\to\mb{R}(\mb{C})$, $D\subset \mb{R}(\mb{C})$. 
Sei $x_0\in D$ ein Häufungspunkt. 
Der Grenzwert von $f$ (falls er existiert) an der Stelle 
$x_0$ ist die einzige Zahl $a\in\mb{R}(\mb{C})$ so dass
  \[F(x)=\begin{cases}
    f(x)&x\in D\setminus \left\{ x_0 \right\}\\
    a&x=x_0
  \end{cases}\]
  stetig in $x_0$ ist.
\end{Def}
\begin{Bem}
  $f(x_0)=a$ und $x_0\in D$ $\implies$ $f$ ist stetig an der Stelle $x_0$.
Aber nicht unbedingt $f(x_0)=a$!
\end{Bem}
\begin{Sat}\label{s:Ch}
  Die folgenden Aussagen sind äquivalent:
  \begin{itemize}
    \item $\lim_{x\to x_0}f(x)=a$
    \item $\forall \left\{ x_n \right\}\subset D\setminus\left\{ x_0 \right\}$ mit $x_n\to x_0$ gilt $\lim_{n\to +\infty} f(x_n)=0$
    \item $\forall \varepsilon >0$ $\exists\delta>0$ so dass $\abs{x-x_0}<\delta$ und $x\in D\setminus\left\{ x_0 \right\}$ $\implies$ $\abs{f(x)-a}=0$
  \end{itemize}
\end{Sat}
\begin{proof}[Beweis] Die sind triviale Folgerungen der 
Definitionen und des Folgenkriteriums f\"ur die Stetigkeit von $f$.
\end{proof}

\begin{Sat}
  (Rechenregeln) $f,g:D\to\mb{R}(\mb{C})$, $x_0$ Häufungspunkt von $D$
  \[\lim_{x\to x_0}(f+g)(x)=\left( \lim_{x\to x_0} f(x) \right) + \left( \lim_{x\to x_0} g(x_0) \right)\]
  falls die Grenzwerte existieren!
  \[\lim_{x\to x_0}(fg)(x)=\left( \lim_{x\to x_0} f(x) \right) \left( \lim_{x\to x_0} g(x_0) \right)\]
  \[\lim_{x\to x_0}\frac{f}{g}(x)=
\frac{\lim_{x\to x_0} f (x)}{\lim_{x\to x_0} f(x)}\s
\text{falls}\s\lim_{x\to x_0}g(x)\neq 0\]
\end{Sat}
\begin{Sat}
  Seien $f:D\to E$, $g:E\to \mb{R} (\mb{C})$ mit
  \begin{itemize}
    \item $x_0$ Häufungspunkt von $D$ und $y_0=\lim_{x\to x_0}f(x)$
    \item $y_0\in E$ und $g$ ist stetig and der Stelle $y_0$
  \end{itemize}
  Dann:
  \[\lim_{x_\to x_0} g\circ f(x)=g(y_0)=g(\lim_{x\to x_0}f(x))\]
\end{Sat}
\begin{proof}[Beweis]
  Wenden Sie die entsprechenden Rechenregeln für Folgen.
Als Beispiel:
Teil 1 von Satz 3.Für $f,g$ wir haben: $\forall \left\{ x_n \right\}\subset D\setminus \left\{ x_0 \right\}$ mit $x_n\to x_0$
  \[\lim_{n\to \infty} f(x_n)=\lim_{x\to x_0}f(x_0)\wedge\lim_{n\to \infty} g(x_n)=\lim_{x\to x_0}g(x_0)\]
  \[\implies \lim_{n\to\infty}(f+g) (x_n)=\overbrace{\lim_{n\to\infty}f(x_n)}^{\lim_{x\to x_0}f(x)}+\overbrace{\lim_{n\to \infty}g(x_n)}^{\lim_{x\to x_0}g(x)}\]
  \[\stackrel{\text{Satz \ref{s:Ch}}}{\implies}\lim_{x\to x_0}(f+g)(x)=\lim_{x\to x_0} f(x)+\lim_{x\to x_0} g(x)\]
\end{proof}
\begin{Def}
  Falls $f:D\to\mb{R}$ und $x_0$ ein Häufungspunkt von $D$ ist, dann:
  \begin{itemize}
    \item $\lim_{x\to x_0}f(x)=+\infty(-\infty)$ falls $\forall \left\{ x_n \right\}\subset D\setminus \left\{ x_0 \right\}$ mit $x_n\to x_0$ gilt $\lim_{n\to +\infty}f(x_n)=+\infty$ (bzw. $-\infty$)
  \end{itemize}
  Ähnlich $f:D\to\mb{C}$ und:
  \begin{itemize}
    \item $D$ ist nicht nach oben beschränkt. Wir schreiben $\lim_{x\to+\infty}f(x)=a$ genau dann, wenn $\forall\left\{ x_n \right\}\subset D$ mit $x_n\to\infty$ gilt $\lim_{n\to+\infty}f(x_n)=a$
  \end{itemize}
  gleich wenn $D$ nicht nach unten beschränkt ist. $\lim_{x\to-\infty}f(x)=a$
\"Ahnlicherweise handelt man die F\"alle
  \[\lim_{x\to+\infty}f(x)=\pm\infty\]
  \[\lim_{x\to-\infty}f(x)=\pm\infty\]
\end{Def}
\begin{Def}
  Seien $D\subset\mb{R}$, $f:D\to\mb{R}(\mb{C})$ und $x_0$ ein Häufungspunkt 
von $\left] -\infty,x_0 \right[\cap D$. Dann $\lim_{x\uparrow x_0}f(x)=a$ falls 
\[\forall \left\{ x_n \right\}\subset D\cap \left] -\infty, x_0 \right[ 
\;\;\mbox{mit } x_n\to x \quad\mbox{gilt}\quad 
\lim_{n\to+\infty}f(x_n)=a\] 
Man schreibt auch
\[
\lim_{x\to x_0^-} f(x)=a
\]
Falls $x_0$ ein Häufingspunkt von $D\cap \left] x_0+\infty \right[$ ist
  \[\lim_{x\downarrow x_0}f(x)=a \qquad \left(\lim_{x\to x_0^+} f(x)\right)\]
  falls $\forall \left\{ x_n \right\}\subset D\cap \left] x_0, +\infty \right[$ 
mit $x_n\to x_0$ gilt $f(x_n)\to a$. Ähnlich definiert
man $\lim_{x\to x_0^\pm}f(x)=\pm\infty$.
\end{Def}
\begin{Bsp}
  Stetigkeit:
  \[\lim_{x\to x_0}f(x)=f(x_0)\]
  \begin{itemize}
    \item $\exists \lim_{x\to x_0}$ $f(x)\neq f(x_0)$ $\implies$ 
$f$ in $x_0$ nicht stetig ist
\item $\lim_{x\to x_0}f(x)=+\infty$: die Funktion $f$ in $x_0$ hat eine Asymptote.
  \end{itemize}
\end{Bsp}
\section{Exponentialfunktion}
Sei $a\in\mb{R}$ $a>0$. Dann $a^q=a^\frac{m}{n}=\sqrt[n]{m}$, f\"ur jede
$q=\frac{m}{n}\in\mb{Q}$. Ziel dieses Kapitel ist die Funktion $a^z$
auf der ganzen komplexen Ebene zu definieren.

\subsection{Existenz und Eindeutigkeit}
\begin{Sat}\label{s:Exp}
  $\exists ! \Exp:\mb{C}\to\mb{C}$ mit folgenden Eigenschaften:
  \begin{itemize}
    \item[(AT)] Additionstheorem $\Exp(z+w)=\Exp(z)\Exp(w)$
    \item[(WT)] Wachstum $\lim_{z\to 0}\frac{\Exp(z)-1}{z}=1$
  \end{itemize}
  Für $\Exp$ wissen wir:
  \begin{itemize}
    \item $\Exp(z)=\sum_{n=0}^\infty\frac{z^n}{n!}$ $\forall z\in\mb{C}$
    \item $\Exp(z)=\lim_{n\to+\infty}\left( 1+\frac{z}{n} \right)^n$ $\forall z\in \mb{C}$
    \item $\Exp$ ist stetig und falls $e=\Exp(1)$ dann $e^q=\Exp(1)$ $\forall q\in \mb{R}$,
wobei 
  \[e=\sum\frac{1}{n!}=\lim_{n\to+\infty}\left( 1+\frac{1}{n} \right)^n\]
 \end{itemize}
\end{Sat}
\begin{Bem}\label{b:Kern}
  Kernidee: Wir suchen eine Funktion $\Exp(z)=f(z)$ mit den Eigenschaften
(AT) und (WT)
\begin{equation}\label{e:100} f(z) = f\left(\frac{nz}{n}\right)=
f\left(\frac{z}{n}+\frac{z}{n}+\cdots+\frac{z}{n}\right)\stackrel{\text{(AT)}}{=}
f\left(\frac{z}{n}\right)^n
\end{equation}
Wir definieren
  \[f\left( \frac{z}{n} \right)=1+\frac{z_n}{n}\qquad
\mbox{d.h.} \qquad z_n = n \left(f\left(\frac{z}{n}\right) -1\right)\]
F\"ur  $n\to+\infty$ $\frac{z}{n}\to 0$ und aus (WT) schliessen wir
\begin{equation}\label{e:Subs}
\lim_{n\to+\infty}z_n=z\lim_{n\to+\infty}
\frac{f\left( \frac{z}{n} \right)-1}{\frac{z}{n}}=z
\end{equation}
Aus \eqref{e:100} folgt
\begin{equation}\label{e:101}
f(z)=\left( 1+\frac{z_n}{n}\right)^n\implies f(z)
=\lim_{n\to+\infty}\left( 1+\frac{z_n}{n}\right)^n
\end{equation}
D\"urfen wir, wegen \eqref{e:Subs}, $z_n$ durch $z$ in \eqref{e:101}
ersetzen? Die Antwort ist im n\"achsten Lemma enthalten
\end{Bem}
\begin{Lem}
  Fundamentallemma: $\forall \left\{ z_n \right\}\subset\mb{C}$ mit $z_n\to z$ gilt:
  \[\Limi{n}\left( 1+\frac{z_n}{n}\right)^n=\Limi{n}\left( 1+\frac{z}{n}\right)^n=\sum \frac{z^n}{n!}\]
\end{Lem}
\begin{Bem}
  $\sum \frac{z^n}{n!}$ konvergiert auf $\mb{C}$ (und konvergiert deswegen absolut)
\end{Bem}
\begin{proof}[Beweis]
  Das Kriterium von Hadamand:
  \[R:=\frac{1}{\limsup_{n\to+\infty}\sqrt{\frac{1}{n!}}}=+\infty\]
  Das bedeutet:
  \[\lim_{n\to +\infty}\sqrt[n]{\frac{1}{n!}}=0\]
  \[\begin{cases}
    n\s\text{gerade}& n!\geq \underbrace{n(n-1)\cdots \left(\frac{n}{2}+1\right)}_{\frac{n}{2}}
\frac{n}{2}\cdot 1 \geq \left(\frac{n}{2}\right)^{\frac{n}{2}}\\
    n\s\text{ungerade}& n!\geq \underbrace{n(n-1)\cdots \frac{n+1}{2}}_{\frac{n+1}{2}}\frac{n-1}{2}\cdot 1
\geq \left(\frac{n+1}{2}\right)^{\frac{n+1}{2}}\geq \left(\frac{n}{2}\right)^{\frac{n}{2}}\\
  \end{cases}\]
Deswegen
  \[\sqrt[n]{n!}\geq\left(\left(\frac{n}{2}\right)^{\frac{n}{2}}\right)^{1/n}
= \frac{\sqrt[2]{n}}{\sqrt[n]{2}} \to \infty\]
\end{proof}

\begin{Bew}
  Die Reihe $\sum^\infty_{n=0}\frac{z^n}{n!}$ konvergiert absolut $\forall z\in\mb{C}$
\end{Bew}
\begin{Beh}
  \[\underbrace{\lim_{n\to+\infty}(sup)\abs{\left( 1-\frac{z_n}{n} \right)^n-\sum^\infty_{n=0}\frac{z^k}{k!}}}_{A_n}=0\]
\end{Beh}
\begin{Bem}
  \[\left( 1-\frac{z_n}{n} \right)^n-\sum^\infty_{n=0}\frac{z^k}{k!}=\abs{\sum^n_{k=0}\binom{n}{k}\frac{z^k_n}{n^k}-\sum^\infty_{k=0}\frac{z^k}{k!}}\]
  Sei $M\in\mb{N}$ und $M\leq n$
  \[\leq \underbrace{\abs{\sum^M_{k=0}\left( \binom{n}{k}\frac{z^k_n}{n^k}-\frac{z^k}{k!} \right)}}_{B_n}+\underbrace{\sum^n_{k\geq M+1}\binom{n}{k}\frac{\abs{z_n}^k}{n^k}}_{C_n}+\underbrace{\sum^\infty_{k=M+1}\frac{\abs{z}^k}{k!}}_{D}\]
  \[\underbrace{\binom{n}{k}\frac{z^k_n}{n^k}}_{z^k_n}=\frac{n!}{(n-k)!k!}\frac{z^k_n}{n^k}=\underbrace{\frac{n(n-1)\cdots(n-k+1)}}_{k\s\text{mal}}{nn\cdots n}\frac{z^k_n}{k!}\]
  \begin{equation}
    \lim_{n\to\infty}a^k_n=\lim_{n\to+\infty}1\left( 1-\frac{1}{n} \right)\cdots\left( 1-\frac{k-1}{n} \right)\frac{z^k_n}{k!}=\frac{z^k}{k!}
    \label{1011034}
  \end{equation}
  \[\limsup_{n\to+\infty}A_n\leq\underbrace{\limsup_{n\to+\infty}B_n}_{=0, aus \ref{1011034}} + \limsup_{n\to+\infty}C_n+D\]
  Abschätzung für $C_n$:
  \[C_n=\sum^n_{k=M+1}\frac{\abs{z_n}^k}{k!}1\left( 1-\frac{1}{n} \right)\cdots\left( 1-\frac{k-1}{n} \right)\]
  $\abs{z_n}$ konvergiert nach $\abs{z}$ $\implies$ $\exists R\geq 0$ mit $\abs{z_n}\leq R$
  \[\leq\sum^n_{k=M+1}\frac{\abs{z_n}^k}{k!}\leq\sum^\infty_{k=M+1}\frac{R^k}{k!}\]
  \begin{equation}
    \limsup_{n\to+\infty}A_n\leq \sum^\infty_{k=M+1}\frac{R^k}{k!}+\sum^\infty_{k=M+1}\frac{\abs{z}^k}{k!}
    \label{1011035}
  \end{equation}
  \begin{equation}
    \limsup_{n\to+\infty}\sum^\infty_{k=M+1}\frac{R^k}{k!}=0
    \label{1011036}
  \end{equation}
  \[(\text{weil}\s\limsup_{n\to+\infty}\sum^M_{k=0}\frac{R^k}{k!}=\sum^\infty_{k=0}\frac{R^k}{k!}\implies\lim_{M\to+\infty}\sum^\infty_{k=M+1}\frac{R^k}{k!}\]
  \[\lim_{M\to+\infty}\left( \sum^\infty_{k=0}\frac{R^k}{k!}- \sum^M_{k=0}\frac{R^k}{k!}\right)=0)\]
  \[\ref{1011035}\ref{1011036}\implies\limsup_{n\to+\infty}A_n=0\]
\end{Bem}
\begin{Bem}
  Lemma + Bemerkung $\implies$ Falls eine Funktion mit der Eigenschaft (AT) und (WT) existiert, dann gilt:
  \[\Exp(z)=\lim_{n\to+\infty}\left( 1+\frac{z}{n} \right)^n=\sum^\infty_{k=0}\frac{z^k}{k!}\]
\end{Bem}
\begin{Bew}
  (Eindeutigkeit haben wir schon $\uparrow$.) Wir definieren
  \[\Exp(z)=\sum^\infty_{k=0}\frac{z^k}{k=0}\left( =\lim_{n\to+\infty}\left( 1+\frac{z}{n} \right)^n \right)\]
  (AT) gilt:
  \[\Exp(z)\Exp(w)=\Limi{n}\left( 1+\frac{z}{n} \right)^n\Limi{n}\left( 1+\frac{w}{n} \right)^n\]
  \[=\Limi{n}\left( \left( 1+\frac{z}{n} \right)\left( 1+\frac{w}{n} \right) \right)^n=\Limi{n}\left( 1+\frac{z+w}{n}+\frac{zw}{n^2} \right)^n\]
  \[=\Limi{n}\left( 1+\frac{\overbrace{\left( z+w+\frac{zw}{n} \right)}^{\alpha_n}}{n} \right)\stackrel{\text{Fundamentallemma}}{=}\Limi{n}\left( 1+\frac{z+w}{n} \right)^n\]
  \[=\Exp(z+w)\s\text{da} \alpha\to(z+w)\]
  Sei
  \[e=\Exp(1)\left( \Limi{n}\left( 1+\frac{1}{n} \right)^n =\sum^\infty_{k=0}\frac{1}{k!}\right)\]
  \[\Exp(q+s)=\Exp(q)\Exp(s)\s\forall q,s\in\mb{Q}\]
  (Zur Erinnerung: Falls $f:\mb{Q}\to\mb{R}$ erfüllt $f(1)=a>0$ und $f(q+s)=f(q)(s)$. Dann $f(q)=a^q$ $\forall q\in\mb{Q}$.) Setze
  \[f:\Exp\implies\Exp{q}=e^q\s\forall q\in\mb{Q}\]
  (Später: $\Exp(z)=e^z$)
\end{Bew}
\begin{Lem}
  Sei $\sum a_nz^n$ eine Potenzreihe mit Konvergenzradius $R>$ ($R=+\infty$ falls die Reihe überall konvergiert). Dann ist $f(z)=\sum a_nz^n$ eine stetige Funktion auf $\left\{ \abs{z}<R \right\}$ ($\mb{C}$ falls $R=+\infty$).
\end{Lem}
\begin{Bem}
  Lemma $\implies$ Stetigkeit von $\Exp$. Ausserdem:
  \[\Limo{z}\frac{\Exp{z}-1}{z}=1\]
  \[\frac{\Exp(z)-1}{z}=\frac{1+\sum^\infty_{k=1}\frac{z^k}{k!}-1}{z}=\sum^\infty_{k=1}\frac{z^{k-1}}{k!}=G(z)\]
  Die Reihe, die $G$ definiert hatKonvergenzradius $+\infty$. Deswegen ist $G$ stetig.
  \[\implies\Limo{z}\frac{\Exp(z)-1}{z}=\Limo{z}G(z)-G(0)=1\]
\end{Bem}
\begin{Bew}
  Zu beweisen: Sei $z_0$ mit $\abs{z_0}<R$
  \[\text{Stetigkeit in}\s x_0\iff \lim{z\to z_0}f(z)=f(z_0)\]
  \[\lim_{z\to z_0}\sum^\infty_{n=0}a_nz^n=\sum^+\infty_{n=0}a_nz^n_0\]
  Finden sie $a_{k,n}\in\mb{R}$ $\forall k\Limi{n}a_{k,n}=a_n (k,n\in\mb{N})$ aber $\Limi{n}\sum^\infty_{n=0}a_{k,n}\neq\sum^\infty_{k=0}a_k$. Sei $z_k\to z_0$.
  \[\limsup_{k\to+\infty}\overbrace{\abs{\sum^\infty_{n=0}a_nz_k^n-\sum^\infty_{n=0}z^n_0}}^{A_k}\]
  \[\leq\limsup_{k\to+\infty}\abs{\sum^M_{n=0}a_nz_k^n-\sum^M_{n=0}a_nz^n_0}+\limsup_{k\to+\infty}\sum^\infty_{n=M}\abs{a_n}\abs{z^n_k}+\sum^\infty_{n=M}\abs{a_n}\abs{z_0}^n\]
  Sei $\rho$ mit $\abs{z_0}<\rho<R$. Da $z_k\to z_0$: $\abs{z_k}<\rho$, falls $k$ gross genug ist.
  \[\limsup_{k\to+\infty}A_k\leq 0+2\sum^\infty_{n=M+1}\abs{a_}n\rho^n\]
  \[\implies\limsup_{k\to+\infty}A_k\leq 2\limsup_{M\to+\infty}\underbrace{\sum\infty_{n=M+1}\abs{a_n}\rho^n}_{\text{konvergiert}}=0
  \[\implies\Limi{k}(z_n)=\left( \sum^\infty_{n=0}z_k^na_n \right)=f(z_0)\left( =\sum^\infty_{n=0}z_0^na_n \right)\implies \lim_{z\to z_0}f(z)=f(z_0)\]
\end{Bew}
\begin{Bem}
  \[\abs{\sum^\infty_{n=0}\xi_n-\sum^\infty_{n=0}\zeta_n}\]
  \[=\abs{\Limi{N}\left( \sum^N_{n=0}\xi_n-\zeta_n \right)}\]
  \[\leq \Limi{N}\left\{ \abs{\sum^M_{n=0}\left( \xi_n-\zeta_n \right) }+\abs{\sum^N_{n=M+1}\left( \xi_n-\zeta_n \right)}\right\}\]
  \[=\abs{\sum_{n=0}^M\left( \xi_n-\zeta_n \right)}+\Limi{N}\abs{\sum_{n=M+1}^\infty\left( \xi_n-\zeta_n \right)}\]
  \[\leq\abs{\sum_{n=0}^M\left( \xi_n-\zeta_n \right)}+\Limi{N}\sum_{n=M+1}^\infty\left( \abs{\xi_n}-\abs{\zeta_n} \right)\]
  \[=\abs{\sum_{n=0}^M\left( \xi_n-\zeta_n \right)}+\sum_{n=M+1}^\infty\abs{\xi_n}-\sum_{n=M+1}^\infty\abs{\zeta_n}\]
\end{Bem}

\subsection{Die Exponentialfunlktion auf der reellen Gerade}

%\begin{itemize}
%  \item $f(z+w)=f(z)f(w)$
%  \item $\Limo{z}\frac{l^z-1}{z}=1$
%\end{itemize}
\begin{Sat} Die Funtion $\mb{R}\ni x \mapsto e^x \in \mb{R}$ ist
  \begin{enumerate}
    \item positiv
    \item monoton steigend
    \item bijektiv (falls $\mb{R}$ durch $\mb{R}^+$ ersetzt wird).
  \end{enumerate}
\end{Sat}
\begin{proof}[Beweis]
  \begin{enumerate}
    \item \[e^x=e^{\frac{x}{2}+\frac{x}{2}} = (e^\frac{x}{2})^2 \geq 0\]
$e^x = 0$ ist nicht m\"oglich, sonst w\"are $e^{xq}=0$ f\"ur alle 
$q\in \mb{Q}$ und, wegen der Dichtheit der rationalen Zahlen und
der Stetigkeit von $f$, $e^x\equiv 0$.
    \item \[\frac{e^{x+h}}{e^x}=e^h=1+\frac{h}{1!}+\cdots > 1\]
    \item z.z.: $\forall y\in\mb{R}^+$
      \[\exists x: e^x=y\]
      Falls $y\geq 1$
      \[e^0=1\leq y\leq e^y\stackrel{\text{ZWS}}{\implies}\exists x:e^x=y\]
      Falls $0<y<1$, dann betrachte $\frac{1}{y}>1$
      \[\exists x: e^x=\frac{1}{y}\implies e^{-x}=y\]
  \end{enumerate}
\end{proof}
\begin{Sat}[vom Wachstum]\label{s:wachstum1}
  \[\Limi{x}\frac{e^x}{x^n}=+\infty\]
  \[\lim_{x\to-\infty}x^ne^x=0\]
\end{Sat}
\begin{proof}[Beweis]
  \[e^x>\frac{x^{n+1}}{(n+1)!}\implies\frac{e^x}{x^n}>\frac{x}{(n+1)!}\stackrel{x\to\infty}{\to}\infty\]
  \[x^ne^x=\frac{x^n}{e^{-x}}=(-1)^n\frac{(-x)^n}{e^{-x}}\stackrel{x\to\infty}{\to}\infty\]
\end{proof}
\subsection{Natürlicher Logarithmus}
\begin{Def}
  $\ln:\mb{R}^+\to\mb{R}$ ist die Inverse der exponentiellen Funktion.
\end{Def}
\begin{Sat}
  \[\ln(xy)=\ln x + \ln y\]
\end{Sat}
\begin{proof}[Beweis]
  \[e^{\ln(xy)}=xy=e^{\ln x}e^{\ln y}=e^{\ln x + \ln y}\]
  \[\implies \ln(xy)=\ln x + \ln y\]
\end{proof}
\begin{Sat}[vom Wachstum 2] \label{s:wachstum2}
  \[\Limi{x}\frac{\ln x}{\sqrt[n]{x}}=0\]
\end{Sat}
\begin{proof}[Beweis]
  \[\frac{\ln x}{\sqrt[n]{x}}=\frac{\ln e^{ny}}{\sqrt[n]{e^{ny}}}=\frac{ny}{e^y}\]
  \[\exists y:x=e^{ny}\]
  \[\underbrace{y\to\infty}_{\iff x\to\infty}\implies \frac{\ln x}{\sqrt[n]{x}}\to 0\]
\end{proof}
\begin{Sat}
  \[\Limo{x}\frac{\ln(1+x)}{x}=1\]
\end{Sat}
\begin{proof}[Beweis]
  \[\Limo{x}\frac{\ln(1+x)}{x}=\Limo{y}\frac{\ln e^y}{e^y-1}=\Limo{y}\frac{y}{e^y-1}=1\]
\end{proof}
\begin{Bem}
  $\ln:\mb{R}^+\to\mb{R}$ ist stetig
\end{Bem}
\begin{Bem}
  $y=\frac{m}{n}\in\mb{Q}, n\in\mb{N}, a>0$
  \[\sqrt[n]{a^n}=\frac{m}{n}=e^{y\ln a}\]
  Warum?
  \[f(1)=e^{\ln a} =a\]
  \[f(q+r)=f(q)f(r)\]
  \[f:\mb{Q}\to\mb{R}\]
  \[\implies a^y=e^{y\ln a}\]
\end{Bem}
\begin{Def}
  $a>0$, $z\in\mb{C}$
  \[a^z:=e^{z\ln a}\]
\end{Def}
\begin{Sat}
  \begin{enumerate}
    \item \[a^{x+y}=a^xa^y\s(x,y\in\mb{C})\]
    \item \[(a^x)^y=a^{xy}\s(x,y\in\mb{R})\]
    \item \[(ab)^x=a^xb^x\]
  \end{enumerate}
\end{Sat}
\begin{proof}[Beweis]
  \begin{enumerate}
    \item \[a^{x+y}=w^{(x+y)\ln a}=e^{x\ln a+y\ln a}=e^{x\ln a}e^{y\ln a}=a^xa^y\]
    \item ähnlich
    \item \"ahnlich
  \end{enumerate}
\end{proof}
\begin{Sat}
  \begin{enumerate}
    \item \[\Limi{x}x^a= \begin{cases}
      \infty&a>0\\
      1&a=0\\
      0&a<0\\
    \end{cases}\] \label{i:101110a}
    \item \[\Limo{x}x^a= \begin{cases}
      0&a>0\\
      1&a=0\\
      \infty&a<0\\
    \end{cases}\]  \label{i:101110b}
    \item \[\Limi{x}x^a= \begin{cases}
      +\infty&a\geq0\\
      0&a<0\\
    \end{cases}\] \label{i:101110c}
    \item \[x^ae^x=+\infty\] \label{i:101110d}
    \item \[\Limo{x}\frac{a^x-1}{x}\ln a\] \label{i:101110e}
  \end{enumerate}
\end{Sat}
\begin{proof}[Beweis]
  \begin{enumerate}
    \item 
      \[\text{Bild}(x\mapsto x^a)=\mb{R}^+\implies\Limi{x}x^a=+\infty\]
      $a=0$ trivial
      $a>0$
      \[x^a=\frac{1}{x^{-a}}\to0\]
      (Wegen $-a>0$ und $x^{-a}\to\infty$)
    \item folgt aus \ref{i:101110a} durch die Substitution $x\mapsto \frac{1}{x}$. \ref{i:101110a} Falls $a>0$, $x^a$ monoton wachsend.
    \item $a\geq 0$ offensichtlich, $a<0$: $\exists n\in\mb{N}$, $a<-\frac{1}{n}$, $-a>\frac{1}{n}$
      \[x^a\ln x=\frac{\ln x}{x^{-a}}<\frac{\ln x}{x^{\frac{1}{n}}}\stackrel{\text{Satz \ref{s:wachstum2}}}{\to} 0\]
    \item $a>0$ trivial, $a<0$, $\exists n\in\mb{N}$ so dass $a>-n$ ($-a<n$)
      \[x^ae^x=\frac{e^x}{a^{-a}}>\frac{e^x}{x^n}\stackrel{\text{Satz \ref{s:wachstum1}}}{\to}\infty\]
    \item $\Limo{x}\frac{a-1}{x}=\ln a$
      \[\frac{a^x-1}{x}=\frac{e^{x\ln a}-1}{x}=\overbrace{\frac{e^{x\ln a}-1}{x\ln a}}^{\to 1}\ln a\to_{x\to 0} \ln a\]
  \end{enumerate}
\end{proof}
\subsection{Trigonometrische Funktionen}
\begin{Def} Falls $\phi$ die Gr\"osse eines Winkels (in Radianten) ist,
dann $\cos (\phi)$ und $\sin (\phi)$ sind die entsprenchenden Werte des
Cosinus und Sinus. Wir erweitern diese Funktionen auf der ganzen reellen
Gerade:
  \[\cos(\phi):=\cos(\phi-2\pi n) \qquad \mbox{falls } 2\pi n \leq \phi < 2\pi (n+1)\]
  \[\sin(\phi):=\sin(\phi-2\pi n) \qquad \mbox{falls } 2\pi n \leq \phi < 2\pi (n+1)\]
\end{Def}
\begin{Sat}
  Für $\phi$ klein genug gilt:
  \begin{enumerate}
    \item 
      \[\abs{\sin\phi}\leq\abs{\phi}\leq\frac{\abs{\sin\phi}}{\cos\phi}\]
    \item
      \[1-\cos\phi\leq\phi^2\]
  \end{enumerate}
\end{Sat}
\begin{proof}
  \begin{enumerate}
    \item Die Gr\"osse des Winkelns in Radianten ist die L\"ange des
entsprechenden Kreissektors (auf einem Kreis mit Radius 1): diese ist
gr\"osser als die L\"ange des (kleineren) Katheten.
    \item
      \[1-\cos\phi=\frac{(1-\cos\phi)(1+\cos\phi)}{1+\cos\phi}=\frac{1-(\cos\phi)^2}{1+\cos\phi}\leq\frac{\sin^2\phi}{1}\leq\phi^2\]
  \end{enumerate}
\end{proof}
\begin{Kor}
  \begin{enumerate}
    \item \[\Limo{\phi}\frac{\sin\phi}{\phi}=1\]
    \item \[\Limo{\phi}\frac{1-\cos\phi}{\phi}=0\]
    \item $\sin$ und $\cos$ sind stetig.
  \end{enumerate}
\end{Kor}
\begin{proof}[Beweis]
  \begin{enumerate}
    \item \[\frac{1}{\cos\phi}\leq\frac{\abs{\sin\phi}}{\phi}\leq 1\]
    \item \[0\leq\frac{1-\cos\phi}{\abs{\phi}}\leq\abs{\phi}\]
    \item Additionsregeln
      \[\cos(x+y)=\cos x\cos y-\sin x\sin y\]
      \[\sin(x+y)=\sin x\cos y+\cos x\sin y\]
  \end{enumerate}
\end{proof}
\begin{Sat}[von Euler]\label{s:Euler}
  \[e^{x+iy}=e^x(\cos x+\sin y) \qquad \forall x,y\in \mb{R}\]
\end{Sat}
\begin{Bew}
  Definiere $f(z)=e^x(\cos x+\sin y)$.
  $f$ erfüllt (AT) und (WT) im Satz \ref{s:Exp}% TODO ref
  \begin{itemize}
    \item[(AT)] folgt aus den Additionsregeln
    \item[(WT)] 2 Spezialfälle:
      \begin{itemize}
        \item $z=x\in\mb{R}$
          \[\Limo{z}\frac{f(z)-1}{z}=\Limo{x}\frac{e^x-1}{x}=1\]
        \item $z=iy$
          \[\Limo{z}\frac{f(z)-1}{z}=\Limo{y}\frac{\cos y+i\sin y-1}{iy}\]
          \[=\frac{1}{i}\Limo{y}\frac{\cos y-1}{y}+\frac{1}{i}\Limo{y}\frac{i\sin y}{y}=\frac{1}{i}0+\frac{1}{i}i=q\]
      \end{itemize}
      Der allgemeine Fall wird im Übungsblatt behandelt.
  \end{itemize}
\end{Bew}
\begin{Bem}(Was hat Euler gemacht?) Wegen der Taylor'schen Reihen:
  \[\cos y=\sum_{k=0}^\infty(-1)^k\frac{y^{2k}}{(2k)!}\]
  \[\sin y=\sum_{k=0}^\infty(-1)^k\frac{y^{2k+1}}{(2k+1)!}\]
das wussten die Mathematikern schon vor Euler seinen Satz entdeckte:
man kann diese Reihen mit der Differentialrechnung bestimmen
(und werden wir sp\"ater lernen).
Wenn man die Formel
  \[e^z:=\sum^\infty_{k=0}\frac{z^k}{k!}\]
f\"ur $z=iy$ anwendet:
  \[e^{iy}=\sum^\infty_{k=0}\frac{(iy)^k}{k!}=\underbrace{\sum_{k=0}^\infty(-1)^k\frac{y^{2k}}{(2k)!}}_{\cos y}+i\underbrace{\sum_{k=0}^\infty(-1)^k\frac{y^{2k+1}}{(2k+1)!}}_{\sin y}\]
  \[\implies e^{iy}=\cos y+i\sin y\]
  $e^{i\pi}=-1$ $\to$ die berühmte Formel von Euler.
\end{Bem}

\subsection{Noch andere spezielle Funktionen}
\[\tan=\frac{\sin}{\cos}\]
\[\tan=\mb{R}\setminus\underbrace{\left\{ \frac{\pi}{2}+k\pi; k\in\mb{N} \right\}}_{\text{Die Nullstellen des Cosinus}}\to\mb{R}\]
Geometrisch leicht zu sehen: 
\[\sin:\left[ -\frac{\pi}{2},\frac{\pi}{2} \right]\to\left[ 1,1 \right]\]
ist injektiv und surjektiv. Die Umkehrfunktion
\[\arcsin:\left[ -1,1 \right]\to\left[ -\frac{\pi}{2},\frac{\pi}{2} \right]\]
\[\cos:\left[ 0,\pi \right]\to\left[ -1,1 \right]\]
ist bijektiv. Die Umkehrfunktion
\[\arccos:\left[ -1,1 \right]\to\left[ 0,\pi \right]\]
\begin{Bem}
  \[\tan:\left] -\frac{\pi}{2},\frac{\pi}{2} \right[\to\mb{R}\]
  \[\lim_{x\to\pm\frac{\pi}{2}}\tan(x)=\pm\infty\]
  Surjektivität des Tagens auf $\left] -\frac{\pi}{2}, \frac{\pi}{2} \right[$ nach $\mb{R}$ ist leicht zu sehen. Injektivität werden wir später sehen.
\end{Bem}
\[\arctan:\mb{R}\to\left] -\frac{\pi}{2},\frac{\pi}{2} \right[\]
ist die Umkehrfunktion.
\[\sinh(t)=\frac{e^t-e^{-t}}{2}\]
\[\cosh(t)=\frac{e^t+e^{-t}}{2}\]
\[\tanh(t)=\frac{\sinh}{\cosh}\]
\begin{Bem}
  \[\cosh^2(t)-\sinh^2(t)=1\]
\end{Bem}
\begin{Bem}
  $\forall t\in\mb{R}$, $\left( \cos t, \sin t \right)$ $\in$ Kreis mit Radius 1 und Mittelpunkt 0. $\forall t\in\mb{R}$, $\left( \cosh t, \sinh t \right)$ $\in$ Hyperbola.
\end{Bem}
\section{Differentialrechnung}
Eine affine Funktion $f:\mb{R}\to\mb{R}$ hat die Gestalt:
\[f(t)=c_0+m_0x\]
\[m_0=\frac{f(t_2)-f(t_1)}{t_2-t_1}\]
$f$ heisst linear wenn $c_0=0$
\subsection{Ableitung}
\begin{Def}
  Die beste Approximation von $f$ in der Nähne von $x_0$ mit einer affinen Funktion $g$ 
\end{Def}
\begin{Bem}
  \[f(x)=\abs{x}\]
  $x_0=0$: $\exists$ keine gute Approximation mit einer affinen Funktion.
\end{Bem}
\begin{Def}
  Sei $f:\left] a,b \right[\to\mb{R}(\mb{C})$. Die Ableitung an der Stelle $x_0$ von $f$ ist
  \[f'(x)=\lim_{h\downarrow 0}\frac{f(x_0+h)-f(x_0)}{x_0+h-x_0}\]
\end{Def}
\begin{Def}
  Die Funktion heisst differenzierbar an der Stelle $x_0$, wenn die Ableitung $f'(x_0=$ existiert.
\end{Def}
\begin{Sat}
  $f:I\to\mb{C}$ ist in $x_0$ genau dann differenzierbar, wenn $\exists L:\mb{R}\to\mb{C}$ linear so dass
  \[\Limo{h}\frac{f(x_0+h)-f(x_0)-L(h)}{h}=0\]
\end{Sat}
\begin{Bew}
  \[L(h)\s\text{linear}\iff\exists m_0\in\mb{C}: L(h)=m_0h\s\forall h\in\mb{R}\]
  \begin{equation}
    \label{e:differential151}
    \Limo{h}\frac{f(x_0+h)-f(x_0)-L(h)}{h}
  \end{equation}
  \begin{equation}
    \label{e:differential152}
    \Limo{h}\frac{f(x_0+h)-f(x_0)}{h}-m_0
  \end{equation}
  \[\ref{e:differential151}=0\iff\ref{e:differential152}=m_0(=f(x_0))\]
\end{Bew}
\begin{Sat}
  $f:I\to\mb{C}$ ist in $x_0\in I$ genau dann differenzierbar, wenn es ein $\phi:I\to\mb{C}$ gibt so dass
  \begin{itemize}
    \item $\phi$ ist stetig in $x_0$
    \item $f(x)-f(x_0)=\phi(x)(x-x_0)$
  \end{itemize}
\end{Sat}
\begin{Bew}
  $\exists\phi$ $\implies$ differenzierbar
  \[\phi(x_0)\lim{x\to x_0}=\lim_{x\to x_0}\frac{f(x)-f(x_0)}{x-x_0}\]
  \[=\Limo{h}\frac{f(x_0+h)-f(x_0)}{h}=f(x)\]
  \[\Leftarrow\]
  \begin{Def}
    \[\phi=\begin{cases}
      f'(x_0)& x=x_0\\
      \frac{f(x)-f(x_0)}{x-x_0}& x\neq x_0
    \end{cases}\implies \phi\s\text{erfüllt die Bedingungen}\]
  \end{Def}
\end{Bew}
\begin{Bsp}
  $f(x)=x^n$
  \[f'(x_0)=\Limo{h}\frac{(x_0+h)^n-x_0^n}{h}=\Limo{h}\frac{\left\{ \left( x_0^n+\binom{n}{1}x_0^{n-1}+\binom{n}{2}x_0^{n-2}h^2+\cdots + h^n \right)-x_0^n \right\}}{h}\]
  \[\Limo{h}\left[ nx_0^{n-1}+\left\{ \binom{n}{2}x_0^{n-2}h+\cdots+h^{n-1} \right\} \right]=nx_0^{n-1}\]
\end{Bsp}
\begin{Bsp}
  $f(x)=e^x$
  \[f'(x_0)=\Limo{h}\frac{e^{x_0+h}-e^{x_0}}{h}\]
  \[=e^{x_0}\Limo{h}\frac{e^h-1}{h}=e^{x_0}\]
\end{Bsp}
\begin{Ueb}
  $f(x)=a^x$
  \[f'(x_0)=\ln(a) a^x\]
\end{Ueb}
\begin{Bsp}
  $f(x)=\ln x$
  \[f'(x_0)=\frac{\ln(x_0+h)-\ln(x_0)}{h}\]
  \[=\frac{\ln\left( \frac{x_0+h}{x_0} \right)}{h}\]
  \[=\left( \Limo{h}\frac{\ln\left( 1+\frac{h}{x_0} \right)}{\frac{h}{x_0}} \right)\frac{1}{x}\]
\end{Bsp}
\begin{Bem}
  Falls $f$ in $x_0$ differenzierbar ist, dann ist $f$ auch stetig in $x_0$.
  \[\lim x_0\iff\lim_{x\to x_0}f(x)f(x_0)\iff\lim{x\to x_0}\left( f(x)-f(x_0) \right)=0\]
  \[\Leftarrow\lim_{x\to x_0}\left( \frac{f(x)-f(x_0)}{x-x_0} \right)(x-x_0)=f'(x_0)0=0\]
\end{Bem}
\begin{Bem}
  Umgekehrt falsch $f(x)=\sqrt[n]{\abs{x}}$
  $n\geq 2$:
  \[\lim_{x\to 0}\frac{f(x)-f(0)}{x-0}=+\infty\]
  $n=1$:
  \[\lim_{x\to 0}\frac{f(x)-f(0)}{x-0}=\pm1\]
  Für $x\neq 0$ ist $\sqrt[n]{\abs{x}}$ differenzierbar
\end{Bem}
\subsection{Rechenregeln}
\begin{Sat}
  Seien $f,g:I\to\mb{C}$ differenzierbar in $x_0$.
  \begin{itemize}
    \item $f+g$ ist auch differenzierbar in $x_0$: 
      \[(f+g)'(x_0)=f'(x_0)+g'(x_0)\]
    \item $fg$ ist auch differenzierbar in $x_0$: 
      \[(fg)'(x_0)=f'(x_0)g(x_0)+f(x)g'(x_0)\]
    \item $\frac{f}{g}$ ist in der Nähne von $x_0$ wohldefiniert wenn $g(x_0)\neq 0$. Ausserdem ist $\frac{f}{g}$ dort differenzierbar.
      \[\left( \frac{f}{g} \right)(x_0)=\frac{f'(x_0)g(x_0)-f(x_0)g'(x_0)}{g(x_0)^2}<\]
  \end{itemize}
\end{Sat}
\begin{Bew}
  \begin{itemize}
    \item
      \[\Limo{h}\frac{(f+g)(x_0+h)-(f+g)(x_0)}{h}=\Limo{h}\left\{ \overbrace{\frac{f(x_0+h)-f(x_0)}{h}}^{f'(x_0)}+\overbrace{\frac{g(x_0+h)-g(x_0)}{h}}^{g'(x_0)}\right\}\]
    \item
      \[\Limo{h}\frac{(fg)(x_0+h)-(fg)(x_0)}{h}\]
      \[=\Limo{h}\frac{f(x_0+h)g(x_0+h)-f(x_0+h)g(x_0)+f(x_0+h)g(x_0)-f(x_0)g(x_0)}{h}\]
      \[=\Limo{h}\left\{ f(x_0+h)\frac{g(x_0+h)-g(x_0)}{h}+g(x_0)\frac{f(x_0+h)-f(x_0)}{h} \right\}\]
      \[=f(x_0)g'(x_0)+g(x_0)f'(x_0)\]
    \item
      \[\Limo{h}\frac{\frac{f(x_0+h)}{g(x_0+h)}-\frac{f(x_0)}{g(x_0)}}{h}=\Limo{h}\frac{f(x_0+h)g(x_0)-f(x_0)g(x_0+h)}{\left[ g(x_0)g(x_0+h) \right]}h\]
      \[=\Limo{h}\frac{1}{g(x_0)g(x_0+h)}\left\{ \frac{f(x_0+h)(g(x_0)-g(x_0+h)}{h}+\frac{f(x_0+h)g(x_0+h)-f(x_0)g(x_0+h)}{h} \right\}\]
      \[=\Limo{h}\frac{1}{g(x_0)g(x_0+h)}\left\{ f(x_0+h)\left[ -\frac{g(x_0+h)-g(x_0)}{h}\right] +g(x_0+h)\frac{f(x_0+h)-f(x_0)}{h} \right\}\]
  \end{itemize}
\end{Bew}
\begin{Sat}{Kettenregel}
  Seien $I\stackrel{f}{\to}J\stackrel{g}{\to}\mb{C}$, mit $I, J\subset\mb{R}$, $f$ und $g$ an der Stelle $x_0$ und $f(x_0)$ differenzierbar sind, dann ist $g\circ f$ an der Stelle $x_0$ differenzierbar und
  \[(g\circ f)'(x_0)=g'(f(x_0))f'(x_0)\]
\end{Sat}
\begin{Bew}
  \[\Limo{h}\frac{g\circ f(x_0+h)-g\circ f(x_0)}{x-x_0}\]
  \[=\Limo{h}\frac{g(f(x_0+h))-g(f(x_0))}{x-x_0}\]
  \[=\Limo{h}\frac{\overbrace{g(f(x_0+h))}^y-\overbrace{g(f(x_0))}^{y_0}}{\underbrace{f(x_0+h)}_y-\underbrace{f(x_0)}_{y_0}}\frac{f(x_0+h)-f(x_0)}{x-x_0}\]
  \[=g'(y_0)f'(x_0)=g'(f(x_0))f'(x_0)\]
  Problem: $y-y_0$ kann null werden. Lösung:
  \[f(x)-f(x_0)=\phi(x)(x-x_0)\]
  \[g(x)-g(x_0)=\gamma(x)(x-x_0)\]
  mit $\phi$ stetig in $x_0$, mit $\phi(x_0)=f'(x_0)$. $\gamma$ stetig in $y_0$ mit $\gamma'(y_0)=g'(y_0)$.
  \[g(f(x))-g(f(x_0))=\gamma(f(x))(f(x)-f(x_0))=\underbrace{\gamma(f(x))\phi(x)}_{\Phi(x)}(x-x_0)\]
  $\Phi$ ist stetig an der Stelle $x_0$. $\implies$ $g\circ f$ ist differenzierbar in $x_0$.
  \[(g\circ f)'(x_0)=\Phi(x_0)=\gamma(f(x_0))\phi(x_0)=g'(f(x_0))f'(x_0)\]
\end{Bew}
\begin{Bsp}
  \[e^{it}=\cos t+i\sin t\]
  \[(\cos x)'=\left(\frac{e^{ix}+e^{-ix}}{2}\right)=\frac{1}{2}\left( (e^{ix})'+(e^{ix})' \right)=\frac{i}{2}(e^{ix}+\frac{i}{2}e^{-ix}=-\frac{1}{2i}(e^{ix}-e^{-ix})=-\sin x\]
  \[(\sin x)'=\left(\frac{e^{ix}-e^{-ix}}{2i}\right)=\frac{1}{2i}\left( (e^{ix})'-(e^{ix})' \right)=\cdots=\cos x\]
  \[\tan'=\left( \frac{\sin}{\cos} \right)'=\frac{\sin'\cos-\sin\cos'}{\cos^2}=\frac{\sin^2+\cos^2}{\cos^2}=\frac{1}{\cos^2}\]
\end{Bsp}

\begin{Sat}\label{s:umkehr}[Differentiation der Umkehrfunktion]
  Sei $g$ die Umkehrfunktion einer streng monotonen Funktion $f:I\to\mb{R}$. 
Falls $f$ in $x_0$ differenzierbar ist und $f'(x_0)\neq 0$, 
dann ist $g$ in $y_0 = f(x_0)$ differenzierbar und
  \[g'(y_0)=\frac{1}{f(x_0)}\left( =\frac{1}{f'(g(y_0))} \right)\]
\end{Sat}
\begin{proof}[Beweis]
  \[f(x)-f(x_0)=\phi(x)(x-x_0)\]
  wobei
  \begin{itemize}
    \item $\phi$ ist stetig in $x_0$
    \item $\phi(x_0) = f'(x_0)$
  \end{itemize}
  \begin{eqnarray*}
    x=g(y)\\
    x_0=g(y_0)
  \end{eqnarray*}
  $\implies$
  \[y-y_0=\phi(g(y))(g(y)-g(y_0))\]
  \[g(y)-g(y_0)=\frac{1}{\phi(g(y))}( y - y_0) \quad \mbox{falls $\phi (g(y))\neq 0$.}\]
  Aber:
  \[\phi(g(y_0))=\phi(x_0)=f'(x_0)\neq 0\]
  $\phi$ ist stetig in $x_0$ und $g$ ist stetig in $y_0$ $\implies$ $\phi(g)$ ist stetig in $y_0$
  \[\exists\varepsilon>0: \abs{y-y_0}<\varepsilon\implies \phi(g(y))\neq 0\]
  Sei
  \[\psi = \begin{cases}
    \frac{1}{\phi(g(y))}&\abs{y-y_0}<\varepsilon\\
    \frac{g(y)-g(y_0)}{y-y_0}&\abs{y-y_0}>\varepsilon
  \end{cases}\]
  \[\implies g(y)-g(y_0)=\psi(y)(y-y_0)\]
  und $\psi$ ist stetig an der Stelle $y_0$. $g$ ist differenzierbar in $y_0$ und deswegen
  \[\psi(y_0)=g'(y_0)\]
  \[=\frac{1}{\phi(g(y_0))}=\frac{1}{\phi(x_0)}=\frac{1}{f'(x_0)}=\frac{1}{f'(g(y_0))}\]
\end{proof}
\begin{Bem}
  Sei $f:I\to\mb{R}$ streng monoton und stetig. Sei $g$ die Umkehrfunktion von $f$ $g:J\to I$.
 Angenommen dass beide Funktionen differenzierbar sind, die Kettenregel impliziert
  \[(f\circ g)'(x_0)=f'(g(x_0))g'(x_0)=1\, .\]
Falls $f' (g(x_0))\neq 0$, wir schliessen $g'(x_0)=\frac{1}{f'(g(x_0))}$. Das ist
aber kein Beweis vom Satz \ref{s:umkehr}, da die Differenzierbarkeit von $g$
angenommen und nicht bewiesen wird.
\end{Bem}
\begin{Bsp}
  (Übung: arcsin', arccos')
  \[\tan'(y_0)=\frac{1}{\cos^2(y_0)}\neq 0\]
  \[(\arctan)'(x_0)=\frac{1}{\tan'(\arctan(x_0))}=\frac{1}{\frac{1}{\cos^2(\arctan(x_0))}}\]
  \[=\cos^2(\arctan(x_0))\]
  \[\cos^2=\frac{1}{1+\tan^2}\left( =\frac{1}{1+\frac{\sin^2}{\cos^2}}=\frac{1}{\frac{\cos^2+\sin^2}{\cos^2}}=\cos^2 \right)\]
  \[\cos^2(\arctan(x_0))=\frac{1}{1+(\tan(\arctan(x_0)))^2}=\frac{1}{1+x_0^2}\]
  \[\implies \arctan'(x)=\frac{1}{1+x^2}\]
\end{Bsp}

\subsection{Die S\"atze von Rolle und Lagrange}

\begin{Sat}
  Sei $f:I\to\mb{R}$ eine überall differenzierbare Funktion. 
Sei $x_0\in I$ ein Maximum (bzw. ein Minium)
  \begin{itemize}
    \item $x_0$ im Inneren $\implies$ $f'(x_0)=0$
    \item $x_0$ ist das rechte Extremum von $I$ $\implies$
      \[f'(x_0)\geq 0\]
      bzw. bei Minima:
      \[f'(x_0)\leq 0\]
    \item $x_0$ ist das linke Extremum von $I$ $\implies$
      \[f'(x_0)\leq 0\]
      bzw. bei Minima:
      \[f'(x_0)\geq 0\]
  \end{itemize}
\end{Sat}
\begin{proof}[Beweis]
  $x_0$ im Innern.
  \[\begin{cases}
    \lim_{x\downarrow x_0} \frac{\overbrace{f(x)-f(x_0)}^{\leq 0}}{\underbrace{x-x_0}_{\geq 0}}\leq 0\\
    \lim_{x\uparrow x_0} \frac{\overbrace{f(x)-f(x_0)}^{\leq 0}}{\underbrace{x-x_0}_{\leq 0}}\geq 0
  \end{cases}\]
Deswegen $f'(x_0)=0$.
  $x_0$ ist das linke Extremum und eine Maximumstelle:
  \[f'(x_0)=\lim_{x\downarrow x_0}\frac{f(x)-f(x_0)}{x-x_0}\leq 0\, .\]
Die anderen F\"alle sind \"ahnlich.
\end{proof}
\begin{Sat}[Mittelwertsatz, Lagrange]\label{s:lagrange}
  Sei $f[a,b]\to\mb{R}$ stetig (überall) und differenzierbar in $]a,b[$. Dann $\exists \xi\in ]a,b[$ mit
  \[f'(\xi)=\frac{f(b)-f(a)}{b-a}\]
\end{Sat}
\begin{Sat}[Rolle]\label{s:rolle}
  Sei $f$ wie oben mit $f(b)=f(a)$. Dann $\exists\underbrace{\xi}_{\in ]a,b[}:f'(\xi)=0$.
\end{Sat}

Der Satz von Rolle ist ein Fall des Satzes von Lagrange. Aber wir werden
zuerst den Satz von Rolle beweisen und dann den von Lagrange daraus schliessen.

\begin{proof}[Beweis vom Satz \ref{s:rolle}]
  \[f(b)=f(a)\implies \begin{cases}
    \exists x\in ]a,b[\s\text{mit}\s f(x)<f(b)\\
    \exists x\in ]a,b[\s\text{mit}\s f(x)>f(b)\\
    f(x)=f(b)\s\forall x\in ]a,b[
  \end{cases}\]
  Dritte Möglichkeit $\implies$ $f$ ist Konstant!
  \[f'(\xi)=0\s\forall \xi\in ]a,b[ \]
  Erste Möglichkeit $\implies$ Sei $x_0$ eine Maximumstelle von $f$ in $[a,b]$
  \[\implies x_0\in ]a,b[ \mbox{ (weil $f (x_0) > f(a) = f(b)$) } \implies f'(x_0)=0\]
  Zweite Möglichkeit: Sei $x_0$ eine Maximumstelle:
  \[x_0\in ]a,b[\implies f'(x_0)=0\]
\end{proof}
\begin{proof}[Beweis vom Satz \ref{s:lagrange}]
  Sei
  \[g(x)=f(a)+\frac{x-a}{b-a}(f(b)-f(a))\]
  $g(b)=f(b)$ und $g(a)=f(a)$ $\implies$ Sei $h:=f-g$. $h(a)=0$, $h(b)=0$. 
  \[\stackrel{\text{Satz von Rolle}}{\implies}\exists \xi\in ]a,b[\s\text{mit}\s h'(\xi)=0\]
  \[\implies f'(\xi)-g'(\xi)=\frac{f(b)-f(a)}{a-b}\, .\]
\end{proof}
\begin{Kor}\label{k:monot}
  Sei $f:[a,b]\to\mb{R}$ eine differenzierbare Funktion.
  \begin{itemize}
    \item $f'\geq 0$ $\implies$ $f$ ist wachsend.
    \item $f'> 0$ $\implies$ $f$ ist wachsend, streng monoton.
    \item $f'\leq 0$ $\implies$ $f$ ist fallend.
    \item $f'< 0$ $\implies$ $f$ ist fallend, streng monoton.
  \end{itemize}
\end{Kor}
\begin{proof}[Beweis]
  Seien $c<d\in [a,b]$
  \[\stackrel{\text{Mittelwertsatz}}{\implies}\exists \xi\in ]c,d[\s\text{mit}\]
  \[f(d)-f(c)= f'(\xi)\underbrace{(d-c)}_{>0}\]
  $\geq 0$ im ersten Fall, $>0$ im zweiten Fall, usw.
\end{proof}
\begin{Kor}
  Sei $f:]a,b[\to\mb{R}$ differenzierbar. Falls:
  \begin{itemize}
    \item $f'(x)<0$ $\forall x>x_0$
    \item $f'(x)>0$ $\forall x<x_0$
  \end{itemize}
  dann ist $x_0$ das Maximum von $f$ auf $]a,b[$.
\end{Kor}
\begin{Kor}
  Sei $f: ]a,b[\to \mb{R}$ differenzierbar mit $f'\equiv 0$. Dann $f=\text{konst}$.
\end{Kor}
\begin{Bsp}\label{b:tan_in}
  $\tan$ ist streng monoton auf $]-\frac{\pi}{2},\frac{\pi}{2}[$:
  \[\tan'=\frac{1}{\cos^2}>0\]
NB: $\tan$ ist nicht monoton auf 
$\mb{R}\setminus\left\{ \frac{\pi}{2}+k\pi:k\in\mb{Z} \right\}$,
weil (z.B.) $-1 = \tan -\frac{\pi}{4} < 1= \tan \frac{\pi}{4}
> -1 = \tan{3\pi}{4}$. In diesem Fall ist Korollar \ref{k:monot}
nicht anwendbar, weil $\{\frac{\pi}{2}+k\pi:k\in\mb{Z}\}$ kein Intervall ist.
\end{Bsp}
\subsection{Anwendugen des Mittelwertsatzes: Schrankensatz und De L'Hôpitalsche Regel}
\begin{Sat}[Schrankensatz]
  Sei $f:[a,b]\to\mb{R}$ stetig (überall) und differenzierbar in $]a,b[$ mit
  \[\abs{f'(\xi)}\leq M\qquad\forall \xi\in ]a,b[\, .\]
  Dann ist $f$ Lipschitzstetig und
  \[\abs{f(y)-f(x)}\leq M\abs{x-y}\qquad\forall x,y\in [a,b]\, .\]
\end{Sat}
\begin{proof}[Beweis]
  $\forall y\neq x$ (OBdA: $y>x$)
  \[\exists \xi\in ]x,y[\subset ]a,b[: f(y)-f(x)=f(\xi)(y-x)\]
  \[\implies \abs{f(y)-f(x)}=\abs{f'(\xi)}\abs{y-x}\leq M\abs{y-x}.\]
\end{proof}

Die bekannte Funktionen die wir schon gesehen haben sind alle
Differenzierbar mit stetigen Ableitungen. Deswegen, wenn eingeschr\"ankt auf
einem geschlossenen Intervall, ist die Ableitung beschr\"ankt. Der Schrankensatz
impliziert dann die Lipschitzstetigkeit.

\begin{Sat}[Cauchy]
  Seien $f,g:[a,b]\to\mb{R}$ überall stetig und differenzierbar in $]a,b[$. Ausserdem $g'(x)\neq 0$ $\forall x\in ]a,b[$. Dann
  \[\exists \underbrace{\xi}_{\in ]a,b[}:\frac{f(b)-f(a)}{g(b)-g(a)}=\frac{f'(\xi)}{g'(\xi)}\, .\]
\end{Sat}
\begin{Bem} Der Mittelwertsatz ist ein Fall des Satzes von Cauchy: setzen wir
$g(x)=x$. Dann $g' (x)=1$ $\forall x$ und deswegen:
\[
\frac{f(b)-f(a)}{b-a} = \frac{f(b)-f(a)}{g(b)-g(a)}
= \frac{f'(\xi)}{g'(\xi)} = \frac{f'(\xi)}{1} = f'(\xi)\, .
\]
\end{Bem}
\begin{proof}[Beweis] Wie der Satz von Lagrange auch der 
Satz von Cauchy kann man auf dem Satz von Rolle herleiten. 
Wie setzten: 
  \[F(x)=f(x)-\frac{f(b)-f(a)}{g(b)-g(a)}(g(x)-g(a))\]
  \[F(a)=f(a)=F(b)\stackrel{Rolle}{\implies} 
\exists \xi: F'(\xi)=0\implies f'(\xi) = \frac{f(b)-f(a)}{g(b)-g(a)}g'(\xi)\, .\]
\end{proof}

\begin{Sat}[De L'Hospitalsche Regel]
  $f,g:]a,b[\to\mb{R}$ überall differenzierbar und mit $g(x), 
g'(x)\neq 0$ $\forall x\in ]a,b[$. In jeder dieser Situationen:
  \begin{enumerate}
    \item $f(x)\to 0, g(x)\to 0$ für $x\downarrow a$
    \item $f(x), g(x) \to +\infty$ (bzw. $-\infty$) für $x\downarrow a$
  \end{enumerate}
  Falls $\lim_{x\downarrow a}\frac{f'(x)}{g'(x)}$ existiert (oder $\pm \infty$ ist), dann
  \[\lim_{x\downarrow a}\frac{f(x)}{g(x)}=\lim_{x\downarrow a}\frac{f'(x)}{g'(x)}\]
Die entsprechenden Aussagen gelten auch f\"ur Grenzprozesse mit $x\uparrow b$ und
$x\to \pm \infty$.
\end{Sat}
Eine grobe Idee wie so dieser Satz gilt: nehmen wir an dass
die Funktionen $f$ und $g$ auch in $a$ definiert und differenzierbar sind, mit
$f(a)=g(a)=0$; dann, wenn $\abs{x-a}$ klein ist,
  \begin{eqnarray*}
    f(x)&=f'(a)(x-a)+R\\
    g(x)&=g'(a)(x-a)+R'
  \end{eqnarray*}
wobei $R$ und $R'$ ziemlich klein im vergleich mit $|x-a|$ sind. Deswegen,
  \[\frac{f(x)}{g(x)}\sim\frac{f'(a)}{g'(a)}\, .\]
Wenn die Ableitungen von $f$ und $g$ stetig w\"aren, dann
\[
 \frac{f(x)}{g(x)}\sim \frac{f'(x)}{g'(x)}.
\]


\begin{Bew}
  \begin{enumerate}
  \item 
      OBdA $f(a)=0$, $g(a)=0$ $\implies$ $f$ und $g$ sind stetig auf $[a,b[$. Verallgemeinerter Mittelwertsatz:
      \[\forall x\in ]a,b[\s\exists \xi\in ]a,x[:\]
      \[\frac{f(x)}{g(x)}=\frac{f(x)-f(a)}{g(x)-g(a)}=\frac{f'(\xi)}{g'(\xi)}\]
      \[x\to a\implies \xi\to a\]
      \[\lim_{x\downarrow a} \frac{f(x)}{g(x)}=\lim_{\xi\downarrow a} \frac{f'(\xi)}{g'(\xi)}\]
    \item Wir nehmen zus\"atzlich an dass 
\[\lim_{\xi\downarrow 0} \frac{f'(\xi)}{g'(\xi)}\in\mb{R}\, .
\]
 Sei $A:=\frac{f'(\xi)}{g'(\xi)}\in\mb{R}$. Wir schätzen $\abs{\frac{f(x)}{g(x)}-A}$ ab für $x$ 
in der Nähe von $a$. F\"ur jede $y<x$ mit $y\in ]a,b[$ schreiben wir
\begin{equation}\label{e:X1}
\frac{f(x)}{g(x)}=\frac{f(x)-f(y)}{g(x)-g(y)}\frac{1-\frac{g(y)}{g(x)}}{1-\frac{f(y)}{f(x)}}
\end{equation}
Sei $\varepsilon$ eine gegebene positive Zahl. Wählen wir ein $\delta>0$ so dass
\begin{equation}\label{e:X2}\left|\frac{f'(\xi)}{f'(\xi)}-A\right|<\varepsilon\qquad\forall 
\xi\in ]a,a+\delta[
\end{equation}
F\"ur jede $a<y<x<a+\delta$, sei $\xi$ die Stelle des Satzes von Cauchy. Dann:     
\[\left|\frac{f(x)}{g(x)}-A\right|\leq \underbrace{\left|
\frac{f(x)}{g(x)}-\frac{f(x)-f(a+\delta)}{g(x)-g(a+\delta)}\right|}_{=B}
+\underbrace{\left|\frac{f(x)-f(a+\delta)}{g(x)-g(a+\delta)}\right|}_{=C}\]
Aus \eqref{e:X2} folgt $C<\varepsilon$. Aus \eqref{e:X1}:
\[B=\left|\frac{f(x)-f(a+\delta)}{g(x)-g(a+\delta}\left( 
\frac{1-\frac{g(a+\delta)}{g(x)}}{1-\frac{f(a+\delta)}{f(x)}} \right)-\frac{f(x)-f(a+\delta)}
{g(x)-f(a+\delta)}\right|\]
      \[=\underbrace{\left|\frac{f(x)-f(a+\delta)}{g(x)-g(a+\delta)}\right|}
_{\leq \abs{A}+\varepsilon}\underbrace{\left|\frac{1-\frac{g(a+\delta)}{g(x)}}
{1-\frac{f(a+\delta)}{f(x)}}-1\right|}_{\to 0\s\text{für}\s x\downarrow a}\]
      $\implies$ $\exists \delta*$ so dass für $\abs{x-a}<\delta*$, $B<\varepsilon$. 
Sei nun $x$ s.d. $x-a<\min\left\{ \delta,\delta* \right\}$. Dann
      \[\left|\frac{f(x)}{g(x)}-A\right|<2\varepsilon\]
  \end{enumerate}
  Um den Beweis zu beenden, es bleibt zu tun:
  \begin{itemize}
    \item $x\downarrow a$, Situation 2., aber
      \[\lim_{x\downarrow a}\frac{f'(x)}{g'(x)}=+\infty (-\infty).\]
Der Beweis ist in diesem Fall ganz \"ahnlich zum obigen Beweis, aber anstatt
\[\left|\frac{f(x)}{g(x)}-A\right|<2\varepsilon \qquad \mbox{f\``ur $x-a$ klein genug}\]
ist das Ziel
\[\frac{f(x)}{g(x)} > M \qquad \mbox{f``ur $x-a$ klein genug}\]
(wobei $M$ eine beliebige reelle Zahl ist).
    \item $x\uparrow b$. Dieser Fall ist trivial.
\item $x\to +\infty$. In diesem Fall, setzen wir
  \[F(y)=f\left( \frac{1}{y} \right) \qquad \mbox{und}\qquad 
G(y)=g\left( \frac{1}{y} \right)\, .\]
Dann
\[\Limi{x}\frac{f(x)}{g(x)}=\lim_{y\downarrow 0 (=:a)}\frac{F(y)}{G(y)}
=\lim_{y\downarrow 0}\frac{F'(y)}{G'(y)}\]
\[=
\lim_{y\downarrow 0} \frac{f'\left(\frac{1}{y}\right)\left( -\frac{1}{y^2} \right)}
{g'\left( \frac{1}{y} \right)\left( -\frac{1}{y^2} \right)}
=\lim_{y\downarrow 0}\frac{f'\left(\frac{1}{y}\right)}{g'\left(
\frac{1}{y}\right)}=\Limi{x}\frac{f'(x)}{g'(x)}\]
\end{itemize}
\end{Bew}
\begin{Bsp}
  \[\Limi{x}\frac{e^x}{x}=\Limi{x}\frac{e^x}{1}=+\infty\]
\end{Bsp}
\begin{Bsp}
  \[\Limi{x}\frac{e^x}{x^n}=\Limi{x}\frac{e^x}{nx^{n-1}}=\Limi{x}\frac{e^x}{n(n-1)x^{n-2}}=\cdots=\Limi{x}\frac{e^x}{n!}=+\infty\]
\end{Bsp}
\begin{Bsp}
  \[\Limo{x}\left( \frac{1}{x}-\frac{1}{\sin x} \right)=\Limo{x}\frac{\sin x-x}{x\sin x}=\Limo{x}\frac{\cos x-1}{\sin x + x\cos x}\]
  \[=\Limo{x}\frac{\overbrace{-\sin x-0}^{\to 0}}{\underbrace{\cos x-x\sin x + \cos x}_{\to 2}}=0\]
\end{Bsp}
\subsection{Differentation einer Potenzreihe}
Aus den Rechenregeln f\"ur die Ableitung wissen wir:
\[P(x)=a_nx^n+a_{n-1}x^{n-1}+\cdots+a_0\]
\[P'(x)=na_nx^{n-1}+(n-1)a_{n-1}x^{n-2}+\cdots+a_1\]
Sei nun $f$ durch eine Potenzreihe definiert, mit einem nichttrivialen Konvergenzradius:
\[f(x)=\sum^\infty_{n=0}a_nx^n\]
K\"onnten wir schliessen dass $f$ auf ihrem Definitionsbereich differenzierbar ist? 
Ausserdem, gilt die Formel
\[f'(x)\stackrel{?}{=}\sum_{n=1}^\infty na_nx^{n-1}\]
\begin{Sat}\label{s:diff}
  Sei $\sum^\infty_{n=0}a_nx^n=f(x)$ eine Potenzreihe mit Konvergenzradius $R$ 
($> 0$, auch $R=+\infty$ m\"oglich). Falls $\abs{x_0}<R$, dann ist $f$ in $x_0$ differenzierbar und
  \[f'(x_0)=\sum^\infty_{n=1}na_nx^{n-1}\]
  (falls $R=+\infty$, $f$ ist überall differenzierbar, auf $\mb{R}$!)
\end{Sat}
\begin{Bem}\label{b:R=R'} Der Satz von Cauchy-Hadamard gibt
  \[R=\frac{1}{\limsup_{n\to\infty}\sqrt[n]{a_n}}\]
Nun, $\sum\infty_{n=1}n a_n x^{n-1}$ konvergiert für $x=0$ und 
für $x\neq 0$ konvergiert genau dann, wenn $\sum^\infty_{n=0} na_nx^n$ konvergiert. 
Der Konvergenzradius ist deswegen:
  \[R'=\frac{1}{\limsup_{n\to\infty}\sqrt[n]{na_n}}
= \frac{1}{\limsup_{n\to\infty} \sqrt[n]{a_n}} =R\, .\]
\end{Bem}

Wir wollen nun noch ein Mal das Lemma \ref{l:Abel} schauen. Dieses Lemma
sagt dass, wenn eine Potenzreihe an einer Stelle $x_0$ konvergiert, dann
konvergiert sie auch an jeder Stelle $x$ mit $|x|< |x_0|$. Aber die Kernidee
des Beweis dieses Lemma hat auch andere Konsequenzen.

\begin{Def}
 Sei $I=[a,b]$ ein abgeschlossenes Intervall und $f: I\to \mb{R}$ eine
stetige Funtion. Dann
\[
\|f\|_{C^0 (I)} \;=\; \max_{x\in I} |f(x)|\, .
\]
\end{Def}

\begin{Def}
Sei $I$ ein abgeschlossenese Intervall und $f_n: I \to \mb{R}$ eine Folge
von Funktionen. Falls $\sum_n f_n (x)$ an jeder Stelle $x\in I$ konvergiert,
koennten wir eine neue Funktion definieren:
\[
I\in x \quad \mapsto \quad f(x) = \sum_{n=0}^\infty f_n (x) \in \mb{R}\, . 
\]
F\"ur diese neue Funktion schreiben wir $f=\sum_n f_n$, d.h. eine
{\em Reihe von Funktionen}.

Falls jede $f_n$ stetig ist und
\[
\sum_{n=0}^\infty \|f_n\|_{C^0 (I)}<\infty\, ,
\]
dann sagen wir dass die Reihe $\sum_n f_n$ konvergiert normal.
\end{Def}

Eine Potenzreihe ist ein dann ein Beispiel einer Reihe von Funktionen.
Der Beweis vom Lemma \ref{l:Abel} impliziert dass eine Potenzreihe
im {\em Inneren} des Konvergenzkreis normal konvergiert.

\begin{Lem}\label{l:Abelbis}
  Sei $\sum a_nx^n$ eine Potenzreihe mit Konvergenzradius $R$. Sei $\rho<R$ und
$I= [-\rho, \rho]$. Die Potenzreihe konvergiert normal auf $I$.
\end{Lem}
\begin{Bew}
  $\rho <R$ Sei $x_0$ mit $\rho<|x_0|<R$. Dann
  \[\sum\abs{a_n}\abs{x_0}^n\quad\text{konvergiert}\]
Deswegen ist $|a_n||x_0|^n$ eine Nullfolge und 
  \[\exists M : \qquad \abs{a_n}\abs{x_0}^n\leq M\quad \forall n\]
Sei nun $f_n (x):= a_n x^n$.
Dann
\[
\|f_n\|_{C^0 (I)} \;=\; \max_{|x|\leq \rho} |f_n (x)|
\;=\; \max_{|x|\leq \rho} |a_n||x|^n = |a_n|\rho^n
\]
\[ 
\leq |a_n| |x_0|^n {\underbrace{\left(\frac{\rho}{|x_0|}\right)}_\gamma}^n
\leq M \gamma^n\, .
\]
Aber $\gamma<1$ und aus dem Majorantenkriterium folgt:
\[
\sum_{n=0}^\infty \|f_n\|_{C^0 (I)} <\infty\, .
\]
\end{Bew}
Sein nun $\sum_n f_n = \sum_n a_n x^n$ eine Potenzreihe wie
im Satz \ref{s:diff}. Sei $R$ der entsprechende Konvergenzradius 
und $\rho$ eine beliebige positive reelle Zahl mit $\rho<R$. 
Aus der Bemerkung \ref{b:R=R'} und dem Lemma \ref{l:Abelbis}
schliessen wir:
  \begin{enumerate}
    \item $\forall f_n$ ist differenzierbar
    \item $\sum f_n$ und $\sum f_n'$ sind beide normal konvergent auf $I=[\rho, \rho]$. 
  \end{enumerate}
Dann Satz \ref{s:diff} folgt aus der folgenden allgemeineren Aussage.

\begin{theorem}\label{t:C1Konv}
  Sei $\sum f_n$ eine Reihe von Funktionen auf einem abgeschlossenen
Intervall $I$. Falls:
  \begin{enumerate}
    \item $\sum f_n (x)$ $\forall x\in I$ konvergiert,
    \item $\sum f_n'$ normal konvergent ist
  \end{enumerate}
  dann ist $f$ überall differenzierbar mit $f'=\sum f_n'$.
\end{theorem}


\begin{Bew}
  Sei $x\in I$. Die Differenzierbarkeit an dieser Stelle bedeutet:
  \[\Limo{h}\left|\frac{f(x+h)-f(x)}{h}-f'(x)\right|=0\, .\]
Deswegen m\"ussen wir beweisen dass
  \[\Limo{h}\underbrace{\left|\sum^\infty_{h=0}\left( \frac{f_n(x+h)-f_n(x)}{h}-f_n'(x) \right)
\right|}=0\, .\]
F\"ur jede $N\in \mb{N}$ und jede $h$ mit $x+h\in I$ gilt:
  
\[D\leq \underbrace{\left|\sum_{n=0}^N\left(\frac{f_n(x+h)-f_n(x)}{h}-f'_n(x)\right)\right|}_{A} + 
\underbrace{\left|\sum_{n=N+1}^\infty\left(\frac{f_n(x+h)-f_n(x)}{h}-f_n(x)\right)\right|}_{B}\]
Sei $\varepsilon>0$ gegeben. Wir zeigen dass $\exists N\in \mb{N}$ und 
$\exists \bar h > 0$ s.d.
\[
A<\frac{\varepsilon}{2} \qquad \mbox{und}\qquad  
B<\frac{\varepsilon}{2} \qquad \forall h \mbox{ mit $|h|< \bar{h}$}.
\] 
Zuerst w\"ahlen wir $N$. 
\[B\leq \sum^\infty_{n=N+1}\left\{ \left|\frac{f_n(x+h)-f_n(x)}{h}\right|+\abs{f_n'(x)} \right\}\]
\[\stackrel{Schrankensatz}{\leq} 
\sum^\infty_{n=N+1}\left\{ \Norm{f_n'}_{C^0 (I)}+\Norm{f_n'}_{C^0} \right\}\]
\[= \sum^\infty_{n=N+1}2\Norm{f_n'}_{C^\circ(I)}=
2\left\{\overbrace{\sum^\infty_{n=0} \Norm{f_n'}_{C^\circ(I)}}^{b} - 
\overbrace{\sum^N_{n=0}\Norm{f_n'}_{C^\circ(I)}}^{b_N\to b}\right\}\]
  \[<\frac{\varepsilon}{2}\s\text{für $N$ gross genug\, .}\]
Eigentlich, diese Wahl von $N$ garantiert dass $A<\eps/2$ f\"ur {\em jede} $h$.

Nun w\"ahlen wir $\bar{h}$.
\[A=\left|\sum_{n=0}^N\underbrace{\left( \frac{f_n(x+h)-f_n(x)}{h}-f_n'(x) \right)}_{\rightarrow 0}\right|
\to 0 \qquad \mbox{f\"ur $h\to 0$}\]
Deswegen, $\exists \bar{h}>0$ s.d. $A< \eps/2$ wenn $|h|< \bar{h}$.
\end{Bew}
\subsection{Ableitungen h\"oherer Ordnung und Taylorreihe}

\begin{Def}
  Eine Funktion $f$ ist 2 mal differenzierbar an einer Stelle $x\in I$ wenn:
  \begin{itemize}
    \item $f'$ existiert $\forall y\in J$, wobei $x\in ]a,b[$ $\leftrightarrow$ $x \in$ Innern von
J, $J=[x,\tilde b[$ falls $I=[x,b[$ und $J= ]\tilde a, x]$ falls $I=]a, x]$)
    \item $f'$ ifferenzierbar in $x$ ist.
  \end{itemize}
  \[(f')' (x)=:f''(x)\qquad s\text{ist die Ableitung zweiter Ordnung}\]
  Induktiv: $f$ $n$-mal differenzierbar in $x$ falls:
  \begin{itemize}
    \item $f^{(n-1)}$ (d.h. die Ableitung $n-1$-ter Ordnung von $f$) 
in einer Umgebung von $x$ existiert
    \item $f^{(n-1)}$ differenzierbar in $x$ ist. 
  \end{itemize}
\[f^{(n)}(x):=\left( f^{(n-1)} \right)'(x) \qquad \mbox{ist die Ableitung $n$-ter Ordnung}.
 \]
Eine Funktion heisst belieb mal differenzierbar auf $I$ falls die Ableitung
aller Ordnungen auf jeder Stelle existieren.
\end{Def}
\begin{Bem}\label{b:Taylor}
  $f(x)=\sum^\infty_{n=0}a_nx^n$ mit Konvergenzradius $R$.
Dann ist $f$ beliebig mal differezierbar auf $]-R,R[$. Ausserdem, k\"onnten wir
$f^{(k)} (x)$ wie folgt bestimmen:
  \[f^{(k)}(x)=\sum^\infty_{n=k}n(n-1)(n-2)\cdots(n-k+1)a_nx^{n-k}\]
Es folgt dass 
\[f(0)=a_0\]
\[f'(0)=a_1\]
\[f^{(k)}(0)=k!a_k\]
\end{Bem}
\begin{Def}
  Eine Funktion $f$ heisst analytisch an einer Stelle $x_0$, falls auf einem Intervall 
$]x_0-\rho, x_0+\rho[$ gilt
\[f(x)=\sum a_n(x-x_0)^n\]
\end{Def}
Die Bemerkung \ref{b:Taylor} hat deswegen die folgende Konsequenz:
\begin{Kor}\label{k:Taylor}
  Sei $f$ analytisch in $x_0$. Dann $\exists \rho>0$ s.d.:
\begin{itemize}
 \item $f$ beliebig mal differenzierbar auf $I=]x-\rho, x+\rho[$ ist
  \item $f(x)=\sum\infty_{n=0}\frac{f^{(n)}(x_0)}{n!}(x-x_0)^n$ $\forall x\in I$.
\end{itemize}
\end{Kor}

Aber Vorsicht:  {\em Beliebig mal differenzierbar $\not\implies$ analytisch!}

\begin{Bsp}
  \[x_0=0\]
  \[f(x)=e^x=f'(x)=f''(x)=\cdots\implies f^{(k)}(x)=e^x\]
  \[\implies f^n(0)=1\implies e^x=\sum\frac{x^k}{k!}\]
\end{Bsp}
\subsection{Konvexität}
\url{http://de.wikipedia.org/wiki/Konvexe_und_konkave_Funktionen}
\begin{Def}
  Eine $f:I\to R$ heisst konvex, wenn:
  $\forall x_1 < x_2\in I$
  \begin{equation}\label{e:kk}
    f(x)\leq\frac{x-x_1}{x_2-x_1}f(x_2)+\frac{x_2-x}{x_2-x_1}f(x_1)=g(x)\s\forall x\in ]x_1,x_2[
  \end{equation}
  \begin{tabular}{l|c|l}
    streng konvex & $<$\\
    konkav & $\geq$& in \eqref{e:kk}\\
    streng konkav & $>$\\
  \end{tabular}
\end{Def}
\begin{Bem}
Allgemein, die Konvexit|'at impliziert nicht die Differenzierbarkeit.
Nehmen Sie z.B. $f(x)= |x|$ auf $\mb{R}$. 
\end{Bem}
\begin{Sat}\label{s:1011242}
  Sei $f:I\to\mb{R}$ stetig und differenzierbar im Inneren
  \[f\s\text{konvex} \iff f'(x_1)\leq f'(x_2)\s\forall x_1<x_2\]
  \[f\s\text{streng konvex} \iff f'(x_1) < f'(x_2)\s\forall x_1<x_2\]
\end{Sat}
\begin{Kor}
  Sei $f$ wie im Satz \ref{s:1011242} aber 2-mal differenzierbar im Inneren
  \[f\s\text{konvex}\iff f''\geq 0\]
  \[f\s\text{streng konvex}\Leftarrow f''> 0\]
\end{Kor}
\begin{Bsp}
Sei  $f(x)=x^4$. $f$ ist streng konvex und $f''(x)=12x^2$. Deswegen $f''(0)=0$
\end{Bsp}
\begin{Bem}
  Sei $f$ differenzierbar überall und 2 mal differenzierbar ain einer Stelle $x_0$ mit $f'(x_0)=0$. 
Falls 
  \begin{itemize}
    \item $f''(x_0)>0$ $\implies$ $x_0$ ist ein lokales Minimum
    \item $f''(x_0)<0$ $\implies$ $x_0$ ist ein lokales Maximum
  \end{itemize}
Nehmen z.B. dass $f'(x_0)=0$, $f''(x_0)>0$. Dann $\exists \varepsilon$ so dass
  \[f'(x)>0\s\forall x\in ]x_0-\varepsilon,x_0[\]
und
  \[f'(x)<0\s\forall x\in ]x_0,x_0+\varepsilon[\]
In der Tat,
  \[\lim_{x\to x_0}\frac{f'(x)-f'(x_0)}{x-x_0}=f''(x_0)\implies\lim_{x\to x_0}
\frac{f`(x)}{x-x_0} =f''(x_0)>0\]
  \[\implies\exists\varepsilon:\frac{f'(x)}{x-x_0}>0\s\forall x
\in]x_0-\varepsilon,x_0+\varepsilon[\setminus\left\{ x_0 \right\}\]
  \[\implies f'(x)>0\s\forall x\in ]x_0,x_0+\varepsilon[\]
  \[\implies f'(x)<0\s\forall x\in ]x_0-\varepsilon,x_0[ .\]
\end{Bem}
\begin{Lem}
  \[\eqref{e:kk}\iff f(\lambda x_1+(1-\lambda)x_2)\leq\lambda f(x_1)+(1-\lambda)f(x_2)\s\forall x_1 < x_2\s\forall \lambda\in ]0,1[\]
\end{Lem}
\begin{Bew}
  $x_1<x_2$
\begin{equation}\label{e:X3}
f(x)=\frac{x_2-x}{x_2-x_1}f(x_1)+\frac{x-x_1}{x_2-x_1}f(x_2)\s
\forall x\in ]x_1,x_2[
\end{equation}
Wir setzen $\lambda = \frac{x_2-x}{x_2-x_1}$
  \[\forall x\in ]x_1,x_2[\implies \lambda=\frac{x_2-x}{x_2-x_1}\in ]0,1[\]
  \[\forall \lambda\in ]0,1[\implies x=\lambda x_1+(1-\lambda)x_2\in ]x_1,x_2[\]
  \[\lambda=\frac{x_2-x}{x_2-x_1}\iff \lambda(x_2-x_1)=x_2-x\iff x=\lambda x_1+(1-\lambda)x_2\]
Wir schliessen dass die Abbildung
  \[ ]0,1[\ni\lambda\mapsto \lambda x_1+(1-\lambda x_2)\in  ]x_1,x_2[\]
bijektiv ist. Deswegen wir k\"onnen $\lambda$ statt $x$ in der Identit\"at \eqref{e:X3}
nutzen.
Aber 
\[\lambda=\frac{x_2-x}{x_2-x_1}\iff 1-\lambda=1-\frac{x_2-x}{x_2-x_1}
=\frac{\not x_2-x_1-\not x_2+x}{x_2-x_1}=\frac{x-x_1}{x_2-x_1}\]
Deswegen ist \eqref{e:X3} equivalent zu
  \[f(\lambda x_1+(1-\lambda)x_2)\leq \lambda f(x_1)+(1-\lambda)f(x_2)\]
\end{Bew}
\begin{Lem}
  $f:I\to \mb{R}$ ist genau dan konvex wenn für jedes Tripel $x_1<x<x_2\in I$ die folgende Ungleichung gilt:
  \[\frac{f(x)-f(x_1)}{x-x_1}\leq \frac{f(x_2)-f(x)}{x_2-x}\]
\end{Lem}
\begin{Bew}
  \[\frac{f(x)-f(x_1)}{x-x_1}\leq \frac{f(x_2)-f(x)}{x_2-x}\]
  \[\iff f(x)\left( \frac{1}{x-x_1}+\frac{1}{x_2-x} \right)\leq \frac{f(x_1)}{x-x_1}+\frac{f(x_2)}{x_2-x}\]
  \[\iff f(x)\frac{x_2-x+x-x_1}{(x-x_1)(x_2-x)}\left( \frac{(x_2-x)(x-x_1)}{x_2-x_1} \right)\]
  \[ \leq f(x_1)\frac{x_2-x}{x_2-x_1}+f(x_2)\frac{x-x_1}{x_2-x_1}\]
  \[\iff f(x)\leq f(x_1)\frac{x_2-x}{x_2-x_1}+f(x_2)\frac{x-x_1}{x_2-x_1}\]
\end{Bew}

\begin{proof}[Beweis vom Satz \ref{s:1011242}]
  {\bf Konvexität $\implies$ $f'$ ist wachsend.}
  \[f'(x)=\lim_{h\downarrow 0}\frac{f(x+h)-f(x)}{(x+h)-x}\]
  \[f'(y)=\lim_{h\downarrow 0}\frac{f(y+h)-f(y)}{(y+h)-y}\]
  $h$ klein $\implies$ $x<x+h<y<y+h$.
In diesem Fall impliziert Lemma \ref{l:1011293} die Ungleichungen:
\[ 
 \frac{f(x+h)-f(x)}{(x+h)-x}\leq \frac{f(y)-f(x+h)}{y-(x+h)}
\leq \frac{f(y+h)-f(y)}{(y+h)-y}
\]
Deswegen
\[
f' (x) = \lim_{h\downarrow 0}\frac{f(x+h)-f(x)}{(x+h)-x}\leq
\lim_{h\downarrow 0}\frac{f(y+h)-f(y)}{(y+h)-y} = f'(y)
\]
{\bf Konvexität $\Leftarrow$ $f'$ wachsend.} 
Sei $x_1<x<x_2$: Der Satz von Lagrange $\implies$ $\exists \xi_1\in]x_1,x[$ mit
  \[\frac{f(x)-f(x_1)}{x-x_1}=f'(\xi_1)\]
  $\exists \xi_2\in]x_1,x[$ mit
  \[\frac{f(x_2)-f(x)}{x_2-x}=f'(\xi_2)\]
  NB: $\xi_2>\xi_1$. Weil $f'(\xi_2)\geq f'(\xi_2)$, gilt das Lemma 
\ref{l:1011293} und Lemma $\implies$ Konvexität.

Der Beweis der zweiten Behauptung des Satzes ist analog.
\end{proof}
\subsection{Die Lagrange Fehlerabschätzung}
\begin{Def}
  Sei $f$ $n$-mal differenzierbar. Das Taylorpolynom mit Ordnung $n$ an der Stelle $x_0$ ist:
  \[T^n_{x_0}=\sum^n_{T=0}\frac{f^{(i)}(x_0)}{i!}(x-x_0)^i\]
\end{Def}
\begin{Sat}[Lagrange Fehlerabsch\"atzung]
  Sei $f$ $(n+1)$-mal differenzierbar in $I$ und $x_0\in$. $\forall x\in I$ 
$\exists\xi$ zwischen $x_0$ und $x$ so dass
  \begin{equation}\label{e:1011292}
    \underbrace{f(x)-T^n_{x_0}(x)}_{R_{x_0}^n (x)}=\frac{f^{(n+1)}(\xi)}{n+1!}(x-x_0)^{n+1}
  \end{equation}
\end{Sat}
\begin{Bem}
  Für $n=0$ \eqref{e:1011292} ist:
  \[f(x)-\underbrace{f(x_0)}_{T^0_{x_0}(x)}=f'(\xi)(x-x_0)\]
  \[\iff\frac{f(x)-f(x_0)}{x-x_0}=f'(\xi)\]
Deswegen die Lagrange Fehlerabsch\"atzung ist eine Verallgemeinerung
des Satzes von Lagrange.
\end{Bem}
\begin{Bew} Seien
\[
h (x) = R^n_{x_0} (x) \qquad \mbox{und} \qquad g(x) = (x-x_0)^{n+1}\, .
\]
Es ist leicht zu sehen dass
\[
g (x_0) = g' (x_0) = \ldots = g^{(n)} (x_0) = 0
\]
\[
h(x_0) = h'(x_0) = \ldots = h^{(n)} (x_0) = 0
\]
und 
\[
h^{(n+1)} (x) = f^{(n+1)} (x) \quad\forall x\, .
\]
Deswegen, wir wenden $n+1$ Mal den verallgemeinerten Mittelwertsatz (Satz von Cauchy)
und schliessen: 
  \[\frac{h(x)}{g(x)} = \frac{h(x)-h(x_0)}{g(x)-g(x_0)}
\stackrel{\text{Cauchy}}{=}\frac{h'(\xi_1)}{g'(\xi_1)}=
\frac{h'(\xi_1)-h'(x_0)}{g'(\xi_1)-g'(x_0)}\stackrel{\text{Cauchy}}{=}
\frac{h''(\xi_2)}{g''(\xi_2)}\]
\[=\cdots\stackrel{\text{Cauchy}}{=}
\frac{h^{(n+1)}(\xi_{n+1})}{g^{(n+1)}(\xi_{n+1})}=\frac{f^{(n+1)}(\xi_{n+1})}{(n+1)!}\, ,\]
wobei: $\xi_1$ eine Stelle zwischen $x$ und $x_0$ ist; $\xi_2$ eine Stelle zwischen $\xi_1$
und $x_0$ ist; \ldots $\xi_{n+1}$ eine Stelle zwischen $\xi_n$ und $x_0$ ist.

Wenn wir $\xi:=\xi_{n+1}$ setzen, dann
\[f(x)-T^n_{x_0}(x) = R^n_{x_0} (x) = h (x)=
\frac{f^{(n+1)} (\xi)}{(n+1)!} g (x) =
\frac{f^{(n+1)}(\xi)}{(n+1)!} (x-x_0)^{n+1}\, .\]
\end{Bew}
\begin{Bsp} Wir wissen schon dass
  \[e^x=\sum^\infty_{j=0}\frac{x^j}{j!}\qquad\forall x\in\mb{R}\, .\]
Diese Identit\"at kann man auch aus der Lagrange Fehlerabsch\"atzung
schliessen. Das Taylor Polynom mit Ordnung $n$ in $0$ ist 
\[T_0^n(x)=\sum^n_{j=0}\frac{x^j}{j!}\, .\]
Sei $x\in\mb{R}$ fixiert. 
  \[\Bigg|\overbrace{e^x-\sum_{j=0}^{n}\frac{x^j}{j!}}^{R^{n+1}_0 (x)}\Bigg|=
\left|\frac{e^{\xi_n}x^{n+1}}{(n+1)!}\right|,\]
wobei $\xi_n$ eine Stelle zwischen $x$ und $0$ ist. 
Deswegen, $\abs{\xi_n}\leq \abs{x}$ und
\begin{equation}\label{e:101129-1}
\left|e^x-\sum_{j=0}^{n}\frac{x^0}{j!}\right|\leq e^{|x|}\frac{\abs{x}^{n+1}}{(n+1)!}
\end{equation}
Wir wissen schon dass
\begin{equation}\label{e:101129-2}
\Limi{n}\frac{\abs{x}^{n+1}}{(n+1)!}=0\, .
\end{equation}
(In der Tat, sei $N$ so dass $N\geq 2\abs{x}$. Dann
 \[\left.\frac{\abs{x}^{n+1}}{(n+1)!} = \frac{\abs{x}^N}{N!}
\frac{\abs{x}}{N+1}\frac{\abs{x}}{N+2}\cdots\frac{\abs{x}}{n+1}
\leq \frac{\abs{x}^N}{N!}\left(\frac{1}{2} \right)^{n+1-N}\, .\right)\]
\eqref{e:101129-1} und \eqref{e:101129-2} implizieren dass
\[
0 \leq \left|f(x) - \sum_{j=0}^\infty \frac{x^j}{j!}\right|
= \left|f (x) - \lim_{n\to\infty} T^n_0 (x)\right| 
\]
\[
= \left|\lim_{n\to\infty} R^n_0 (x)\right|
= \lim_{n\to \infty} |R^n_0 (x)| \leq \lim_{n\to\infty}
 e^{|x|}\frac{\abs{x}^{n+1}}{(n+1)!} = 0\, .
\]
\end{Bsp}
\begin{Bsp} Sei
  \[f(x)=\ln(x+1)\]
  (Bem: das Taylorpolynom (bzw. die Taylorreihe) in 0 von $f$ ist 
das Taylorpolynom (bzw. die Taylorreihe) von $\ln x$ an der Stelle 1.)
Dann
  \[T^n_0(x)=\sum^n_{j=0}\frac{f^{(j)}(0)}{j!}x^j=
\sum^n_{j=1}\frac{(-1)^{j-1}}{j}x^j=x-\frac{x^2}{2}+\frac{x^3}{3}-\frac{x^4}{4}+\cdots
+ (-1)^{n-1} \frac{x^n}{n}\]
Wir wollen zeigen dass
\begin{equation}\label{e:logreihe}
\ln(x+1)= \sum_{j=1}^n (-1)^{n-1} \frac{x^j}{j}
\quad \left( = x-\frac{x^2}{2}+\frac{x^3}{3}-\frac{x^4}{4}+\cdots\right)
\end{equation}
in einer Umgebung von $0$.

Sei $x> -1$. Die Lagrange Fehlerabsch\"atzung impliziert: 
\[\big|\overbrace{f (x) -T_0^n(x)}^{R^n_0 (x)}\big|=
\frac{\abs{f^{(n+1)}(\xi_n)}}{(n+1)!}\abs{x}^{n+1}=
\frac{\frac{n!}{\left( 1+\xi_n \right)^{n+1}}}{(n+1)!}\abs{x}^{n+1}\]
wobei $\xi_n$ zwischen 0 und $x$ liegt. Deswegen $\xi_n> -1$
und $1+\xi_n \geq 1-|\xi_n|\geq 1-|\xi|$.
  % TODO unsichere Reihenfolge
Wir schliessen
\[|R^n_0 (x)|\leq \frac{1}{(n+1)}\frac{\abs{x}^{n+1}}{(1-\abs{x})^{n+1}}\, .\]
Aber $\abs{x}\leq \frac{1}{2}$ $\implies \frac{\abs{x}}{1-\abs{x}}\leq 1$
und
\[\Lim_{n\to\infty} \abs{R^n_0 (x)}\leq\lim_{n\to\infty} \frac{1}{n+1} = 0
\qquad \forall x\in \left[\frac{1}{2}, \frac{1}{2} \right]\]

Falls $x\in ]0,1]$, dann $\xi_n>0$ und
\[|R^n_0 (x)| \leq \frac{|x|^{n+1}}{n+1} \leq \frac{1}{n+1}\, .\]
Deswegen gilt die Gleichung \eqref{e:logreihe} auch f\"ur $x\in ]1/2, 1]$.
In der Tat, gilt diese Gleichung auch f\"ur $x\in ]-1, -1/2[$, aber diesen
Fall ist keine einfache Folgerung der Lagrange Fehlerabsc\"atzung.

Ausserdem, die Reihe $\sum_{j=1}^\infty\frac{x^j}{j}(-1)^j$ 
hat Konvergenzradius $R=1$. Deswegen ist die Gleichung \eqref{e:logreihe}
falsch wenn $x>1$.
\end{Bsp}
\section{Integralrechnung}
Sei $f: [x_0, x_1]= I \to \mb{R}$ eine stetig nichtnegative Funktion.
Das Ziel der Integralrechnung ist den Inhalt der folgenden Fl\"ache zu finden: 
\[G=\left\{ (x,y): x\in I \;\;\mbox{und}\;\; 0\leq y \leq f(x) \right\}\]
\subsection{Treppenfunktion}
\begin{Def}
  Eine $\phi:[a,b]\to\mb{R}$ heisst Treppenfunktion wenn $\exists a=x_0<x_1<\cdots<x_n=b$ so dass $\phi$ in jedem Intervall $]x_{k-1},x_k[$ konstant ist.
\end{Def}
\begin{Def} Sei $\phi$ eine Treppenfunktion und $x_0<x_1< \ldots < x_n$ wie oben.
Falls $c_k$ der Wert von $\phi$ in $]x_{k-1}, x_k[$ ist, dann
\[\int^b_af(x)\md x:=\sum^n_{k=1} (x_k-x_{k-1})c_k\, .\]
\end{Def}
\begin{Bem} Es ist leicht zu sehen dass
die Zahl $\int^b_af(x)\md x$ unabhängig von der Verteilung ist.
\end{Bem}

\begin{Lem}
  Für Treppenfunktionen $\phi, \psi$ und Zahlen $\alpha, \beta\in\mb{R}$ gilt:
  \begin{enumerate}
    \item
      \[\int^b_a(\alpha\phi, \beta\psi)\md x=\alpha\int_a^b\psi\md x+\beta\int^b_a\psi\md x\]
    \item
      \[\abs{\int_a^b\psi\md x}\leq \int_a^b\abs{\psi}\md x\leq (b-a)\max \phi(x)\s x\in[a,b]\]
    \item Falls $\phi\leq\psi$ (d.h. $\phi(x)\leq \psi(x)\forall x\in [a,b]$), dann
      \[\int_a^b\phi\md x\leq\int_a^b\psi\md x\]
  \end{enumerate}
\end{Lem}
\begin{Bew}
  \begin{enumerate}
    \item $\exists 0\leq x_0<x_1<\cdots<x_n=b$ so dass $\phi|_{]x_k, x_{k+1}}\equiv$konst, \item $\exists 0\leq y_0<y_1<\cdots<y_n=b$ so dass $\phi|_{]y_k, y_{k+1}}\equiv$konst
      \[\left\{ x_0,\cdots,x_n,y_0,\cdots y_m \right\}=\left\{ z_0<z_1<\cdots<z_N \right\}\]
      Dann: $\forall k:$
      \[\phi|_{]z_{k-1},z_k[}\equiv c_k\in\mb{R}\s k\geq 1\]
      \[\psi|_{]z_{k-1},z_k[}\equiv d_k\in\mb{R}\s\]
      F: $\alpha\phi+\beta\psi$
      \[F|_{]z_{k-1},z_n[}=\alpha c_k\beta d_k\]
      \[\int_a^b\phi=\sum_{k=1}^N(z_k-z_{k-1})c_k\]
      \[\int_a^b\psi=\sum_{k=1}^N(z_k-z_{k-1})d_k\]
      \[\int_a^bF=\sum^N_{k=1}(z_k-z_{k-1})(\alpha c_k+\beta d_k)\]
      \[=\alpha\sum_{k=1}^N(z_k-z_{k-1})c_k+\beta\sum_{k=1}^N(z_k-z_{k-1})c_k\]
      \[=\alpha\int_a^b\phi+\beta\int_a^b\psi\]
    \item
      Seien $a=x_0<x_1<\cdots<x_n=b$ mit
      \[\phi|_{]x_{k-1},x_k}=c_k\in\mb{R}\]
      \[\abs{\phi}|_{]x_{k-1},x_k}=\abs{c_k}\in\mb{R}\]
      \begin{equation}\label{e:101201A}
        \abs{\int_a^b\phi}=\abs{\sum^n_{k=1}(x_k-x_{k-1})c_n}
      \end{equation}
      \begin{equation}\label{e:101201B}
        \int_a^b\abs{\phi}=\sum^n_{k=1}(x_k-x_{k-1})\abs{c_n}=\sum^n_{k=1}\abs{(x_k-x_{k-1})c_k}
      \end{equation}
      $\ref{e:101201A}\leq\ref{e:101201B}$ wegen Dreiecksungleichung
    \item
      Die beiden oberen Prinzipien $\implies$ Beweis
  \end{enumerate}
\end{Bew}
\subsection{Regelfunktion}
\begin{Def}\label{d:1012011}
  Eine Abbildung $f:[a,b]\to\mb{R}$ heisst Regelfunktion falls $\exists f_x:[a,b]\to\mb{R}$ (Folge von Funktionen), so dass:
  \begin{itemize}
    \item Jede $f_n$ eine Treppenfunktion ist
    \item
      \[\Limi{k}\underbrace{\left( \sup_{x\in[a,b]}\abs{f_k-f} \right)}_{:=\Norm{f_k-f}}=0\]
  \end{itemize}
\end{Def}
\begin{Sat}
  Sei $f$ eine Regelfunktion. Seien $\left\{ f_n \right\}$ und $\left\{ g_n \right\}$ zwei Folgen von Treppenfunktionen, welche die zweite Bedinung in \ref{d:1012011} erfüllen. Dann:
  \[\Limi{k}\int_a^bf_k=\Limi{k}\int_a^bg_k\s(\in\mb{R})\]
\end{Sat}
\begin{Def}
  Sei $f$ eine Regelfunktion und $\left\{ f_k \right\}$ eine Folge $f_k:[a,b]\to\mb{R}$. Dann
  \[\int_a^nf(x)\md x=\Limi{k}\int_a^bf_k(x)\md x\]
\end{Def}
\begin{Bem}
  \[ \begin{cases}
    \int_a^bf\geq 0& f\geq 0\\
    \int_a^bf\leq 0& f\leq 0
  \end{cases}\]
\end{Bem}
\begin{Bew}
  $\forall \varepsilon, \exists N$ so dass:
  \[\abs{\int_a^bf_k-\int_a^bf_i}=\abs{\int_a^b(f_k-f_i)}\leq\]
  \[\sup_{x\in[a,b]}\abs{f_k-f_i}(x)\leq \sup_{x\in[a,b]}\left\{ \abs{f_n-f}(x)+\abs{f-f_i}(x) \right\}\]
  \[\leq \sup_{x\in[a,b]}\abs{f_k-f}(x)+\sup_{x\in[a,b]}\abs{f-f_i}(x)\]
  \[=\Norm{f_k-f}+\Norm{f-f_i}<2\varepsilon\s\text{wenn}\s k,j\geq N\]
  $\implies (a_k)=\left( \int_a^bf_k \right)$ ist eine Cauchyfolge (d.h. $\forall \varepsilon\exists N$ mit $\abs{a_j-a_k}<2\varepsilon$ $\forall k,j\geq N$) $\implies$ $\exists \Limi{k}\int_a^bf_k\in\mb{R}$
  \[\Norm{f+g}\leq \Norm{f}+\Norm{g}\]
  \[\abs{\sup_xf(x)+g(x)}\leq \sup_x\abs{f(x)}+\abs{g(x)}\leq\sup_x\abs{f(x)}+\sup_x\abs{g(x)}\]
  \[\abs{\Limi{k}\int_a^bf_k-\Limi{k}\int_a^bg_k}\]
  \[=\abs{\Limi{k}\left( \int_a^bf_k-\int_a^bg_k \right)}\]
  \[=\Limi{k}\int_a^b(f_k-g_k)\]
  \[\leq \Limi{k}\abs{\int_a^b(f_k-g_k)}\]
  \[\leq\Limi{k}\Norm{f_n-g_k}\]
  \[\leq\Limi{k}\left\{ \underbrace{\Norm{f_k-f}}_{\to 0}+\underbrace{\Norm{g_k-f}}_{\to 0} \right\}=0\]
  \[\implies \Limi{k}\int_a^bf_k=\Limi{k}\int_a^bg_k\]
\end{Bew}
\begin{Sat}
  Eine stetige Funktion $f:[a,b]\to\mb{R}$ ist eine Regelfunktion.
\end{Sat}
\begin{Bew}
  Sei $k\in\mb{N}\setminus\left\{ 0 \right\}$, $f$ stetig, $[a,b]$ kompakt. $f$ ist gleichmässig stetig. $\varepsilon=\frac{1}{k}$ in der Definition der gleichmässigen Stetigkeit.
  \[\implies\exists\delta>0\s\abs{x-y}<\delta\implies\abs{f(x)-f(y)}<\frac{1}{k}\]
  Sei $a=x_0$, $a+\delta=x_1$, $a+2\delta=x_2$, \ldots, $a+N\delta=x_n<b$, $a+(N+1)\delta\geq b$, $N=\max\left\{ k,a+k\delta<b \right\}$ $\forall j\in\left\{ 1,\cdots,N+1 \right\}$. Sei $y_j=\frac{x_{j-1}+x_j}{2}$ (der Mittelpunkt von $I=[x_{j-1},x_j]$).
  \[ f_k(x)= \begin{cases}
    f_k(x)=f(y_j)& x\in [x_{j-1},x_j[\\
    f_k(x)=f(y_{N+1})&x=b
  \end{cases}\]
  \[\Norm{f-f_k}=\sup_{x\in I}\abs{f_k(x)-f(x)}<\frac{1}{k}\]
  $x\in [x_{j-1},x_j[$ oder $x\in [x_N,x_{N+1}]$, $\abs{x-y_j}\leq \frac{\delta}{2}$ oder $\abs{x-y_{N+1}}\leq \frac{\delta}{2}$.
  \[\implies\]
  \[\abs{f(x)-f_k(x)}=\abs{f(x)-f(y_j)}<\frac{1}{k}\]
  \[\text{oder}\s\abs{f(x)-f_k(x)}=\abs{f(x)-f(y_{N+1})}<\frac{1}{k}\]
  $\forall k$ ist $f_k$ eine Treppenfunktion $\Norm{f_k-f}\to 0$ für $k\to+\infty$
\end{Bew}
\begin{Bem}
  \[\int_a^bf_k=\sum_{j=1}^{N+1}(x_j-x_{j-1})f(y_i)\]
\end{Bem}
\begin{Kor}
  Eine ``stückweise stetige'' Funktion auf $[a,b]$ ist auch eine Regelfunktion. D.h. $\exists a=x_0<x_1<\cdots<x_n=b$ mit
  \begin{itemize}
    \item $f$ ist stetig überall auf $]x_{j-1}, x_j[$
    \item $\forall j\geq 1, j\leq n$, $\forall j\in \left\{ 0,\cdots,n \right\}$
      \[\lim_{x\downarrow x_j}f(x)\in\mb{R}\]
      \[\lim_{x\uparrow x_j}f(x)\in\mb{R}\]
  \end{itemize}
\end{Kor}
\begin{theorem}
  Seien $f,g:[a,b]\to\mb{R}$ Regelfunktionen und $\alpha,\beta\in\mb{R}$
  \begin{enumerate}
    \item[Linearität]
      \[\int_a^b(\alpha + \beta g)=\alpha\int_a^bf+\beta\int_a^bg\]
    \item[Abschätzung]
      \[\abs{\int_a^bf}\leq\abs{b-a}\Norm{f}\]
    \item[Monotonie]
      \[\int_a^bf\leq\int_a^bg\s\text{falls}\s f\leq g\]
    \item
      \[\int_a^bf=\int_a^cf+\int_c^bf\s\forall x\in ]a,b[\]
    \item[Mittelwertsatz]
      falls $f$ stetig $\implies$ $\exists\xi ]a,b[$ so dass
      \[\int_a^bf=f(\xi)(b-a)\]
  \end{enumerate}
\end{theorem}
\begin{Bew}
  \begin{enumerate}
    \item $f_k, g_k$ Treppenfunktion mit $\Norm{f_k} \to 0$, $\Norm{g-g_k}\to 0$. $\alpha f_k+\beta g_k$ ist auch eine Treppenfunktion.
      \[\Norm{(\alpha f+\beta g)-(\alpha f_k-\beta g_k)}\leq\abs{\alpha}\Norm{f-f_k}+\abs{\beta}\Norm{g-g_k}\to 0\]
      \[\int_a^b(\alpha f+\beta g)=\Limi{k}\int_a^b(\alpha f_k+\beta g_k)=\Limi{k}(\alpha \int_a^bf_k+\beta\int_a^b g_k)\]
      \[=\Limi{k}\alpha\int_a^b f_k+\Limi{k}\beta\int_a^b g_n=\alpha\Limi{k}\int_a^b f_k+\beta\Limi{k}\int_a^bg_k\]
      \[=\alpha\int_a^b f+\beta^a_b g\]
    \item sehe oben
    \item sehe oben
    \item $\tilde{f_k}=f_k+\Norm{f-f_k}$ $\implies$ $\tilde{f_k}\geq f$, $\tilde{g_k}=g_k+\Norm{g-g_k}$, $\implies$ $g\geq\tilde{g_k}$ $\implies$ $\tilde{f_k}\geq f \geq g \geq \tilde{g_k}$
    \[\int_a^bf=\Limi{k}\int_a^b\tilde{f_k}\geq\Limi{k}\int_a^b\tilde{g_k}=\int_a^bg\]
     \item
       \[(b-a)\min f\leq \underbrace{\int_a^b f}_{(b-a)}\leq(b-a)\max f\]
       Zwischenwertsatz $\implies$ $\exists\xi\in ]a,b[$ mit $f(\xi)=\frac{\int_a^bf}{b-a}$
  \end{enumerate}
\end{Bew}


\newpage

%= Stichwortverzeichnis ======================================================================
\rhead{}
\addcontentsline{toc}{section}{Stichwortverzeichnis}
\printindex

\end{document}
