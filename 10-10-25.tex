\section{Stetige Funktionen und Grenzwerte}
\subsection{Stetigkeit}
In diesem Kapitel $D$ ist immer eine Teilmenge von $\mb{R}$ oder 
von $\mb{C}$. $D\subset\mb{R}$.
\begin{Def}
Seien $f:D\mapsto \mb{R}(\mb{C})$ und $x_0\in D$. $f$ heisst stetig in $x_0$ falls $\forall \varepsilon >0$, $\exists\delta>0$ mit
\begin{equation}\label{e:S}
\abs{x-x_0}<\delta, x\in D\implies \abs{f(x)-f(x_0)}<\varepsilon
 \end{equation}
\end{Def}
Deswegen, an einer ``Unstetigkeitstelle'' $x_0$ von $f$ gibt es einer
$\varepsilon > 0$ die die folgende Bedingung erf\"ullt:
\[\forall\delta>0\quad 
\exists x\in\left] x_0-\delta,x_0+\delta  \right[\text{ mit } 
\abs{f(x)-f(x_0)}\geq \varepsilon\]
\begin{Bsp}
  Die Polynome sind stetige Funktionen, weil
 Summe und Produkte stetiger Funktionen sind auch stetig (siehe 
Satz \ref{s:rech_stet}).
\end{Bsp}
\begin{Bem}
  \begin{itemize}
    \item Die Bedingung \eqref{e:S} ist trivial für die Funktion $f=const$
    \item Die Bedingung \eqref{e:S} ist trivial für die Funktion $f(x)=x$
      \[\abs{x-x_0}<\delta=\varepsilon\implies\abs{f(x)-f(x_0)}=\abs{x-x_0}<\varepsilon\]
  \end{itemize}
\end{Bem}
\begin{Def}
  Eine Funktion $f:D\to \mb{R}(\mb{C})$ heisst Lipschitz(-stetig) falls $\exists L\geq 0$ mit
  \begin{equation}\label{e:L}
\abs{f(x)-f(y)}\leq L\abs{x-y}, \forall x,y\in D
\end{equation}
\eqref{e:L} $\implies$ \eqref{e:S}: wähle $\delta=\frac{\varepsilon}{L}$
\end{Def}
\begin{Kor}
  $g(x):=\abs{x}$ ist stetig.
  \[\abs{g(x)-g(y)}=\abs{\abs{x}-\abs{y}}\leq\abs{x-y}\]
  d.h. \eqref{e:L} mit $L=1$
\end{Kor}
\begin{Bsp}
$\frac{f}{g}$ ist stetig falls $f$,$g$ stetig und $g(x)\neq 0$ $\forall x\in D$
(siehe Satz \ref{s:rech_stet}). $\implies$ Rationale Funktionen $\frac{P(x)}{Q(x)}$ sind stetig auf $d=\mb{C}\setminus\left\{ x:Q(x)=0 \right\}$
\end{Bsp}
\begin{Bsp}
  $f(x)=x^k$, $k\in\mb{N}$ ist ein Polynom $\implies$ $f$ ist stetig. 
Sei $g(x):=x^\frac{1}{k}=\sqrt[k]{x}$, 
$k\in\mb{N}\setminus\left\{ 0 \right\}$ ( $g(x)$ ist die einzige relle Zahl $y\in\mb{R}$ mit $y\geq 0$ und $y^k=x$ ). 
$x_0\in\mb{R}$, $x_0\geq 0$. Wir behaupten dass $f$ stetig in $x_0$ ist.
In der Tat:
  \[\abs{\underbrace{\sqrt[k]{x}}_y-\underbrace{\sqrt[k]{x_0}}_{y_0}} \leq \sqrt[k]{\abs{x-x_0}}\]
  \[\iff \abs{y-y_0}^k\leq\abs{y^k-y_0^k}\]
  oBdA $y\geq y_0$
  \[{\underbrace{\left( y-y_0 \right)}_a}^k\leq {\underbrace{y}_c}^k-
{\underbrace{y_0}_b}^k\]
  \[\iff a^k+b^k\leq c^k=(a+b)^k\]
  \[a^k+b^k\leq (a+b)^k=a^k+\overbrace{\binom{k}{1}a^{k-1}b+\cdots}^{\geq 0}+b^k\]
  Sei nun $\varepsilon$ eine gegebene positive Zahl
und w\"ahle $\delta=\varepsilon^{k}$. 
\[\abs{x-x_0}, x\geq 0 \implies
\abs{\sqrt[k]{x}-\sqrt[k]{x_0}}\leq\sqrt[k]{\abs{x-x_0}}
< \sqrt[k]{\varepsilon^k}=\varepsilon\]
\end{Bsp}
\begin{Bsp}
  Sei $a>0$ und $f(x)=a^x$ $\forall x\in\mb{Q}$ $f:\mb{Q}\to\mb{R}$ ist stetig!
Das wird sp\"ater bewiesen, wenn wir die Exponentialfunktion auf
der ganzen komplexen Ebene definieren.
\end{Bsp}
\begin{Sat}
  Sei $f:D\to\mb{R}(\mb{C})$. Sei $x_0\in D$. Diese zwei Aussagen sind equivalent:
  \begin{itemize}
    \item $f$ ist stetig an der Stelle $x_0$.
    \item $\forall(x_n)\subset D$ mit $x_n\to x_0$ haben wir $f(x_n)\to f(x_0)$
  \end{itemize}
\end{Sat}
\begin{proof}[Beweis]
  Sei $\varepsilon >0$. $f$ stetig in $x_0$ $\implies$ $\exists\delta>0$ mit $\abs{f(x)-f(x_0)}<\varepsilon$ falls $\abs{x-x_0}<\delta$. $x_n\to x_0$ $\implies$ $\exists N$:
  \[\abs{x_n-x_0}<\delta\;\;\forall n\geq N\implies \abs{f(x_n)-f(x_0)}<\varepsilon\]
  Andere Richtung: Nehmen wir an dass $f$ nicht stetig ist.
  \[\implies \exists \varepsilon>0:\; \forall \delta>0 
\quad \exists x: \abs{x-x_0}<\delta\;\mbox{ und }\abs{f(x)-f(x_0)}\geq \varepsilon\]
Sei $n\in \mb{N}\setminus \{0\}$. Wir setzen $\delta=\frac{1}{n}$ und w\"ahle
$x_n$ mit $\abs{x_n-x_0}<\frac{1}{n}$ und 
$\abs{f(x_n)-f(x_0)}\geq \varepsilon$. Dann $x_n\to x_0$ und $f(x_n)\not\to f(x_0)$.
\end{proof}

\subsection{Rechenregeln f\"ur stetige Funktionen}

\begin{Sat}\label{s:rech_stet}
  Seien $f,g:D\to\mb{R}(\mb{C})$ zwei stetige Funktionen in $x_0$. Dann:
  \begin{itemize}
    \item $f+g$, $fg$ sind stetig in $x_0$
    \item $\frac{f}{g}$ ist stetig in $x_0$ falls $f(x_0)\neq 0$.
  \end{itemize}
\end{Sat}
\begin{proof}[Beweis]
  Sei $x_0\in D$, $(x_n)\subset D$ $x_n\to x_0$ (für $\frac{f}{g}$ 
$g(x_n)\neq 0, g(x_0)\neq 0$ weil $(x_n), {x_0}\subset D\setminus\left\{ x:g (x)=0 \right\})$
  \begin{align*}
    f(x_n)+g(x_n)&\to& f(x_0)+g(x_0)\\
    f(x_n)g(x_n)&\to& f(x_0)g(x_0)\\
    \frac{f(x_n)}{g(x_n)}&\to& \frac{f(x_0)}{g(x_0)}\\
  \end{align*}
\end{proof}
\begin{Sat}
  $f:D\to A$, $g:A\to B$. $f$ stetig $x_0$ und $g$ stetig
auf $f(x_0)=y_0$ $\implies g\circ f:D\to B$ stetig auf $x_0$.
\end{Sat}
\begin{proof}[Beweis]
  $x_0, (x_n)\subset D$ mit $x_n\to x_0$ 
$\implies \underbrace{f(x_n)}_{y_n}\to \underbrace{f(x_0)}_{y_0}$ $(y_n), y_0\in A$ $\implies$ 
  \begin{itemize}
    \item $g(y_n)\to g(y_0)$
    \item $g(f(x_n))\to g(f(x_0))$
  \end{itemize}
  $\implies$ $g\circ f(x_n)\to g\circ f(x_0)$ $\implies$ Stetigkeit von $g\circ f$
\end{proof}

\begin{Def}
Eine Funktion $f:D \to \mb{R} (\mb{C})$ heisst stetig falls
$f$ stetig an jeder Stelle $x_0\in D$ ist.
\end{Def}


\begin{Sat}
  Sei $f:\left[ a,b \right]\to\mb{R}(\mb{C})$ injektiv. Sei
  \[B:=f\left( \left[ a,b \right] \right)\left( =\left\{ z:\exists x\in\left[ a,b \right]\text{ mit } f(x)=z \right\} \right)\]
  \begin{Bem}
    $f:\left[ a,b \right]\to B$ ist bijektiv und deswegen umkehrbar.
  \end{Bem}
  Sei $f^{-1}:B\to\left[ a,b \right]$ die Umkehrfunktion. Dann ist $f^{-1}B\to \left[ a,b \right]$ stetig, falls $f$ stetig ist.
\end{Sat}
\begin{proof}[Beweis]
  Sei $x_0\in B$, $(x_n)$ mit $(x_n)\subset B$ und $x_n\to x_0$. Die Folge
  \[\underbrace{f^{-1}(x_n)}_{=y_n}\stackrel{?}{\to}\underbrace{f^{-1}(x_0)}_{=y_0}\]
  $(y_n)\subset\left[ a,b \right]$, $y_0\in\left[ a,b \right]$. Falls $y_n\not\to y_0$, dann:
  \[\exists\varepsilon>0: \forall N\in\mb{N}\quad 
\exists \underbrace{n}_{n_k}\geq \underbrace{N}_{k}: \abs{y_n-y_0}\geq\varepsilon\]
  \[n_k\geq n_{k-1}\implies\text{Teilfolge } (y_{n_k}): \abs{y_{n_k}-y_0}\geq 
\varepsilon\quad \forall k\in \mb{N}\]
  \[\text{Bolzano-Weiterstrass}\implies \exists \mbox{ Teilefolge }y_{n_{k_j}}\to\bar{y}\implies \bar{y}\neq y_0\]
  \[f(y_{n_{k_j}})=x_{n_{k_j}}\]
Die Stetigkeit von $f$ implies $f(y_{n_{k_j}})\to f(\bar y)$. 
Und da $x_{n_{k_j}}\to x_0$ sowie  $x_{n_{k_j}} = f(y_{n_{k_j}}$, heisst 
dass das $f(\bar y)=x_0$. Aber $f(y_0)=x_0$ und deswegen $ f(\bar y)=f(y_0)$,
mit $\bar y\neq y_0$. Widerspruch mit der Injektivität von $f$. 
Deswegen $f^{-1}(x_n)=y_n\to y_0=f^{-1}(x_0)$ $\implies$ $f^{-1}$ ist stetig.
\end{proof}
\begin{Bem}
  Aus diesem Satz schliessen Sie die Stetigkeit von $x\mapsto x^{\frac{1}{k}}$ von der Stetigkeit $x\mapsto x^k$.
\end{Bem}
\begin{Bem}
  Für Satz 1 genügt die Stetigkeit der beiden Funktionen ander der Stelle $x_0$. Für Satz 2 ähnlich. Für Satz 3 ist die Stetigkeit auf dem ganzen $D$ wichtig.
\end{Bem}
