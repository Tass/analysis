\begin{Def}{(Absolute Integrierbarkeit)}
Sei $I$ wie oben. Eine Abbildung $f:I\to\mb{R}$  ist absolut integrierbar falls
  \begin{itemize}
    \item $f|_{[\alpha,\beta]}$ eine Regelfunktion ist $\forall a<\alpha<\beta<b$
    \item 
      \[\int_a^b\abs{f}\infty =\lim_{\alpha\downarrow a, \beta\uparrow b} \int_\alpha^\beta\abs{f}
<\infty\]
  \end{itemize}
\end{Def}
\begin{Bem}
  $f$ Regelfunktion auf $[\alpha,\beta]$ $\implies$ $\abs{f}$ Regelfunktion auf $[\alpha,\beta]$. $f$ Regelfunktion: $\forall \varepsilon>0$ $\exists g$ Treppenfunktion mit $\Norm{f-g}<\varepsilon$. $\abs{g}$ ist eine Treppenfunktion $\Norm{\abs{f}-\abs{g}}<\varepsilon$.
  \[\left( \abs{\abs{f}(x)-\abs{g}(x)}\leq\abs{f(x)-g(x)}\implies \Norm{\abs{f}-\abs{g}}\leq\Norm{f-g} \right)\]
\end{Bem}
\begin{Sat}
  Absolute Integrierbarkeit $\implies$ $\int_a^b f$ existiert.
\end{Sat}
\begin{Bew}
  $\int_a^b\abs{f}$ existiert $\implies$ $\forall x_0\in I$
  \begin{equation}\label{e:1012081}
    \Limd{\alpha}{a}\int_\alpha^{x_0}\abs{f}\in\mb{R}\implies\Limd{\alpha}{a}
\int_\alpha^{x_0}f\s\text{existiert}
  \end{equation}
  \begin{equation}\label{e:1012082}
    \Limu{\beta}{b}\int_\beta^{x_0}\abs{f}\in\mb{R}\implies\Limu{\beta}{b}\int_\beta^{x_0}f\s\text{existiert}
  \end{equation}

\medskip

{\bf Beweis von \eqref{e:1012081}} Sei $F(\alpha):=\int_\alpha^{x_0}\abs{f}$. Die
Existenz von $\Limd{\alpha}{a} F(\alpha)$ $\implies$ die Cauchy Eigenschaft:
  \[\forall \varepsilon>0\s\exists\delta\s\text{so dass, wenn}\s\tilde a,\bar a\in]a,a+\delta[\s\text{dann}\s\abs{F(\tilde a)-F(\bar a)} < \varepsilon\]
Aber
  \[|F(\tilde a) - F(\bar a)| = \left|\int_{\tilde{a}}^{x_0}\abs{f}-\int_{\bar{a}}^{x_0} \abs{f}\right|
=\int_{\tilde{a}}^{\bar{a}}\abs{f}<\varepsilon\]
Sei nun $G(\alpha):=\int_\alpha^{x_0}f$. F\"ur  $a<\tilde{a}\leq \bar{a}<a+\delta$:
\[
|G(\tilde a)- G (\bar a)|= \left|\int_{\tilde a}^{\bar a} f\right|
\leq \int_{\tilde a}^{\bar a} |f| <\eps\, .
\]
\[\implies \text{$G$ erf\"ullt die Cauchy Bedingung}
\implies \Limd{\alpha}{a}G(\alpha)\in\mb{R}\]

\medskip

\eqref{e:1012082} folgt aus der gleichen Idee.
\end{Bew}

\begin{Bsp} Es gibt $f$ mit $\int_a^bf < +\infty$ die aber nicht absolut integrierbar sind.
 
\[f(x)=(-1)^nn
\qquad
\mbox{für } x\in\left( \frac{1}{n+1},\frac{1}{n} \right] \qquad (n\in\mb{N}\setminus\left\{ 0 \right\})
\]
$f:]0,1]\to\mb{R}$ und $\forall \alpha>0$ ist $f|_{[\alpha,1]}$ offenbar 
eine Regelfunktion ($f|_{[\alpha, 1]}$ ist in der Tat eine Treppenfunktion!). 
Sei nun $\alpha\in\left] \frac{1}{N+2},\frac{1}{N+1} \right]$. Dann gilt:
  \[\int_\alpha^1f(x)\md x=\sum^N_{n=1}\left( \frac{1}{n}-\frac{1}{n+1} \right)n(-1)^n+(N+1)\left( \frac{1}{N+1}-\alpha \right)(-1)^{N+1}\]
  Beachte, dass:
  \begin{itemize}
    \item 
      \[\sum^N_{n=1}\left( \frac{1}{n}-\frac{1}{n+1} \right)n(-1)^n=\sum^N_{n=1}\frac{1}{n+1}(-1)^n\]
    \item $\sum^\infty_{n=1}\frac{(-1)^n}{n+1}$ konvergiert
    \item 
      \[0\leq (N+1)\left( \frac{1}{N+1}-\alpha \right)\leq(N+1)\left( \frac{1}{N+1}-\frac{1}{N+2} \right)=\frac{1}{N+2}\stackrel{N\to\infty}{\rightarrow}0\]
  \end{itemize}
  Somit existiert $\Limd{\alpha}{0}\int_\alpha^1f$. ABER:
  \[\int_\alpha^1\abs{f(x)}\md x\geq \sum^N_{n=1}\left( \frac{1}{n}-\frac{1}{n+1} \right)n=\sum_{n=1}^N\frac{1}{n+1}\]
  und $\sum\frac{1}{n+1}$ divergiert! (Harmonische Reihe). Also ist $f$ integrierbar aber nicht absolut integrierbar.
\end{Bsp}
\begin{Kor}{(Majorantenkriterium)}\label{k:1012085}
  Sei $f:I\to\mb{R}$ so dass
  \begin{itemize}
    \item $f|_{[\alpha,\beta]}$ ist eine Regelfunktion $\forall \alpha<\beta\in I$
    \item $\abs{f}\leq g$ und $g$ ist integrierbar auf $I$. Dann ist $f$ auch absolut integrierbar
  \end{itemize}
\end{Kor}
\begin{Bem}\ref{k:1012085} ist sehr n\"utzlich, um die Integrierbarkeit einer Funktion zu beweisen.
\end{Bem}
\begin{Bsp}
  \[\int_{-\infty}^\infty e^{-x^2}\md x\in\mb{R}\]
  Tats\"achlich ist auf $[1,+\infty[$: $e^{-x^2}\leq xe^{-x^2}$ und
  \[\int_1^{-\infty}xe^{-x^2}\md x\]
  \[=\Limi{R}\int_1^Rxe^{-x^2}\md x\]
  \[=\Limi{R}\left[ -\frac{1}{2}e^{-x^2} \right]_{x=1}^R\]
  \[=\Limi{R}\left( \frac{1}{2}e^{-1}-\frac{1}{2}\underbrace{e^{-R}}_{\to 0} \right)\]
  \[=\frac{1}{2e}\]
  Analog benutzt man nun $e^{-x^2}\leq -xe^{-x^2}$ für $x\in]-\infty,-1]$.
\end{Bsp}
\begin{Bem} Korollar \ref{k:1012085} kann auch benutzt werden, um die Konvergenz von Reihen zu beweisen.
\end{Bem}
\begin{Kor}[``Integralkriterium'' f\"ur Reihen]\label{k:1012088} 
Sei $\sum^\infty_{n=0} a_n$ eine Reihe. Definiere $f:[0,+\infty]$ durch $f(x)=a_n$, falls $x\in[n,n+1[$ $(n\in \mb{N})$. Dann gilt:
  \[\int_0^{+\infty}f \mbox{ existiert} \iff \sum_{n=0}^\infty a_n \mbox{ konvergiert}\]
  und so:
  \[f \mbox{ ist absolut integrierbar} \iff \sum_{n=0}^\infty a_n\s\text{konvergiert absolut}\]
\end{Kor}
\begin{Bew}
  ``$\Leftarrow$''
  \[\int_0^Rf=\int_0^Nf+\int_N^Rf\s(N=\Floor{R})\]
  und
  \[\int_0^N f=\sum_{n=0}^{N-1}a_n\rightarrow\sum^\infty_{n=0}a_n<\infty\]
  Ausserdem, da $\sum a_n$ konvergiert, ist $a_n$ eine Nullfolge. Und so
  \[\left|\int_N^Rf\right|=\left|(R-N)a_N\right|\leq\abs{a_N}\rightarrow 0\s\text{für}\s R\to +\infty\]
  ``$\implies$'' $\int_0^{N+1}f=\sum^\infty_{n=0}a_n$, und da $\Limi{N}\int_0^{N+1}f$ existiert, folgern wir die Konvergenz von $\sum a_n$.
\end{Bew}
\begin{Bsp}
  \[\sum_{n\geq 2}\frac{1}{n(\ln n)^2}<\infty\]
  Wir werden Korollar \ref{k:1012088} zwischen 3 und $\infty$ statt $0$ und $\infty$ anwenden. Seien $f(x)=\frac{1}{n(\ln n)^2}$ falls $x[n,n+1[$ $(n\geq 3)$ und $g(x)=\frac{1}{x(\ln x)^2}$. Beachte: $\ln x>0$
$\forall x>1$. Somit, falls $x\in[n,n+1[$ mit $n\geq 3$ gilt $n>x-1$,
  \[\ln n>\ln(x-1)\geq \ln(n-1)\geq\ln 2>0\]
  \[\implies f(x)=\frac{1}{n(\ln n)^2}<\frac{1}{(x-1)(\ln(x-1))^2}=g(x-1)\]
  Aber:
  \[\int_3^{+\infty}g(x-1)\md x=\int_2^{+\infty}g(x)\md x\]
  \[=\int_2^{+\infty}\frac{1}{x(\ln x)^2}\md x=\Limi{R}\int_2^R\frac{1}{x(\ln x)^2}\md x\]
  \[=\Limi{R}\left[ -\frac{1}{\ln x} \right]^R_{x=2}=\frac{1}{\ln 2}-\underbrace{\Limi{R}\frac{1}{\ln R}}_{=0}\]
  \[=\frac{1}{\ln 2}< +\infty\]
  und so:
  \[\sum_{n\geq 2}\frac{1}{n(\ln n)^2}=\frac{1}{2(\ln 2)^2}+\sum_{n\geq 3}\frac{1}{n(\ln n)^2}\]
  \[\leq\frac{1}{2(\ln 2)^2}+\frac{1}{\ln 2}< +\infty\]
\end{Bsp}
\subsection{Integration einer Potenzreihe}
  
Zur Erinnerung:

\begin{Sat}\label{s:1012080}
 Ist $\sum f_n=\sum a_nx^n$ eine Potenzreihe mit Konvergenzradius $R$, so konvergiert $\sum f_n$ normal auf jedem Intevall $[-\rho, \rho]$ mit $0<\rho<R$
\end{Sat}

Zun\"achst beweisen wir den folgenden Satz. Als Korollar erhalten wir eine
entsprechende Potenzreihendarstellung f\"ur das Integral einer Potenzreihe.

\begin{Sat}\label{s:1012081}
  Ist $f=\sum f_n$ eine Reihe von Regelfunktionen auf $[a,b]$, welche auf $[a,b]$ normal konvergiert, so ist $f$ selber eine Regelfunktion auf $[a,b]$ und es gilt:
\begin{equation}\label{e:gliederweise}
\int_a^bf=\sum\int_a^bf_n
 \end{equation}

\end{Sat}
\begin{Bew}
  Sei $\varepsilon>0$. W\"ahle $N$ so dass
  \[\sum^{+\infty}_{n=N+1}\Norm{f_n}<\frac{\varepsilon}{2}\]
  sowie Treppenfunktion $g_n$ mit
  \[\Norm{f_n-g_n}<\frac{\varepsilon}{2^{n+1}}\s\forall n\in\left\{ 0,\dots,N \right\}\]
  Dann ist $g:=\sum_{n=0}^Ng_n$ eine Treppenfunktion und es gilt:
  \[\abs{f(x)-g(x)}=\Limi{n}\left|\sum_{k=0}^nf_k(x)-\sum^N_{k=0}g_k(x)\right|\]
  \[\leq\sum_{k=0}^N\abs{f_k(x)-g_k(x)}+\Limi{n}\sum^n_{k=N+1}\abs{f_k(x)}\]
  \[\leq\sum^N_{n=0}\Norm{f_k-g_k}+\sum^{+\infty}_{k=N+1}\Norm{f_k}\]
  \[<\sum_{k=0}^N\varepsilon 2^{-k-1}+\frac{\varepsilon}{2}<\sum^{+\infty}_{k=0}\varepsilon2^{-k-1}+\frac{\varepsilon}{2}=\frac{\varepsilon}{2}+\frac{\varepsilon}{2}=\varepsilon\]
  $\implies$ $f$ ist eine Regelfunktion. Es gilt damit auch:
  \[\left|\int_a^bf-\sum^N_{n=0}\int_a^bf_n\right|
 \leq\left|\int_a^b(f-g)\right|+\left|\sum_{n=0}^N\int_a^b(f_n-g_n)\right|
 <(b-a)\varepsilon+(b-a)\frac{\varepsilon}{2}\]
  aber auch:
  \[\left|\sum_{n=0}^\infty\int_a^bf_n-\sum_{n=0}^N\int_a^bf_n\right|
  =\Limi{k}\abs{\sum_{n=N+1}^k\int_a^bf_n}\]
  \[\leq\sum^k_{n=N+1}(b-1)\Norm{f_n} <(b-a)\frac{\varepsilon}{2}\]
  Folglich:
  \[\left|\int_a^bf-\sum_{n=0}^\infty\int_a^bf_n\right|<2\varepsilon(b-a)\]
  Da $\varepsilon$ beliebig war, folgt \eqref{e:gliederweise}.
\end{Bew}
\begin{Kor}
  Sei $f(x)=\sum_{n=0}^{+\infty}a_nx^n$ eine analytische Funktion mit Konvergenzradius $R$. Dann ist $F(x)=\int_0^xf(y)\md y$ ebenfalls analytisch. Ihre Potenzreihe ist gegeben durch
  \begin{equation}\label{e:1012083}
    F(x)=\sum^{+\infty}_{n=0}\frac{a_n}{n+1}x^{n+1}
  \end{equation}
  und der Konvergenzradius ist $R$.
\end{Kor}
\begin{Bew}
  Gleichung \eqref{e:1012083} für $\abs{x}<R$ ist ein Korollar von Satz \ref{s:1012080} und 
Satz \ref{s:1012081}. Der Konvergenzradius ist
  \[R'=\frac{1}{\limsup_{n\rightarrow+\infty}\sqrt[n]{\frac{a_n}{n+1}}}\]
  \[=\frac{1}{\limsup_{n\rightarrow+\infty}\sqrt[n]{a_n}}=R\]
\end{Bew}
\begin{Bsp}
  \[\ln(1+x)=\int_0^x\frac{\md y}{1+y}\]
  aber für $\abs{x}<1$
  \[\frac{1}{1+x}=\sum_{n=0}^{+\infty}(-1)^nx^n\]
  und so:
  \[\ln(1+x)=\sum^{+\infty}_{n=0}\frac{(-1)^n}{n+1}x^{n+1}\]
\end{Bsp}
\begin{Bsp}
  \[\arctan x=\int_0^x\frac{1}{1+y^2}\md y\]
  und für $\abs{x}<1$
  \[\frac{1}{1+x^2}=\sum_{n=0}^{+\infty}(-1)^nx^{2n}\]
  und so
  \[\arctan x=\sum^{+\infty}_{n=0}\frac{(-1)^n}{2n+1}x^{2n+1}\]
\end{Bsp}
