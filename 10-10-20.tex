\subsection{Wurzel- und Quotientenkriterium}
\begin{Def}
  Eine Reihe $\sum^\infty_{n=0} a_n$ heisst \ul{absolut konvergent}, falls $\sum_{a=0}^\infty \abs{a_n}$ eine konvergente Reihe ist.
\end{Def}
\begin{Bem}
  Majorantenkriterium $\iff$ die absolute Konvergent impliziert die Konvergent.
\end{Bem}
\begin{Sat} (Quotientenkriterium)
  Sei $\sum a_n$ eine Reihe mit $a_n\neq 0$ für fast alle $n$ und s.d. $\lim_{n\to+\infty}\abs{\frac{a_{n+1}}{a_n}}=q$ existiert. Falls
  \begin{itemize}
    \item $q<1$ konvergiert die Reihe absolut.
    \item $q>1$ divergiert die Reihe.
    \item $q=1$ unentschieden.
  \end{itemize}
\end{Sat}
\begin{proof}[Beweis]
  \begin{itemize}
    \item $q>1$. $1<\tilde q=\frac{1}{2}+\frac{q}{2}<q$.Dann $\exists N$ so dass $\abs{a_{n+1}}\geq \tilde q\abs{a_n}$ 
falls $n\geq N$. Deswegen:
      \[\abs{a_n}\geq \tilde q\abs{a_{n-1}}\geq \tilde{q}^2\abs{a_{n-2}}\cdots>\tilde{q}^{n-N}\abs{a_N}\]
      OBdA $\abs{a_N}\neq 0$
      \[\implies \lim_{n\to+\infty}\abs{a_n}=+\infty\implies\sum a_n\text{ divergiert}\]
    \item $q<1$. $1<\tilde q=\frac{1}{2}+\frac{q}{2}<q$. $\exists N$ so dass $\abs{a_n}\leq \tilde{q}^{n-N}\abs{a_N}$ (das gleiche Argument wie vorher).
      \[b_n=\tilde{q}^{n-N}\abs{a_N}=C\tilde{q}^n \qquad \mbox{falls $n\geq N$}\]
      \[b_n=\abs{a_n} \qquad \mbox{falls $n<N$}\]
      $\sum b_n$ majorisiert $\sum a_n$
      \[\sum b_n \text{ konv }\stackrel{\text{Maj.}}{\implies}\sum\abs{a_n} \text{ konvergiert}\]
  \end{itemize}
\end{proof}
\begin{Sat} (Wurzelkriterium)
  Sei $\sum a_n$ eine Reihe und $L:=\limsup_{n\to+\infty}\sqrt[n]{\abs{a_n}}$ 
(``$L=+\infty$'' falls $\sqrt[n]{\abs{a_n}}$ unbeschränkt ist!) Dann:
  \begin{itemize}
    \item $L<1$ konvergiert die Reihe absolut
    \item $L>1$ divergiert die Reihe
    \item $L=1$ unentschieden
  \end{itemize}
\end{Sat}
\begin{proof}[Beweis]
  \begin{itemize}
    \item $L<1$ 
      \[L<\tilde L=\frac{L}{2}+\frac{1}{2}<1 \implies \exists N: 
\sqrt[n]{\abs{a_n}}\leq\tilde L\; \forall n\geq N\, . \]
Dann $|\abs{a_n}|\leq\tilde{L}^n$ für $n\geq N$ und wir haben
(wie oben) die absolute Konvergenz.
    \item $L>1$
      \[\exists k_n: \sqrt[k_n]{\abs{a_{k_n}}}\to L\]
      \[1<\tilde{L}=\frac{L}{2}+\frac{1}{2}<L\]
      \[\exists N: k_n\geq N \implies \sqrt[k_n]{\abs{a_{k_n}}}\geq \tilde L\]
      \[\implies \abs{a_{k_n}}\geq \tilde{L}^{k_n}\to+\infty\text{ für } n\to+\infty\]
      \[\implies a_n\not\to 0\implies \sum a_n\text{ divergiert}\]
  \end{itemize}
\end{proof}
\begin{Bsp}
  Sei $s\geq 1$ $\sum\frac{1}{n^s}$
  \begin{itemize}
    \item $s=1$ harmonische Reihe: divergiert.
    \item $s>1$ konvergiert! (eine bekannte Erfolg von Euler
ist die Formel $\sum\frac{1}{n^2} =\frac{\pi^2}{6}$).
      \[\sum\frac{1}{n^{2k}}\sim \underbrace{c_k}_{\in\mb{Q}}\pi^{2k}\]
\end{itemize} 
Bermerken Sie dass ($a_n=\frac{1}{n^s}$):
\[\lim_{n\to+\infty}\frac{a_{n+1}}{a_n}=1\forall s\geq 1\]
\[\lim_{n\to+\infty}\sqrt[n]{a_n}=1\forall s\geq 1\]
Deswegen sind die F\"alle $q=1$, bzw. $L=1$, f\"ur das
Quotientekriterium, bzw. Wurzelrkiterum, unentschieden.

Wir habe schon gesehen dass die harmonische Reihe divergiert.
Wir beweisen die Konvergenz von $\sum a_n$ im Fall $s>1$.
Betrachten wir die Reihe:
\[\sum_n b_n = \frac{1}{1^s}+\underbrace{\frac{1}{2^s}+\frac{1}{2^s}}_
{\mbox{$2$ mal}}+ \underbrace{\frac{1}{4^s}+\cdots+
\frac{1}{4^s}}_{\mbox{$4$ mal}}+\cdots + \underbrace{\frac{1}{2^{ks}}+\cdots
+ \frac{1}{2^{ks}}}_{\mbox{$2^k$ mal.}}
\]
und die entsprechenden Partialsummen $s_n$. $\sum b_n$ majorisiert $\sum a_n$.
Da $b_n\geq 0$, es bleibt die beschr\"anktheit von $s_n$ zu zeigen.
In der Tat
\begin{eqnarray*}
s_{2^k-1}&=&\frac{1}{1^s}+\frac{2}{2^s}+\cdots+\frac{2^{k-1}}{2^{(k-1)s}}\\
&\leq& \sum_{j=0}^\infty \frac{1}{2^{j(s-1)}} = \sum_{j=0}^\infty
\left(\frac{1}{2^{s-1}}\right)^j = \frac{1}{1-2^{-(s-1)}} <\infty\, .
\end{eqnarray*}
Die Konvergenz von $\sum a_n$ folgt aus dem Majorantenkriterium.
\end{Bsp}
\subsection{Das Cauchyprodukt}
\begin{Def}
  $\sum a_n$ und $\sum b_n$. Das CP ist die Reihe $\sum c_n$
  \[c_n=a_0b_n+a_1b_{n-1}+\cdots+a_nb_0=\sum^n_{j=0}a_jb_{n-j}=\sum_{j+k=n}a_jb_k\]
\end{Def}
\begin{Sat}
  Falls $\sum a_n$ und $\sum b_n$ absolut konvergieren, dann konvergiert das CP absolut.
Ausserdem  \[\sum c_n=\left( \sum a_n \right)\left( \sum b_n \right)\]
\end{Sat}
NB:
\begin{align*}
&c_0=a_0b_0\\
&c_1=a_0b_1+a_1b_0\\
&c_1=a_0b_2+a_1b_1+a_2b_0\\
& \ldots 
\end{align*}

\begin{proof}[Beweis]
  \[s_k=\sum^k_{j=0}a_j,\qquad \sigma_k=\sum_{i=0}^k b_i\]
  \[s_k\sigma_k=\sum^n_{j=0}\sum^n_{i=0}b_ia_j\]
  \[c_n=\sum_{j+i=n}a_ib_j,\qquad \beta_k=\sum^k_{n=0}c_k\]
  \[\sum^k_{n=0}\sum_{i+j=n}a_ib_j=\sum_{i+j\leq n}a_ib_j\]
  Unseres Ziel ist die Differenz  \[\beta_k-\sigma_ks_k\]
abzusch\"atzen.

Zuerst wir zeigen die Absolte Konvergenz: $\sum\abs{c_k}<+\infty$.
Wir setzten $B_n=\sum^\infty_{k=0}\abs{c_k}$ und zeigen dass
$\left( B_n \right)$ eine beschränkte Folge ist.
  \[B_n=\sum^N_{k=0}\abs{\sum_{i+j\geq k}a_ib_j}\leq\sum^N_{k=0}\sum_{i+j=k}\abs{a_i}\abs{b_j}\]
  \[=\sum_{i+j\leq N}\abs{a_i}\abs{b_j}\leq\sum^N_{i=0}\sum^N_{j=0}\abs{a_i}\abs{b_j}\]
  \[=\left( \sum^N_{i00}\abs{a_i} \right)\left( \sum_{j=0}^N\abs{b_j} \right)\leq\left( \sum\abs{a_i} \right)\left( \sum\abs{b_j} \right) = LM\]
  Wobei $L=\sum\abs{a_i}$ und $M=\sum\abs{b_j}$. $\implies$ $\left( B_n \right)$ 
konvergiert $\implies$ $\sum c_n$ konvergiert absolut.
Non wir beweisen dass $\beta_n - \sigma_n s_n \to 0$.  
\[|\beta_n - \sigma_n s_n| \;=\; \left| \sum^N_{i=0}a_i \sum^N_{j=0}b_j-\sum^N_{k=0} c_k
\right|\]
  \[=\left|\sum^N_{i=0, j=0}a_ib_j-\sum_{i+j\leq N}a_ib_j\right|\]
  \[=\left|\sum_{i+j>N, i , j\leq N}a_ib_j\right|\leq
\sum_{i+j>N, i , j\leq N}\abs{a_i}\abs{b_j}\]
  \[\leq \sum_{i\leq N, j\leq N, i\geq \frac{N}{2}, j\geq\frac{N}{2}}\abs{a_i}\abs{b_j}=\sum_{i,j\leq N}\abs{a_i}\abs{b_j}- \sum_{i,j<\frac{N}{2}}\abs{a_i}\abs{b_j}\]
  \[=\underbrace{\left( \sum^N_{i=0}\abs{a_i} \right)\left( \sum^N_{j=0}\abs{b_j} \right)-\left( \sum_{i=0}^{\floor{\frac{N}{2}}}\abs{a_i} \right)\left( \sum^{\floor{\frac{N}{2}}}_{j=0}\abs{b_j} \right)}_{\Gamma_N}\]
Aber:
  \[\lim_{N\to+\infty}\Gamma_N=\sum^\infty_{i=0}\abs{a_i}\sum^\infty_{j=0}\abs{b_j}-\sum^\infty_{i=0}\abs{a_i}\sum^\infty_{j=0}\abs{b_j}\]
Deswegen,
\[\limsup_{n\to\infty} |\beta_n - s_n\sigma_n| \leq \lim_{n\to\infty} \Gamma_n 
= 0\]
und
\[ \sum_n c_n = \lim_n \beta_n = \lim_n s_n\sigma_n = \sum_n a_n \sum_n b_n\, .
\]
\end{proof}
\subsection{Potenzreihen}
\begin{Def}
  Die Potenzreihen: $\sum a_nz^n$, $z\in\mb{C}$.
\end{Def}
\begin{Lem}\label{l:Abel}
  Falls $a_nz_0^n$ eine konvergente Reihe ist, 
dann $\forall z$ mit $\abs{z}<\abs{z_0}$ konvergiert $\sum a_nz^n$ absolut.
\end{Lem}
\begin{proof}[Beweis]
  $a_nz_0^n$ ist eine Nullfolge.
  \[\implies \exists C:\abs{a_nz_0^n}\leq C\qquad\forall n\]
  \[\abs{a_nz^n}\leq\abs{a_nz^n_0}\underbrace{\frac{\abs{z}^n}{\abs{z_0}^n}}_\alpha\leq C\alpha^n\]
  \[\abs{z}<\abs{z_0}\implies \alpha<1\]
  $\implies$ $\sum C\alpha^n$ eine konvergente Majorante.
\end{proof}
  Sei $(a_n)$ eine Folge von Koeffizienten $a_n\in\mb{C}$. Sei 
\[K:=\left\{ z\in\mb{C}:\sum a_nz^n\text{ konvergiert} \right\}\] 
Die Werte der Potenzreihen geben eine wohldefinierte Funktion:
\[K\ni z\mapsto f(z)=\sum^\infty_{n=0}a_nz^n.\]

\begin{Sat}
Seien
\[f(z)=\sum a_nz^n, g(z)=\sum b_nz^n\]
mit Definitionsbreiche $K$ and $K'$.
Auf $K\cap K'$:
\[f(z)+g(z)=\sum(a_n+b_n)z^n\]
\[f(z)g(z)=\underbrace{\sum c_nz^n}_{\text{falls $z$ absolute Konvergenz garantiert}}\]
wobei $\sum c_n$ das Cauchy Produkt von $\sum_n a_n$ und $\sum b_n$ ist.
\end{Sat}
\begin{proof}[Beweis]
  Sei $\sum \gamma_n$ das CP von $\sum a_nz^n$ und $b_nz^n$.
  \[\sum a_nz^n\sum b_nz^n = sum \gamma_n\]
  \[=\sum\sum_{i+j=n}\left( a_iz^i \right)\left( b_jz^j \right)=\sum^\infty_{n=0}\sum_{i+j=n}a_ib_jz^{i+j}\]
  \[=\sum_{n=0}^\infty z^n\underbrace{\sum_{i+j=n}a_ib_j}_{=c_n}\]
\end{proof}
Als ein einfaches Korollar des Wurzelkriteriums erhalten wir:
\begin{Sat}[Cauchy-Hadamard] $\sum a_nz^n$. Sei $L:=\limsup \sqrt[n]{\abs{a_n}}$. Dann 
  \begin{itemize}
    \item $\abs{z}<\frac{1}{L}\implies$ $\sum a_nz^n$ konvergiert absolut
    \item $\abs{z}>\frac{1}{L}\implies$ $\sum a_nz^n$ divergiert
    \item $\abs{z}=1$ unentschieden
  \end{itemize}
\end{Sat}
