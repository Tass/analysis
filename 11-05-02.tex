\subsection{Satz der lokalen Umkehrbarkeit}
\begin{Sat}
  Sei $\Phi:\underbrace{U}_{\subset\mb{R}^n}\to\mb{R}^n$ ($U$ offene Menge) eine $\mb{C}^1$-Abbildung und sei $a\in U$ so dass $\md \Phi|_a$ umkehrbar ist. Dann $\exists U_0$ offene Umgebung von $a$ so dass $V:=\Phi(U_0)$ eine offene Umgebung von $\Phi(a)$ und die Einschränkung
  \[\Phi:U_0\to V\]
  ein Diffeomorphismus ist.
\end{Sat}
\begin{Lem}(Banachscher Fixpunktsatz)
  Sei $C\subset\mb{R}^n$ eine abgeschlossene Menge und sei $\phi:C\mapsto C$ eine Abbildung mit folgender Eigenschaft:
  \begin{align*}
    \Norm{\phi(x)-\phi(y)}\leq \lambda\Norm{x-y}& &\forall x,y\in C
  \end{align*}
  wobei $0\geq \lambda < 1$ (unabhängig von $x,y$).\\
  Dann $\exists x\in C$ so dass $\phi(x)=x$ (d.h. $x$ ein Fixpunkt von $\phi$ ist.
\end{Lem}
\begin{Def}
  Eine Abbildung
  \[\phi:X\mapsto X\s(\text{mit $X$ metrischer Raum})\]
  heisst Kontraktion falls $\exists \lambda <1$ so dass
  \begin{align*}
    \md (\phi(x),\phi(y))\leq\lambda\md(x,y)& &\forall x,y\in X
  \end{align*}
\end{Def}
\subsubsection{Allgemeine Form des Fixpunktsatzes von Banach}
\begin{Sat}
  Jede Kontraktion auf einem \ul{vollständigen} metrischen Raum besitzt einen Fixpunkt.
\end{Sat}
\begin{Bew}
  Sei $x_0\in X$ (bzw. in $C\subset\mb{R}^n$)
  \[ \begin{pmatrix}
    x_1=\phi(x_0)\\
    x_2=\phi(x_1)\\
    \vdots \\
    x_k=\phi(x_{k-1})
  \end{pmatrix}\]
  Behauptungen:
  \begin{enumerate}
    \item $\left\{ x_k \right\}$ ist eine Cauchyfolge 
      \[\xRightarrow{\text{Vollständigkeit von $X$}}\exists x\Limi{k}x_k\]
    \item $\phi(x)=x$
  \end{enumerate}
  1 $\implies$ 2 weil
  \[\phi(x)=\Limi{k}\phi(x_k)=\Limi{x_{k+1}}=x\]
\end{Bew}
\begin{Bew}
    \[\md(x_0,x_1)=M\geq 0\]
  \begin{eqnarray*}
    \md(x_{k+1},x_k)=\md\left( \phi(x_k),\phi(x_{k-1}) \right)\\
    \leq \lambda\md(x_k,x_{k-1})\leq \cdots \leq\lambda^2\md(x_{k-1},x_{k-2})\\
    \cdots\leq \lambda^k\md(x_1,x_0)=\lambda^kM
  \end{eqnarray*}
  \begin{eqnarray*}
    \md(x_{k+j}x_k)\\
    \leq\md(x_{k+j},x_{k+j-1})+\md(x_{k+j-1},x_{k+j-2})+\cdots+\md(x_{k+1},x_{k})\\
    \leq\lambda^{k+j-1}M+\lambda^{k+j-2}M+\cdots+M\lambda^k
  \end{eqnarray*}
  \begin{eqnarray*}
    \md(x_{k+j},x_k)\leq M\lambda^k(1+\lambda+\cdots+\lambda^{j-1})\\
    <M\lambda^k\sum^\infty_{i=0}\lambda^i\\
    =\frac{M\lambda^k}{1-\lambda}
  \end{eqnarray*}
  Deswegen $\forall m>n\geq N$ ($\lambda^N\to$ für $N\to+\infty$
  \[\md(x_m,x_n)\leq\frac{M}{1-\lambda}\lambda^N\]
  $\forall \varepsilon>0$ $\exists N$ so dass
  \[\frac{M\lambda^N}{1-\lambda}<\varepsilon\]
  \begin{align*}
    \implies\md(x_m,x_n)<\varepsilon& & \forall n>m\geq N
  \end{align*}
  Das ist die Cauchyeigenschaft $\implies$ $\left\{ x_k \right\}$ ist eine Cauchyfolge
\end{Bew}
\paragraph{Beweis des Satzes}
\subparagraph{Schritt 1}
Wir suchen eine Umgebung von $W$ von $\Phi(a)$, wo wir immer ein Urbild von $\in W$ finden. D.h. 
\begin{equation}
  \label{e:1105021}
  \Phi(x)=y
\end{equation}
besitzt eine Lösung $x$.\\
OBdA nehmen wir an $a=0$ und $\md\Phi|_a=\id$ % identität
(In der Tat, nehmen wir an dass
\[L=\md\Phi|_a\neq \id\]
Sei 
\[\Phi'=L^{-1}\circ \Phi\] und 
\[\md\Phi'|_x=L^{-1}\circ \md\Phi|_x\]
$\implies$ $\Phi'$ ist eine $\mb{C}^1$-Funktion.
\[\md\Phi|_0=L^{-1}\circ\md\Phi|_0=L^{-1}\circ L=\id\]
$\implies$ Satz an $\Phi'$ anwenden
\[\Psi'(\Phi'(x))=x\implies \Psi'(L^{-1}(\Phi(x)))=x\]
\[\implies\Psi:=\Phi'\circ L^{-1}\]
die gesuchte Umkehrung von $\Phi$ ist $V:=(V')$)
Wir wollen zeigen dass, wenn $\Norm{y-\Phi(0)}<\delta$, dann die Gleichung \ref{e:1105021} lösbar ist.
\[\ref{e:1105021}\iff \underbrace{y+x-\Phi(x)}_{x\mapsto \phi_y(x)}=x\]
$\phi_y:U\to\mb{R}^n$ $\exists \eta>0$ so dass
\[\phi_y:\ol{B_\eta}(0)\mapsto \ol{B_\eta}(0)\]
eine Kontraktion ist.
\begin{enumerate}
  \item $\phi_y$ bildet $\ol{B_y}(0)$ in $\ol{B_\eta}(0)$
  \item $\Norm{\phi_y(z)-\phi_y(w)}\leq\frac{1}{2}\Norm{z-w}$
\end{enumerate}
Das zweite:
\begin{eqnarray*}
  \Norm{\phi_y(z)-\phi_y(w)}\\
  =\Norm{y+z-\Phi(z)-y-w+\Phi(w)}\\
  =\Norm{(\Phi(w)-\Phi(z))-(w-z)}\\
  =\Norm{\underbrace{\Phi(w)-w}_{\Lambda(w)}-\underbrace{\Phi(z)-z}_{\Lambda(z)}}
\end{eqnarray*}
$\Lambda$ ist $\mb{C}^1$
\[\md\Lambda|_0=\md\Phi|_0-\id=0\]
\[\Norm{\md|_0}_{HS}=0\]
$\implies$ $\exists \eta>0$ so dass
\[B_{\leq \eta}(0)\ni x\implies\Norm{\md\Lambda|_x}_{HS}\leq\frac{1}{2}\]
$z,w\in \ol{B_\eta}(0)$ und $\in B_\eta(0)$
\begin{eqnarray*}
  \Norm{\phi_y(z)-\phi_y(w)}=\Norm{\Lambda(z)-\Lambda(w)}\\
  \stackrel{\text{Schrankensatz}}{\leq}\left( \max_{\ol{B_\eta}(0)}\Norm{\md\Lambda}_O \right)\Norm{z-w}\\
  \frac{1}{2}\Norm{z-w}
\end{eqnarray*}
\[\phi_y(0)=y-\Phi(0)+0=y-\Phi(0)\]
$\delta=\frac{\eta}{2}$, $\Norm{\phi_y(0)}\leq\frac{1}{2}$. Sei $z\in\ol{B_\eta}(0)$
\begin{eqnarray*}
  \Norm{\phi_y(z)}\Norm{\phi_y(z)-\phi_y(0)}+\Norm{\phi_y(0)}\\
  <\Norm{\phi_y(z)-\phi_y(0)}+\frac{\eta}{2}\\
  \leq\frac{1}{2}\Norm{z-0}+\frac{\eta}{2}\\
  \leq \frac{1}{2}\eta+\frac{1}{2}\eta\\
  =\eta\\
  \implies\Norm{\phi_y(z)}<\eta
\end{eqnarray*}
So 
\[\phi_y:\ol{B_\eta}(0)\mapsto B_\eta(0)\]
Banach: $\forall y\in B_{\frac{\eta}{2}}(\Phi(0))$, $\exists x \in B_\eta(0)$ und $\in \ol{B_\eta}(0)$ mit
\[\phi_y(x)=x\iff\Phi(x)=y\]
Sei $V:=B_\delta(\Phi(0))$ (offen und Umgebung von $\Phi(0)$)
\begin{align*}
  \underbrace{B_\eta(0)\cap\Phi^{-1}(V)}_{\text{ist eine offene Menge}}=U_0 & & (\text{offen und Umgebung von 0})
\end{align*}
\[Phi:U_0\to V\]
\begin{enumerate}
  \item $\Phi$ ist surjektiv: $\forall y\in V$, $\exists x\in B_\eta(0)$ mit $\Phi(x)=y$
    \[\implies x\in \Phi^{-1}(V)\cap B_\eta(0)=U_0\]
  \item $\Phi$ ist injektiv
    \begin{eqnarray*}
      \Norm{\Phi(x)-\Phi(z)}=\Norm{(x+\Lambda(x))-(z+\Lambda(z))}\\
      \implies\Norm{\Phi(x)-\Phi(z)}\\
      \leq\Norm{x-z}-\Norm{\Lambda(x)-\Lambda(z)}\\
      \leq\Norm{x-z}-\frac{1}{2}\Norm{x-z}\\
      \leq\frac{1}{2}\Norm{x-z}\\
    \end{eqnarray*}
    $\implies$ $\Phi$ ist injektiv. (Alternativerweise wenn $\phi$ eine Kotraktion ist, der Fixpunkt ovn $\phi$ ist eindeutig: $\phi(p)=p$, $\phi(q)=q$
    \begin{eqnarray*}
      \md(p,q)=\md(\phi(p),\phi(q))\\
      \leq\lambda\md(p,q)\\
      (1-\lambda)\md(p-q)\leq 0\\
      \xRightarrow{\lambda<1}\md(p,q)=0\\
      \implies p=q
    \end{eqnarray*}
\end{enumerate}
\subparagraph{Schritt 2}
Sei $\Phi:V\mapsto U_0$ die Umkehrfunktion von $\Phi$. $\Psi$ ist stetig. Seien $\xi, \zeta\in V$, $x=\Phi(\xi), z=\Phi(\zeta)$ $\implies$ $\Phi(x)=\xi$, $\Phi(z)=\zeta$. Aber:
\begin{eqnarray*}
  \Norm{\Phi(x)-\Phi(z)}\leq\frac{1}{2}\Norm{x-z}\\
  \implies \underbrace{2\Norm{\xi-\zeta}\geq \Norm{\Psi(\xi)-\Phi(\zeta)}}_\text{Lipschitz-Bedingung für $\Phi$: stetig}
\end{eqnarray*}
\subparagraph{Schritt 3}
\begin{Bem}
  $\Phi:U_0\to V$ ist differenzierbar und $\md\Phi|_x$ ist umkehbar $\forall x\in U_0$.
  \[\Phi(x)=x-\Lambda(x)\]
  \[\md\Phi|_x=\id-\md\Lambda|_x\]
  Wir wissen, dass
  \begin{align*}
    \Norm{\md\Lambda|_x}_{HS}\leq\frac{1}{2}& &\forall x\in U_0\subset B_\eta(0)
  \end{align*}
  \[\md\Phi|_x(v)=v-\md\Lambda|_x(v)\]
  \begin{eqnarray*}
    \Norm{\md\Phi|_x(v)}\\
    \geq \Norm{v}-\Norm{\md\Lambda|_x(v)}\\\
    \geq\Norm{v}-\frac{1}{2}\Norm{v}\\
    \geq\frac{1}{2}\Norm{v}
  \end{eqnarray*}
  $\implies$ $\Ker(\md\Phi|_x)=\left\{ 0 \right\}$ $\implies$ $\md\Phi|_x$ ist injektiv $\implies$ Surjektivität $\implies$ $\md\Phi|_x$ ist umkehrbar
\end{Bem}
\begin{Lem}
  Falls $\Phi:U_0\to V$ eine $\mb{C}^1$ umkehrbare Abbildung so dass
  \begin{itemize}
    \item $\md\Phi|_x$ umkehrbar $\forall x\in U_0$ ist
    \item die Umkehrfunktion $\Psi:V\to U_0$ stetig ist, dann ist auch $\Psi$ eine $\mb{C}^1$ Abbildung.
  \end{itemize}
\end{Lem}
