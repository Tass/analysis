\subsection{Konvergenz}
Sei $\left\{ x_k \right\}_{k\in\mb{N}}$ $x_k\in\mb{R}$ $x_k=\left( x_{k1}, \cdots, x_{kn} \right)$
\begin{Def}
  Die Folge $\left\{ x_k \right\}$ konvergiert gegen $x_\infty\in\mb{R}^n$ falls
  \[\Limi{k}d(x_k,x_\infty)=0\]
  \[\left( \Limi{k}\Norm{x_k,x_\infty}=0 \right)\]
  Dann schreiben wir
  \[x_\infty=\Limi{k}x_k\]
\end{Def}
\begin{Sat}
  \[x_k\to x_\infty\iff x_{ki}\to x_{\infty_i}\s\forall i\in\left\{ 1,\cdots,n \right\}\]
\end{Sat}
\begin{Bew}
  \[\Norm{x_k - x_\infty}=\sqrt{\sum_{i=1}^n\left( x_{ki}-x_{\infty_i} \right)^2}\geq \Abs{x_{ki}-x_{k\infty}}\geq 0\]
  \[\implies 0\leq \Limi{k}\Abs{x_{ki}-x_{k\infty}}\leq \lim\Norm{x_k-x_\infty}=0\]
  \[\Norm{x_k-x_\infty}=\underbrace{\sqrt{\sum_{i=1}^n\underbrace{(x_{ki}-x_{\infty_i})^2}_{\to 0}}}_{\to 0}\leq \sum_{i=1}^n\Abs{x_{ki}-x_{\infty_i}}\]
  \[\implies \Norm{x_k-x_\infty}\to 0\]
  Eine alternative Formulierung: $\Limi{k}x_k=\left( \Limi{k} x_{k1},\cdots,\Limi{k} x_{kn} \right)$
\end{Bew}
\begin{Bem}
  \[\forall \varepsilon>0 \exists N: \Norm{x_k-x_\infty}<\varepsilon\s\text{falls}\s k\geq N\]
  Für jede Umgebung $U$ von $x_\infty$ fast alle $x_k\in U$.
\end{Bem}
\begin{Def}
  Eine Folge $\left\{ x_k \right\}\subset\mb{R}^n$ heisst Cauchy falls:
  \[\forall \varepsilon>0\s\exists N: m,k\geq N\implies \Norm{x_k-x_m}<\varepsilon\]
\end{Def}
\begin{Lem}
  $\left\{ x_k \right\}\subset\mb{R}^n$ konvergiert genau dann, wenn $\left\{ x_k \right\}$ Cauchy ist.
\end{Lem}
\begin{Bew}
  $\left\{ x_k \right\}$ ist Cauchy $\implies$ $\left\{ x_{k_{\underbrace{i}_{\left\{ \text{fixiert} \right\}}}} \right\}$ Cauchy!
  \[\Abs{x_{ki}-x_{m_i}}\leq\Norm{x_k-x_m}\]
  $\implies$ $\left\{ x_k \right\}$ ist eine Cauchyfolge $\stackrel{\text{Erstes Semester}}{\implies}$ $x_{ki}$ konvergiert $\stackrel{\text{Lemma 2}}{\implies}$ $x_k$ konvergiert. $x_k$ konvergiert $\implies$ Cauchyfolge
  \[x_\infty=\Limi{k} x_k\s\forall \varepsilon>0\s\exists N:\Norm{x_k-x_\infty}<\frac{\varepsilon}{2}\s\forall k\geq N\]
  \[k,m\geq N\s \Norm{x_k-x_m}\leq \Norm{x_k-x_\infty}+\Norm{x_\infty-x_m}\leq d(x_k,x_\infty)+(x_\infty,x_m)\]
  \[<\frac{\varepsilon}{2}+\frac{\varepsilon}{2}=\varepsilon\]
\end{Bew}
\begin{Bem}
  In einem metrischen Raum, Cauchy $\Leftarrow$ Konvergenz. Aber allgemein: Cauchy $\not\implies$ Konvergenz. Falls Cauchy $\implies$ Konvergenz, dann ist der metrische Raum vollständig.
\end{Bem}
\begin{Def}
  Eine Folge $\left\{ x_k \right\}\subset\mb{R}^n$ heisst beschränkt falls $\Norm{x_k}$ beschränkt ist.
\end{Def}
\begin{Sat}
  \begin{enumerate}
    \item Eine konvergente Folge ist beschränkt
    \item (Bolzano-Weierstrass) $\left\{ x_k \right\}$ beschränkt $\implies$ $\exists \left\{ x_{k_j} \right\}$ die konvergiert.
  \end{enumerate}
\end{Sat}
\begin{Bew}
  \[\left\{ x_k \right\}\s\text{beschränkt}\implies \left\{ x_{k1} \right\}_{k\in\mb{N}}\s\text{beschränkt}\]
  \[\implies \exists x_{k_j}: x_{k_j1}\to x_1\]
  Ich definiere $y_j=x_{k_j}$ $y_{j1}\to x_1$
  \[y_j\s\text{beschränkt}\implies\exists j_l: y_{j_l2}\to x_2\]
  \[z_l:=y_{j_l}\s\text{und}\s z_{l1}\to x_1, \s x_{l2}\to x_2\]
  \ldots $(n-2)$ Schritte. $w_r$ Teilfolge von $x_k$ mit $w_{ri}\to x_i$
  \[w_r\to(x_1,\cdots,x_n)\]
\end{Bew}
\subsection{Ein bisschen mehr Topologie}
\begin{Def}
  Eine Menge $G\subset\mb{R}^n$ heisst geschlossen falls $G^c(:=\mb{R}^n\setminus G)$ eine offene Menge ist.
\end{Def}
\begin{Bem}
  \[(A\cup B)^c = A^c\cap B^c\]
  \[(A\cap B)^c = A^c\cup B^c\]
\end{Bem}
\begin{Sat}
  \begin{enumerate}
    \item $\varnothing, \mb{R}^n$ sind abgeschlossen
    \item $G_1,\cdots,G_N$ abgeschlossen $\implies$ $G_1\cup G_2\cup \cdots\cup G_N$ abgeschlossen
    \item $\left\{ G_\lambda \right\}_{\lambda\in\Lambda}$ abgeschlossen $\implies$ $\bigcap_{\lambda\in\Lambda} G_\lambda$ abgeschlossen.
  \end{enumerate}
\end{Sat}
\begin{Sat}
  $G\subset\mb{R}^n$ $G$ ist abgeschlossen $\iff$ $\forall$ jede konvergente $\left\{ x_k \right\}\subset G$ gehört der Grenzwert zu $G$ (gilt auch für metrische Räume).
\end{Sat}
\begin{Bew}
  \begin{itemize}
    \item[$\Leftarrow$] Die rechte Eigenschaft gilt. Ziel: $G^c$ ist offen. Sei $x\in G^c$: das Ziel ist eine Kugel $K_r(x)\in G^c$ zu finden. Widerspruchsbeweis: $K_{\frac{1}{j}}(x)\not\subset G^c$, $j\in\mb{N}\setminus\left\{ 0 \right\}$
      \[\implies \exists x_j\in K_{\frac{1}{j}}(x)\cap G\implies \left\{ x_j \right\}\subset G\s\text{und}\s x_j\to x \]
      \[\left\{ x_j \right\}\subset G\s x_j\to x\s x\not\in G\]
      $\implies$ d.h. $G^c$ offen $\implies$ falls $\left\{ x_k \right\}\subset G$ und $x_k\to x$ dann $x\in G$
      Widerspruch: $G^c$ offen, aber $\exists \left\{ x_k \right\}\subset G$ mit Grenzwert $x\not\in G$, d.h. $x\in G^c$. Offenheit von $G^c$.
      \[\implies \exists K_r(x)\subset G^c\implies K_r(x)\cap=\varnothing\]
      d.h. $\exists N$ mit
      \[\Norm{x_N-x}<r\implies x_N\in K_r(x)\cap G\]
  \end{itemize}
\end{Bew}
\begin{Bsp}
  Eine offene Kugel ist nicht geschlossen.
  \[K_r(x)=\left\{ y:\Norm{y-x}<r \right\}\]
  Sei $\left\{ y_k \right\}\in K_r(x)$, (d.h. $\Norm{y_k-x}<r$) mit $y_k\to y$ und $\Norm{y-x}=r$.
\end{Bsp}
\begin{Def}
  Sei $\ol{K_r(x)}:=\left\{ y\in\mb{R}^n:\Norm{y-x}\leq r \right\}$.
\end{Def}
\begin{Ueb}
  $\ol{K_r(x)}$ ist abgeschlossen
\end{Ueb}
\begin{Def}
  $x\in\mb{R}^n$ ist ein Randpunkt von $M$ falls
  \[\forall K_r(x)\s\exists y\in K_r(x)\cap M\s\text{und}\s \exists z\in K_r(x)\cap M^c\]
\end{Def}
\begin{Def}
  Sei $M$ eine Menge in $\mb{R}^n$, dann ist der Rand von $M$
  \[\partial M=\left\{ x\in\mb{R}^n, \s\text{Randpunkt von}\s M \right\}\]
\end{Def}
\begin{Sat}
  $\partial M^c=\partial M$
  \begin{enumerate}
    \item $M\setminus \partial M$ ist die grösste offene Menge die in $M$ enthalten ist.
    \item $M\cup \partial \partial M$ ist die kleinste geschlossene Menge die $M$ enthält.
  \end{enumerate}
\end{Sat}
\begin{Bew}
  $M\setminus \partial M$ ist offen. 
  \[x\in M\setminus \partial M \implies x\in M\s\text{und}\s \exists K_r(x)\s\text{mit}\s K_r(x)\cap M^c=\varnothing\]
  \[\implies K_r(x)\subset M\]
  Sei $y\in K_r(x)$
  \[\implies \Abs{y-x}=\rho<r\]
  \[\implies K_{r-\rho}(y)\subset K_r(x)\subset M\implies y\in M,y\not\in \partial M\]
  \[K_r(x)\subset M\setminus \partial M\]
  $x$ ist beliebig $\implies$ $M\setminus \partial M$ ist offen.\\
  Sei $A\subset M$ eine offene Menge. Das Ziel ist $A\subset M\setminus\partial M$. Sei $x\in A$. Ziel:($x\in M\setminus\partial M$) $x\not\in \partial M$.
  \[A\s\text{offen}\implies \exists K_r(x)\subset A\subset M\implies x\not\in \partial M\implies A\subset M\setminus\partial M\]
\end{Bew}
