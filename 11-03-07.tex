%\begin{Sat}
% $E\subset \mb{R}^n$
% \[E\s\text{kompakt}\iff E\s\text{folgenkompakt}\]
% d.h.
%  \[\forall \left\{ x_k \right\} \subset E\s\exists\s\text{Teilfolge}\s \left\{ x_{k_l} \right\}\s\text{die gegen $x\in E$ %konvergiert}\]
%\end{Sat}
Wir geben noch eine zweite Characterisierung der kompaken Teilmenge von $\mb{R}^n$.

\begin{Def}
  (Überdeckungseigenschaft)  EIne Familie $\left\{ U_\lambda \right\}_{\lambda\in\Lambda}$ von Teilmengen von $\mb{R}^n$ ist eine \"Uberdeckung einer Menge $E$ falls 
\[\bigcup_{\lambda\in\Lambda} U_\lambda\supset E\]
Eine Teilüberdeckung ist eine Teilfamilie von $\left\{ U_\lambda \right\}$ die noch eine Überdeckung von $E$ ist.

Eine Teilmenge $E\subset\mb{R}^n$ besitzt die Überdeckungseigenschaft falls:
 \begin{itemize}
    \item $\forall$ Überdeckung $\left\{ U_\lambda \right\}_{\lambda\in\Lambda}$ von $E$ mit offenen Mengen $\exists$ endliche Teilüberdeckung.
 \end{itemize}
 
\end{Def}
\begin{Bsp}\label{b:off}
  Eine offene Kugel hat diese Eigenschaft nicht.
  \[\forall x\in K_r(0)\s\text{sei}\s K_{\frac{r-\Norm{x}}{2}}(x)=U_x\]
  \begin{enumerate}
    \item $\left\{ U_x \right\}_{x\in K_r(0)}$ ist eine Überdeckung von $K_r(0)$.
    Einfach weil $x\in U_x$! 
\item Keine endliche Teilfamile von $\{U_x\}$ ist eine \"Uberdeckung von $K_r (0)$.
In der Tat, sei $\{U_{x_1},\cdots,U_{x_N}\}$ eine beliebige endliche Teilfamilie. Sei 
  \[\rho:=\max_{i\in\left\{ 1,\cdots,N \right\}}\Norm{x_i}<r\]
  $\implies$ falls $\Norm{y}\geq \frac{\Norm{x_i}+r}{2}$ dann $y\not\in U_{x_i}$. So, wenn $\Norm{y}\geq \frac{\rho+r}{2}$ dann
  \[y\not\in U_{x_1}\cup\cdots\cup U_{x_N}\, .\]
Aber $\frac{\rho+r}{2}< r$. So, wenn
 $\Norm{y}=\frac{p+r}{2}$, dann $y\in K_r(0)$.
\end{enumerate}
Jede geschlossene Kugel hat die \"Uberdeckungseigenschaft: das ist eine Konsequenz
des n\"achsten Satzes.
\end{Bsp}

\begin{Sat} \label{s:k_ub}
  Sei $E\subset\mb{R}^n$
  \[E\s\text{kompakt}\iff E\s\text{hat die Überdeckungseigenschaft}\]
\end{Sat}
\begin{Bem} Satz \ref{s:k_ub} kann auch so formuliert werden:
\[
(\mbox{$E$ beschr\"ankt und abgeschlossen}) \iff E\s\text{hat die Überdeckungseigenschaft}\, .\]
Das Beispiel \ref{b:off} erkl\"art wie so die Abgeschlossenheit n\"otig ist.
Sei nun $E=\mb{R}^n$ und $U_n=K_{n+1}(0)$.
  \[E\subset \bigcup_{n\in\mb{N}} U_n\]
  Aber $\forall N\in\mb{N}$
  \[\mb{R}^n=E\not\subset \bigcup_{n=0}^N U_n\, .\]
Dieses Beispiel zeigt wie so die Beschr\"ankheit n\"otig ist.
\end{Bem}
\begin{Bew}[Beweis des Satzes \ref{s:k_ub}] 
{\bf $E$ ist nicht kompakt $\implies$ Überdeckungseigenschaft gilt nicht.}
Da $E$ nocht kompakt ist,   $\exists \left\{ x_i \right\}\subset E$ ohne konvergente Teilfolge in $E$. 
$\implies$  Zwei Möglichkeiten:
  \begin{enumerate}
    \item $\exists$ eine beschr\"ankte Teilfolge von $\{x_i\}$. Bolzano-Weierstrass $\implies$
$\exists$ Teilfolge $\left\{ y_i \right\}\subset \{x_i\}\subset E$ die gegen $y\in \mb{R}^n$
konvergiert. $y\not\in E$.
    \item $\{x_k\}$ besitzt beschr\"ankte Teilfolge $\implies$ $\|x_i\|\to \infty$.
  \end{enumerate}
  Beim ersten ist die folgende Menge offen: 
  \[U_0:=\mb{R}^n\setminus \underbrace{\left( \left\{ y_i \right\}\cup\left\{ y \right\} \right)}_{E\s\text{ist abgeschlossen}}\]
  Beim zweiten gilt:
  \[U_0=\mb{R}^n\setminus\underbrace{\left\{ x_i \right\}}_{F}\s\text{ist offen}\]
  \[U_n=U_0\cup \left\{ y_1,\cdots,y_{n-1} \right\} \s n\geq 0\]
  $U_n$ ist auch offen.
  \[\bigcup_{n=0}^{\infty}U_n= \begin{cases}
    \mb{R}^n\setminus \left\{ y \right\}& \text{im Fall 1}\\
    \mb{R}^n & \text{im Fall 2}
  \end{cases}\]
  Aber jede endliche Familie
  \[U_0\cup U_1\cup \cdots\cup U_n\not\supset E\]
  in beiden Fällen lassen wir unendlich viele Punkte weg.

\bigskip

{\bf $E$ kompakt $\implies$ $E$ besitzt die Überdeckungseigenschaft.} 
$E$ ist beschränkt und abgeschlossen und sei $\left\{ U_\lambda \right\}_{\lambda\in\Lambda}$ eine Familie von offenen Mengen mit $E\subset\left\{ U_\lambda \right\}_{\lambda\in\Lambda}$. Wir decken die Menge $U$ mit Würfel. Jeder W\"urfel hat die Form
 \begin{equation}\label{e:wurf}
\left[k_1,k_1+1\right]\times \left[ k_2,k_2+1 \right]\times \cdots\times \left[ k_n,k_n+1 \right]
\end{equation}
wobei $k_1, \ldots k_n \in \mb{Z}$.
Nun, da $E$ beschr\"ankt ist, $\exists N\in \mb{N}$ so dass $[-N, N]^n \supset E$. Aber
$[-N, N]^n$ k\"onnen wir mit $M=(2N)^n$ W\"urfel der Form \eqref{e:wurf} \"uberdecken:
 \[E\subset W_1\cup\cdots\cup W_M\]
  Falls jedes $E\cap W_i$ mit einer endlichen Familie von $\left\{ U_\lambda \right\}$ überdeckt wird, dann finde ich eine endliche Überdeckung von $E$ wenn ich die Vereinung der entsprechenden endlichen Teil\"uberdeckungen
von $E\cap W_i$ nehme. So, angenommen dass die Überdeckungseigenschaft nicht gilt,
  $\exists E_1:= E\cap W_i$ s.d.
  \begin{enumerate}
    \item $\left\{ U_\lambda \right\}_{\lambda\in\Lambda}$ eine Überdeckung von $E_1$
    \item keine endliche Teilfamilie deckt $E_1$.
  \end{enumerate}
  Teilen wir $W_i$ in $2^n$ Würfel mit Seite $\frac{1}{2}$
  \[\tilde W_1,\cdots,\tilde W_{2^n}\, .\]
Mit dem obigen Argument finden wir
  \[E_2:= E\cap \tilde W_i:\s\text{die Eigenschaften 1. und 2. mit $E_2$ statt $E_1$ noch gelten}\]
  Induktiv
  \[E\supset E_1\supset E_2\supset\cdots\]
  jede $E_i\subset W^i$ Würfel mit Seite $2^{-i+1}$ und die beiden Eigenschaften 1. und 2. gelten mit $E_i$ statt $E_1$.

Ausserdem, $E_i$ ist nicht leer. F\"ur jede $i$ w\"ahlen wir $x_i\in E_i$.
Dann $\left\{ x_k \right\}\subset E$. Aber $\left\{ x_k \right\}$ ist eine Cauchy-Folge: falls $j,k>i$, 
$x_k,x_j\subset E_i$ und $E_i$ ist in einem W\"urfel mit Seite $2^{-i+1}$ enthalten.
Deswegen $\Norm{x_j-x_k}\leq \sqrt{n}2^{-i+1}$. Die Vollestendigkeit von $\mb{R}^n$ garantiert
die Existenz von $x\in \mb{R}^n$ s.d. $x_k\to x$. Da $E$ abgeschlossen ist, $x\in E$. Deswegen $\exists U_\mu
\in \{U_\lambda\}_\lambda$
s.d. $x\in U_\mu$. Da $U_\mu$ offen ist,
  \[\exists  K_r(x)\supset U\]
Aber, $x\in E_i$ f\"ur jedes $i$ (weil $\{x_k\}_{k\geq i}\subset E_i$ und $E_i$ ist abgeschlossen!).  
Sei nun $k\in \mb{N}$ s.d. $\sqrt{n}2^{-k+1} < r$. Falls $y\in E_k$, dann $\|y-x\|\leq \sqrt{n}2^{-k+1} < r$.
Deswegen $E_k\subset K_r (x)\subset U_\mu$. So, die Familie $\{U_\mu\}$ ist endlich (ent\"ahlt
sogar einen einzigen Element!) und \"uberdeckt  $E_k$. Widerspruch!
\end{Bew}
\begin{Bem}
  $f$ stetig $\implies$ $f^{-1}(U)$ offen falls $U$ offen: diese m\"achtige Characterisierung 
der Stetigkeit werden wir nun nutzen!
\end{Bem}
\begin{Kor} Sei $E\subset \mb{R}^n$ kompakt und $f\mb{R}^m\to \mb{R}^k$ stetig. 
Dann $f(E)$ ist kompakt.
\end{Kor}
\begin{Bew}
  Sei $\left\{ U_\lambda \right\}$ eine Überdeckung (mit offenen Mengen) von $f(E)$, dann ist $\left\{ f^{-1}\left( U_\lambda \right) \right\}$ ein Überdeckung von $E$.
  \[\exists f^{-1}(U_{\lambda_1}),\cdots,f^{-1}(U_{\lambda_N}\s\text{Teilüberdeckung von $E$}\]
  $U_{\lambda_i},\cdots,U_{\lambda_N}$ ist eine Überdeckung von $f(E)$ $\implies$ $f(E)$ ist kompakt
\end{Bew}
\begin{Kor}\label{k:max_und_min}
  Wenn $f:\mb{R}^n\to \mb{R}$ stetig ist und $E\subset\mb{R}^n$ kompakt ist, besitzt $f$ ein Maximum und ein Minimum auf $E$.
\end{Kor}
\begin{Bew}
  $f(E)\subset\mb{R}$ ist kompakt.
  \[s=\sup f(E)<+\infty\]
  \[\exists \left\{ x_k \right\}\subset f(E)\s\text{mit}\s x_k\to s\xRightarrow{\text{abgeschlossen}}s\in s\in f(E)\]
  \[\left( s-\frac{1}{k}\implies \exists x_k\in f(E)\s\text{mit}\s x_k>s-\frac{1}{k},x_k\leq s \right)\]
  $\implies$ $s$ ist ein Maximum.
\end{Bew}

Ohne Beweis:

\begin{Lem}\label{l:tietze}[Lemma von Tietze]
Sei $E\subset \mb{R}^{m}$ kompakt und $f:E\to \mb{R}$ stetig. Dann $\exists g: \mb{R}^n\to
\mb{R}$ stetig s.d. $g|_E = f$.
\end{Lem}

Wenn wir Lemma \ref{l:tietze} und Korollar \ref{k:max_und_min} kombinieren, erhalten wir
den folgenden Satz:

\begin{Sat}
 Wenn $E\subset\mb{R}^n$ kompakt ist und $f:E\to \mb{R}$ stetig ist, besitzt $f$ ein Maximum und ein Minimum.
\end{Sat}

Wir geben auch einen alternativen Beweis, unabh\"angig von Tietzes Lemma

\begin{Bew}
Sei $s = \sup \{f(x):x\in R\}$ (es kann sein dass $s=\infty$). Dann $\exists \{x_k\}\subset E$
s.d. $f(x_k)\to s$. Die Kompaktheit von $E$ impliziert die Existenz einer Teilfolge
$\{x_{k_i}\}$ die gegen einen Element $x\in E$ konvergiert. Deswegen
\[
s = \lim_{i\to\infty} f(x_{k_i}) = f(x)\, .
\]
Ein \"ahnlichens Argument beweist die Existenz einer Minimumstelle.
\end{Bew}

Zur Erinnerung: das Intervallschachtelungsprinzip in $\mb{R}$. 

Sei $I_j$ eine Intervallschachtelung d.h.:
  \begin{enumerate}
    \item \[I_j=\left[ a_j,b_j \right]\]
    \item \[I_0\supset I_1\supset \cdots \supset I_j\supset I_{j+1}\]
    \item \[b_j-a_j\to 0\]
  \end{enumerate}
Dann
  \[\bigcap^\infty_{j=0}E_j\neq\varnothing\]

Ein Verallgemeinerung dieses Prinzips ist der Folgende

\begin{Sat}
  Sei $E_j$ eine Folge von kompakten Mengen mit $E_j\supset E_{j+1}$ $\forall j$ ($E_0\subset\mb{R}^n$).
Dann
  \[\bigcap_{j=1}^\infty E_j\neq\varnothing\s\text{falls}\s E_j\neq\varnothing \s\forall j\]
\end{Sat}
\begin{Bew}
  Sei $E_j$ wie im Satz mit $E_j\neq\varnothing$, aber $\bigcap_{j=0}^\infty E_j=\varnothing$. Sei $U_j:=\mb{R}^n\setminus E_j\implies U_j$ ist offen. $\bigcup_{j=1}^\infty U_j=\mb{R}^n$ und deswegen
ist $\left\{ U_j \right\}$ eine Überdeckung von $E_0$. Aber $U_1\cup\cdots\cup U_N=U_N$ (weil $U_{j+1}\supset U_j$)
  \[U_N\not\supset E_N\neq \varnothing\s E_N\subset E_0\]
  Keine endliche Teilfamilie von $\left\{ U_j \right\}$ ist eine Überdeckung von $E_0$. Widerspruch wegen der Kompaktheit von $E_0$.
\end{Bew}

Wir geben endlich eine Zusammenfusassung der Eigenschaften der stetigen Funktionen $f:\mb{R}^n\to 
\mb{R}^k$:
\begin{itemize}
\item $E\subset \mb{R}^k$ offen $\implies$ $f^{-1} (E)$ offen;
\item $E\subset \mb{R}^k$ geschlossen $\implies$ $f^{-1} (E)$ geschlossen;
\item $E\subset \mb{R}^n$ kompakt $\implies$ $f(E)$ kompakt.
\end{itemize}
Aber {\em Vorsicht!}
\begin{itemize}
\item $E\subset \mb{R}^n$ offen impliziert {\bf nicht} $f(E)$ offen;
\item $E\subset \mb{R}^n$ geschlossen impliziert {\bf nicht} $f(E)$ geschlossen;
\item $E\subset \mb{R}^k$ kompakt impliziert {\bf nicht} $f^{-1} (E)$ kompakt.
\end{itemize}
