% headers by Alexander Berthold van der Bourg / Pirmin Weigele 

%= Document-Class ==================================================================================
\documentclass[10pt,a4paper]{article}

%= Packages ========================================================================================
\usepackage[utf8x]{inputenc}
\usepackage{ngerman,amsmath,amssymb,amsfonts,mathrsfs}
\usepackage{amsthm}
\usepackage{bbm}
\usepackage{colortbl}
\usepackage{graphicx}
\usepackage{makeidx}
\usepackage{fancyhdr}
\usepackage{latexsym}
\usepackage{psfrag}
\usepackage{enumerate}
\usepackage{float}
\usepackage{dsfont}
\pagestyle{fancy}
\usepackage{multirow, bigdelim, bigstrut}
\usepackage{rotating}
\usepackage{ifthen}
\usepackage{boxedminipage}
\usepackage{mathtools}
\usepackage{trfsigns}
\usepackage{url}
%\usepackage{savetrees}

%= Seiten-Layout =========================================================================
\voffset-22mm \textheight715pt 

%Seitenbreite==============================================================

%\oddsidemargin=-0.2in
%\evensidemargin=-0.4in
%\textwidth=5.2in
%\headwidth=5.2in

%= Index-Befehle ========================================================================
\renewcommand{\indexname}{Stichwortverzeichnis}
\makeindex

%= Befehl-Overwriting =======================================================================
\makeatletter
\makeatother

%= Strings ================================================================
\newcommand{\mainfold}{.}
\newcommand{\prefix}{A1-}

%= Eigene Befehle ==========================================================================
\DeclareMathOperator{\id}{Id}
\DeclareMathOperator{\arccot}{arccot}
\DeclareMathOperator{\arcsinh}{arcsinh}
\DeclareMathOperator{\arccosh}{arccosh}
\DeclareMathOperator{\arctanh}{arctanh}
\DeclareMathOperator{\md}{d}
\DeclareMathOperator{\Grad}{grad}
\DeclareMathOperator{\Spur}{Spur}
\DeclareMathOperator{\Graph}{Graph}
\DeclareMathOperator{\sign}{sign}
\DeclareMathOperator{\Hom}{Hom}
\DeclareMathOperator{\rot}{rot}
\DeclareMathOperator{\Ker}{Ker}
\DeclareMathOperator{\Exp}{Exp}
\DeclareMathOperator{\Sym}{Sym}
\DeclareMathOperator{\diam}{diam}

\newcommand{\Diff}[2]{\displaystyle\frac{\mathrm{d}#1}{\mathrm{d}#2}}
\newcommand{\End}{\hfill{\hbox{$\Box$}}\par\vspace{2mm}}
\newcommand{\eps}{\varepsilon}
\newcommand{\ePic}[1]{\input{\mainfold/graphics/\prefix#1.eepic}}
\newcommand{\pst}[1]{\input{\mainfold/graphics/\prefix#1.pst}}
\newcommand{\pic}[1]{\input{\mainfold/graphics/\prefix#1.pic}}
\newcommand{\Mx}[1]{\begin{pmatrix}#1\end{pmatrix}}
%\newcommand{\im}[1]{\operatorname{Im}(#1)}
%\newcommand{\Include}[4]{\rhead{#2.#3.20#4}\input{\mainfold/lectures/#1-#4-#3-#2.tex}}
\newcommand{\Index}[1]{\emph{#1}\index{#1}}
\newcommand{\Int}[4]{\displaystyle\int\limits_{#1}^{#2}#3\,\mathrm{d}#4}
\newcommand{\diff}[1]{\operatorname{d}\!#1}
\newcommand{\Limi}[1]{\displaystyle\lim_{#1\rightarrow\infty}}
\newcommand{\Limo}[1]{\displaystyle\lim_{#1\rightarrow0}}
\newcommand{\Limu}[2]{\displaystyle\lim_{#1\uparrow #2}}
\newcommand{\Limd}[2]{\displaystyle\lim_{#1\downarrow #2}}
\newcommand{\Lim}[2]{\displaystyle\lim_{#1\rightarrow#2}}
\newcommand{\mb}[1]{\mathbb{#1}}
\newcommand{\ds}{\displaystyle}
\newcommand{\ol}[1]{\overline{#1}}
\newcommand{\Part}[2]{\dfrac{\partial #1}{\partial #2}}
\newcommand{\QED}{\hfill{\hbox{(QED)}}\par\vspace{2mm}}
\newcommand{\re}[1]{\operatorname{Re}(#1)}
\newcommand{\s}{\hspace{2mm}}
\newcommand{\vsa}{\vspace{1mm} \\}
\newcommand{\vsb}{\vspace{2mm} \\}
\newcommand{\vsc}{\vspace{3mm} \\}
% \newcommand{\tr}[1]{\textrm{#1}}
\newcommand{\tr}[1]{\text{#1}}
\newcommand{\ra}{\rightarrow}
\newcommand{\Ra}{\Rightarrow}
\newcommand{\Lra}{\Leftrightarrow}
\newcommand{\La}{\Leftarrow}
\newcommand{\ul}[1]{\underline{#1}}
\newcommand{\rsa}{\rightsquigarrow}
\newcommand{\ara}[2]{\autorightarrow{\ensuremath{#1}}{\ensuremath{#2}}}
\newcommand{\dcp}[2]{\begindc{\commdiag}[#1] #2 \enddc}
\renewcommand{\to}{\rightarrow}

%\newcommand{\detmx}{\left| \begin{array} #1 \end{array} \right|}

\newcommand{\grad}[1]{\Grad(#1)}
\newcommand{\fr}[2]{\displaystyle\frac{#1}{#2}} % fertiger bullshit, daf�r gibts \dfrac{}{}
\renewcommand{\Re}{\operatorname{Re}}
\renewcommand{\Im}{\operatorname{Im}}

% ---- DELIMITER PAIRS ----
\def\floor#1{\lfloor #1 \rfloor}
\def\ceil#1{\lceil #1 \rceil}
\def\seq#1{\langle #1 \rangle}
\def\set#1{\{ #1 \}}
\def\abs#1{\mathopen| #1 \mathclose|}	% use instead of $|x|$ 
\def\norm#1{\mathopen\| #1 \mathclose\|}% use instead of $\|x\|$ 

% --- Self-scaling delmiter pairs ---
\def\Floor#1{\left\lfloor #1 \right\rfloor}
\def\Ceil#1{\left\lceil #1 \right\rceil}
\def\Seq#1{\left\langle #1 \right\rangle}
\def\Set#1{\left\{ #1 \right\}}
\def\Abs#1{\left| #1 \right|}
\def\Norm#1{\left\| #1 \right\|}

%Adrians Abbildungs-Environment ==============================================

\newcommand{\Sidein}{\begin{rotate}{90}\small$\in$\end{rotate}}

\newcommand{\Abb}[5][]{\ensuremath{
    \begin{array}{lc}
      \ifthenelse{\equal{#1}{}}{}{#1:}\;\; & 
      \begin{xy}
        \xymatrixrowsep{1em}\xymatrixcolsep{2em}%
        \xymatrix{ #2 \ar[r] \ar@{}[d]^<<<<{\hspace{0.001em} \Sidein}
          & #3  \ar@{}[d]^<<<<{\hspace{0.001em} \Sidein} \\
          #4 \ar@{|->}[r] & #5} \end{xy}
    \end{array}
  }%
}

%= Environments ========================================================================
\def\thechapter{\Roman{chapter}}
\def\thesection{\arabic{section}}
\newtheorem{theorem}{Theorem}[section]
\newenvironment{Bew}{\begin{proof}[Beweis]}{\end{proof}}
\newtheorem{Axi}[theorem]{Axiom}
\newtheorem{Lem}[theorem]{Lemma}
\newtheorem{Kor}[theorem]{Korollar}
\newtheorem{Sat}[theorem]{Satz}
\newtheorem{Prop}[theorem]{Proposition}
\newtheorem{Beh}[theorem]{Behauptung}
\theoremstyle{definition}
\newtheorem{Bsp}[theorem]{Beispiel}
\newtheorem{Def}[theorem]{Definition}
\newtheorem{Ueb}[theorem]{\"Ubung}
\theoremstyle{remark}
\newtheorem{Bem}[theorem]{Bemerkung}
\newtheorem{Eig}[theorem]{Eigenschaften}
\newtheorem{Not}[theorem]{Notation}

\def\pstexInput#1{%
  \begin{center}
    \begin{picture}(0,0)%
      \special{psfile=\mainfold/graphics/A2-#1.pstex}%
    \end{picture}%
    \input{\mainfold/graphics/A2-#1.pstex_t}%
  \end{center}
}

%= Titelseite ===========================================================================
\begin{document}
\headheight15pt
\begin{titlepage}
\hfill
\vspace{20mm}
\pagenumbering{roman}
\begin{center}
{\LARGE Analysis II - Vorlesungs-Script} \vskip 3em {\large Prof. Dr. Camillo De Lellis} \vskip 1.5em
{\large Basisjahr 11 Semester I}\vspace{30mm}\\
{\large {\bf Mitschrift:} \vspace{2mm}\\
Simon Hafner}\vspace{5mm}\\ %30mm
%{\large {\bf Graphics:} \vspace{2mm}\\
%Pirmin Weigele }\vspace{30mm}\\ %30mm
\author{Simon Hafner}

\end{center}
\vfill

\end{titlepage}


%= Inhaltsverzeichnis ==========================================================================
\lhead{}
\rhead{}
\tableofcontents
\newpage
\pagenumbering{arabic}
\setcounter{page}{1}

%= Vorlesung-Skripts ==========================================================================
\cfoot{\thepage}
\fancyhead[L]{\nouppercase{\leftmark}}
\newpage

%Analysis II
\section{Metrik und Topologie des euklidischen Raumes}
$\mb{R}^n=\left\{ \left( x_1,\cdots,x_n \right), x\in\mb{R} \right\}$.
Wir f\"uhren verschiedene neue Begriffe in $\mb{R}^n$ ein:
\begin{itemize}
  \item die Euklidische Norm
  \item der Euklidische Abstand
  \item die entsprechende Topologie.
\end{itemize}
Wir betrachten gleichzeitig die entsprechenden Verallgemeinerungen,
d.h. die ``Abstrakte Theorien'' der
\begin{itemize}
  \item Normierten Vektorr\"aume
  \item Metrischen R\"aume
  \item Topologischen R\"aume.
\end{itemize}
\begin{Def}
  Sei $x\in\mb{R}^n$ ($x=(x_1,\cdots,x_n)$, $x_i\in\mb{R}$). Die Euklidische Norm
von $x$ ist 
  \[\Norm{x}_e=\sqrt{x_1^2+\cdots+x_n^2}=\sqrt{\sum_{i=1}^nx_i^2}\]
(wir schreiben oft $\|x\|$ anstatt $\|x\|_e$).
\end{Def}

Intuitiv: $\Norm{x}=$''der Abstand zwischen $x$ und 0``.  In der Tat, wenn $n=2$,
das Pytaghoras Theorem zeigt dass $\|x\|_e$ die L\"ange des Segments mit Extrema
$x$ and $0$ ist. 

\begin{Lem}\label{l:norm}
  $\Norm{.}$ erf\"ullt die Regeln
  \begin{enumerate}
    \item $\Norm{x}\geq 0$ und $\Norm{x}=0\iff x=0$
    \item $\Norm{\lambda x}=\Abs{\lambda}\Norm{x}$ $\forall \lambda\in\mb{R}$, $\forall x\in\mb{R}$
    \item $\Norm{x+y}\leq\Norm{x}+\Norm{y}$ $\forall x,y\in\mb{R}$
  \end{enumerate}
\end{Lem}
\begin{Bew}
  \begin{enumerate}
    \item $\geq 0$ trivial
      \[x=0\implies \sum x_i^2=0\implies \Norm{x}=0\]
      \[x=0\Leftarrow x_i=0\;\; \forall i\Leftarrow \sum x_i^2=0\Leftarrow \Norm{x}=0\]
    \item \[\Norm{\lambda x}=\sqrt{\sum^n_{i=1}(\lambda x_i)^2} = \sqrt{\lambda^2\sum x^2}=\Abs{\lambda}\sqrt{\sum x^2}=\Abs{\lambda}\Norm{x}\]
%\[\Abs{\lambda}=\frac{\Norm{x}\Abs{\lambda}}{\Norm{x}}\]
    \item Diese Aussage ist \"aquivalent zu
      \[\iff \underbrace{\Norm{x+y}^2}\leq \Norm{x}^2+\Norm{y}^2+2\Norm{x}\Norm{y}\]
Wir rechnen
      \[\sum_{i=1}^n(x_i+y_i)^2=\sum_{i=1}^n\left( x_i^2+y_i^2+2x_iy_i \right)=\Norm{x}^2+\Norm{y}^2\overbrace{2\sum_i x_iy_i}^{Skalarprodukt}\]
Wir definieren 
\[\langle x,y\rangle := \sum_{i=1}^n  x_iy_i\]
Wir brauchen dann die ber\"uhmte Cauchy-Schwartz Ungleichung, d.h.
      \[\langle x,y\rangle\leq \Norm{x}\Norm{y}\, .\]
Diese Ungleichung ist der Inhalt des n\"achsten Satzes.
  \end{enumerate}
\end{Bew}
\begin{Sat}{Cauchy-Schwartzsche Ungleichung}
  \[\sum^n_{i=1}x_iy_i\leq\sqrt{\sum_{i=1}^nx_i^2}\sqrt{\sum_{i=1}^ny_i^2}\]
\end{Sat}
\begin{Bew}
  OBdA $y\neq 0$ ($y=0$ trivial)
  \[t\to g(t)=\sum_{i=1}^n(x_i+ty_i)^2=\left( \sum x_i^2 \right)+2t\sum x_iy_i+t^2\sum y_i^2\]
  \[=\Norm{x}^2+2t\seq{x,y}+\Norm{y}^2t^2\]
  Sei $t_0=\frac{\seq{x,y}}{\Norm{y}^2}$, dann
  \[0\leq g(t_0)=\Norm{x}^2-2\frac{\seq{x,y}^2}{\Norm{y}^2}+\Norm{y}^2\frac{\seq{x,y}^2}{\Norm{y}^4}
 =\Norm{x}^2-\frac{\seq{x,y}^2}{\Norm{y}^2}\]
  \[\implies \seq{x,y}^2\leq\Norm{x}^2\Norm{y}^2\implies \Abs{\seq{x,y}}\leq\Norm{x}\Norm{y}\]
\end{Bew}
\begin{Def}
  Ein normierter Vektorraum ist ein reeller Vektorraum $V$ mit einer Abbildung $\Norm . :V\to\mb R$ so dass:
  \begin{enumerate}
    \item $\Norm x\geq 0$ und $\Norm x=0\iff x=0$ (Nullvektor)
    \item $\Norm{\lambda x}=\abs\lambda\Norm x$ $\forall \lambda\in\mb R$, $\forall x\in V$
    \item $\Norm{x+y}\leq\Norm{x}+\Norm{y}$ $\forall x,y\in V$
  \end{enumerate}
\end{Def}
\begin{Bsp}
  $V=\mb{R}^n$
  \[\Norm{x}_p=\left(\sum\Abs{x_i}^p\right)^{\frac{1}{p}}\s p\geq 1\, .\]
$\|\cdot\|_2$ ist die Euklidische Norm.
\end{Bsp}
\begin{Def}
  Seien $x,y\in\mb{R}^n$. Die Euklidische Metrik ist $d(x,y):=\Norm{x-y}$.
\end{Def}
\begin{Lem}\label{l:euk_met}
  \begin{enumerate}
    \item $d(x,y)\geq 0$ und $d(x,y)=0\iff x=y$
    \item $d(x,y)=d(y,x)$
    \item $d(x,z)\leq d(x,y)+d(y,z)$ (Dreiecksungleichung)
  \end{enumerate}
\end{Lem}
\begin{Bew} Die erste Zwei Aussagen sind trivial. Um die letzte zu beweisen:
  \[\Norm{x-z}\leq\|\underbrace{x-y}_{=:v}\|+\|\underbrace{\Norm{y-z}}_{=:w}\|\, .\]
Aber $x-z=v+w$. Wir wenden die dritte Aussage von Lemma \ref{l:norm} an:
  \[d (x,z) = \Norm{v+w}\leq\Norm v+\Norm w = d (x,y)+d(y,z)\, .\]
\end{Bew}
\begin{Def}
  Ein metrischer Raum ist eine Menge $X$ mit einer Abbildung
  \[d:X\times X\to\mb{R}\s (x,y)\mapsto d(x,y)\in\mb{R}\]so dass
  \begin{enumerate}
    \item $d(x,y)\geq 0$ und $d(x,y)=0\iff x=y$ $\forall x,y\in X$
    \item $d(x,y)=d(y,x)$ $\forall x,y\in X$
    \item $d(x,z)=d(x,y)+d(y,z)$ $\forall x,y,z\in X$
  \end{enumerate}
\end{Def}
\begin{Lem}
  Sei ($V$, $\Norm .$) ein normierter Vektorraum. Dann sind $V$ und $d(x,y)=\Norm{x-y}$ ein metrischer Raum.
\end{Lem}
\begin{Bew} Wir nutzen das gleiche Argument vom Lemma \ref{l:euk_met}.
\end{Bew}
\begin{Def}
  Die offene Kugel mit Radius $r>0$ und Mittelpunkt $x\in\mb{R}^n$ ist die Menge
  \[K_r(x)=\left\{ y\in\mb{R}^n, d(x,y)<r \right\}\]
(Wir werden auch oft $B_r (x)$ statta $K_r (x)$ nutzen.)
\end{Def}
\begin{Def}
  Eine Menge heisst ''Umgebung`` von $x$, wenn $V$ eine offene Kugel mit Mittelpunkt $x$ enth\"alt.
\end{Def}
\begin{Def}
  Eine Menge $U\subset\mb{R}^n$ heisst offen falls $\forall x\in U$ ist $U$ eine Umgebung von $x$, d.h.
  \[\forall x\in U\s\exists \s\text{eine Kugel}\s K_r(x)\subset U\]
\end{Def}
\begin{Bem}
Die Dreiecksungleichung impliziert dass jede offene Kugel eine offene Menge ist. In der Tat,
sei $y\in K_r (x)$. Dann $\rho:=d(x,y) < r$. Sei $\tau:= r-\rho>0$. Falls $z\in K_\tau (y)$,
dann $d(x,z)\leq d(x,y) + d(y,z) = \rho + d (y,z) < \rho +\tau =r$. D.h., $K_\tau (y)\subset K_r (x)$.
Das beweist dass $K_r (x)$ eine Ungebung ihrer ganzen Elementen ist, d.h. $K_r (x)$ ist offen. 
\end{Bem}
\begin{Sat}
  \begin{enumerate}
    \item $\varnothing$ und $\mb{R}^n$ sind offen
    \item Der Schnitt \ul{endlich vieler} offener Mengen ist auch offen.
    \item Die Vereinigung einer \ul{beliebigen} Familie offener Mengen ist auch offen.
  \end{enumerate}
\end{Sat}
\begin{Bew}
  \begin{enumerate}
    \item $\mb{R}^n$ trivialerweise offen, auch $\varnothing$
    \item Sei $x\in U\cap\dots\cap U_N$
      \[\forall i\in\left\{ 1,\dots,N \right\}\s \quad \exists r_i>0 \;\;\mbox{so dass}\;\; K_{r_i}(x)\subset U_i\]
      Sei $r=\min\left\{ r_i,\dots,r_N \right\} > 0$;
      \[\implies K_r(x)\subset U_i\quad\forall i\implies K_r(x)\subset U_1\cap\dots\cap U_N\]
    \item $\left\{ U_\lambda \right\}_{\lambda\in \Lambda}$. Sei $U=\bigcup_{\lambda\in\Lambda}U_\lambda$
      \[x\in U\implies x\in U_\lambda\s\text{f\"ur ein}\s\lambda\in\Lambda\]
      \[\implies \exists K_r(x)\subset U_\lambda\subset U.\]
  \end{enumerate}
\end{Bew}
\begin{Def}
  Ein topologischer Raum ist eine Menge $X$ und eine Menge $O$ von Teilmengen von $X$ so dass:
  \begin{enumerate}
    \item $\varnothing, X\in O$
    \item $U_1\cap\dots\cap_N\in O$ falls $U_i\in O$
    \item $\bigcap_{\lambda\in\Lambda}U_\lambda\in O$ falls $U_i\in O$
  \end{enumerate}
$O$ heisst die {\em Topologie}.
\end{Def}
\begin{Sat}
  Sei $(X,d)$ ein metrischer Raum. Wir definieren die entsprechende offene Kugel mit Mittelpunkt $x\in X$
und Radius $r>0$:
  \[K_r(x)=\left\{ y=X: d(x,y)<r \right\}\]
  Umgebungen und offene Mengen sind wie im Euklidischen Fall definiert. $O=\left\{ \text{offene Menge} \right\}$ definiert eine Topologie.
\end{Sat}

\subsection{Konvergenz}
Sei $\left\{ x_k \right\}_{k\in\mb{N}}$ $x_k\in\mb{R}$ $x_k=\left( x_{k1}, \cdots, x_{kn} \right)$
\begin{Def}
  Die Folge $\left\{ x_k \right\}$ konvergiert gegen $x_\infty\in\mb{R}^n$ falls
  \[\Limi{k}d(x_k,x_\infty)=0\]
  \[\left( \Limi{k}\Norm{x_k,x_\infty}=0 \right)\]
  Dann schreiben wir
  \[x_\infty=\Limi{k}x_k\]
\end{Def}
\begin{Sat}
  \[x_k\to x_\infty\iff x_{ki}\to x_{\infty_i}\s\forall i\in\left\{ 1,\cdots,n \right\}\]
\end{Sat}
\begin{Bew}
  \[\Norm{x_k - x_\infty}=\sqrt{\sum_{i=1}^n\left( x_{ki}-x_{\infty_i} \right)^2}\geq \abs{x_{ki}-x_{kinfty}}\geq 0\]
  \[\implies 0\leq \Limi{k}\abs{x_{ki}-x_{kinfty}}\leq \lim\Norm{x_k-x_\infty}=0\]
  \[\Norm{x_k-x_\infty}=\underbrace{\sqrt{\sum_{i=1}^n\underbrace{(x_{ki}-x_{\infty_i})^2}_{\to 0}}}_{\to 0}\leq \sum_{i=1}^n\abs{x_{ki}-x_{\infty_i}}\]
  \[\implies \Norm{x_k-x_\infty}\to 0\]
  Eine alternative Formulierung: $\Limi{k}x_k=\left( \Limi{k} x_{k1},\cdots,\Limi{k} x_{kn} \right)$
\end{Bew}
\begin{Bem}
  \[\forall \varepsilon>0 \exists N: \Norm{x_k-x_\infty}<\varepsilon\s\text{falls}\s k\geq N\]
  Für jede Umgebung $U$ von $x_\infty$ fast alle $x_k\in U$.
\end{Bem}
\begin{Def}
  Eine Folge $\left\{ x_k \right\}\subset\mb{R}^n$ heisst Cauchy falls:
  \[\forall \varepsilon>0\s\exists N: m,k\geq N\implies \Norm{x_k-x_m}<\varepsilon\]
\end{Def}
\begin{Lem}
  $\left\{ x_k \right\}\subset\mb{R}^n$ konvergiert genau dann, wenn $\left\{ x_k \right\}$ Cauchy ist.
\end{Lem}
\begin{Bew}
  $\left\{ x_k \right\}$ ist Cauchy $\implies$ $\left\{ x_{k_{\underbrace{i}_{\left\{ \text{fixiert} \right\}}}} \right\}$ Cauchy!
  \[\abs{x_{ki}-x_{m_i}}\leq\Norm{x_k-x_m}\]
  $\implies$ $\left\{ x_k \right\}$ ist eine Cauchyfolge $\stackrel{\text{Erstes Semester}}{\implies}$ $x_{ki}$ konvergiert $\stackrel{\text{Lemma 2}}{\implies}$ $x_k$ konvergiert. $x_k$ konvergiert $\implies$ Cauchyfolge
  \[x_\infty=\Limi{k} x_k\s\forall \varepsilon>0\s\exists N:\Norm{x_k-x_\infty}<\frac{\varepsilon}{2}\s\forall k\geq N\]
  \[k,m\geq N\s \Norm{x_k-x_m}\leq \Norm{x_k-x_\infty}+\Norm{x_\infty-x_m}\leq d(x_k,x_\infty)+(x_\infty,x_m)\]
  \[<\frac{\varepsilon}{2}+\frac{\varepsilon}{2}=\varepsilon\]
\end{Bew}
\begin{Bem}
  In einem metrischen Raum, Cauchy $\Leftarrow$ Konvergenz. Aber allgemein: Cauchy $\not\implies$ Konvergenz. Falls Cauchy $\implies$ Konvergenz, dann ist der metrische Raum vollständig.
\end{Bem}
\begin{Def}
  Eine Folge $\left\{ x_k \right\}\subset\mb{R}^n$ heisst beschränkt falls $\Norm{x_k}$ beschränkt ist.
\end{Def}
\begin{Sat}
  \begin{enumerate}
    \item Eine konvergente Folge ist beschränkt
    \item (Bolzano-Weierstrass) $\left\{ x_k \right\}$ beschränkt $\implies$ $\exists \left\{ x_{k_j} \right\}$ die konvergiert.
  \end{enumerate}
\end{Sat}
\begin{Bew}
  \[\left\{ x_k \right\}\s\text{beschränkt}\implies \left\{ x_{k1} \right\}_{k\in\mb{N}}\s\text{beschränkt}\]
  \[\implies \exists x_{k_j}: x_{k_j1}\to x_1\]
  Ich definiere $y_j=x_{k_j}$ $y_{j1}\to x_1$
  \[y_j\s\text{beschränkt}\implies\exists j_l: y_{j_l2}\to x_2\]
  \[z_l:=y_{j_l}\s\text{und}\s z_{l1}\to x_1, \s x_{l2}\to x_2\]
  \ldots $(n-2)$ Schritte. $w_r$ Teilfolge von $x_k$ mit $w_{ri}\to x_i$
  \[w_r\to(x_1,\cdots,x_n)\]
\end{Bew}
\subsection{Ein bisschen mehr Topologie}
\begin{Def}
  Eine Menge $G\subset\mb{R}^n$ heisst geschlossen falls $G^c(:=\mb{R}^n\setminus G)$ eine offene Menge ist.
\end{Def}
\begin{Bem}
  \[(A\cup B)^c = A^c\cap B^c\]
  \[(A\cap B)^c = A^c\cup B^c\]
\end{Bem}
\begin{Sat}
  \begin{enumerate}
    \item $\varnothing, \mb{R}^n$ sind abgeschlossen
    \item $G_1,\cdots,G_N$ abgeschlossen $\implies$ $G_1\cup G_2\cup \cdots\cup G_N$ abgeschlossen
    \item $\left\{ G_\lambda \right\}_{\lambda\in\Lambda}$ abgeschlossen $\implies$ $\bigcap_{\lambda\in\Lambda} G_\lambda$ abgeschlossen.
  \end{enumerate}
\end{Sat}
\begin{Sat}
  $G\subset\mb{R}^n$ $G$ ist abgeschlossen $\iff$ $\forall$ jede konvergente $\left\{ x_k \right\}\subset G$ gehört der Grenzwert zu $G$ (gilt auch für metrische Räume).
\end{Sat}
\begin{Bew}
  \begin{itemize}
    \item[$\Leftarrow$] Die rechte Eigenschaft gilt. Ziel: $G^c$ ist offen. Sei $x\in G^c$: das Ziel ist eine Kugel $K_r(x)\in G^c$ zu finden. Widerspruchsbeweis: $K_{\frac{1}{j}}(x)\not\subset G^c$, $j\in\mb{N}\setminus\left\{ 0 \right\}$
      \[\implies \exists x_j\in K_{\frac{1}{j}}(x)\cap G\implies \left\{ x_j \right\}\subset G\s\text{und}\s x_j\to x \]
      \[\left\{ x_j \right\}\subset G\s x_j\to x\s x\not\in G\]
      $\implies$ d.h. $G^c$ offen $\implies$ falls $\left\{ x_k \right\}\subset G$ und $x_k\to x$ dann $x\in G$
      Widerspruch: $G^c$ offen, aber $\exists \left\{ x_k \right\}\subset G$ mit Grenzwert $x\not\in G$, d.h. $x\in G^c$. Offenheit von $G^c$.
      \[\implies \exists K_r(x)\subset G^c\implies K_r(x)\cap=\varnothing\]
      d.h. $\exists N$ mit
      \[\Norm{x_N-x}<r\implies x_N\in K_r(x)\cap G\]
  \end{itemize}
\end{Bew}
\begin{Bsp}
  Eine offene Kugel ist nicht geschlossen.
  \[K_r(x)=\left\{ y:\Norm{y-x}<r \right\}\]
  Sei $\left\{ y_k \right\}\in K_r(x)$, (d.h. $\Norm{y_k-x}<r$) mit $y_k\to y$ und $\Norm{y-x}=r$.
\end{Bsp}
\begin{Def}
  Sei $\ol{K_r(x)}:=\left\{ y\in\mb{R}^n:\Norm{y-x}\leq r \right\}$.
\end{Def}
\begin{Ueb}
  $\ol{K_r(x)}$ ist abgeschlossen
\end{Ueb}
\begin{Def}
  $x\in\mb{R}^n$ ist ein Randpunkt von $M$ falls
  \[\forall K_r(x)\s\exists y\in K_r(x)\cap M\s\text{und}\s \exists z\in K_r(x)\cap M^c\]
\end{Def}
\begin{Def}
  Sei $M$ eine Menge in $\mb{R}^n$, dann ist der Rand von $M$
  \[\partial M=\left\{ x\in\mb{R}^n, \s\text{Randpunkt von}\s M \right\}\]
\end{Def}
\begin{Sat}
  $\partial M^c=\partial M$
  \begin{enumerate}
    \item $M\setminus \partial M$ ist die grösste offene Menge die in $M$ enthalten ist.
    \item $M\cup \partial \partial M$ ist die kleinste geschlossene Menge die $M$ enthält.
  \end{enumerate}
\end{Sat}
\begin{Bew}
  $M\setminus \partial M$ ist offen. 
  \[x\in M\setminus \partial M \implies x\in M\s\text{und}\s \exists K_r(x)\s\text{mit}\s K_r(x)\cap M^c=\varnothing\]
  \[\implies K_r(x)\subset M\]
  Sei $y\in K_r(x)$
  \[\implies \abs{y-x}=\rho<r\]
  \[\implies K_{r-\rho}(y)\subset K_r(x)\subset M\implies y\in M,y\not\in \partial M\]
  \[K_r(x)\subset M\setminus \partial M\]
  $x$ ist beliebig $\implies$ $M\setminus \partial M$ ist offen.\\
  Sei $A\subset M$ eine offene Menge. Das Ziel ist $A\subset M\setminus\partial M$. Sei $x\in A$. Ziel:($x\in M\setminus\partial M$) $x\not\in \partial M$.
  \[A\s\text{offen}\implies \exists K_r(x)\subset A\subset M\implies x\not\in \partial M\implies A\subset M\setminus\partial M\]
\end{Bew}

\subsection{Stetigkeit}
\begin{Def}
  Sei $f:\Omega_{\subset\mb{R}^n}\to\mb{R}^k$. $f$ ist stetig an der Stelle $x\in\Omega$ falls $\forall \left\{ x_k \right\}\subset\Omega$ mit $x_k\to x$.
  \[\Limi{k} f(x_k)=f(x)\]
\end{Def}
\begin{Lem}
  \label{l:1102282}
  Eine equivalente Definition:
  \[\forall \varepsilon>0\s\exists \delta>0: f\left( K_\delta(x)\cap \Omega \right)\subset K_\varepsilon(f(x))\]
\end{Lem}
\begin{Bew}
  $\varepsilon$-$\delta$ $\implies$ Folgendefinition. Sei $x_k\to x$. Ziel: $f(x_k)\to f(x)$
  \[\forall \varepsilon>0\s\exists \s\text{mit}\s \underbrace{\Norm{f(x_k)-f(x)}}_{\underbrace{d(f(x_k),f(x))}_{f(x_k)\in K_\varepsilon(f(x))}}<\varepsilon\s\forall k\geq N\]
  \[\exists \delta>0\s \underbrace{f(K_\delta(x))\subset K_\varepsilon(f(x))}\]
  \[\exists\s \Norm{x_k-x}<\delta\s k\geq N\]
  \[x_k \in K_\delta(x)\implies f(x_k)\in K_\varepsilon(f(x))\]
  Folgendefinition $\implies$ ($\varepsilon$-$\delta$)-Defintion. Widerspruchsannahme: 
  \[\exists \varepsilon>0: f(K_\delta(x)\cap \Omega)\not\subset K_\varepsilon(f(x))\s\forall \delta>0\]
  \[\implies\forall \delta>0\s\exists y_\delta\in K_\delta(x)\s\text{und}\s \Norm{f(y_\delta)-f(x)}\geq \varepsilon\]
  Nehmen wir $\delta=\frac{1}{j}$ und $x_j=\frac{y_1}{j}$
  \[\Norm{x_j-x}<\frac{1}{j}\s(\text{weil}\s x_j\in K_{\frac{1}{j}}(x))\]
  \[\Norm{f(x_j)-f(x)}=\Norm{f(y_{\frac{1}{j}}-f(x))}\geq\varepsilon\]
  $x_j\to x$ aber $f(x_j)\not\to f(x)$
\end{Bew}
\begin{Def}
  Die allgemeine Defintion der Stetigkeit für metrische Räume: Seien $(X,d)$ und $(Y,\ol d)$ zwei metrische Räume. Sei $f:X\to Y$. $f$ ist stetig an der Stelle $x$ falls:
  \[\forall \varepsilon>0\s\exists \delta>0\s\text{mit}\s d(y,x)<\delta\implies d(f(y),f(x))<\varepsilon\]
  \[\iff f(K\delta(x))\subset K_\varepsilon(f(x))\]
\end{Def}
\begin{Def}
  Eine $f:X\to Y$ heisst stetig falls $f$ stetig an jeder Stelle $x\in X$ ist.
\end{Def}
\begin{Sat}
  Sei $f:X\to Y$ ($(X,d), (Y\ol d)$ metrische Räume) Dann:
  \begin{enumerate}
    \item Die Stetigkeit in $x$ $\iff$ $\forall$ Umgebung $U$ von $f(x)$ ist $f^{-1}(U)$ eine Umgebung von $x$.
    \item Stetigkeit von $f$ $\iff$ $f^{-1}(U)$ ist offen $\forall U$ offen.
  \end{enumerate}
\end{Sat}
\begin{Bew}
  \begin{enumerate}
    \item
      \begin{itemize}
        \item Stetigkeit $\implies$ Umgebung.
          $U$ Umgebung von $f(x)\implies \exists \delta>0$ mit $K_\delta(f(x))\subset U$
          \[\implies \exists \varepsilon>0:f(K_\varepsilon(x))\subset K_\delta(f(x))\]
          \[\implies f^{-1}(U)\supset f^{-1}(K_\delta(f(x)))\supset K_\varepsilon(x)\implies f^{-1}(U)\s\text{Umgebung von}\s U\]
        \item Umgebung $\implies$ Stetigkeit. Sei $\delta>0$ $U=K_\delta(f(x))$. $U$ Umgebung von $f(x)$. $f^{-1}(U)$ ist eine Umgebung von $x$.
          \[\implies \exists\varepsilon>0:K_\varepsilon(X)\subset f^{-1}(U)\]
          \[\implies f(K_\varepsilon(x))\subset U=K_\delta(f(x))\]
      \end{itemize}
    \item
      \begin{itemize}
        \item 
          Stetigkeit $\implies$ offen. Sei $U$ offen $\iff$ $\forall y\in U$ ist $U$ eine Umgebung von $y$
          \[f^{-1}U\ni x\implies f(x)\in U\stackrel{\text{Stetigkeit in}\s x}{\implies} f^{-1}(U)\s\text{ist eine Umgebung von}\s x\]
          \[\implies f^{-1}(U)\s\text{ist offen}\]
        \item offen $\implies$ Stetigkeit an jedem $x\in X$. Sei $x\in X$, $\delta>0$, $K_\delta(f(x))$ ist offen
          \[f^{-1}(K_\delta(f(x)))\s\text{ist offen}\implies x\in f^{-1}(K_\delta(f(x)))\]
          \[\implies \exists \varepsilon>0: K_\varepsilon(x)\subset f^{-1}(K_\delta(f(x)))\]
          \[f(K_\varepsilon(x))\subset K_\delta(f(x))\]
      \end{itemize}
  \end{enumerate}
\end{Bew}
\subsection{lineare Abbildungen}
\begin{Def}
  Eine Abbildung $L:V\to W$ ($V$, $W$ Vektoren) heisst linear, falls
  \[L(\lambda_1v_1+\lambda_2v_2)=\lambda_1L(v_1)+\lambda_2L(v_2)\s\forall v_1,v_2\in V,\s\forall \lambda_1,\lambda_2\in\mb{R}\]
  \[L:\mb{R}^n\to\mb{R}^k\iff \exists\s\text{eine Matrix}\s L_{ij}:\]
  \[L(x)=\left( \sum^n_{j=1}L_{1j}x_j,\sum^n_{j=1}L_{2j}x_j,\cdots,\sum^n_{j=1}L_{kj}x_j \right)\]
\end{Def}
\begin{Def}
  Sei $L_{ij}$ eine Matrix die die lineare Abbildung $L:\mb{R}^n\to\mb{R}^k$ darstellt. Die Hilbert-Schmidt Norm von $L$ ist
  \[\Norm{L}_{\text{HS}}=\sqrt{\sum^k_{i=1}\sum^n_{j=1}L_{ij}^2}\]
\end{Def}
\begin{Bem}
  $\left\{ L:(L_{ij} n\times k\s\text{Matrixen} \right\}\sim \mb{R}^{nk}$ $\Norm{.}_{\text{HS}}$ ist die euklidische Norm.
\end{Bem}
\begin{Bem}
  Sei $L:\mb{R}^n\to\mb{R}^k$ eine lineare Abbildung und $x\in\mb{R}^n$. Dann $\Norm{L(x)}\leq\Norm{x}\Norm{L}_\text{HS}$.
\end{Bem}
\begin{Kor}
  Sei $L$ wie oben, dann ist $L$ stetig.
\end{Kor}
\begin{Bew}
  Sei $x_k\to x$. Ziel $L(x_k)\to L(x)$
  \[\Norm{L(x_k)-L(x)}=\Norm{L(x_k-x)}\leq\Norm{x_k-x}\Norm{L}_\text{HS}\to 0\]
  \[\implies \Norm{L(x_k)-L(x)}\to 0\]
  \[\implies \text{Stetigkeit}\]
\end{Bew}
\begin{Bew}
  Beweis von \ref{l:1102282}: $L(x)=y$
  \[\Norm{L(x)}^2=\sum^k_{i=1}y_i^2\]
  \[=\sum^k_{i=1}\left( \sum^n_{j=1}L_{ij}x_j \right)^2\stackrel{\text{Cauchy-Schwartz}}{\leq}\sum^k_{i=1}\left( \sum^n_{j=1}L_{ij}^2 \right)\left( \sum_{j=1}x_j \right)^2\]
  \[=\sum^k_{i=1}\sum^n_{j=1}L_{ij}^2\Norm{x}^2=\Norm{x}^2\left( \sum^k_{i=1}\sum^n_{j=1}L_{ij}^2 \right)\]
  \[\Norm{x}^2\Norm{L}^2_\text{HS}\implies \Norm{L(x)}\leq\Norm{x}\Norm{L}_\text{HS}\]
\end{Bew}
\begin{Def}
  Sei $L:V\to W$ eine lineare Abbildung wobei $(V,\Norm{.}_V)$ und $(W,\Norm{.}_W)$ zwei endlich-dimensionierte Vektorräume sind. Die Operatornorm von $L$ ist:
  \[\Norm{L}_{L(V,W)}:=\sup_{\Norm{v}_V\leq 1}\Norm{L(v)}_W\]
\end{Def}
\begin{Sat}
  $\Norm{.}_{L(V,W)}$ ist eine Norm und
  \[\Norm{L(v)}_W\leq \Norm{L}_{L(V,W)}\Norm{v}_V\]
  Deswegen: jede lineare Abbildung $L:V\to W$ ist stetig.
\end{Sat}
\begin{Bew}
  Der Kern ist die folgende Eigenschaft:
  \[\Norm{L}_{L(V,W)}<+\infty\]
  Wenn das gilt dann:
  \begin{enumerate}
    \item 
      \[\underbrace{\Norm{L}_{L(V,W)}}_\text{Kern}\s\text{und}\s\Norm{L}_{L(V,W)}=0\iff L=0\]
      $\Leftarrow$ einfach. Sei $\Norm{L}_{L(V,W)}=0$. Dann sei $v\in V$.
      \[v=0\implies L(v)=0\]
      \[v\neq 0\s z\frac{v}{\Norm{v_V}}\implies \Norm{z}_V=1\]
      \[\Norm{L(z)}_W\leq\sup_{\Norm{y}_V\leq 1}\Norm{L(v)}_W=0\]
      \[\implies L(z)=0\implies L(v)=L\left( \Norm{v}_V z \right)=\Norm{v}_VL(z)=0\]
    \item
      \[\Norm{\lambda L}_{L(V,W)}=\abs{\lambda}\Norm{L}_{L(V,W)}\]
      \[\Norm{\lambda L}_{L(V,W)}=\sup_{\Norm{y}_V\leq 1}\Norm{\lambda L(v)}_W\]
      \[=\sup_{\Norm{y}_V\leq 1}\abs{\lambda}\Norm{L(v)}_W\]
      \[=\abs{\lambda}\sup_{\Norm{y}_V\leq 1}\Norm{L(v)}_W\]
      \[=\abs{\lambda}\Norm{L}_{L(V,W)}\]
    \item
      \[\Norm{L+L'}_{L(V,W)}\]
      \[=\sup_{\Norm{y}_V\leq 1}\Norm{(L+L')(v)}_{L(V,W)}\]
      \[=\sup_{\Norm{y}_V\leq 1}\Norm{L(v)+L'(v)}_{L(V,W)}\]
      \[\leq\sup_{\Norm{y}_V\leq 1}\left( \Norm{L(v)}_W+\Norm{L'(v)}_W\right)\]
      \[\leq\sup_{\Norm{y}_V\leq 1}\Norm{L(v)}_W+\sup_{\Norm{y}_V\leq 1}\Norm{L'(v)}_W\]
      \[=\Norm{L}_{L(V,W)}+\Norm{L'}_{L(V,W)}\]
      Wenn $v_1,\cdots,v_n$ Basis für $V$, $w_1,\cdots,w_k$ Basis für $W$. Die lineare Abbildung $E_{ij}(v_i)=w_j$, $E_{ij}(v_l)=0$ falls $l\neq i$ ist eine Basis für $L(V,W)\implies L=\sum_{i,j}\lambda_{ij}E_{ij}$
  \end{enumerate}
\end{Bew}


\begin{Bem}
Aus der Definition von $\|\cdot\|_{L(V,W)}$ folgt
  \begin{equation}
    \label{e:1103022}
    \Norm{L(v)}_W\leq\Norm{L}_{L(V,W)}\Norm{v}_V \qquad \forall v\in V\, .
  \end{equation}
  Falls $\Norm{v}_V=1$, dann
  \[\Norm{L(v)} \leq \sup_{\Norm{v}_V\leq 1}\Norm{L(v)}_W = \|L\|_{L(V,W)}\]
F\"ur $v=0$ ist $L(v)=0$ und deswegen ist die Ungliechung \eqref{e:1103022} trivial.
Falls $\Norm{v}_V>0$,
  \[\tilde v:=\frac{v}{\Norm{v}_V}\implies\Norm{\tilde v}_V = \frac{\Norm{v}_V}{\Norm{v}_V}=1
  \implies \Norm{L(\tilde v)}_W\leq \Norm{L}_{L(V,W)}\]
  \[\implies \Norm{\frac{1}{\Norm{v}_V}L(v)}_W=\frac{1}{\Norm{v}_V}\Norm{L(v)}_W\]
  \[\implies\frac{\Norm{L(v)}_W}{\Norm{v}_V}\leq \Norm{L}_{L(V,W)}\]
In der Tat,  $\Norm{L}_{L(V,W)}$ ist die {\em optimale Konstante} in \eqref{e:1103022}. D.h.,
f\"ur jede $C<\Norm{L}_{L(V,W)}$ $\exists v\in V$ mit $\|L(v)\|_W > C \|v\|_V$.
\end{Bem}
\begin{Kor}
Seien $V$ und $W$ zwei endlichdimensionierte Vektorr\"aume und $L:V\to W$ eine
lineare Abbildung. Dann $L$ ist stetig.
\end{Kor}

\begin{Bew}
  $\varepsilon-\delta$ Stetigkeit. $v,\varepsilon>0$. Suche $\delta>0$ mit
  \[\Norm{v'-v}_V<\delta\implies\Norm{L(v')-L(v)}_W<\varepsilon\]
  Linearität von $L$
  \[\implies \Norm{L(v')-L(v)}_W=\Norm{L(v'-v)}_W\]
  und aus \eqref{e:1103022}
  \[\Norm{L(v'-v)}\leq\underbrace{\Norm{L}_{L(V,W)}\overbrace{\Norm{v'-v}_V}^{<\delta}}_{<\varepsilon}\]
  \[\implies \delta=\frac{\varepsilon}{\Norm{L}_{L(V,W)}}\]
  $\implies$ Ungleichung erfüllt.
\end{Bew}
\begin{Bem} Seien
  $V=\mb{R}^n$ und $\Norm{.}_V$ die euklidische Norm, $W=\mb{R}^k$ und $\|\cdot\|_W$ die euklidische Norm.
Dann \eqref{e:CS2} ist einfach die folgende Aussage:  
\[\Norm{L}_{L(V,W)}\leq \Norm{L}_{\text{HS}}\]
In Matrixdarstellung:
  \[\Norm{L}_\text{HS}=\sqrt{\sum_{i,j}L_{ij}^2}\]
  \[\Norm{L}_{L(V,W)}:=\sup_{\sum^n_{i=1}v_i^2\leq 1}\sqrt{\sum^k_{j=1}\left( \sum_{i=1}^nL_{ji}v_i \right)^2}\, .\]
In diesem Fall wir nutzen die Notation $\|\cdot\|_O$ f\"ur die Operatornorm.
\end{Bem}
\subsection{Mehr über stetige Funktionen}
\paragraph{Regeln} für stetige Funktionen
\subparagraph{Regel 1}
Seien $f:X\to Y$, $g:X\to Z$ zwei stetige Funktionen ($X$, $Y$ und $Z$ topologische R\"aume).
Dann
  \begin{itemize}
\item falls $Y=Z$ ein normierter Vektorraum ist, $f+g$ ist auch stetig;
\item falls $Y$ ein normierter Vektorraum und $Z=\mb{R}$, $gf$ ist auch stetig;
\item falls $Y=Z=\mb{R}^n$ auch
\[ x\mapsto f(x)\cdot g(x)=\sum_{i=1}^nf_i(x)g_i(x)\]
ist stetig.
  \end{itemize}
\begin{Bew} Wir geben den Beweis
  f\"ur den Fall $X\subset \mb{R}^m$. Der allgemeine Fall lassen wir als eine \"ubung.
In diesem Fall k\"onnen wir die Folgendefinition der Stetigkeit anwenden.
  \[\underbrace{\left\{ x^k \right\}}_{\subset X} x^k\to x\in X\]
  Stetigkeit von $f$ und $g$: $g(x^k)\to g(x)$, $f(x^k)\to f(x)$.
  \[g(x^k)=(g_1(x^k),\cdots,g_m(x^k))\]
  \[g(x)=(g_1(x),\cdots,g_m(x))\]
  \[f(x^k)=(f_1(x^k),\cdots,f_m(x^k))\]
  \[f(x)=(f_1(x),\cdots,f_m(x))\]
  \[(g+f)(x^k)=\left( g_1(x^k)+f_1(x^k),\ldots,g_m(x^k)+f_m(x^k) \right)\] 
\[ \to (g_1 (x)+f_1 (x), \ldots, g_m (x) + f_m (x)) = (g+f) (x)\, .\]
 D.h.
  \[x^k\to x\in X\implies (f+g)(x^k)\to (f+g)(x).\]
DIe anderen Regeln folgen aus \"ahnlichen Argumente.
\end{Bew}
\subparagraph{Regel 2}
Seien $X,Y,Z$ topologische Räume. Seien $f:X\to Y$ und $g:Y\to Z$ stetig. Dann
\[g\circ f:\underbrace{X\to Z}_{x\mapsto g(f(x))}\]
ist stetig.
\begin{Bew}
  Sei $U$ eine offene Menge in $Z$.
  \[(g\circ f)^{-1}(U)=\underbrace{f^{-1}(\underbrace{g^{-1}(U)}_{\text{offen}})}_{\text{offen}}\]
\end{Bew}
\begin{Def}
  Sei $f:X\to \mb{R}$.
  \[\Norm{f}=\sup_{x\in X}\Norm{f(x)}\]
  $f:X\to V$, $V,\Norm{.}_V$ normierter Vektorraum
  \[\Norm{f}=\sup_{x\in X}\Norm{f(x)}_V\]
\end{Def}
\begin{Bem}
  $X$ Menge, $V,\Norm{.}$ ein normierter Vektorraum.
  \[F:=\left\{ f:X\to V\right\} \s\text{mit}\s\Norm{f}\]
  Dann ist $F,\Norm{.}$ ist ein normierter Vektorraum.
\end{Bem}
\begin{Def}
  Eine Folge von Funktionen
  \[f^k:X\to V\]
  konvergiert gleichmässig gegen $f$ falls
  \[\Norm{f^k-f}\to 0\]
\end{Def}
\begin{Bem}
  $x\in X$
  \[\Norm{f^k(x)-f(x)}_V\leq \Norm{f^k-f}\]
  Folgerung $f^k$ konvergiert gleichmässig
  \[\implies f^k(x)\to f(x)\s\forall x\]
\end{Bem}
\begin{Sat}
  Sei $X$ ein metrischer Raum und $f^k:X\to V$ eine Folge die gleichmässig gegen $f$ konvergiert. Dann ist $f$ stetig.
\end{Sat}
\begin{Bew}
  Seien $x\in X$ und $\varepsilon>0$. Wir suchen $\delta>0$ so dass
  \begin{equation}\label{e:ziel}
d(x,y)<\delta\implies\Norm{f(x)-f(y)}<\varepsilon\, .
\end{equation}
Aus der gleichm\"assigen Konvergenz folgt die Existenz von  $N$ so dass
  \[\Norm{f-f^k}<\frac{\varepsilon}{3}\s\text{falls}\s k\geq N\]
  $f^N$ ist stetig: $\exists \delta>0$:
  \[d(x,y)<\delta\implies \Norm{f^N(x)-f^N(y)}<\frac{\varepsilon}{3}\]
 Siene nun $x,y$ s.d. $d(x,y)<\delta$. Dann
  \[\Norm{f(x)-f(y)}=\Norm{(f(x)-f^N(x))+(f^N(x)-f^N(x))+(f^N(y)-f(y))}_V\]
  \[\leq\Norm{f(x)-f^N(x)}_V+\Norm{f^N(x)-f^N(y)}_V+\Norm{f^N(y)-f(y)}_V\]
  \[<\Norm{f^N-f}+\frac{\varepsilon}{3}+\Norm{f^N-f}\]
  \[<\frac{\varepsilon}{3}+\frac{\varepsilon}{3}+\frac{\varepsilon}{3}=\varepsilon\]
\end{Bew}
\subsection{Kompakte Menge}
\begin{Def}
  Eine Menge $K\subset \mb{R}^n$ heisst kompakt falls $K$ abgeschlossen und beschränkt ($\iff \exists B_R(0):K\subset B_R(0)$) ist.
\end{Def}
\begin{Sat}\label{s:KiffFK}
  Sei $K\subset\mb{R}^n$.
 \begin{equation}\label{e:KiffFK}
K \s\text{kompakt}\s \iff\forall \left\{ x^j \right\}\subset K\s\exists \mbox{ Teilfolge} x^{j_l}
\mbox{ die gegen $x\in K$ konvergiert.}
\end{equation}
\end{Sat}
Die Eingeschaft in der rechten Seite von \eqref{e:KiffFK} heisst {\em Folgenkompatkheit}.
Der Satz \ref{s:KiffFK} ist also die folgende Behauptung:
\[
\mbox{falls $K\subset \mb{R}^n$ dann }\quad \mbox{$K$ kompakt} \iff
\mbox{$K$ folgenkompakt.}
\]
\begin{Bew} {\bf Kompaktheit $\implies$ Folgenkompaktheit.}
Sei $K$ kompakt und $\left\{ x^j \right\}\subset K$ eine Folge.
  \[x^j\in K\subset B_R(0)\implies \Norm{x^j}<R\]
Aus der Bolzano-Weiertsrass Eigenschaft
  $\exists x^{j_l}\to x\in\mb{R}^n$. Die abgeschlossenheit von $K$ $\implies$ $x\in K$. 

\medskip

{\bf Folgenkompaktheit $\implies$ Abgeschlossenheit und Beschränktheit.}
  \[\text{$K$ nicht abgeschlossen}\implies \exists x^j\subset K\s\text{mit}\s x^j\to x \not\in K\]
  \[\text{Folgenkompaktheit}\implies \exists x^{j_l}\to y\in K\]
  Widerspruch (weil $x = y$!).

  Sei $K$ nicht beschränkt.
  \[\forall j\in\mb{N}\s B_j(0)\not\supset K\]
  \[\exists x^j\in K\setminus B_j(0)\implies \Norm{x^j}\geq j\]
  Wenn $x^{j_l}\to x$. Aber das impliziert dass $\{\|x^{j_l}\|\}$ eine beschr\"ankte Folge
ist. (Wir wiederlegen das Argumebnt:
  \[\Norm{x^{j_l}}\leq \Norm{x}+\Norm{x^{j_l}-x}\]
  \[\Norm{x}\leq\Norm{x^{j_l}}+\Norm{x-x^{j_l}}\]
  \[\Abs{\Norm{x}-\Norm{x^{j_l}}}\leq\Norm{x-x^{j_l}}\]
  \[\implies\Norm{x^{j_l}}\to\Norm{x})\]
Aber $\Norm{x^{j_l}}=j_l\to+\infty$ $\implies$ Widerspruch.
\end{Bew}

\begin{Sat}
  $E\subset \mb{R}^n$
  \[E\s\text{kompakt}\iff E\s\text{folgenkompakt}\]
  d.h.
  \[\forall \left\{ x_k \right\} \subset E\s\exists\s\text{Teilfolge}\s \left\{ x_{k_l} \right\}\s\text{die gegen $x\in E$ konvergiert}\]
\end{Sat}
\begin{Def}
  (Überdeckungseigenschaft) Eine Teilmenge $E\subset\mb{R}^n$ besitzt die Überdeckungseigenschaft falls:
  \begin{itemize}
    \item $\forall$ Überdeckung $\left\{ U_\lambda \right\}_{\lambda\in\Lambda}$ von $E$ mit offenen Mengen $\exists$ endliche Teilüberdeckung.
      \[\left\{ U_\lambda \right\}_{\lambda\in\Lambda}\s\text{Überdeckung}\iff \bigcup_{\lambda\in\Lambda} U_\lambda\supset E\]
  \end{itemize}
  Teilüberdeckung ist eine Teilfamilie von $\left\{ U_\lambda \right\}$ die noch eine Überdeckung von $E$ ist.
\end{Def}
\begin{Bsp}
  Eine offene Kugel hat diese Eigenschaft nicht.
  \[\forall x\in K_r(0)\s\text{sei}\s K_{\frac{r-\Norm{x}}{2}}(x)=U_x\]
  \begin{enumerate}
    \item $\left\{ U_x \right\}_{x\in K_r(0)}$ ist eine Überdeckung von $K_r(0)$.
  \end{enumerate}
  Einfach weil $x\in U_x$! Sei $U_{x_1},\cdots,U_{x_N}$ eine beliebige endliche Teilfamilie. Sei 
  \[p:=\max_{i\in\left\{ 1,\cdots,N \right\}}\Norm{x_i}<r\]
  $\implies$ falls $\Norm{y}\geq \frac{\Norm{x_i}+r}{2}$ dann $y\not\in U_{x_i}$. So, wenn $\Norm{y}\geq \frac{p+r}{2}$ dann
  \[y\not\in U_{x_1}\cup\cdots\cup U_{x_N}\s\frac{p+r}{2}<r\]
  falls $\Norm{y}=\frac{p+r}{2}$, dann $y\in K_r(0)$. Mit einer geschlossenen Kugel ist das anders.
\end{Bsp}
\begin{Sat}
  Sei $E\subset\mb{R}^n$
  \[E\s\text{kompakt}\iff E\s\text{hat die Überdeckungseigenschaft}\]
\end{Sat}
\begin{Bsp}
  $E=\mb{R}^n$, $U_n=K_{n+1}(0)$.
  \[E\subset \bigcup_{n\in\mb{N}} U_n\]
  Aber $\forall N\in\mb{N}$
  \[\mb{R}^n=E\not\subset \bigcup_{n=0} U_n\]
\end{Bsp}
\begin{Bew}
  $\exists \left\{ x_i \right\}\subset E$ ohne konvergente Teilfolge in $E$ $\implies$ $E$ ist nicht kompakt $\implies$ Überdeckungseigenschaft gilt nicht. Zwei Möglichkeiten:
  \begin{enumerate}
    \item $\exists$ eine Teilfolge $\left\{ y_i \right\}\subset E$ $y_i\to y$ $y\not\in E$
    \item $\exists$ eine Teilfolge $\left\{ y_i \right\}\subset E$ $y_i\to +\infty$
  \end{enumerate}
  Beim ersten ist die Menge offen. 
  \[U_0:=\mb{R}^n\setminus \underbrace{\left( \left\{ y_1 \right\}\cup\left\{ y \right\} \right)}_{E\s\text{ist abgeschlossen}}\]
  Beim zweiten gilt:
  \[U_0=\mb{R}^n\setminus\underbrace{\left\{ x_i \right\}}_{F}\s\text{ist offen}\]
  \[U_n=U_0\cup \left\{ y_1,\cdots,y_{n-1} \right\} \s n\geq 0\]
  $U_n$ ist auch offen.
  \[\bigcup_{n=0}^{\infty}U_n= \begin{cases}
    \mb{R}^n\setminus \left\{ y \right\}& \text{im Fall 1}\\
    \mb{R}^n\setminus & \text{im Fall 2}
  \end{cases}\]
  Aber jede endliche Familie
  \[U_1\cup \cdots\cup U_n\not\supset E\]
  in beiden Fällen lassen wir unendlich viele Punkte weg. $E$ kompakt $\implies$ Überdeckungseigenschaft. $E$ ist beschränkt und abgeschlossen und sei $\left\{ U_\lambda \right\}_{\lambda\in\Lambda}$ eine Familie von offenen Mengen mit $E\subset\left\{ U_\lambda \right\}_{\lambda\in\Lambda}$. Wir decken die Menge $U$ mit Würfel:
  \[\left[k_1,k_1+1\right]\times \left[ k_2,k_2+1 \right]\times \cdots\times \left[ k_n,k_n+1 \right]\]
  \[W_1\cup\cdots\cup W_M\]
  Falls jedes $E\cap W_i$ mit einer endlischen Familie von $\left\{ U_\lambda \right\}$ überdeckt wird, dann finde ich eine endliche Überdeckung von $E$ wenn $N$ gross genug ist. So, angenommen dass die Überdeckungseigenschaft nicht gilt.
  \[\exists E_i:= E\cap W_i:\]
  \begin{enumerate}
    \item $\left\{ U_\lambda \right\}_{\lambda\in\Lambda}$ eine Überdeckung von $E_1$
    \item keine endliche Teilfamilie deckt $E_1$
  \end{enumerate}
  Teilen wir $W_i$ in $2^n$ Würfel mit Seite $\frac{1}{2}$
  \[\tilde W_1,\cdots,\tilde W_2\]
  \[\exists E_2:= E\cap \tilde W_i:\s\text{so dass die beiden Eigenschaften noch gelten}\]
  Induktiv
  \[E\supset E_1\supset E_2\supset\cdots\]
  jede $E_i\subset W^i$ Würfel mit Seite $2^{-i+1}$ und die beiden Eigenschaften gelten mit $E_j$ statt $E_i$.\\
  $\left\{ x_k \right\}\subset E$. $\left\{ x_k \right\}$ ist eine Cauchy-Folge. $j,k>i$, $x_k,x_j\in W$ mit Seite $w^{-i+1}$ $\Norm{x_j-x_k}\leq \sqrt{n}2^{-i+1}$
  \[\implies x_j\to x\in E\to x\in U\in \left\{ U_\lambda \right\}_{\lambda\in\Lambda}\implies K_r(x)\supset U\]
  \[x\in E, x\in E^i\s\forall i\implies x\in W^i\]
  \[\implies W^i\subset B_r(x)\subset U\]
  für $i$ gross genug
  \[\implies E_i\subset U\]
  $\implies$ wir haben eine endliche Teilüberdeckung $\left\{ U \right\}\subset\left\{ U_\lambda \right\}$ gefunden $\implies$ Widerspruch mit den beiden Eigenschaften.
\end{Bew}
\begin{Bem}
  $f$ stetig $\implies$ $f^{-1}(U)$ offen falls $U$ offen.
\end{Bem}
\begin{Bew}
  Sei $\left\{ U_\lambda \right\}$ eine Überdeckung (mit offenen Mengen) von $f(E)$, dann ist $\left\{ f^{-1}\left( U_\lambda \right) \right\}$ ein Überdeckung von $E$.
  \[\exists f^{-1}(U_{\lambda_1}),\cdots,f^{-1}(U_{\lambda_N}\s\text{Teilüberdeckung von $E$}\]
  $U_{\lambda_i},\cdots,U_{\lambda_N}$ ist eine Überdeckung von $f(E)$ $\implies$ $f(E)$ ist kompakt
\end{Bew}
\begin{Kor}
  Wenn $F:E\to \mb{R}$ stetig ist und $E\subset\mb{R}^n$ kompakt ist, besitzt $f$ ein Maximum und ein Minimum.
\end{Kor}
\begin{Bew}
  $f(E)\subset\mb{R}$ ist kompakt.
  \[s=\sup f(E)<+\infty\]
  \[\exists \left\{ x_k \right\}\subset f(E)\s\text{mit}\s x_k\to s\xRightarrow{\text{abgeschlossen}}s\in s\in f(E)\]
  \[\left( s-\frac{1}{k}\implies \exists x_k\in f(E)\s\text{mit}\s x_k>s-\frac{1}{k},x_k\leq s \right)\]
  $\implies$ $s$ ist ein Maximum.
\end{Bew}
\begin{Def}
  Das Intervallschachtelungsprinzip in $\mb{R}$. Sei $I_j$ eine Intervallschachtelung:
  \begin{enumerate}
    \item \[I_j=\left[ a_j,b_j \right]\]
    \item \[I_0\supset I_1\supset \cdots \supset I_j\supset_{j+1}\]
    \item \[b_j-a_j\to 0\]
  \end{enumerate}
  \[\implies \bigcap^\infty_{j=0}E_j\neq\varnothing\]
\end{Def}
\begin{Sat}
  Sei $E_j$ eine Folge von kompakten Mengen mit $E_j\supset E_{j+1}$ $\forall j$ ($E_0\subset\mb{R}^n$)
  \[\bigcap_{j=1}^\infty E_j\neq\varnothing\s\text{falls}\s E_j\neq\varnothing \s\forall j\]
\end{Sat}
\begin{Bew}
  Sei $E_j$ wie im Satz mit $E_j\neq\varnothing$, aber $\bigcap_{j=0}^\infty E_j=\varnothing$. Sei $U_j:=\mb{R}^n\setminus E_j\implies U_j$ ist offen. $\bigcup_{j=1}^\infty U_j=\mb{R}^n$ $\left\{ U_j \right\}$ ist eine Überdeckung von $E_0$. Aber $U_1\cup\cdots\cup U_N=U_N$ (weil $U_{j+1}\supset U_j$)
  \[U_N\not\supset E_N\neq \varnothing\s E_N\subset E_0\]
  Keine endliche Teilfamilie von $\left\{ U_j \right\}$ ist eine Überdeckung von $E_0$. Widerspruch wegen Kompaktheit von $E_0$.
\end{Bew}

\subsection{Differenzierbare Funktionen}
\paragraph{Erinnerung} $f:\mb{R}\to \mb{R}$ heisst differenzierbar in $a\in \mb{R}$ falls
\[f'(a)=\Limo{h}\frac{f(a+h)-f(a)}{h}\]
existiert. Was geschieht mit Funktionen von mehrere Variablen? Die ``Tangentensteigung'' hängt auch von der Richtung ab. D.h. Es gibt eine lineare Abbildung $L:\mb{R}^2\to\mb{R}$
\begin{Def}
  $f:U\to\mb{R}$, $U\subset\mb{R}^n$ offen, heisst differenzierbar in $a\in U$, falls
  \begin{equation}
    \label{e:1103091}
    \Limo{h}\frac{f(a+h)-f(a)-Lh}{\Norm{h}}=0
  \end{equation}
  wobei $L:\mb{R}^n\to\mb{R}$ eine lineare Abbildung ist.
\end{Def}
\begin{Bem}
  $n=1$: Es gilt $Lh = f'(a)h$
\end{Bem}
\begin{Bem}
  Die lineare Abbilung $L$ in \ref{e:1103091} ist eindeutig definiert. Annahme $L'\neq L$ erfüllt die Bedungung. Sei $v\in\mb{R}^n$ mit $\Norm{v}=1$. Es gilt:
  \[(L-L')(v)\stackrel{\text{linear und}\s \Norm{v}=1}{=}\lim_{t\downarrow 0}\frac{(L-L')(tv)}{\Norm{tv}}\stackrel{\ref{e:1103091}\s h=tv}{=}\implies L=L'\]
\end{Bem}
\begin{Bem}
  Wir können \ref{e:1103091} auch anders beschreiben:
  \[f(a+h)-f(a)=Lh+\underbrace{R(h)}_{\text{Restglied}}\]
  Dann gilt
  \begin{equation}
    \label{e:1103092}
    \ref{e:1103091} \iff \Limo{h}\frac{R(h)}{\Norm{h}}=0
  \end{equation}
\end{Bem}
\begin{Def}
  $L$ heisst Differential von $f$ in $a$. Man schreibt $\md f(a)$. Sei nun $\left\{ e_1,\cdots,e_n \right\}$ die Standardbasis $\mb{R}^n$, $h=(h_1,\cdots,h_n)\in\mb{R}^n$
  \[\implies \md f(a)h=\md f(a)\left( \sum_{i=1}^kh_i-e_i \right)=\sum_{i=1}^nh_i\md f(a)e_i\]
\end{Def}
\begin{Def}
  \[f'(a)=(\md f(a)e_1,\cdots,\md f(a)e_n)\]
  heisst Ableitung
\end{Def}
\begin{Def}
  \[Tf(x,a)=f(a)+f'(a)(x-a)\s\text{(Ebene (tangential))}\]
  lineare Approximation
\end{Def}
\begin{Sat}
  $f$ differentierbar in $a$ $\implies$ $f$ ist stetig in $a$
\end{Sat}
\begin{Bew}
  \[\Abs{f(a+b)-f(a)}=\Abs{\md f(a)h+R(h)}\leq\Abs{\md f(a)}+\underbrace{\Abs{R(h)}}_{\to 0}\]
\end{Bew}
\begin{Bsp}
  $f(x)=Ax+b$, $A\in M_a(1,n,\mb{R})$, $b\in\mb{R}$
  \begin{Beh}
    $Lh:=ah$ ist linear
    \[\md f(a)h=Ah, \s f'(a)=A\]
  \end{Beh}
  \begin{Bew}
    \[f(a+h)-f(a)-Lh=\not{R(h)}=0\]
  \end{Bew}
\end{Bsp}
\begin{Bsp}
  $f(x):=x^TAx$, $A=(a_{ij})\in\Sym(n,\mb{R})$
  \[f(a+h)-f(a)-\underbrace{2a^TAh}_{\md f(a)h}+\underbrace{h^TAh}_{R(h)}\]
  $Lh:=2a^TAh$ ist linear (in $h$), $R(h)=h^TAh$ ($=\sum h_ia_{ik}h_l$)
  z.z.: $\Abs{Rh}\leq\sum^h_{i,j=1}\Abs{a_{ij}}\Norm{h}_{\infty}^2$, d.h. $\frac{R(h)}{\Norm{h}}\to 0$ (falls $\Norm{h}\to 0$)
\end{Bsp}
\paragraph{Ziel}
Wir wollen $\md f(a)h$ berechnen. sei $t\in\mb{R}$
\[f(a+th)=f(a)+\md f(a)th+R(th)\]
\begin{equation}
  \label{e:1103094}
  \implies \md f(a)h=\Limo{t}\frac{f(a+th)-f(a)}{t}
\end{equation}
\begin{Def}
  $f:U\to\mb{R}$, $a\in U$. Die Richtungsableitung von $f$ in Richtung $h\in\mb{R}^n$ ist der Grenzwert (falls er existiert)
  \[\partial_nf(a):=\Limo{t}\frac{f(a+th)-f(a)}{t}\]
  Die Ableitungen in Richtung $e_1,\cdots,e_n$ heissen partielle Ableitungen in $a$. Wir schreiben
  \[\partial_{ei}f(a)=\partial_if(a)=\frac{\partial f}{\partial x_i}(a)=f_{xi}(a)\]
\end{Def}
\begin{Bem}
  Wir haben \ul{nicht} vorausgesetzt, dass $f$ differenzierbar ist in $a$!
\end{Bem}
\begin{Sat}
  Sei $f$ in $a$ differenzierbar. Dann existieren die Richtungsableitungen in jede Richtung. Insbesondere existieren die aprtiellen Ableitungen. Es gelten:
  \begin{equation}
    \label{e:1103095}
    \md f(a)h=f'(a)h=\partial_nf(a)=\sum_{i=1}^n\partial_if(a)h_i
  \end{equation}
  und
  \[f'(a)=\left( \partial_1f(a),\cdots,\partial_nf(a) \right)\]
\end{Sat}
\begin{Bew}
  Existenz der Richtungsableitung oke (Herleitung von \ref{e:1103094})
\end{Bew}
\paragraph{Frage}
Wie berechnet man die partielle Ableitung effizient? Es gilt:
\[\partial_if(a)=\Limo{t}\frac{f(a+t_{ei})-f(a)}{t},\s a=(a_1,\cdots,a_n)\]
\[g_i(x):=f(a_1,\cdots,a_{i-1},x,a_{i+1},\cdots,a_n)\]
\[\partial_if(a)=\Limo{t}\frac{g(a_i+t)-f(a_i)}{t}=g'(a_i)\]
\begin{Bsp}
  \[f(x,y):=\sin(2x)e^{3y}\]
  \[\partial_xf=2e^{3y}\cos(2x)\]
  \[\partial_yf=\sin(2x)e^{3y}3\]
\end{Bsp}
\paragraph{Frage}
Wann folgt aus der Existenz der partiellen Ableitung (Richtungsableitung) die Differenzierbarkeit?
\begin{Bsp}
  \[f(x,y)= \begin{cases}
    \frac{x^2y}{x^2+y^2}&(x,y)\neq(0,0)\\
    0&(x,y)=(0,0)
  \end{cases}\]
  Es gilt: $f(tx,ty)=tf(x,y)$, d.h. der Graph von $f$ besteht aus Geraden durch $0$, für $h=(h_1,h_2)\in\mb{R}^2$
  \[\implies \partial_hf(0,0)=\Limo{t}\frac{f(th_1,th_2)-f(0,0)}{k}=\Limo{t}\frac{t}{t}f(h_1,h_2)=f(h_1,h_2)\]
  \[\implies \partial f(0,0)=f(h_1,h_2)\]
  \[\partial_{e_1}f(0,0)=f(1,0)=0\]
  \[\partial_{e_2}f(0,0)=f(0,1)=0\]
  \paragraph{Annahme}
  $f$ ist in $(0,0)$ differenzierbar
  \[\xRightarrow{\text{aus}\s\ref{e:1103095}}\underbrace{\partial_nf(0,0)}_{=\md f(a)h=0}=\underbrace{\partial_1f(a)}_{0}(h_1)+\underbrace{\partial_2f(a)}_0(h_2)=0\]
  \[\implies \md f(a)=0\]
  \paragraph{Test}
  $L=0$
  \[\frac{f(h_1,h_1)-\overbrace{f(a_0)-L(h_1,h_1)}}{\Norm{(h_1,h_1)}_\infty}=\frac{h_1^3}{2h_1^2\Abs{h_1}}\to\pm\frac{1}{2}\]
  $\implies$ $f$ ist in $(0,0)$ nicht differenzierbar.
\end{Bsp}

\subsubsection{Das Differenzial}
$f:\Omega\to\mb{R}$, $\Omega\subset\mb{R}^n$, Umgebung von $x$.
\[f\s\text{diff in $x$}\iff \exists L:\mb{R}^n\to\mb{R}\s\text{linear s.d.}\]
\begin{equation}
  \label{e:1103141}
  \lim_{h\downarrow 0}\frac{f(x+h)-f(x)-L(h)}{\Norm{h}}=0
\end{equation}
\[\lim_{h\downarrow 0}G(h)=0\iff \forall \varepsilon>0\exists \delta>0\s\Norm{h}<\delta\implies\abs{G(h)}<\varepsilon\]
\[\iff \forall h_k=0\s G(h_k)\to 0\]
Wenn $f$ differenzierbar ist und \ref{e:1103141} erfüllt, heisst $L$ das Differential von $f$.
\[L=\md f\]
\[\md f_x \s \text{das Differential an der Stelle $x$}\]
\subsubsection{Richtungsableitung}
$x\in\Omega$, $h\in\mb{R}^m$, $g(t)=f(x+th)$ (wohldefiniert für $\abs{t}$ klein)
\[\partial_n f(x)=g'(0)=\Limo{t}\frac{f(x+th)-f(x)}{t}\]
\subsubsection{Partielle Ableitung}
$(x_1,\cdots,x_n)$ Kond. in $\mb{R}^n$ $y\in \Omega$ so dass $\Omega$ eine Umgebung von $y$ ist
\[\Part{f}{x_i}(y)\left( =\partial_{x_i}f(y) \right)=\Limo{t}\frac{y_1,\dots,y_i+t,\dots,y_n-f(y)}{t}\]
Falls $e_i=(0,\dots,0,\underbrace{1}_i,0,\dots,0)$
\[=\Limo{t}\frac{f(y+te_i)-f(y)}{t}=\partial_{e_i}f(y)\]
\begin{Sat}
  (Hauptkriterium der Differenzierbarkeit) Sei $f:U\to \mb{R}$ und $U$ eine Umgebung von $y$. Falls $\Part{f}{x_1},\dots,\Part{f}{x_n}$ \ul{in $U$} existieren und stetig in \ul{in $y$} sind, dann ist $f$ in $y$ differenzierbar.
\end{Sat}
\begin{Bew}
  $h=(h_1,\dots,h_n)\in\mb{R}^n$
  \[L(h)=\sum^n_{i=1}\Part{f}{x_i}(y)h_i\]
  \paragraph{Ziel} $L$ ist das Differential von $f$
  \[\Limo{h}\frac{f(x+h)-f(x)-L(h)}{\Norm{h}}=0\]
  \[f(x+h)-f(x)=f(x+(h_1,\dots,h_n))-f(y+(h_1,\dots,h_{n-1},0)+f(y+(h_1,\dots,h_{n-1}, 0)-\dots\]
  \[+\dots\s(\text{$i$te Zeile})\]
  \begin{equation}
    \label{e:1103143}
    +f(y+(k,0,\dots,0))-f(y)
  \end{equation}
  $i\in\left\{ 1,\dots,n \right\}$
  \[g(t))=f(y+(h_1,\dots,h_{i-1},th_i,0,\dots,0)\]
  \[\text{$i$te Zeile}=g_i(1)-g_i(0)=g_i'(\xi_i)\s\xi\in \left[ 0,1 \right]\]
  \[g_i'(t)=\Limo{\varepsilon}\frac{g_i(t+\varepsilon)-g_i(t)}{\varepsilon}\]
  \[=h_i\Limo{\varepsilon}\frac{f(y_1+h_1,\dots,y_{i-1},y_i+(t+\varepsilon)h_i,y_{i+1}\dots,y_n)-f(y_1+h_1,\dots,y_i+th_i,\dots,y_n}{\varepsilon h_i}\]
  \[=h_i\Part{f}{x_i}\left( y_1+h_i,\dots,y_i+th_1,y_{i+1},\dots,y_n \right)\]
  \[\text{$i$te Zeile}=h_i\Part{f}{x_i}(y_1+h_1,\dots,y_{i-1}h_{i-1},y_i+\xi_ih_i,y_{i+1},\dots,y_n)\]
  \[\zeta_i=\left( h_1,\dots,h_{i-1},\xi h_i,0,\dots,0 \right)\]
  \begin{equation}
    \label{e:1103144}
    =h_i\Part{f}{x_i}(y+\zeta_i)
  \end{equation}
  \ref{e:1103144} in \ref{e:1103143}:
  \begin{equation}
    \label{e:1103145}
    f(y+h)-f(y)=\sum_{i=1}^nh_i\Part{f}{x_i}(y+\zeta_i)
  \end{equation}
  \[f(x+h)-f(x)-L(h)\]
  \begin{equation}
    \label{e:1103146}
    =\sum_{i=1}^nh_i\left( \Part{f}{x_i}(y+\zeta_i)-\Part{f}{x_i}(y) \right)
  \end{equation}
  \[\frac{\abs{f(x+h)-f(x)-L(h)}}{\Norm{h}}\]
  \begin{equation}
    \label{e:1103147}
    \stackrel{\ref{e:1103146}}{\leq}\sum_{i=1}^n\frac{\abs{h_i}\abs{\Part{f}{x_i}(y+\zeta_i)-\Part{f}{x_i}(y)}}{\Norm{h}}
  \end{equation}
  Wenn $\Norm{h}\to 0$, $\Norm{\zeta}\to 0$. Die Stetigkeit von $\Part{f}{x_i}$ in $y$ impliziert
  \[\Part{f}{x_i}(y+\zeta_i)\to\Part{f}{x_i}\]
  Die rechte Seite von \ref{e:1103147} $\to 0$ wenn $h\to 0$ $\implies$ \ref{e:1103142}.
\end{Bew}
\begin{Def}
  Der Gradient an der Stelle $x_0$ist der Vektor
  \[\left( \Part{f}{x_1}(x_0),\dots,\Part{f}{x_i}(x_0) \right)=\nabla f(x_0)\]
\end{Def}
\begin{Bem}
  \[df|_{x_0}(h)\left( \partial_nf(x_0) \right)=\sum_{i=1}^nh_i\Part{f}{x_i}(x_0)\]
  \[\left(\seq{\nabla f(x_0), h}\right)=\nabla f(x_0)h\]
  \[\abs{\partial_nf(x_0)}\stackrel{\text{Cauchy-Schwartz}}{\leq}\Norm{\nabla f(x_0)}\Norm{h}\]
  Falls $\Norm{h}=1$, dann
  \[\abs{\partial_nf(x_0)}\leq\Norm{\nabla f(x_0)}\]
  Fall $\Norm{\nabla f(x_0)}\neq 0$, wenn wir 
  \[K=\frac{\nabla f(x_0)}{\Norm{\nabla f(x_0)}}\]
  bekommen wir $\Norm{K}=1$ und
  \[\partial_Kf(x_0)=\Norm{\nabla f(x_0)}\]
  Deswegen:
  \[K=\frac{\nabla f(x_0)}{\Norm{\nabla f(x_0)}}\]
  ist die Richtung der maximalen Steigung und
  \[\Norm{\nabla f(x_0)}\]
  ist die maximale Steigung.
\end{Bem}
\subsection{Rechenregeln}
\begin{Sat}
  Sei $U$ eine Umgebung von $x\in\mb{R}^n$ und $f,g:U\to\mb{R}$ in $x$ differenzierbar. Dann sind $f+g$ und $fg$ auch differenzierbar in $x$ und
  \[\md (f+g)|_x=\md f|_x+\md g|x\]
  \[\md(fg)=f(x)\md g|x+g(x)\md f|_x\]
  Falls $f(x)\neq 0$ ist auch $\frac{1}{f}$ in $x$ differenzierbar
  \[\md \left( \frac{1}{f} \right)|_x=-\frac{1}{(f(x))^2}\md f|_x\]
\end{Sat}
\begin{Kor}
  $g(x)\neq 0$, dann
  \[\md \left( \frac{f}{g} \right)|_x=\frac{1}{g(x)}\md f|_x-\frac{f(x)}{g(x)^2}\md g|_x\]
  \[=\frac{g(x)\md f|_x-f(x)\md g|_x}{g(x)^2}\]
\end{Kor}
\begin{Bew}
  Das Ziel ist eine lineare Abbildung $L$ zu finden so dass
  \[\Limo{h}\frac{\frac{1}{f(x+h)-\frac{1}{f(x)}-L(h)}}{\Norm{h}}\]
  \[L=-\frac{1}{f(x)^2}\md f|_x\]
  \[\Limo{h}\frac{\overbrace{\frac{1}{f(x+h)-\frac{1}{f(x)}-\frac{1}{f(x)^2}(h)\md f|_x(h)}}^A}{\Norm{h}}=\frac{B+C}{\Norm{h}}\]
  \[\frac{1}{f(x+h)}-\frac{1}{f(x)}=\frac{f(x)-f(x+h)}{f(x)f(x+h)}\]
  $f(x+h)\neq 0$ falls $\Norm{h}$ klein genug
  \[\frac{f(x+h)-f(x)-\md f|_x(h)}{\Norm{h}}\to 0\]
  \[A=\left[ \frac{-(-f(x)+f(x+h))}{f(x)f(x+h)} \frac{\md f|_x(h)}{f(x)f(x+h)}\right]=C\]
  \[+\frac{-\md f|_x(h)}{f(x)f(x+h)} + \frac{\md f|_x(h)}{f(x)^2}=B\]
  \[\frac{B}{\Norm{h}}=-\frac{1}{f(x)f(x+h)}\underbrace{\frac{f(x+h)-f(x)-\md f|_x(h)}{\Norm{h}}}_{\to 0}\]
  \[\Limo{h} f(x+h)=f(x)\neq 0\]
  \[\Limo{h}\frac{B}{\Norm{h}}=0\]
  Diff von $f$ für $\Norm{h}\to 0$
  \[\frac{C}{\Norm{h}}=\underbrace{\frac{\md f|_x(h)}{\Norm{h}}}_{\text{ist beschränkt}}\frac{1}{f(x)}\underbrace{\left( \frac{1}{f(x)}-\frac{1}{f(x+h)} \right)}_{\to 0}\]
  Sei $L=\md f|_x$ und $\Norm{L}_O$ ihre Operatornorm
  \[\abs{\md f|_x(h)}=\abs{L(h)}\leq\Norm{K}_O\Norm{h}\]
  \[\implies \frac{\abs{\md f|_x(h)}}{\Norm{h}}\leq\Norm{L}\]
\end{Bew}

\begin{Def}
  Eine Kurve ist eine Abbildung $\gamma:[a,b]\to\mb{R}^n$ (d.h. $\forall t\s\gamma(t)\in\mb{R}^n$ 
  \[\gamma(t)=(\gamma_1(t),\cdots,\gamma_n(t))\]
  deswegen $t\to\gamma_i(t)\in\mb{R}$. Die Kurve $\gamma$ heisst differenzierbar wenn jede $\gamma_i$ differenzierbar ist.
  \[\gamma'=(\gamma'(t),\cdots,\gamma_n'(t))\]
\end{Def}
\begin{Sat}(Kettenregel 1. Version) Sei $f:U\to\mb{R}$ mit $U$ Umgebung von $x$ und $f$ differenzierbar in $x$. Sei $\gamma:[a,b]\to U$ eine differenzierbare Kurve mit $\gamma(t_0)=x$. Sei $g=f\circ \gamma$
  \[g(t)=f(\gamma(t))\]
  Sei $g$ in $t_0$ differenzierbar. Dann
  \[g'(t_0)=\md|_\gamma(t_0)(\dot\gamma(t_0))=\seq{\nabla f(\gamma(t_0)),\dot\gamma(t_0)}\]
  % TODO
\end{Sat}
\begin{Bew}
  Das Ziel:
  \[\Limo{h} \frac{g(t_0+h)-g(t_0)-h\left[ \md f|_{\gamma(t_0)}(\dot\gamma(t_0)) \right]}{h}=0\]
  \begin{equation}
    \label{e:1103161}
    R(h)=g(t_0+h)-g(t_0)-g(t_0)-h\left[ \md f|_{\gamma(t_0)}(\dot\gamma(t_0)) \right]
  \end{equation}
  \begin{equation}
    \label{e:1103162}
    \Limo{h}\frac{R(h)}{h}=0
  \end{equation}
  Neue Notation
  \[\ref{e:1103162}\iff R(h)=o(h)\]
  \[x_0=\gamma(t_0)\]
  Annahmen: Differenzierbarkeit von $f$
  \[\Limo{k}\frac{f(x_0+k)-f(x_0)-\md f|_{x_0}(k)}{\Norm{k}}\left( =\frac{r(k)}{\Norm{k}} \right)=0\]
  \[\left( r(k)=o(\Norm{k}) \right)\]
  Differenzierbarkeit von $\gamma$:
  \[\Limo{k}\frac{\gamma(x_0+k)-\gamma(x_0)-\md h\gamma'|_{x_0}(k)}{h}\left( =\frac{p(k)}{\Norm{k}} \right)=0\]
  \[p(h)=o(h)\]
  \[\gamma(t_0+h)=\gamma(t_0)+k\left( =\gamma(t_0+h)-\underbrace{\gamma(t_0)}_{x_0} \right)\]
  \[g(t_0+h)-g(t_0)=f(\gamma(t_0+h))-g(\overbrace{\gamma(t_0)}^{x_0})\]
  \[=f(\gamma(t_0)-k)-f(\gamma(t_0))=\md f|_{\gamma(t_0)}(k)+r(k)\]
  \[=\md f|_{\gamma(t_0)}(\gamma(t_0+h)-\gamma(t_0))+r(k)\]
  \[=\md f|_{\gamma(t_0)}(h\dot\gamma(t_0)+p(h))+r(k)\]
  \[\stackrel{\text{Linearität von $\md f$}}{=}h\md f|_{\gamma(t_0)}(\dot\gamma(t_0))+\md f|_{\gamma(t_0)}(p(h))+r(k)\]
  \[g(t_0+h)-g(t_0)-h\md f|_{\gamma(t_0)}(\dot\gamma(t_0))\]
  \[=f|_{\gamma(t_0)}(p(h))+r(\gamma(t_0+h)-\gamma(t_0))=R(h)\]
  \[\abs{R(h)}\leq \frac{\underbrace{\overbrace{\abs{f|_{\gamma(t_0)}(p(h))}}^{L}+r(\gamma(t_0+h)-\gamma(t_0))}}{\Norm{h}}\]
  \[\leq\Norm{L}\frac{p(h)}{\Norm{h}}+\frac{r(\gamma(t_0+h)-\gamma(t_0)}{\Norm{h}}\]
  Ziel
  \[\Limo{h}\frac{r(\gamma(t_0+h)-\gamma(t_0)}{\abs{h}}\]
  Falls 
  \[r(\gamma(t_0+h)-\gamma(t_0)=0\]
  dann $r(0)=0$. Wenn 
  \[r(\gamma(t_0+h)-\gamma(t_0)\neq 0\]
  \[=\frac{r(\gamma(t_0+h)-\gamma(t_0)}{\Norm{\gamma(t_0+h)-\gamma(t_0)}}\frac{\Norm{t_0+h)-\gamma(t_0)}}{\abs{h}}\]
  \[\frac{r(\gamma(t_0+h)-\gamma(t_0)}{\Norm{\gamma(t_0+h)-\gamma(t_0)}}=\frac{r(k)}{\Norm{k}}\to 0\]
  \ldots wenn $\Norm{k}\to 0$ und $h\to 0$. Es fehlt die Beschränktheit von
  \[\frac{\Norm{t_0+h)-\gamma(t_0)}}{\abs{h}}\]
  \[\frac{t_0+h)-\gamma(t_0)}{h}-\frac{\not h\dot \gamma(t_0)}{\not h}=\frac{p(h)}{h}\]
  \[\frac{\gamma(t_0+h)-\gamma(t_0)}{h}=\underbrace{\dot\gamma(t_0)}_{\text{konstant}}+\underbrace{\frac{p(h)}{h}}_{\to 0}\]
  Deswegen
  \[\Limo{h}\frac{\Norm{\gamma(t_0+h)-\gamma(t_0)}}{\abs{h}}=\Norm{\dot\gamma(t_0)}\]
  \[\implies \frac{\abs{R(h)}}{\Norm{h}}\to 0\]
  $\implies$ Differenzierbarkeit und Kettenregel!
\end{Bew}
\begin{Bem}
  Der Gradient ist orthogonal zur Niveaumenge (Höhenlinien).
\end{Bem}
\begin{Def}
  Sei $\gamma:[a,b]\to U$ eine differenzierbare Kurve, $U$ offen. Sei $f:U\to\mb{R}$ differenzierbar. Wenn $f(\gamma(t))=c_0$ ($c_0$ hängt nicht von $t$ ab). Dann
  \[\nabla f(\gamma(t))\bot \dot\gamma(t)\]
  d.h.
  \[\seq{\nabla f(\gamma(t)), \dot\gamma(t) }=0\]
  \[0=g'(t)=(f(\gamma(t)))'\stackrel{\text{Kettenregel}}{=}\seq{\nabla f(\gamma(t)), \dot\gamma(t)}\]
\end{Def}
\subsection{Mittelwertsatz und Schrankensatz}
  $f:[a,b]\to\mb{R}$, $\xi\in ]a,b[$
  \[f(b)-f(a)=f'(\xi)(b-a)\]
  Sei nun:
  \[f:U\mapsto\mb{R}\s\text{differenzierbar auf $U$}\]
  \[x,y\in U\s\text{so dass das Segment}\s[x,y]\subset U\]
  Was ist ein Segment? Gerade durch $x$ und $y$
  \[\left\{ x+t(y-x)|t\in \mb{R} \right\}\]
  \[\left[ [x,y] \right]=\left\{ x+t(y-x)|t\in \left[ 0,1 \right] \right\}\]
  \[\gamma(t):= x+t(y-x)\]
  \[f(y)-f(X)=\left( x_1+t(y_1-x_1),\cdots,x_n+t(y_n-x_n) \right)\]
  $\gamma$ ist differenzierbar.
  \[g=f\circ \gamma g(t)=f(\gamma(t))\]
  \[g(1)-g(0)=g'(\tau)\s\text{für}\s \tau\in ]0,1[\]
  \[f(y)-f(x)=\md f|_{\gamma(\tau)}(\dot\gamma(\tau))\]
  \[\dot\gamma(\tau)=(\gamma_1'(\tau),\cdots,\gamma_n'(\tau))\]
  \[=(y_1-x_1,\cdots,y_n-x_n)=y-x\]
  \[\gamma(\tau)=\xi\]
  \begin{equation}
    \label{e:1103163}
    f(y)-f(x)=\md f|_\xi(y-x)=\partial_{y-x}f(\xi)
  \end{equation}
\begin{Sat}
  (Mittelwertsatz) $U$ offen, $[x,y]\subset U$ und $f:U\to\mb{R}$ differenzierbar. Dann $\exists \xi\in ]x,y[$ so das \ref{e:1103163} gilt.
\end{Sat}
\begin{Def}
  $U$ sternförmig: wenn $0\in U$ und $[x,0]\subset U$ $\forall x\in U$. Sternförmig mit Zentrum $x_0$ wenn $x_0\in U$ $[x,x_0]\subset U$ $\forall x\in U$
\end{Def}
\begin{Sat}
  (Schrankensatz) Sei $U$ eine offene Menge, die sternförmig ist und $f:U\to\mb{R}$ eine differenzierbare Funktion mit
  \[\sup_{x\in U}\Norm{\md f|_x}_{O}=S<\infty \left( =\sup_{x\in U}\Norm{\nabla f(x)} \right)\]
  Dann
  \[\abs{f(x)-f(0)}\leq S\Norm{x}\]
  Wenn $U$ konvex ist, d.h. das Segment $[x,y]\subset U$ $\forall x,y\in U$, dann
  \[\abs{f(x)-f(y)}\leq S\Norm{y-x}\]
\end{Sat}
\begin{Def}
  $f:\underbrace{K}_{\in \mb{R}^n}\to\mb{R}$ heisst Lipschitz wenn $\exists L[0,+\infty[$ so dass
  \[\abs{f(y)-f(x)}\leq L\Norm{y-x}\s\forall x,y\in K\]
  Wenn $f:(X,d)\to\mb{R}$ Lipschitz bedeutet die Existenz eines $L$ so dass
  \[\abs{f(y)-f(x)}\leq L d(y-x)\s\forall x,y\in K\]
\end{Def}

\subsection{Höhere (partielle) Ableitungen}
Sei
\[f:\Omega_{\subset\mb{R}}\to\mb{R}\]
Die partiellen Ableitungen von $f$:
\[\Part{f}{x_i}(x)=\Limo{\varepsilon}\frac{f(x+\varepsilon e_i)-f(x)}{\varepsilon}\]
\[e_i=(0,\cdots,i,\cdots,0)\]
\[\Part{f}{x_i}:\Omega\to\mb{R}\]
\[\Part{\left( \Part{f}{x_i} \right)}{x_j}(x)\left(=\frac{\partial^2f}{\partial x_j\partial x_i} \right)(x))\]
\[=\lim_{\varepsilon\downarrow 0}\frac{\Part{f}{x_i}(x+\varepsilon e_j)-\Part{f}{x_i}(x)}{\varepsilon}\]
\[\frac{\partial^3f}{\partial x_k\partial x_j \partial x_i}(x)\]
\[=\lim_{\varepsilon\downarrow 0}\frac{\frac{\partial^2f}{\partial x_i}(x+\varepsilon e_j)-\frac{\partial ^2f}{\partial x_j \partial x_i}(x)}{\varepsilon}\]
\[\left( \frac{\partial f}{\partial x_i \partial x_i} \right)=\frac{\partial^2 f}{\partial x_i^2}\]
\[\frac{\partial^3 f}{\partial x_i \partial x_i \partial x_i}=\frac{\partial^3 f}{\partial x_i^3}\]
\begin{Sat}
  (von Schwarz) Sei $f:\Omega\to\mb{R}$ eine Funktion die in einer Umgebung von $p\in\Omega$ die partielle Ableitungen $\Part{f}{x_i}$, $\Part{f}{x_j}$ und $\frac{\partial^2}{\partial x_i \partial x_j}$ besitzt. Falls $\frac{\partial^2 f}{\partial x_i \partial x_j}$ stetig in $p$ ist, dann existiert $\frac{\partial^2 f}{\partial x_j\partial x_i}(p)$ und
  \[\frac{\partial^2 f}{\partial x_i\partial x_j}(p)=\frac{\partial^2 f}{\partial x_j\partial x_i}(p)\]
\end{Sat}
\begin{Bsp}
  \[f(x_1,x_2)=\sum_{i=1}^{N_1}\sum_{j=1}^{N_2}a_{ij}x_1^ix_2^j\]
  \[\Part{f}{x_1}=\sum_{i=1}^{N_1}\sum_{j=1}^{N_2}ia_{ij}x_1^{i-1}x_2^j\]
  \[\frac{\partial^2 f}{\partial x_2\partial x_1}=\sum_{i=1}^{N_1}\sum_{j=1}^{N_2}ija_{ij}x_1^{i-1}x_2^{j-1}\]
  \[\Part{f}{x_2}=\sum_{i=1}^{N_1}\sum_{j=1}^{N_2}ja_{ij}x_1^ix_2^{j-1}\]
  \[\frac{\partial^2 f}{\partial x_1\partial x_2}=\sum_{i=1}^{N_1}\sum_{j=1}^{N_2}ija_{ij}x_1^{i-1}x_2^{j-1}\]
\end{Bsp}
\begin{Bsp}
  Sei $V:\mb{R}\to\mb{R}$.
  \[v:\mb{R}^2\to\mb{R}\]
  \[v(x_1,x_2)=V(x_2)\]
  \[\Part{f}{x_1}=0\]
  \[\frac{\partial^2f}{\partial x_2\partial x_+}=0\]
\end{Bsp}
\begin{Bew}
  Die Idee ist den Mittelwertsatz zu benutzen.
  \paragraph{Schritt 1} Von Dimension $n\to 2$
  \[f(x_1,\cdots,x_i,\cdots,x_j,\cdots,x_n)\]
  \[p=(p_1,\cdots,p_i,\cdots,p_j,\cdots,p_n)\]
  \[g:U_{\subset\mb{R}}\to\mb{R}\]
  \[g(y,z)=g(p_1,\cdots,p_{i-1},y,p_{i+1},\cdots,p_{j-1},z,p_{j+1},\cdots,p_n\]
  \[\Part{f}{x_i}(p)=\Part{g}{y}(p_i,p_j)\]
  \[\Part{f}{x_j}(p)=\Part{g}{y}(p_i,p_j)\]
  \[\frac{\partial f}{\partial x_j\partial x_i}=\frac{\partial^2 g}{\partial z\partial y}(p_i,p_j)\]
  \[\frac{\partial f}{\partial x_i\partial x_j}(p)=\lim_{\varepsilon\downarrow 0}\frac{\Part{f}{x_j}(p_1,\ldots,p_i+\varepsilon,\ldots,p_j,\ldots,p_n)-\Part{f}{x_j}(p)}{\varepsilon}\]
  \[\Part{g}{z}(p)=\lim_{\varepsilon\downarrow 0}\frac{\Part{f}{x_j}(p_i+\varepsilon p_j)-\Part{g}{z}(p)}{\varepsilon}\]
  \[=\frac{\partial g}{\partial y \partial z}(p_i,p_j)\]
  Falls
  \[\frac{\partial g}{\partial y \partial z}(p_i,p_j)\]
  existiert und
  \[\frac{\partial g}{\partial z \partial y}(p_i,p_j)\]
  gleicht, dann ist das Theorem bewiesen.
  \paragraph{Deswegen}
  Nun, 
  \[f:\Omega_{\subset\mb{R}^2}\to\mb{R}\]
  $\Part{f}{x_1}$, $\Part{f}{x_2}$ und $\frac{\partial^2 f}{\partial x_2\partial x_1}$ existieren in einer Umgebunv von $p=(a,b)$, dann ist $\frac{\partial^2 f}{\partial x_2\partial x_1}$ stetig auf $p$. Zu beweisen: $\frac{\partial^2 f}{\partial x_2\partial x_1}(p)$ existiert und
  \[\frac{\partial^2 f}{\partial x_2\partial x_1}(p)=\frac{\partial^2 f}{\partial x_2\partial x_1}(p)\]
  $p=(a,b)$, $h,k\in\mb{R}\setminus\left\{ 0 \right\}$, $Q=$ Rechteck mit Ecken $(a,b)$, $(a+h, b)$, $(a, b+k)$, $(a+h, b+k$.
  \[D_Qf=f(a+h,b+k)-f(a+h, b)-f(a,b+k)+f(a,b)\]
  \[\Limo{k} \Limo{h}\frac{D_Qf}{hk}\]
  \[=\Limo{k} \Limo{h}\frac{f(a+h,b+k)-f(a,b+k)}{hk}-\frac{f(a+h,b)-f(a,b)}{hk}\]
  \[=\Limo{k}\frac{\Part{f}{x_1}(a,b+k)-\Part{f}{x_1}(a,b)}{k}\]
  \[=\frac{\partial^2f}{\partial x_2\partial x_1}(a,b)\]
  \[\Limo{h}\left( \Limo{k}\frac{D_Qf}{hk} \right)\]
  \[=\Limo{h}\frac{\Part{f}{x_2}(a+h,b)-\Part{f}{x_2}(a,b)}{h}=?\]
  wenn der Limes existiert
  \[=\frac{\partial^2f}{\partial x_1\partial x_2}(a,b)\]
  \paragraph{Ziel}
  $\Limo{h}\Limo{k}\frac{D_Qf}{hk}$ existiert und gleicht $\Limo{k}\Limo{h}\frac{D_Qf}{hk}$. $\implies$ Satz von Schwarz
  \paragraph{Zuerst}
  Wir behaupten ($\forall h,k$ klein genug) die Existenz von einer Stelle $(\xi,\zeta)\in Q$ so dass
  \begin{equation}
    \label{e:1103231}
    \frac{D_Qf}{hk}=\frac{\partial^2f}{\partial x_2\partial x_1}(\xi, \zeta)
  \end{equation}
  \[\frac{D_Qf}{hk}\neq \frac{1}{h}\left\{ \frac{f(a+h,b+k)-f(a+h,b)}{k}-\frac{f(a,b+k)-f(a,b)}{k} \right\}\]
  \[=\frac{1}{h}\left\{ g(a+h)-g(a) \right\}\stackrel{\text{Mittelwertsatz}}{=}g'(\xi)\]
  $x\in ]a,a+h[$, $\zeta\in]b,b+k[$ OBdA: $h,k>0$
  \[g(z)=\frac{f(z,b+k)-f(z,b)}{k}\]
  \[g'(z)=\left( \Part{f}{x_1}(z,b+k)-\Part{f}{x_1}(z,b) \right)\frac{1}{k}\]
  \ldots
  \[=\frac{1}{k}\left( \Part{f}{x_1}(\xi,b+k)-\Part{f}{x_1}(\xi,b) \right)\]
  \[=\Part{f}{x_2}\left( \Part{f}{x_1} \right)(\xi,\zeta)\]
  Womit wir beim zweiten Teil von \ref{e:1103231} wären.
  \[\frac{D_Qf}{hk}-\frac{\partial^2f}{\partial x_2\partial x_1}(a,b)=\frac{\partial^2f}{\partial x_2\partial x_1}(\xi,\zeta)-\frac{\partial^2f}{\partial x_2\partial x_1}(a,b)\]
  \[\Limo{h}\Limo{k}\left( \frac{D_Qf'}{hk}-\frac{\partial^2f}{\partial x_2\partial x_1}(a,b) \right)\]
  \[=\Limo{h}\Limo{k}\left( \frac{\partial^2f}{\partial x_2\partial x_1}(\xi,\zeta)-\frac{\partial^2 f}{\partial x_2\partial x_1}(a,b) \right)\]
  $\forall \varepsilon$ $\exists \delta$ so dass wenn $\sqrt{h^2+k^2}<\delta$
  \begin{equation}
    \label{e:1103233}
    \implies \Abs{\frac{\partial^2f}{\partial x_2\partial x_1}(\xi,\zeta)-\frac{\partial^2f}{\partial x_2\partial x_1}(a,b)}<\varepsilon
  \end{equation}
  \[\limsup_{h\to 0}\Abs{\Limo{k}\frac{D_Qf}{hk}-\frac{\partial^2 f}{\partial x_2\partial x_1}(a,b)}\]
  \[\leq \sup_{h\in ]0,\frac{\delta}{2}[}\Abs{\Limo{k}\frac{D_Qf}{hk}-\frac{\partial f}{\partial x_2 \partial x_1}(a,b)}\]
  \[\leq \sup_{h\in ]0,\frac{\delta}{2}[}\sup_{k\in ]0,\frac{\delta}{2}[}\Abs{\frac{D_Qf}{hk}-\frac{\partial f}{\partial x_2 \partial x_1}(a,b)}\stackrel{\ref{e:1103233}}{\leq} \varepsilon\]
  \[\implies \limsup_{k\to 0}\cdots=\Limo{k}\cdots=0\]
  \[\implies\Limo{h}\Limo{k}\frac{D_+f}{hk}\]
  \begin{equation}
    \label{e:110323r}
    =\frac{\partial^2f}{\partial x_2\partial x_1}(a,b)
  \end{equation}
  \[\left( =\Limo{k}\Limo{h}\frac{D_Qf}{hk} \right)\]
  \[\Limo{h}\Limo{k}\frac{1}{hk}\left\{ f(a+h,b+k)-f(a+h,b)-f(a,b+h)+f(a,b) \right\}\]
  \[=\Limo{h}\\frac{1}{h}\left\{ \Limo{k} \frac{f(a+h,b+k)-f(a+h,b)}{k}-\frac{f(a,b+h)+f(a,b)}{k} \right\}\]
  \[\Limo{h}\frac{1}{h}\left\{ \Part{f}{x_2}(a+h,b)-\Part{f}{x_2}(a,b) \right\}\]
  \[=\Part{}{x_1}\left( \Part{f}{x_2} \right)(a,b)\]
  \begin{equation}
    \label{e:110323b}
    =\frac{\partial^2f}{\partial x_1\partial x_2}(a,b)
  \end{equation}
  \ref{e:110323r} = \ref{e:110323b}
\end{Bew}

%TODO
Sei $a\in\Omega$ und $w\in\mb{R}$. Dann
\[\md^{(k)}f(a)w^k=\sum^n_{i_1=1}\cdots\sum^n_{i_k=1}\frac{\partial^kf(a)}{\partial x_{i_1}\cdots\partial x_{i_k}}w_{i_1}\cdots w_{i_k}\]
\[T^k_xf(z)=f(x)+\md f|_x(z-x)+\cdots+\frac{1}{k!}\md f^{(k)}|_x(z-x)^k\]
\[R^k_xf(z)=\frac{1}{(k+1)!}\md f^{(k+1)}|_\zeta (z-x)^{k+1}\]
Falls $f$ beliebig mal differenzierbar ist ($f\in C^{\infty}(\Omega)$ d.h. die ganze partielle Ableitung existieren und sind stetig) können wir die Taylorreihe schreiben.
\[\sum_{k=0}^\infty\frac{1}{k!}\md f^{(k)}|_x(z-x)^k\]
Konvention:
\[\frac{1}{0!}\md f^{(0)}|_x(z-x)^0=f(x)\]
\begin{Def}
  Eine Funktion $f\in C^\infty(\Omega)$ heisst analytisch wenn $\forall x\in \Omega$ $\exists B_r(x)\subset\Omega$ mit der Eigenschaft dass:
  \[T_x(z)=f(z)\s\forall z\in B_r(x)\]
  \[(f\in C^\omega (\Omega))\]
\end{Def}
\subsection{Das Taylorpolynom zweiter Ordnung}
\[f(z)=f(x)+\underbrace{\sum^n_{i=1}\Part{f}{x_i}(x)(z_i-x_i)}_{\seq{\nabla f(x), z-x}}\]
\begin{equation}
  \label{e:110328gelb}
  +\frac{1}{2}\sum^n_{i=1}\sum^n_{j=1}\frac{\partial^2f}{\partial x_i\partial x_j}(x)(z_i-w_i)(z_j-w_j)
\end{equation}
\[+ \text{Fehler} = \sum\sum\sum\cdots(z_{i_1}-x_{j_1})(z_{i_2}-x_{i_2})(z_{i_3}-x_{i_3}\]
Die Hessche Matrix
\[Hf(x)=\left( \frac{\partial f}{\partial x_i\partial x_j}(x) \right)\]
\[ \begin{pmatrix}
  \frac{\partial^2 f}{\partial x_1^2} & \frac{\partial^2 f}{\partial x_1\partial x_2} & \frac{\partial^2 f}{\partial x_1\partial x_3} \\
  \frac{\partial^2 f}{\partial x_2\partial x_1} & \frac{\partial^2 f}{\partial x_2^2} & \frac{\partial^2 f}{\partial x_2\partial x_3} \\
  \frac{\partial^2 f}{\partial x_3\partial x_1} & \frac{\partial^2 f}{\partial x_3\partial x_2}& \frac{\partial^2 f}{\partial x_3^2}
\end{pmatrix}\]
\begin{Bem}
  Schwarz $\implies$ $Hf(x)$ ist symmetrisch wenn alle Ableitungen zweiter Ordnung stetig sind.
  \[\underbrace{\sum_i\frac{\partial^2 f}{\partial x_1\partial x_j}(x)(z_i-x_i),\cdots,\sum_i\frac{\partial^2 f}{\partial x_n\partial x_i}(x)(z_i-x_i)}_{=Hf(x)(z-x)}\]
  Deswegen
  \[\sum^n_{j=1}(z_j-x_j)\sum^n_{i_1}\frac{\partial^2 f}{\partial x_j\partial x_i}(x)(z_j-x_j)=\ref{e:110328gelb}\]
  \[=\seq{z-x, Hf(x)(z-x)}\]
  \[=(z-x)^THf(x)(z-x)\]
  $H$ $n\times n$ Matrix, die Abbildung
  \[w\mapsto w^T A W (=\seq{w,Aw})\]
  ist eine ``quadratische Form''.
\end{Bem}
Das Taylorpolynom zweiter Ordnung
\[T^2_xf(z)=f(x)+\seq{\nabla f(x), z-x}+\frac{1}{2}(z-x)^T Hf(x)(z-x)\]
\begin{Kor}
  Falls $f\in C^3(\Omega)$ und $B_r(x)\subset\Omega$
  \[f(z)=T^2_x+O(\Norm{x-z}^3)\]
  d.h.
  \[\Abs{f(z)-T_x^2f(z)}\leq C\Norm{z-x}^3\]
\end{Kor}
\begin{Kor}
  Falls $f\in C^2(\Omega)$ und $B_r(x)\subset\Omega$, dann
  \[f(z)=T^2_xf(z)+o(\Norm{z-x}^2)\]
  d.h.
  \[\lim_{z\to x}\frac{f(z)-T_x^2f(z)}{\Norm{z-x}^2}=0\]
\end{Kor}
\begin{Bew}
  Taylorapprozetamation mit Ordnung 1
  \[f(z)=T^1_xf(z)+\frac{1}{2}(z-x)^THf(\zeta)(z-x)\]
  \[f(z)-T_x^2f(z)=\frac{1}{2}(z-x)^THf(\zeta)(z-x)-\frac{1}{2}(z-x)^THf(x)(z-x)\]
  \[=\frac{1}{2}(z-x)^T(Hf(\zeta)-Hf(x))(z-x)\]
  \[\leq \frac{1}{2}\Norm{z-x}\Norm{Hf(\zeta)-Hf(x)(z-x)}\]
  \[\leq \frac{1}{2}\Norm{z-x}\Norm{Hf(\zeta)-Hf(x)}_O\Norm{z-x}\]
  \[= \frac{1}{2}\Norm{z-x}^2\Norm{Hf(\zeta)-Hf(x)}_O\]
  \[\frac{\Abs{f(z-)-T_x^2f(z)}}{\Norm{z-x}^2}\leq\frac{1}{2}\Norm{Hf(\zeta)-Hf(x)}_O\]
  \[\Norm{\zeta-x}\leq\Norm{z-x}^x\]
  Stetigkeit der Ableitungen 2. Ordnung
  \[\implies \lim_{\zeta\to x}\Norm{Hf(\zeta)-Hf(x)}_O=0\]
\end{Bew}

\begin{Def}
  $X\subset \mb{R}^n$, $\exists:X\to \mb{R}$. $f$ hat in $a\in X$ ein lokales Minimum/Maximum 
  \[\iff\exists a\in V\s\text{(Umgebung)}\s. f(a)\leq f(x) (\text{bzw.}\s \geq f(x)) \forall x\in V\]
  Man sagt das Minimum/Maximum ist \ul{strikt} (oder \ul{isoliert})
  \[\iff f(a)<f(x)(\text{bzw.}\s>f(x))\forall x\in V\setminus \left\{ a \right\}\]
\end{Def}
\begin{Sat}
  (Notwendiges Kriteroium für lokale Extrema). Sei $U\subset\mb{R}^n$ offen, $f:U\to\mb{R}$ haben ein lokales Extremum in $a\in U$ und sei partiell differenzierbar. Dann gilt
  \[\partial_1f(a)=\cdots=\partial_nf(a)=0\]
  D.h. wenn $f$ differenzierbar ist, dann gilt $\md f|_a=0$
\end{Sat}
\begin{Bew}
  $F(t)=f(a+t_{e_i})$ (für $t$ sehr klein, so dass $a+t_{e_i}\in U$ $F$ hat lokale Extrema in 0, d.h. $F'(0)=\partial_1 f(a)=0$
\end{Bew}
\begin{Def}
  $f$ differenzierbar, dann heisst $a$ mit $\md f|_a=0$ \ul{kritischer Punkt}. Man sagt auch $f$ ist \ul{stationär} in $a$. 
\end{Def}
\begin{Bem}
  lokale Extremum $\implies$ $\not\Leftarrow$ kritischer Punkt.
\end{Bem}
\begin{Sat}
  (Hinreichendes Kriterium für lokale Extrema) $U\subset\mb{R}^n$ offen, $f\in C^2(U,\mb{R})$ it $\md f|_a=0$. Dann
  \begin{eqnarray*}
    H_f(a)>0\implies a\s\text{lokales Minimum}\\
    H_f(a)<0\implies a\s\text{lokales Maximum}\\
    H_f(a)\s\text{indefinit}\implies a\s\text{kein Extremum}
  \end{eqnarray*}
  Im indefiniten Fall gilt: $\exists$ Geraden $G_1$, $G_2$ durch $a$ so dass $f|_{G_1\cap U}$ in $a$ ein lokales Minimum und $f|_{G_2\cap U}$ in $a$ ein lokales Maximum hat, d.h. $a$ ist ein Sattelpunkt.
\end{Sat}
\begin{Bem}
  \begin{itemize}
    \item 
      $H_f(a)>0$ bedeutet $H_f(a)$ \ul{positiv definit}, d.h. 
      \[v^TH_f(a)v>0\s\forall v\in\mb{R}\setminus\left\{ 0 \right\}\]
    \item
      $H_f(a)$ indefinit, $\exists v,w\in\mb{R}^n\setminus\left\{ 0 \right\}$ mit
      \[v^tH_f(a)v>0\]
      \[w^tH_f(a)w<0\]
  \end{itemize}
\end{Bem}
\begin{Bew}
  \begin{itemize}
    \item[$H_f(a)>0$]
      \[\md f|_a=0\xRightarrow{\text{Taylor}}f(a+h)=f(a)+\frac{1}{2}h^TH_f(a)h+R(h)\\\]
      mit
      \[\frac{R(h)}{\Norm{h}^2}\to 0 \s(\Norm{H}\to 0)\]
      $f\in C^2$
      \begin{itemize}
        \item $\implies$ $h\mapsto h^TH_f(a)h$ ist stetig
        \item $\implies$ \ldots hat ein Minimum auf $\left\{ \Norm{h}=1 \right\}$ (kompakt), $m > 0$ (da $H_f(a)>0$).
        \item $\implies$ $h^TH_f hh\geq m\Norm{h}^2$ (da $h=\Norm{h}\frac{h}{\Norm{h}}$, $h\neq 0$
      \end{itemize}
      Wähle $\varepsilon>0$ so klein, dass $B_\varepsilon(a)\subset U$
      \[\Abs{R(h)}\leq \frac{m}{4}\Norm{h}^2\s\forall h\in B_\varepsilon(a)\]
      \[\implies f(a+h)\geq f(a)+\frac{m}{4}\Norm{h}^2>f(a)\s\forall h\in B_\varepsilon(a)\]
      d.h. $f$ hat in $a$ ein lokales Minimum
    \item[$H_f(a)<0$]
       Betrachte $-f$ wie oben.
    \item[$H_f(a)<>0$] % TODO
      \[\exists v,w :\s v^T H_f(a) v>0, w^T H_f(a) w<0\]
      \[F_v(t):=f(a+tv), F_w(t)=f(a+tw)\]
      \[\implies F_v''(0)>0 \implies \text{lokales Maximum}\]
      \[\implies F_v''(0)<0 \implies \text{lokales Minimum}\]
      $\implies$ Beh
  \end{itemize}
\end{Bew}
\begin{Bem}
  Mit diesem Satz lässt sich keine Aussage machen, falls $H_f(a)$ semidefinitiv ist, d.h. $H_f(a)\geq 0$, $H_f(a)\leq 0$.
\end{Bem}
\begin{Bsp}
  $f(x,y)=y^2(x-1)+x^2(x+1)$
  \[\md f|_{(x,y)} = (y^2+3x^2+2x, 2(x-1)y)\]
  $\implies$ $\md f|_{(x,y)}=(0,0)$ $\implies$ kritische Punkte:
  \[P_1=(0,0), P_2(-\frac{2}{3}, 0)\]
  \[\implies H_f(x,y)= \begin{pmatrix}
    6x+2&2y\\
    2y&2(x-1)
  \end{pmatrix}\]
  d.h.
  \[\implies H_f(P_1)= \begin{pmatrix}
    2&0\\
    0&-2
  \end{pmatrix}\]
  indefinit, d.h. Sattelpunkt.
  \[\implies H_f(P_2)= \begin{pmatrix}
    -2&0\\
    0&-\frac{10}{3}
  \end{pmatrix}<0\]
  d.h. lokales Maximum
\end{Bsp}
\begin{Bsp}
  $f(x,y)=x^2+y^3$, $g(x,y)=x^2+y^4$
  % TODO insert graph here
  Beim Punkt 0 ist die Hesse-Matrix in beiden Fällen $\begin{pmatrix} 2&0\\ 0&0 \end{pmatrix}$. Daraus kann man nichts schliessen (sehe Graphen (Freiwilliger gesucht))
\end{Bsp}
\subsection{Konvexität}
\begin{Def}
  $U\subset\mb{R}^n$ heisst \ul{konvex}
  \[\iff \forall x,y\in U: \s \left[ x,y \right]\subset U\]
\end{Def}
\begin{Def}
  $f:U\to\mb{R}$ heisst \ul{konvex}
  \[\iff \forall x,y\in U:\s f(tx+(1-t)y)\leq tf(x)+(1-t)f(y)\]
  \begin{itemize}
    \item Falls $\forall x,y\in U$ $\forall t\in (0,1)$ ``$<$'', heisst die Funktion \ul{strikt} konvex.
    \item $f$ heisst (streng) \ul{konkav}, falls $-f$ (streng) \ul{konvex}
  \end{itemize}
\end{Def}
\begin{Bem}
  $f$ ist konvex
  \[\iff \forall x\neq y\in U:\s F_{x,y}(t)=f(x+t(y-x))\s\text{konvex (auf $\left[ x,y \right]$)}\]
\end{Bem}
\begin{Sat}
  (Konvexitätskriterium)
  Sei $f:U\to\mb{R}, C^2$ $U\subset\mb{R}^n$ offen, konvex. Es gilt:
  \begin{enumerate}
    \item $f$ konvex $\iff$ $H_f(x)\geq 0$ $\forall x\in U$
      \label{i:1103301}
    \item $H_f(x)>0$ $\forall x\in U$ $\implies$ $f$ streng konvex
      \label{i:1103302}
  \end{enumerate}
\end{Sat}
\begin{Bem}
  Umkehrung von \ref{i:1103301} gilt nicht, z.B. $f(x,y)=x^4+y^4$
\end{Bem}
\begin{Bew}
  \begin{enumerate}
    \item $f$ konvex: $\forall x\in U$ wähle $r>0$: $B_r(x)\subset U$
      \[\implies F_{x,x+h}(t)\s\text{konvex}\s\forall h\in B_r(0)\]
      \[\implies h^TH_f(x)h=F_{x,x+h}''(0)\underbrace{\geq}_{\text{Konvexität in 1-Dim}} 0\forall h\in B_r(0)\]
      \[\xRightarrow{\text{homogenität}}h^TH_f(x)h\geq 0\s\forall h\in \mb{R}^n, \text{d.h.}\s H_f(x)\s\text{positiv semidefinit}\]
      $H_f(x)\geq 0$ $\forall x\in U$:
      \[a,b\in U\implies F_{a,b}''(t)=(b-a)^TH_f(a+t(b-a))(b-a)\geq 0\]
      \[\implies F_{a,b} \s\text{konvex}\s\forall a,b\in U\implies\text{Behauptung}\]
    \item Analog wie die zweite Richtung im Ersten.
  \end{enumerate}
\end{Bew}
\subsection{Differentation parameterabhängiger Integrale}
$f:U\times [a,b]\to\mb{R}$, $U\subset\mb{R}^n$ offen. Sei $t\to f(x,t)$ stetig. $\forall x\in U$. Definiere
\[F(X):=\int_a^bf(x)\md t\s x\in U\]
\begin{Sat}
  Sei $f$ wie oben und es gelte:
  \begin{enumerate}
    \item $\forall t\in [a,b]$: $x\mapsto f(x,t)$ nach $x_i$ partiell differenzierbar
    \item $(x,t)\mapsto \partial_if(x,t)$ ist stetig auf $U\times [a,b]$
  \end{enumerate}
  $\implies$ $F$ ist nach $x_i$ stetig partiell differenzierbar, und es gilt:
  \[\Part{F}{x_i}(x)=\int_a^b\Part{f}{x_i}(x,t)\md t\]
\end{Sat}

Sei $f:\underbrace{U}_{\subset \mb{R}}\times [a,b]\to\mb{R}$ stetig ($U$ offen) $\forall x\in U$ sei
\[F(x)=\int_a^bf(x,t)\md t\]
\begin{Sat}
  (Differentationssatz) Falls
  \begin{enumerate}
    \item $\forall t\in \left[ x,b \right]$ ist $x\mapsto f(x,t)$ nach $x_i$ partiell differenzierbar
      \[\exists\Part{f}{x_i}(x,t)\s\forall (x,t)\in U\times \left[ a,b \right]\]
    \item und $\Part{f}{x_i}$ ist stetig
  \end{enumerate}
  dann $\exists$ auch $\Part{F}{x_i}(x)$ und
  \[\Part{F}{x_i}(x)=\int_a^b\Part{f}{x_i}(x,t)\md t\]
  \[\Part{}{x_i}\int_a^bf(x,t)\md t=\int_a^b\Part{}{x_i}f(x,t)\md t\]
\end{Sat}
\begin{Bew}
  Sei $x\in U$ und $e_i=(0,\cdots,\underbrace{1}_{i\text{te Stelle}},\cdots,0)$
  \[\Part{F}{x_i}=\Limo{\varepsilon}\frac{F(x+\varepsilon e_i)-F(x)}{\varepsilon}\]
  \[=\Limo{\varepsilon}\frac{1}{\varepsilon}\left\{ \int_a^bf(x+\varepsilon e_i ,t)\md t-\int_a^bf(x,t)\md t \right\}\]
  \[\Limo{\varepsilon}\int_a^b\frac{f(x+\varepsilon e_i,t)-f(x,t)}{\varepsilon}\md t\]
  \[\Part{F}{x_i}(x,t)=\int_a^b\Part{f}{x_i}(x,t)\md t\iff\]
  \[\iff \Limo{\varepsilon}\left\{ \int_a^b\frac{f(x+\varepsilon e_i,t)-f(x,t)}{\varepsilon}\md t-\int_a^b\Part{f}{x_i}(x,t)\md t \right\}=0\]
  \[\iff \Limo{\varepsilon}\left\{ \int_a^b\left[ \frac{f(x+\varepsilon e_i,t)-f(x,t)}{\varepsilon}-\Part{f}{x_i}(x,t)\md t \right] \right\}=0\]
  Wir behaupten mehr, d.h.
  \[\int_a^b\abs{ \underbrace{\frac{f(x+\varepsilon e_i,t)-f(x,t)}{\varepsilon}}_{\Part{f}{x_i}(\xi_\varepsilon,t)}-\Part{f}{x_i}(x,t)}\md t\stackrel{\varepsilon\to 0}{\to}0\]
  wobei $\xi_\varepsilon(t)\in \left[ x,x+\varepsilon e_i \right]$
  \[\int_a^b\abs{\cdots}=\int_a^b\abs{\Part{f}{x_i}\left( \xi(t),t \right)-\Part{f}{x_i}(x,t)}\md t\]
  \[\Limo{e}\xi_\varepsilon(t)=x\]
  und (wegen der Stetigkeit von $\Part{f}{x_i}$)
  \[\Part{f}{x_i}\left( \xi_\varepsilon(t),t \right)\to \Part{f}{x_i}(x,t)\]
  \begin{Beh}
    $\forall \varepsilon>0$ $\exists \varepsilon_0>0$ so dass
    \[\abs{\varepsilon}\leq \varepsilon_0\implies \sup_{t\in [a,b]}\abs{\Part{f}{x_i}(\xi(t),t)-\Part{f}{x_i}(x,t)}<\delta\]
  \end{Beh}
  $\implies$
  \[\limsup_{\varepsilon\to 0} A(\varepsilon)\leq \sup_{\abs{\varepsilon}<\varepsilon_0}A(\varepsilon)\]
  \[\leq \int_a^b\delta\md t=\delta(b-a)\]
  $\delta$ ist beliebig
  \[\Limo{\varepsilon}A(\varepsilon)=0\]
  \begin{Lem}
    Sei $g:U\times [a,b]\to\mb{R}$ stetig (wobei $U\subset\mb{R}^n$ offen ist). Sei $x\in U$ Dann $\forall \delta >0$ $\exists \varepsilon>0$ mit
    \[\sup_{y\in B_\varepsilon(x)}\abs{g(y,t)-g(x,t)}<\delta\s\forall t\in [a,b]\]
    % TODO Zeichnung
    Betrachte $x$ als ``Parameter'' $\forall y$ sei $t\mapsto g(y,t)=g_y(t)$. Dann $g_y\to g_x$ gleichmässig für $x\to x$.
  \end{Lem}
  \begin{Bem}
    Das Lemma nutzt nur die Kompaktheit von $[a,b]$ (in der Behauptung können wir $[a,b]$ durch eine beliebige kompakte Menge $K\subset\mb{R}$ ersetzen)
  \end{Bem}
  \begin{Bew}
    Sei $\varepsilon>0$ gegeben $\forall (x,t)$ $\exists \delta(x,t)>0$ so dass
    \[\abs{g(\xi, \tau)-g(x,t)}<\frac{\varepsilon}{10}\s\forall (\xi, \tau)\]
    mit
    \[\Norm{\underbrace{(\xi,\tau)}_{\in \mb{R}^n}-\underbrace{(x,t)}_{\in \mb{R}^n}}<\delta(x,t)\]
    \[\Norm{(\xi,\tau)-(x,t)}=\sqrt{\Norm{\xi-x}^2+(t-\tau)^2}\]
    Nun $\forall t\in [a,b]$
    \[\left\{ B_{\delta(x,t)}(x,t):t\in [a,b] \right\}\]
    ist eine Überdeckung von
    \[K = \left\{ (x,t):t\in [a,b] \right\}\]
    kompakt weil
    \[\left[ a,b \right]\ni t\mapsto(x,t)\]
    stetig von $[a,b]$ nach $\mb{R}^n\times\mb{R}$. $K$ ist das Bild von $[a,b]$ durch diese Abbildung.
  \end{Bew}
  \begin{Bem}
    $f$ stetig
    \begin{tabular}{lcl}
      $K$ kompakt & $\implies$ & $f(K)$ kompakt\\
      $A$ abgechlossen & $\implies$ & $f^{-1}(A)$ abgeschlossen\\
      $O$ offen & $\implies$ & $f^{-1}(A)$ offen
    \end{tabular}
    Alle anderen Implikationen stimmen NICHT.
  \end{Bem}
  $\forall (x,t)$ Sei
  \[U_{x,t}=\underbrace{B_{\frac{\sqrt{2}}{2}\delta(x,t)}}_{\subset\mb{R}^n}(x)\times \left] t-\frac{\sqrt{2}}{2}\delta(x,t),t+\frac{\sqrt{2}}{2}\delta(x,t) \right[\]
  $(y,\tau)\in U_{x,t}$
  \[\implies \Norm{y-x}\leq\frac{\sqrt{2}}{2}\delta(x,t)\s\text{und}\s\abs{t-\tau}<\frac{\sqrt{2}}{2}\delta(x,t)\]
  \[\Norm{(y,t)-(x,\tau)}<\sqrt{\frac{1}{2}\delta^2(x,t)+\frac{1}{2}\delta^2(x,t)}=\delta(x,t)\]
  \[\implies (y,t)\in B_{\delta(x,t)}(x,t)\]
  \[\implies U_{x,t}\subset B_{\delta(x,t)}(x,t)\]
  $\left\{ U_x,t:t\in [a,b] \right\}$ ist eine offene Überdeckung von $K$. Kompaktheit $\implies$ $\exists \left\{ U_{x_i,t_i}:i\in \left\{ 1,\cdots,N \right\} \right\}$ Überdeckung von $K$. Sei 
  \[\delta=\min\left\{ \frac{\sqrt{2}}{2}\delta(x_i,t_i):i\in\left\{ 1,\cdots,N \right\} \right\}>0\]
  Sei $t\in [a,b]$, $(x,t)\in U_{x_i,t_i}$ für mindestens ein $i\in \left\{ i,\cdots,N \right\}$. Sei $y$ so dass $y-x<\delta$
  \[(x,t),(y,t)\in U_{x_i,t_i}\subset B_{\delta(x_i,t_i}(x_i,t_i)\]
  \[\implies \abs{g(y,t)-g(x_i,t_i)}<\frac{\varepsilon}{10}\]
  und
  \[\implies \abs{g(x,t)-g(x_i,t_i)}<\frac{\varepsilon}{10}\]
  \[\implies \abs{g(x,t)-g(y,t)}<\frac{\varepsilon}{5}\]
  \[\implies \sup_{y\in B_\delta(x)}\abs{g(x,t)-g(y-t)}\leq\frac{\varepsilon}{5}<\varepsilon\s\forall t\in [a,b]\]
\end{Bew}
\begin{Kor}
  Sei $g:U\times [a,b]\to\mb{R}$ stetig. Dann
  \[F(x)=\int_a^bg(x,t)\md t\]
  ist eine stetige Funktion
\end{Kor}
\begin{Bew}
  Seien $x\in U$ und $\varepsilon >0$. Das letzte Lemma $\implies$ $\exists \delta >0$ so dass
  \[\abs{g(x,t)-g(y,t)}\frac{\varepsilon}{b-a}\]
  $\forall t$ und $\forall y,x$ mit $\Norm{y-x}<\delta$. Deswegen für $\Norm{y-x}<\delta$
  \[\abs{F(y)-F(x)}=\abs{\int_a^b(g(x,t)-g(y,t))\md t}\]
  \[\leq\int_a^b\abs{g(x,t)-g(y,t)}\md t\]
  \[<\int_a^b\frac{\varepsilon}{b-a}\md t=\varepsilon\]
\end{Bew}
\begin{Bem}
  Im Differentiationssatz ist $\Part{f}{x_i}$ eine stetige Funktion. Da
  \[\Part{f}{x_i}(x)=\int_a^b\Part{f}{x_i}(x,t)\md t\]
  ist $\Part{F}{x_i}$ stetig.
\end{Bem}
\begin{Bem}
  Eine sehr wichtige Konsequenz: Sei $f:\underbrace{U}_{\subset\mb{R}^2}\to \mb{R}$ eine stetige Funktion. Sei $\underbrace{[a,b]\times [c,d]}_R\subset U$
  % TODO Zeichnung
  \[s\mapsto F(s)=\int_a^b f(t,s)\md t\]
  \[\int_c^dF(s)\md s=\int_c^d\left( \int_a^bf(t,s)\md s \right)\md t\]
  \[\int_c^d\int_a^bf(t,s)\md t\md s\]
  \[t\mapsto G(t)=\int_c^df(t,s)\md s\]
  \[\int_a^bG(t)\md t=\int_a^b\int _c^df(t,s)\md s\md t\]
\end{Bem}
\begin{Sat}
  $f$ stetig $\implies$
  \[\int_a^b\int_c^df(s,t)\md s\md t=\int_c^d\int_a^b f(s,t)\md t\md s\]
\end{Sat}
\begin{Bew}
  $(x,y)\in [a,b]\times [c,d]$
  \[F(x,y)=\int_a^x\int_c^yf(s,t)\md s\md t\]
  \[G(x,y)=\int_c^y\int_a^x f(t,s)\md t\md s\]
\end{Bew}

\begin{Sat}
  Sei $f:\underbrace{U}_{\mb{R}^2}\to\mb{R}$ eine stetige Funktion. Sei $R=[a,b]\times [c,d]\subset U$. Dann:
  \[\int_a^b\int_c^d f(s,t)\md t\md s=\int_c^d\int_a^b f(s,t)\md s\md t\]
\end{Sat}
\begin{Bew}
  Wir definieren
  \[\Phi(x,y)=\int_a^x\int_c^yf(s,t)\md t\md s\]
  \[\Psi(x,y)=\int_c^y\int_a^x f(s,t)\md s\md t\]
  Konvention: $\int_\alpha^\beta = -\int_\beta^\alpha$ falls $\beta<\alpha$ und $\int_\alpha^\alpha=0$ \\
  % TODO separate
    $\Phi$ und $\Psi$ sind stetig differenzierbar und $\nabla \Phi = \nabla\Psi$ \\
  % TODO separate
    $\Phi=\Psi$ (Kein Problem mit Definition. Die FUnktion sind wohldefiniert fur $(x,y)\in ]a-\varepsilon,b+\varepsilon[\times ]c-\varepsilon,d+\varepsilon [$ wobei $\varepsilon>0$ klein genug ist)
  % TODO end
  Sei $y$ fixiert
  \[\Part{\Phi}{x}(x,y) = ?\]
  \[\phi(x)=\int_c^yf(x,t)\md t\]
  $\phi$ stetig wegen der letzten Vorlesung. Fundamentalsatz der Int.:
  \[\Part{\Phi}{x}(x,y)=\phi(x)=\int_c^y f(x,t)\md t\]
  % TODO separate
    $\Part{\Phi}{x}$ ist eine stetige Funktion.
    Sei $(x_0,y_0)$, $\varepsilon>0$. Dann (aus der letzten Vorlesung stetig in $x$) $\exists \delta$
    \[\Abs{\Part{\Psi}{x}(x,y_0)-\Part{\Psi}{x}(x_0,y_0)}<\frac{\varepsilon}{2}\]
  % TODO end
  Sei $x$ fixiert:
  \[\Abs{\Part{\Psi}{x}(x,y_0)-\Part{Psi}{x}(x,y)}\]
  \[=\Abs{\int_c^yf(x,t)\md t-\int_c^{y_0}f(x,t)\md t}\]
  \[=\Abs{\int_{y_0}^yf(x,t)\md t}\]
  \[\leq \int_{y_0}^y\Abs{f(x,t)}\md\]
  \[\leq M\Abs{y-y_0}\]
  Deswegen
  für $\bar\delta \leq \frac{\varepsilon}{2H}$
  \[\Abs{y-y_0}<\bar\delta\]
  \[\implies \Abs{\Part{\Psi}{x}(x,y_0)-\Part{\Psi}{x}(x,y)}<\frac{\varepsilon}{2}\]
  Wenn
  \[\Norm{(x,y)-(x_0,y_0)}]<\min \left\{ \delta,\bar\delta \right\} \]
  $\implies$ $\Abs{x-x_0}<\delta$ und $\Abs{y-y_0}<\bar\delta$
  \[\Abs{\Part{\Phi}{x}(x,y)-\Part{\Phi}{x}(x_0,y_0)}\]
  \[\leq \Abs{\Part{\Phi}{x}(x,y)-\Part{\Phi}{x}(x,y_0)}+\Abs{\Part{\Phi}{x}(x,y_0)-\Part{\Phi}{x}(x_0,y_0)}<\frac{\varepsilon}{2}\]
  Das gleiche Argument: $\Part{\Psi}{y}$ exisiert und ist stetig.
  \[\psi(x,y):=\int_a^xf(s,y)\md s\]
  \[\Part{\Psi}{x}=\Part{}{x}\int_c^y\psi(x,t)\md t\]
  \[\stackrel{?}{=}\int_c^y\Part{\psi}{x}(x,t)\md t\]
  Wir brauchen hier die Stetigkeit von $\psi$. Das haben wir mit dem letzten Argument!
  \begin{equation}
    \label{e:1104062}
    \Part{\psi}{x}(x,t)=\Part{}{x}\int_a^xf(s,t)\md s\stackrel{\text{Fundamentalsatz}}{=}f(x,t)
  \end{equation}
  \[\Part{\Psi}{x}=\int_c^yf(x,t)\md t\stackrel{!}{=}\Part{\Phi}{x}\]
  Das gleiche Argument $\Part{\Psi}{x}=\Part{\Phi}{x}$ sind stetig. Sei $\alpha:=\Phi-\psi$ $\implies$ $\alpha$ ist differenzierbar und $\md \alpha=0$
  % TODO Zeichnung
  \[=[a-\varepsilon, b+\varepsilon[\times ]c-\varepsilon,d+\varepsilon [\]
  \[\Abs{\alpha(x_0,y_0)-\alpha(x_1,y_1)}\leq\Norm{(x_1,y_1)-(x_0,y_0)}\max\Norm{\nabla\alpha}=0\]
  Schrankensatz? da $[(x_0,y_0)(x_1,y_1)]$ ist im Definitionsbereich
  \[\Phi-\Psi=\alpha=\text{konstant}=\Phi(a,c)-\Psi(a,c)=0-0=0\]
  \[\implies \Phi(x,y)=\Psi(x,y) \s\forall (x,y)\in ]a-\varepsilon,b+\varepsilon [\times ]c-\varepsilon, d+\varepsilon [\]
  $y=d,x=b$ $\implies$ den Satz.
\end{Bew}
\section{Differenzierbare Abbildungen}
$f:\underbrace{\subset\mb{R}^n}\to\mb{R}^m$
\begin{Def}
  $f$ ist in $x_0$ differenzierbar falls $\exists L:\mb{R}^n\to\mb{R}^m$ lineare Abbildung:
  \[\Limo{h}\frac{f(x_0+h)-f(x_0)-L(h)}{\Norm{h}}=0\]
  d.h. wenn 
  \[R(h):=f(x_0+h)-f(x_0)-L(h)\]
  dann
  \[\Limo{h}\frac{\Norm{R(h)}}{\Norm{h}}=0\]
  \[\forall \varepsilon>0\s\exists\delta>0\s\text{so dass}\s0<\Norm{h}<\delta\implies \frac{\Norm{R(h)}}{\Norm{h}}<\varepsilon\]
  oder auch ``$\Norm{R(h)}\to 0$ schneller als $\Norm(h)$'' (in ``klein-o-Notation'': $R(h)=o(\Norm{h})$) Deswegen
  \begin{equation}
    \label{e:1104063}
    f\s\text{diff in }\s x_0\iff \exists L \lim\s\text{mit}\s f(x_0+h)-f(x_0)+L(h)+o(\Norm{h})
  \end{equation}
\end{Def}
\begin{Bem}
  $f$ differenzierbar in $x_0$ $\implies$ stetig in $x_0$\\
  $f$ differenzierbar in $x_0$ $\implies$ $\exists !$ lineare Abbildung die \ref{e:1104063} erfüllt. Wir nennen $L$ das Differential von $f$. $\md f|_{x_0}$
\end{Bem}
\begin{Bem}
  $f:U\to\mb{R}^m$
  \[f(x)=\underbrace{(f(x),\cdots,f_m(x))}_{m\s\text{Funktionen}}\]
  $\forall i \Part{f_i}{x_j}$ $n$ partielle Ableitungen
  \[L:\mb{R}^n\to\mb{R}^m\]
  \[L = \begin{pmatrix}
    L_{11} & \cdots & L_{1n}\\
    L_{21} & \cdots & L_{2n}\\
    \vdots & & \vdots \\
    L_{m1} & \cdots & L_{mn}
  \end{pmatrix} = \begin{pmatrix}
    L_1 \\ L_2 \\ \vdots \\ L_m
  \end{pmatrix}\]
  \[L(x) = \begin{pmatrix}
    L_{11}+L_{12}+\cdots +L_{1n}x_n\\
    L_{21}+\cdots +L_{2n}x_n\\
    \vdots \\
    L_{m1}+\cdots +L_{mn}x_n\\
  \end{pmatrix} = \begin{pmatrix}
    L_1 x\\ L_2 x\\ \vdots \\ L_m x
  \end{pmatrix}\]
  $\exists m$ lineare Abbildungen $\mb{L}:\mb{R}^n\to\mb{R}$
  \[L(x)= \begin{pmatrix}
    \mb{L}_1(x)\\
    \mb{L}_2(x)\\
    \vdots \\
    \mb{L}_n(x)\\
  \end{pmatrix}\]
  \[\mb{L}_i(x)=L_i x\]
\end{Bem}
\begin{Bem}
  Sei $f:U\to\mb{R}^m$ differenzierbar in $x_0$ und sei $L=\md f|_{x_0}$. Dann:
  \begin{equation}
    \label{e:1104064}
    \frac{\overbrace{f(x_0+h)-f(x_0)-L(h)}^A}{\Norm{h}}\to 0
  \end{equation}
  \[A:= \begin{pmatrix}
    f_1(x+h) \\ \vdots \\ f_m(x_0+h)
  \end{pmatrix} - \begin{pmatrix}
    f_1(x_0)\\ \vdots \\ f_m(x_0)
  \end{pmatrix} - \begin{pmatrix}
    \mb{L}_1(h) \\ \vdots \\ \mb{L}_m(h)
  \end{pmatrix}\]
  \[ = \begin{pmatrix}
    f_1(x_0+h)-f_1(x_0)-\mb{L}_1(h)\\
    \vdots \\
    f_m(x_0+h)-f_m(x_0)-\mb{L}_m(h)\\
  \end{pmatrix} \]
  \[\frac{A}{\Norm{h}}= \begin{pmatrix}
    \frac{f_1(x_0+h)-f_1(x_0)-\mb{L}_1(h)}{\Norm{h}} \\
    \vdots \\
    \frac{f_m(x_0+h)-f_m(x_0)-\mb{L}_m(h)}{\Norm{h}} \\
  \end{pmatrix}\]
  Deswegen
  \[\ref{e:1104064} \iff \Limo{h}\frac{f_i(x_0+h)-f_i(x_0)-\mb{L}_i(h)}{\Norm{h}} = 0\s\forall i\in \left\{ 1,\cdots,m \right\}\]
  $\iff$ $f_i$ ist differenzierbar in $x_0$ und $\mb{L}_i=\md f_i|_{x_0}$
\end{Bem}
\begin{Sat}
  Sei $f:\underbrace{U}_{\subset\mb{R}^n}\to\mb{R}^m$ mit $U$ offen und $f=(f_1,\cdots,f_m)$
  \begin{enumerate}
    \item $f$ ist differenzierbar in $x_0$ $\iff$ $f_i$ differenzierbar in $x_0$ $\forall i\in \left\{ 1,\cdots,m \right\}$
    \item
      \[\md f|_{x_0}(h)= \begin{pmatrix}
        \md f_1|_{x_0}(h)\\
        \vdots \\
        \md f_m|_{x_0}
      \end{pmatrix}\]
    \item 
      \[\md f|_{x_0}(h)= \begin{pmatrix}
        \nabla f_1+(x_0)h
        \vdots \\
        \nabla f_n+(x_0)h
      \end{pmatrix} = 
      \begin{pmatrix}
        \Part{f_1(x_0)}{x_1} & \Part{f_1}{x_2} & \cdots & \Part{f_1}{x_n} \\
        \vdots & \ddots & \ddots & \vdots \\
        \vdots & \ddots & \Part{f_i}{x_j} & \vdots \\
        \Part{f_m}{x_1} & \cdots & \cdots & \Part{f_m}{x_n} \\
      \end{pmatrix} \begin{pmatrix}
        h_1 \\ & \vdots & h_n
      \end{pmatrix}
      \]
      Das ist die Jacobi Matrix.
  \end{enumerate}
\end{Sat}
\begin{Bem}
  $f,g;U\to\mb{R}^m$ beide differenzierbar in $x_0$, dann
  \[f+g\left( = \begin{pmatrix}
    f_1+g_1\\
    \vdots \\
    f_m+g_m
  \end{pmatrix} \right)\]
  ist differenzierbar in $x_0$ und $\md f|_{x_0}+\md g|_{x_0}$
\end{Bem}
\begin{Bem}
  $f:U\to\mb{R}^m$ und $g:U\to\mb{R}^m$ differenzierbar in $x_0$
  \[(gf)(x)=g(x)f(x)= \begin{pmatrix}
    g(x)f_1(x) \\
    \vdots \\
    g(x)f_m(x) \\
  \end{pmatrix}\]
  \[\frac{?}{\md (gf)}=\md \begin{pmatrix}
    gf_1 \\ \vdots g f_m
  \end{pmatrix} \Big|_{x_0}(h) = \begin{pmatrix}
    \md (gf_1)|_{x_0}(h)\\
    \vdots \\
    \md (gf_m)|_{x_0}(h)\\
  \end{pmatrix} \]
  \[ = \begin{pmatrix}
    \md g|_{x_0}(h)f_1(x_0)+g(x_0)\md f_1|_{x_0}(h)\\
    \vdots \\
    \md g|_{x_0}(h)f_m(x_0)+g(x_0)\md f_m|_{x_0}(h)\\
  \end{pmatrix} \]
  Jacobi-Matrix
  \[ \begin{pmatrix}
    \Part{g}{x_1}(x_0)f_1(x_0)+g(x_0)\Part{f_1}{x_1}(x_0) & \cdots & \Part{g}{x_n}(x_0)f_1(x_0)+g(x_0)\Part{f_1}{x_n}(x_0) \\
    \vdots & \ddots & \vdots \\
    \Part{g}{x_1}(x_0)f_m(x_0)+g(x_0)\Part{f_m}{x_1}(x_0) & \cdots & \Part{g}{x_n}(x_0)f_m(x_0)+g(x_0)\Part{f_m}{x_n}(x_0) \\
  \end{pmatrix} \]
  \[ = \begin{pmatrix}
    \Part{g}{x_1}(x_0)f_1(x_0) & \cdots & \Part{g}{x_n}(x_0)f_1(x_0) \\
    \vdots & \ddots & \vdots \\
    \Part{g}{x_1}(x_0)f_m(x_0) & \cdots & \Part{g}{x_n}(x_0)f_m(x_0) \\
  \end{pmatrix}\]
  \[+ \begin{pmatrix}
    g(x_0)\Part{f_1}{x_1}(x_0) & \cdots & g(x_0)\Part{f_1}{x_n}(x_0)\\
    \vdots & \ddots & \vdots \\
    g(x_0)\Part{f_m}{x_1}(x_0) & \cdots & g(x_0)\Part{f_m}{x_n}(x_0)\\
  \end{pmatrix} \]
  \[ \underbrace{\begin{pmatrix}
    \Part{g}{x_j}(x_0) f_i(x_0)
  \end{pmatrix}}_{ = f(x_0)\otimes \nabla g(x_0)} + \overbrace{\underbrace{g(x_0) \left( \Part{f_i}{x_j}(x_0) \right)}_{\text{Jacobi}}}^A\]
  \[\md (gf)|_{x_0}=\underbrace{g(x_0)\md f|_{x_0}}_A+\overbrace{f(x_0)\otimes \md g|_{x_0}}^{\text{lineare Abbildung mit Rang 1}}\]
  \[\md (gf)|_{x_0}(h)=g(x_0)\left[ \md f|_{x_0}(h) \right]+\left[ f(x_0) \right]\md g|_{x_0}(h)\]
\end{Bem}

\begin{Sat}
\[f:\underbrace{U}_{\subset \mb{R}^n}\to\underbrace{V}_{\subset \mb{R}^n}\]
\[g:V\to\mb{R}^k\]
  Falls $f$ in $a$ differenzierbar ist und $g$ in $b=f(a)$ differenzierbar ist, dann ist $g\circ f$ in $a$ differenzierbar und
  \begin{equation}
    \label{e:1104111}
    \md(g\circ f)|_{a}\stackrel{?}{=}\md g|_b\circ\md f|_a
  \end{equation}
\end{Sat}
\begin{Bew}
  Differential von $f$ in $a$ mit $\frac{R(h)}{\Norm{h}}\to 0$
  \[f(a+h)=f(a)+\md f|_a(h) +\overbrace{R(h)}^{o(\Norm{h})}\]
  Differential von $g$ in $b$ mit $\frac{R(k)}{\Norm{k}}\to 0$
  \[g(b+k)=g(b)+\md g|_b(k) +\underbrace{\bar R(k)}_{o(\Norm{k})}\]
  \[g(f(a+h))=g(\underbrace{f(a)}_b+k)=g(b)+\md g|_b (k)+\bar R(k)\]
  Linearität von $\md g|_b$
  \[=g(b)+\md g|_b\left( \md f|_a(h)+R(h) \right)+\bar R(k)\]
  \[=\underbrace{g(b)}_{g\circ f(a)}+\underbrace{\md g|_b(\md f|_a(h))}_{\text{ist linear in}\s h}+\underbrace{\md g|_b(R(h))+\bar R(k)}_{:=\rho(h)}\]
  \[\rho(h)=o(\Norm{h})\]
\end{Bew}
\begin{Lem}
  Linearität:
  \[\md g|_b\circ\md f|_a(\lambda_1h_1+\lambda_2h_2)\s\lambda_1,\lambda_2\in\mb{R}, h_1,h_2\in\mb{R}^n\]
  \[=\md g|_b\left( \md f|_a(\lambda_1h_+\lambda_2h_2 \right))\]
  \[=\md g|_b\left( \lambda_1\overbrace{\md f|_a(h_1)}^{\in\mb{R}^m}+\lambda_2\overbrace{\md f|_a(h_2)}^{\in\mb{R}^m} \right)\]
  \[=\lambda_1\md g|_b\left( \md f|_a(h_1) \right)+\lambda_2\md g|_b\left( \md f(h_2) \right)\left( \md f(h_2) \right)\]
  \[=\lambda_1\md g|_b\circ \md f|_a(h_1)+\lambda_2\md g|_b\circ \md f|_a(h_2)\]
\end{Lem}
\begin{Lem}
  \[\frac{\rho(h)}{\Norm{h}}\leq\frac{\abs{\md g|_b (R(h))}}{\Norm{h}}+\frac{\abs{\bar R(j)}}{\Norm{h}}\]
  \[\leq\frac{\Norm{\md g|_b}_0\Norm{R(h)}}{\Norm{h}}+\frac{\Norm{\bar R(h)}}{\Norm{h}}\]
  \[\frac{\Norm{R(h)}}{\Norm{h}}\to 0\]
  \[\frac{\Norm{\bar R(h)}}{\Norm{h}} = \begin{cases}
    0 & \text{falls}\s k=0\\
    \frac{\Norm{\bar R(k)}}{\Norm{k}}\frac{\Norm{k}}{\Norm{h}}
  \end{cases}\]
  \[\Norm{k}=\Norm{\md f|_a(h)+R(h)}\leq \Norm{\md f|_a(h)}+\Norm{R(h)}\]
  \begin{equation}
    \label{e:1104110}
    \leq \Norm{\md f|_a}_0\Norm{h}+\Norm{R(h)}
  \end{equation}
  \[\frac{\Norm{R(h)}}{\Norm{h}}\to 0\]
  Sei $\varepsilon=1$, $\exists \delta>0$ so dass
  \[\Norm{h}<\delta\implies \frac{\Norm{R(h)}}{\Norm{h}}\]
  Falls $\Norm{h}<\delta$
  \[\ref{e:1104110}\leq (\Norm{\md f|_a}_0+1)\Norm{h}\]
  Deswegen: wenn $\Norm{h}\to 0$, dann $\Norm{k}\to 0$ und für $\Norm{h}<\delta$
  \[\frac{\Norm{\bar R(k)}}{\Norm{h}}\leq \underbrace{\frac{\Norm{\bar R(k)}}{\Norm{k}}}_{\to 0\s\text{für}\s\Norm{h}\to 0}(\Norm{\md f|_a}_0+1)\]
  Deswegen:
  \[0\leq \limsup_{\Norm{h}\to 0}\frac{\Norm{\rho(h)}}{\Norm{h}}\]
  \[\leq\Limo{\Norm{h}}\frac{\Norm{\bar R(k)}}{\Norm{h}}+\Limo{\Norm{h}}\frac{\Norm{\md g|_b(R(h))}}{\Norm{h}}=0+0=0\]
  \[\implies \Limo{\Norm{h}}\frac{\Norm{\rho(h)}}{\Norm{h}}=0\]
\end{Lem}
\begin{Bem}
  $n=m=k=1$
  \[f(a+h)=f(a)+\underbrace{\md f|_a}+o(\Norm{h})\]
  $b=f(a)$
  \[\md f|_a(h)=f'(a)h\]
  \[\md g|_b(k)=g'(b)k\]
  \[\md g|_b=\md f|_a(k)=\md g|_b(\md f|_a(h))=\md g|_b(f'(a)h)\]
  \begin{equation}
    \label{e:1104112}
    =g'(b)f'(a)h=g'(f(a))f'(a)h
  \end{equation}
  $\phi=g\circ f$
  \[\md \phi|_a(h)=\phi'(a)h=(g\circ f)'(a)h\]
  \ref{e:1104111} d.h. die allgemeine Kettenregel
  \begin{eqnarray*}
    \md \phi|_a(h)=\md (g\circ f)|_a(h)\\
    =\md g|_b\circ \md f|_a(h)\stackrel{\ref{e:1104112}}{=}g'(f(a))f'(a)h
  \end{eqnarray*}
  \begin{eqnarray*}
    \implies(g\circ f)'(a)\not h=g'(f(a))f'(a)\not h\\
    \implies \underbrace{(g\circ f)'(a)=g'(f(a))f'(a)}_{\text{alte Kettenregel}}
  \end{eqnarray*}
\end{Bem}
\begin{Bem}
  Kettenregel für die Jacobi-Matrizen: Formel \ref{e:1104111} $\iff$ Sei $M$ die Jacobi-Matrix für $\md g|_{b(=f(a)}$ und $N$ düe für $\md f_a$. Dann ist die Jacobi für $\md (g\circ f)|_a$ ist $MN$ \\
  $\implies$ $g=(g_1,\cdots,g_k)$ $f=(f_1,\cdots,f_m)$ Es gibt eine Formel für $\Part{(g\circ f)_i}{x_j}$
  \[\md g_b\circ \md g|_a(w)=\md g|_b(\underbrace{\md f|_a(w)}_v)\]
  \begin{eqnarray*}
    \md g|_b\circ \md f|_a(w)=\md g|_b(v)\\
    =\left( \sum^m_{i=1}M_{1i}v_i,\sum^m_{i=1}M_{2i}v_i,\cdots,\sum^m_{i=1}M_{ki}v_i \right)\\
    =\left( \sum^m_{i=1}M_{1i}\sum^n_{j=1}N_{ij}w_j,\cdots,\sum^m_{i=1}M_{ki}\sum^n_{j=1}N_{ij}w_j \right)\\
  \end{eqnarray*}
  \begin{eqnarray*}
    v=\md f|_a(w)=\left( \sum^n_{j=1}N_{1j}w_j,\cdots,\sum^n_{j=1}N_{mj}v_j \right)\\
    \iff v_i=\sum^n_{j=1}N_{ij}w_j
  \end{eqnarray*}
  \[\md g|_b\circ \md f|_a(v)=\left( \sum^m_{i=1}\sum^n_{j=1}M_{1i}N_{ij}v_j,\cdots,\sum^m_{i=1}\sum^n_{j=1}M_{ki}N_{ij}v_j \right)\]
  (Sei $A$ die Matrix 
  \[A_{lj}=\sum_{i=1}^mM_{li}N_{ij} \iff A=M\cdot N\]
  \[=\left( \sum^n_{j=1}A_{1j}v_j,\cdots,\sum_{j=1}^nA_{kj}v_j \right)\]
  Deswegen ist $A$ die Matrixdarstellung von
  \[\md g|_b\circ \md f|_a=\md (g\circ f)|_a\]
  $\iff$ $A$ ist die Jacobi-Matrix für $\md (g\circ f)|_a$
\end{Bem}
\begin{Bem}
  $f:U\to V\subset\mb{R}^m$ $f=(f_1,\cdots,f_m)$, $f_i(x)=f(x_1,\cdots,x_n)$\\
  $g:V\to \mb{R}^k$ $g=(g_1,\cdots,g_k)$, $g_j(x)=g(y_1,\cdots,y_m)$\\
  \[g\circ f(x)=\left( g_1(f(x)),\cdots,g_k(f(x)) \right)\]
  \[g_j(x)=g_j(f_1(x),\cdots,f_m(x))\]
  \[g_j(y)=g_j(f(x_1,\cdots,x_n),\cdots,f_m(x_1,\cdots,x_n))\]
  \[A_{lj}=\Part{}{x_j}(g_l\circ f)(a)\]
  \[M_{li}=\Part{g_l}{y_i}(b)=\Part{g_l}{y_i}(f(a))\]
  \[N_{ij}=\Part{f_i}{x_j}(a)\]
  \begin{eqnarray*}
    \Part{}{x_j}(g_l\circ f)(a)=A_{lj}=\sum_{i=1}M_{li}N_{ij}\\
    =\sum_{i=1}^m\Part{g_l}{y_i}(f(a))\Part{f_i}{x_j}(a)
  \end{eqnarray*}
\end{Bem}
\begin{Kor}
  Sei $f:U\to V(\subset\mb{R}^m)$ und $\phi:V\to\mb{R}$ mit:
  \begin{itemize}
    \item $a\in U$ und $U$ offen
    \item $b\in V$, $V$ offen und $b=f(a)$
    \item $f$ differenzierbar in $a$ und $\phi$ differenzierbar in $b$
  \end{itemize}
  Dann ist $\phi\circ f$ differenzierbar in $a$ und
  \[\Part{\phi\circ f}{x_j}(a)=\sum_{i=1}^m\Part{\phi}{y_i}(f(a))\Part{f_i}{x_j}(a)\]
  Das ist die ``konkrete'' allgemeine Kettenregel.
\end{Kor}

\subsection{Schrankensatz}

\begin{Def} sei $f:U\to\mb{R}^m$ eine Abbildung. Wir schreiben
  $f\in C^k(U,\mb{R}^m)$ falls die partielle Ableitungen jeder $f_i$ mit Ordnung $\leq k$ existieren und stetig sind ($f=(f_1,\cdots,f_m)$). 
\end{Def}

\begin{Sat}
  Sei $\Omega\subset\mb{R}^n$ eine offene Menge, $f\in C^1 (\Omega, \mb{R}^k)$
und $\gamma [a,b]\to\Omega$ eine $\mb{C}^1$ Kurve.
Dann:
  \[\Norm{f(\gamma(b))-f(\gamma(a))}\leq \left[ \sup_{t\in [a,b]}\Norm{\md f|_{\gamma(t)}}_O \right]\underbrace{\int_a^b\Norm{\dot \gamma(t)}\md t}_{\text{Länge der Kurve}}\]
\end{Sat}
Zur Erinnerung:  $\gamma:[a,b]\to\Omega\subset\mb{R}^n$, $\gamma=(\gamma_1,\cdots,\gamma_n)$, $\dot\gamma=(\gamma_1',\cdots,\gamma_n')$.
\begin{Bew}
  Sei $\phi:[a,b]\to\mb{R}^k$ die Funktion
  \[\phi(t):=f(\gamma(t))=f\circ\gamma\]
  Kettenregel
  \begin{equation}
    \label{e:1104131}
    \md \phi|_t=\md f|_{\gamma(t)}\md \gamma|_t
  \end{equation}
  \[\phi:[a,b]\to\mb{R}^k\]
  \[\md \phi|_t:\mb{R}\to\mb{R}^k\s\text{lineare Abbildung}\]
  $\phi=(\phi_1,\cdots,\phi_k)$
  \[ \begin{pmatrix}
    \Part{\phi_1}{t} \\ \vdots \\ \Part{\phi_k}{t} 
  \end{pmatrix} = \begin{pmatrix}
    \phi_1' \\ \vdots \\ \phi_k'
  \end{pmatrix} = \dot\phi \]
  Sei $A (x)$ die Jacobi-Matridx für $\md f|_x$ (d.h.  $A_{ij} (x) = \Part{f}{x_i} (x)$). 
Kettenregel:
  \[\underbrace{\dot\phi(t)=A(\gamma(t))\cdot \dot\gamma(t)}_{\text{Matrix-Darstellung von \eqref{e:1104131}}}\]
  \[f(\gamma(b))-f(\gamma(a))=\phi(b)-\phi(a)= \begin{pmatrix}
    \phi_1(b)-\phi_1(a) \\
    \vdots \\
    \phi_k(b)-\phi_k(a)
  \end{pmatrix}\]
  $\phi_i'$ ist eine stetige Funktion:
 \[
    \phi_i'(t)=\sum_{j=1}^nA_{ij}(\gamma(t))\gamma_j'(t) =
    \sum_{j=1}^n\Part{f_i}{x_j}(\gamma(t))\gamma_j'(t)
 \]
Nun
  \[\phi(b)-\phi(a)= \begin{pmatrix}
    \int_a^b\phi_1'(t)\md t\\
    \vdots \\
    \int_a^b\phi_k'(t)\md t
  \end{pmatrix} \]
und
 \[
    \Norm{f(\gamma(b))-f(\gamma(a))}= \Norm{\phi(b)-\phi(a)}^2
    =\sqrt{\sum^k_{i=1}\left( \int_a^b\phi_i'(t)\md t \right)^2}
\]
Wir brauchen nun die folgende``Dreiecksungleichun'':
\begin{equation}\label{e:int_drei}
\sqrt{\sum^k_{i=1}\left( \int_a^b\phi_i'(t)\md t \right)^2}
\leq \int_a^b \|\dot\phi (t)\| \md t\, .
\end{equation}
Diese Ungleichung folgt aus dem Lemma \ref{l:int_drei} unten.
Mit der schreiben wir
\begin{eqnarray*}
   \Norm{f(\gamma(b))-f(\gamma(a))}&\leq& \int_a^b \|\dot\phi (t)\| \md t
\;=\;  \int_a^b \| A (\gamma (t))\cdot \dot{\gamma} (t)\|\md t\\
&\leq& \int_a^b \|A (\gamma (t)\|_O \|\dot{\gamma} (t)\md t
\;=\ \int_a^b \|df|_{\gamma (t)}\|_O \|\dot{\gamma} (t)\md t\\
&\leq& \sup_{t\in [a,b]}\Norm{\md f|_{\gamma(t)}}_O
\end{eqnarray*}
\end{Bew}
\begin{Bem} In der Tat $\sup_{t\in [a,b]}\Norm{\md f|_{\gamma(t)}}_O$ ist
ein Maxim wegen der Stetigkeit der Abbildung $t\mapsto \Norm{\md f|_{\gamma(t)}}_O$.
\end{Bem}

 \begin{Lem}\label{l:int_drei}
  Sei $g:[a,b]\to\mb{R}^k$ eine stetige Funktion. Dann
  \[\sqrt{\sum_{i=1}^k\left( \int_a^bg_i \right)^2}\leq\int_a^b\Norm{g}\, .\]
  \ul{Dreiecksungleichung}
\end{Lem}
\begin{Bew}
  Sei $\varepsilon>0$ und Treppenfunktion $\alpha_i$ so dass $g_i-\varepsilon\leq\alpha_i\leq g_i+\varepsilon$, $\alpha_i-\varepsilon\leq g_i\leq \alpha_i+\varepsilon$. Dann
 \[\int_a^b\alpha_i-(b-a)\varepsilon
    \leq\int_a^b g_i \leq \int_a^b\alpha_i+(b-a)\varepsilon\, 
\]
d.h.
  \[\Abs{\int_a^bg_i-\int_a^b\alpha_i}\leq (b-a)\varepsilon\]
Deswegen
\begin{equation}\label{e:abs1}  
 \Abs{\sqrt{\sum_{i=1}^k(\int g_i)^2}-\sqrt{\sum_{i=1}^k \int\alpha_i^2}}
    \leq\sqrt{\sum_{i=1}^k \left(\int g_i-\int \alpha_i\right)^2}
    \leq \sqrt{k}(b-a)\varepsilon
  \end{equation}
Sei nun $\alpha = (\alpha_1, \ldots, \alpha_n)$. Dann
  \begin{equation}\label{e:abs2}
    \Abs{\int_a^b\Norm{g}-\int_a^b\Norm{\alpha}}
    \leq\int_a^b\Abs{\Norm{g}-\Norm{\alpha}}
    \leq\int_a^b\Norm{g-\alpha}
    \leq\int_a^b\sqrt{k}\varepsilon
    =\sqrt{k} (b-a)\varepsilon\, .
  \end{equation}
 Wir werden bewesein dass
    \begin{equation}
      \label{e:1104134}
      \sqrt{\sum\left(\int_a^b\alpha_i\right)^2}\leq\int_a^b\Norm{\alpha}
    \end{equation}
 \eqref{e:abs1}, \eqref{e:abs2} und \eqref{e:1104134} implizieren
  \begin{eqnarray*}
    \sqrt{\sum\left(\int_a^b g_i\right)^2}&\leq&
\sqrt{\sum\left(\int_a^b\alpha_i\right)^2}+(b-a)\sqrt{k}\varepsilon \\
    &\leq&\int_a^b\Norm{\alpha}+(b-a)\sqrt{k}\varepsilon
    \leq\int_a^b\Norm{g}+2(b-a)\sqrt{k}\varepsilon
  \end{eqnarray*}
  Wenn $\varepsilon\downarrow 0$:
  \[\sqrt{\sum\left( \int_a^bg_i \right)^2}\leq\int_a^b\Norm{g}\]

\medskip

{\bf Beweis von \eqref{e:1104134}.}  Ohne Beschränkung der Allgemeinheit: $\exists$ eine Zerteilung von $[a,b]$ 
  \[a=c_0<c_1<\cdots<c_N=b\]
so dass  jedes $\alpha_i$ ist konstant auf $[c_{j-1},c_j]=I_j$. Die Konstante ist $a_{i,j}$.
  \[\alpha = \begin{pmatrix}
    \alpha_1\\ \vdots \\ \alpha_k
  \end{pmatrix}\]
  ist konstant auf $I_j$ mit Wert
  \[a_j = \begin{pmatrix}
    a_{1,j}\\ \vdots \\ a_{k,j}
  \end{pmatrix}\]
  \[\sqrt{\sum_{i=1}^k\left( \int_a^b\alpha_i \right)^2}=\sqrt{\sum_{i=1}^k\left( \sum_{j=1}^N\abs{I_j}\alpha_{i,j} \right)^2}=\Norm{a}\]
wobei
\[
    a:=\sum_{j=1}^N\abs{I_j}a_j
    = \begin{pmatrix}
      \sum_{j=1}^N\abs{I_j}\alpha_{1,j}\\
      \vdots \\
      \sum_{j=1}^N\abs{I_j}\alpha_{1,j}\\
    \end{pmatrix}\, .
\]
Deswegen
\begin{eqnarray*}
\|a\|   &=& \Norm{\sum_{j=1}^N\abs{I_j}a_j}
\;\stackrel{\text{Dreiecksungleichung}}{\leq}\;\sum_{j=1}^N\Norm{\abs{I_j}a_j}\nonumber\\
& =&\sum_{j=1}^N\abs{I_j}\Norm{a_j}=\int_a^b\Norm{\alpha}
 \end{eqnarray*}
\end{Bew}

\begin{Kor}
  $f\in C^1(\Omega, \mb{R}^k)$ und $[p,q]\subset\Omega$. Dann:
  \[\Norm{f(p)-f(q)}\leq\max_{z\in [p,q]}\Norm{\md f|_z}_O \Norm{p-q}\]
\end{Kor}
\begin{Bew}
  Wenden den Schrankensatz an $f$ und $\gamma:[0,1]\to\Omega$ ist $\gamma(a)=(1-s)p+sq$.
Da $\dot \gamma=q-p$,
  \begin{eqnarray*}
    \Norm{f(p)-f(q)}\leq
    \max_{s\in [0,1]}\Norm{\md f|_{\gamma(s)}}_O \underbrace{\int_0^1\Norm{\dot\gamma(s)}\md s}_{\Norm{p-q}}
= \max_{z\in [p,q]}\Norm{\md f|_z}_O \Norm{p-q}\, .
  \end{eqnarray*}
\end{Bew}

\subsection{Satz der lokalen Umkehrbarkeit}
\begin{Sat}
  Sei $\Phi:\underbrace{U}_{\subset\mb{R}^n}\to\mb{R}^n$ ($U$ offene Menge) eine $\mb{C}^1$-Abbildung und sei $a\in U$ so dass $\md \Phi|_a$ umkehrbar ist. Dann $\exists U_0$ offene Umgebung von $a$ so dass $V:=\Phi(U_0)$ eine offene Umgebung von $\Phi(a)$ und die Einschränkung
  \[\Phi:U_0\to V\]
  ein Diffeomorphismus ist.
\end{Sat}
\begin{Lem}(Banachscher Fixpunktsatz)
  Sei $C\subset\mb{R}^n$ eine abgeschlossene Menge und sei $\phi:C\mapsto C$ eine Abbildung mit folgender Eigenschaft:
  \begin{align*}
    \Norm{\phi(x)-\phi(y)}\leq \lambda\Norm{x-y}& &\forall x,y\in C
  \end{align*}
  wobei $0\geq \lambda < 1$ (unabhängig von $x,y$).\\
  Dann $\exists x\in C$ so dass $\phi(x)=x$ (d.h. $x$ ein Fixpunkt von $\phi$ ist.
\end{Lem}
\begin{Def}
  Eine Abbildung
  \[\phi:X\mapsto X\s(\text{mit $X$ metrischer Raum})\]
  heisst Kontraktion falls $\exists \lambda <1$ so dass
  \begin{align*}
    \md (\phi(x),\phi(y))\leq\lambda\md(x,y)& &\forall x,y\in X
  \end{align*}
\end{Def}
\subsubsection{Allgemeine Form des Fixpunktsatzes von Banach}
\begin{Sat}
  Jede Kontraktion auf einem \ul{vollständigen} metrischen Raum besitzt einen Fixpunkt.
\end{Sat}
\begin{Bew}
  Sei $x_0\in X$ (bzw. in $C\subset\mb{R}^n$)
  \[ \begin{pmatrix}
    x_1=\phi(x_0)\\
    x_2=\phi(x_1)\\
    \vdots \\
    x_k=\phi(x_{k-1})
  \end{pmatrix}\]
  Behauptungen:
  \begin{enumerate}
    \item $\left\{ x_k \right\}$ ist eine Cauchyfolge 
      \[\xRightarrow{\text{Vollständigkeit von $X$}}\exists x\Limi{k}x_k\]
    \item $\phi(x)=x$
  \end{enumerate}
  1 $\implies$ 2 weil
  \[\phi(x)=\Limi{k}\phi(x_k)=\Limi{x_{k+1}}=x\]
\end{Bew}
\begin{Bew}
    \[\md(x_0,x_1)=M\geq 0\]
  \begin{eqnarray*}
    \md(x_{k+1},x_k)=\md\left( \phi(x_k),\phi(x_{k-1}) \right)\\
    \leq \lambda\md(x_k,x_{k-1})\leq \cdots \leq\lambda^2\md(x_{k-1},x_{k-2})\\
    \cdots\leq \lambda^k\md(x_1,x_0)=\lambda^kM
  \end{eqnarray*}
  \begin{eqnarray*}
    \md(x_{k+j}x_k)\\
    \leq\md(x_{k+j},x_{k+j-1})+\md(x_{k+j-1},x_{k+j-2})+\cdots+\md(x_{k+1},x_{k})\\
    \leq\lambda^{k+j-1}M+\lambda^{k+j-2}M+\cdots+M\lambda^k
  \end{eqnarray*}
  \begin{eqnarray*}
    \md(x_{k+j},x_k)\leq M\lambda^k(1+\lambda+\cdots+\lambda^{j-1})\\
    <M\lambda^k\sum^\infty_{i=0}\lambda^i\\
    =\frac{M\lambda^k}{1-\lambda}
  \end{eqnarray*}
  Deswegen $\forall m>n\geq N$ ($\lambda^N\to$ für $N\to+\infty$
  \[\md(x_m,x_n)\leq\frac{M}{1-\lambda}\lambda^N\]
  $\forall \varepsilon>0$ $\exists N$ so dass
  \[\frac{M\lambda^N}{1-\lambda}<\varepsilon\]
  \begin{align*}
    \implies\md(x_m,x_n)<\varepsilon& & \forall n>m\geq N
  \end{align*}
  Das ist die Cauchyeigenschaft $\implies$ $\left\{ x_k \right\}$ ist eine Cauchyfolge
\end{Bew}
\paragraph{Beweis des Satzes}
\subparagraph{Schritt 1}
Wir suchen eine Umgebung von $W$ von $\Phi(a)$, wo wir immer ein Urbild von $\in W$ finden. D.h. 
\begin{equation}
  \label{e:1105021}
  \Phi(x)=y
\end{equation}
besitzt eine Lösung $x$.\\
OBdA nehmen wir an $a=0$ und $\md\Phi|_a=\id$ % identität
(In der Tat, nehmen wir an dass
\[L=\md\Phi|_a\neq \id\]
Sei 
\[\Phi'=L^{-1}\circ \Phi\] und 
\[\md\Phi'|_x=L^{-1}\circ \md\Phi|_x\]
$\implies$ $\Phi'$ ist eine $\mb{C}^1$-Funktion.
\[\md\Phi|_0=L^{-1}\circ\md\Phi|_0=L^{-1}\circ L=\id\]
$\implies$ Satz an $\Phi'$ anwenden
\[\Psi'(\Phi'(x))=x\implies \Psi'(L^{-1}(\Phi(x)))=x\]
\[\implies\Psi:=\Phi'\circ L^{-1}\]
die gesuchte Umkehrung von $\Phi$ ist $V:=(V')$)
Wir wollen zeigen dass, wenn $\Norm{y-\Phi(0)}<\delta$, dann die Gleichung \ref{e:1105021} lösbar ist.
\[\ref{e:1105021}\iff \underbrace{y+x-\Phi(x)}_{x\mapsto \phi_y(x)}=x\]
$\phi_y:U\to\mb{R}^n$ $\exists \eta>0$ so dass
\[\phi_y:\ol{B_\eta}(0)\mapsto \ol{B_\eta}(0)\]
eine Kontraktion ist.
\begin{enumerate}
  \item $\phi_y$ bildet $\ol{B_y}(0)$ in $\ol{B_\eta}(0)$
  \item $\Norm{\phi_y(z)-\phi_y(w)}\leq\frac{1}{2}\Norm{z-w}$
\end{enumerate}
Das zweite:
\begin{eqnarray*}
  \Norm{\phi_y(z)-\phi_y(w)}\\
  =\Norm{y+z-\Phi(z)-y-w+\Phi(w)}\\
  =\Norm{(\Phi(w)-\Phi(z))-(w-z)}\\
  =\Norm{\underbrace{\Phi(w)-w}_{\Lambda(w)}-\underbrace{\Phi(z)-z}_{\Lambda(z)}}
\end{eqnarray*}
$\Lambda$ ist $\mb{C}^1$
\[\md\Lambda|_0=\md\Phi|_0-\id=0\]
\[\Norm{\md|_0}_{HS}=0\]
$\implies$ $\exists \eta>0$ so dass
\[B_{\leq \eta}(0)\ni x\implies\Norm{\md\Lambda|_x}_{HS}\leq\frac{1}{2}\]
$z,w\in \ol{B_\eta}(0)$ und $\in B_\eta(0)$
\begin{eqnarray*}
  \Norm{\phi_y(z)-\phi_y(w)}=\Norm{\Lambda(z)-\Lambda(w)}\\
  \stackrel{\text{Schrankensatz}}{\leq}\left( \max_{\ol{B_\eta}(0)}\Norm{\md\Lambda}_O \right)\Norm{z-w}\\
  \frac{1}{2}\Norm{z-w}
\end{eqnarray*}
\[\phi_y(0)=y-\Phi(0)+0=y-\Phi(0)\]
$\delta=\frac{\eta}{2}$, $\Norm{\phi_y(0)}\leq\frac{1}{2}$. Sei $z\in\ol{B_\eta}(0)$
\begin{eqnarray*}
  \Norm{\phi_y(z)}\Norm{\phi_y(z)-\phi_y(0)}+\Norm{\phi_y(0)}\\
  <\Norm{\phi_y(z)-\phi_y(0)}+\frac{\eta}{2}\\
  \leq\frac{1}{2}\Norm{z-0}+\frac{\eta}{2}\\
  \leq \frac{1}{2}\eta+\frac{1}{2}\eta\\
  =\eta\\
  \implies\Norm{\phi_y(z)}<\eta
\end{eqnarray*}
So 
\[\phi_y:\ol{B_\eta}(0)\mapsto B_\eta(0)\]
Banach: $\forall y\in B_{\frac{\eta}{2}}(\Phi(0))$, $\exists x \in B_\eta(0)$ und $\in \ol{B_\eta}(0)$ mit
\[\phi_y(x)=x\iff\Phi(x)=y\]
Sei $V:=B_\delta(\Phi(0))$ (offen und Umgebung von $\Phi(0)$)
\begin{align*}
  \underbrace{B_\eta(0)\cap\Phi^{-1}(V)}_{\text{ist eine offene Menge}}=U_0 & & (\text{offen und Umgebung von 0})
\end{align*}
\[Phi:U_0\to V\]
\begin{enumerate}
  \item $\Phi$ ist surjektiv: $\forall y\in V$, $\exists x\in B_\eta(0)$ mit $\Phi(x)=y$
    \[\implies x\in \Phi^{-1}(V)\cap B_\eta(0)=U_0\]
  \item $\Phi$ ist injektiv
    \begin{eqnarray*}
      \Norm{\Phi(x)-\Phi(z)}=\Norm{(x+\Lambda(x))-(z+\Lambda(z))}\\
      \implies\Norm{\Phi(x)-\Phi(z)}\\
      \leq\Norm{x-z}-\Norm{\Lambda(x)-\Lambda(z)}\\
      \leq\Norm{x-z}-\frac{1}{2}\Norm{x-z}\\
      \leq\frac{1}{2}\Norm{x-z}\\
    \end{eqnarray*}
    $\implies$ $\Phi$ ist injektiv. (Alternativerweise wenn $\phi$ eine Kotraktion ist, der Fixpunkt ovn $\phi$ ist eindeutig: $\phi(p)=p$, $\phi(q)=q$
    \begin{eqnarray*}
      \md(p,q)=\md(\phi(p),\phi(q))\\
      \leq\lambda\md(p,q)\\
      (1-\lambda)\md(p-q)\leq 0\\
      \xRightarrow{\lambda<1}\md(p,q)=0\\
      \implies p=q
    \end{eqnarray*}
\end{enumerate}
\subparagraph{Schritt 2}
Sei $\Phi:V\mapsto U_0$ die Umkehrfunktion von $\Phi$. $\Psi$ ist stetig. Seien $\xi, \zeta\in V$, $x=\Phi(\xi), z=\Phi(\zeta)$ $\implies$ $\Phi(x)=\xi$, $\Phi(z)=\zeta$. Aber:
\begin{eqnarray*}
  \Norm{\Phi(x)-\Phi(z)}\leq\frac{1}{2}\Norm{x-z}\\
  \implies \underbrace{2\Norm{\xi-\zeta}\geq \Norm{\Psi(\xi)-\Phi(\zeta)}}_\text{Lipschitz-Bedingung für $\Phi$: stetig}
\end{eqnarray*}
\subparagraph{Schritt 3}
\begin{Bem}
  $\Phi:U_0\to V$ ist differenzierbar und $\md\Phi|_x$ ist umkehbar $\forall x\in U_0$.
  \[\Phi(x)=x-\Lambda(x)\]
  \[\md\Phi|_x=\id-\md\Lambda|_x\]
  Wir wissen, dass
  \begin{align*}
    \Norm{\md\Lambda|_x}_{HS}\leq\frac{1}{2}& &\forall x\in U_0\subset B_\eta(0)
  \end{align*}
  \[\md\Phi|_x(v)=v-\md\Lambda|_x(v)\]
  \begin{eqnarray*}
    \Norm{\md\Phi|_x(v)}\\
    \geq \Norm{v}-\Norm{\md\Lambda|_x(v)}\\\
    \geq\Norm{v}-\frac{1}{2}\Norm{v}\\
    \geq\frac{1}{2}\Norm{v}
  \end{eqnarray*}
  $\implies$ $\Ker(\md\Phi|_x)=\left\{ 0 \right\}$ $\implies$ $\md\Phi|_x$ ist injektiv $\implies$ Surjektivität $\implies$ $\md\Phi|_x$ ist umkehrbar
\end{Bem}
\begin{Lem}
  Falls $\Phi:U_0\to V$ eine $\mb{C}^1$ umkehrbare Abbildung so dass
  \begin{itemize}
    \item $\md\Phi|_x$ umkehrbar $\forall x\in U_0$ ist
    \item die Umkehrfunktion $\Psi:V\to U_0$ stetig ist, dann ist auch $\Psi$ eine $\mb{C}^1$ Abbildung.
  \end{itemize}
\end{Lem}

\begin{Lem}
  Sei $\Phi:U\to V$ eine $\mb{C}^1$ umkehrbare Abbildung ($U,V\subset\mb{R}^n$ offene Menge). Sei $\Psi:V\to U$ die Umkehrfunktion von $\Phi$. Wenn $\Psi$ stetig ist und $\md \Phi|_x$ umkehrbar $\forall x$ ist, dann ist auch $\Psi$ eine $\mb{C}^1$-Abbildung.
\end{Lem}
Mit dem Lemma schliessen wir den dritten Schritt des Beweises der lokalen Umkehrbarkeit.
  Es genügt die differenzierbarkeit von $\Psi$ zu zeigen. In diesem Fall:
  \[\md\Psi|_y= (\md\Phi|_{\Psi(y)})^{-1}\]
  Falls $M(y)$ die Jacobimatrix von $\md\Psi|_y$ und $N(x)$ die Jacobimatrix von $\md\Phi|_x$ ist
  \[M_{ij}(y)=(N(\Psi(y))^{-1}_{ij}=\frac{(-1)^{i+j}}{\det N(\Psi(y))}\det C\circ f^{ij}(N(\Psi(y)))\]
  \[C\circ f^{ij}= \begin{pmatrix} \cdots \end{pmatrix}\]
  $C\circ f^{ij}$ ist die Matrix $\in ( (n-1)\times (n-1))$ die wir von $N$ erhalten, wenn wir die $i$-te ZEile und die $j$-te Spalte löschen. $\implies$ die Koeffizienten $M_{ij}(y)$ ist eine stetige Funktion.
  \[M_{ij}(y)=\Part{\Psi_i}{y_j}\]
  Diff von $\Psi$. Wir fixieren $y\in V$
  \[\Psi(y_0)\implies\Phi(x_0)=y_0\]
  Wir setzen
  \[L:=(\md\Phi|_{x_0})^{-1}\]
  Sei $\Phi'=L\circ\Phi$, $\Phi\in\mb{C}^1$
  \[\md\Phi'|_{x_0}L\circ\md\Phi|_{x_0}=\id\]
  \[\md\Phi'|_x=\underbrace{L\circ\md\Phi|_x}_{\text{umkehrbar}}\]
  Dann ist $\Psi'=\Psi\circ L^{-1}$ die Umkehrfunktion von $\Phi'$
\begin{Beh}
  $\Phi'$ ist an der Stelle 
  \[y_0'=\Phi'(x_0)=L(\Phi(x_0))=L(y_0)\]
  differenzierbar $\implies$ $\Psi=\Psi'\circ L$ ist an der Stelle $y_0$ differenzierbar
  \[\md\Psi|_{y_0}=\md\Psi'|_{L(y_0)}\]
\end{Beh}
OBdA können wir zusätzlich annehmen $\md\Phi|_{x_0}=\id$
\begin{Bem}
  $\exists B_\delta(y_0)$ so dass 
  \[\Norm{\Psi(z)-\Psi(w)}\leq 2\Norm{z-w}\]
  Falls $\delta$ klein genug ist, $\underbrace{\Psi(z)}_\zeta$ und $\underbrace{\Psi(w)}_\omega$ in einer Umgebung von $x_0$
  \begin{eqnarray*}
    \Norm{(\Phi(\zeta)-\Phi(\omega))-(\zeta-\omega)}=\Norm{\underbrace{(\Phi(\zeta)-\zeta)}_{\Lambda(\zeta)}-\underbrace{(\Phi(\omega)-\omega)}_{\Lambda(\omega)}}\\
    \md\Lambda|_{x_0}=\md\Phi|_{x_0}-\id =0\\
    \implies\Norm{\md\Lambda|_x}_O\leq\frac{1}{2}\s\forall x\in \ol{B_\varepsilon(x_0)}\\
    \leq\max_{x\in \ol{B_\varepsilon(x_0)}}\Norm{\md\Lambda|_x}_O\Norm{\zeta-\omega}\\
    \implies \Norm{(\Phi(\zeta)-\Phi(\omega))-(\zeta-\omega)}\leq\frac{1}{2}\Norm{\zeta-\omega}\\
    % TODO fit that in somewhere: \Norm{\zeta-\omega}-\Norm{\Phi(\zeta)-\Phi(\omega)}
    \implies \frac{1}{2}\Norm{\zeta-\omega}\leq\Norm{\Phi(\zeta)-\Phi(\omega)}\\
    \implies \frac{1}{2}\Norm{\Psi(z)-\Psi(w)}\leq\Norm{z-w}\\
    \Norm{\Psi(z)-\Psi(w)}\leq2\Norm{z-w}\\
    \Phi(x)-\Phi(x_0)=\underbrace{\md\Phi(x_0)}_{\id} (x-x_0)+\underbrace{R(x)}_{o(\Norm{x-x_0}}\\
    \overbrace{\Phi(x)}^{y\iff x=\Psi(y)}-\Phi(x_0)=(x-x_0)+R(x)\\
    y-x_0=\Psi(y)-\Psi(y_0)+R(\Psi(y))\s\forall y\in B_\delta(y_0)\\
    \implies (\Psi(y)-\Psi(y_0))=\id (y-y_0)-R(\Psi(y))\\
  \end{eqnarray*}
  Für die differenzierbarkeit brauchen wir $R(\Psi(y))=o(\Norm{y-y_0})$
  \begin{eqnarray*}
    \frac{\abs{R(\Psi(y))}}{\Norm{y-y_0}}=\frac{\abs{\Psi(y))}}{\Norm{\Psi(y)-\Psi(y_0)}} \frac{\Norm{\psi(y)-\Psi(y_0)}}{\Norm{y-y_0}}\leq 2\frac{\abs{R(\Psi(y))}}{\Norm{y-x_0}}\\
    \lim_{y\to y_0}2\frac{\abs{R(\Psi(y))}}{\Norm{y-x_0}}=\lim_{x\to x_0}\frac{\abs{R(x)}}{\Norm{x-x_0}}=0\\
    y\to y_0\implies \Psi(y)\to\Psi(y_0)
  \end{eqnarray*}
\end{Bem}
\subsection{Lösungen von Gleichungen}
Der Satz über implizite Funktionen
\[x^2+bx+c=0\iff(x,b,c)\in \s\text{Nullstellen von}\s f\]
\[f(x,b,c)=x^2+bx+c\]
Die Gleichung zu lösen $\iff$ Es gibt eine Funktion $g$ so dass $x=g(b,c)$ die Gleichung löst.
\begin{align*}
  f(x,a)=0 & & x\in\mb{R}, a\in\mb{R}
\end{align*}
% TODO Skizze
Sei $(x_0,a_0)$ Nullstelle. Gibt es in einer Umgebung von $(x_0,a_0)$ eine ``Formel'' $x=g(a)$ für die Lösungen?
\paragraph{Allgemein}
Sei $f:\underbrace{U}_{\mb{R}^n=\mb{R}^{k+m}}\to\mb{R}^k$. $z\in\mb{R}^{k+m}$ schreiben wir als 
\[z=(\underbrace{x_1,\cdots,x_n}_x, \underbrace{y_1,\cdots,y_m}_y)=(x,y)\]
Das System
\[\begin{cases}
  f_1(x_1,\cdots,x_k,y_1,\cdots,y_m)=0\\
  f_2(x_1,\cdots,x_k,y_1,\cdots,y_m)=0\\
  \vdots \\
  f_k(x_1,\cdots,x_k,y_1,\cdots,y_m)=0
\end{cases} \iff f(x,y)=0\]
\begin{Sat}
  Sei $f:U\to\mb{R}^k$ eine $\mb{C}^1$ Abbildung und $(\bar x, \bar y)$ eine Nullstelle von $f$. Falls $\md_xf|_{\bar x,\bar y}$ umkehrbar ist dann $\exists$ $U,V$ Umgebungen von $\bar y$ und $\bar x$ und eine $\mb{C}^1$ Funktion
  \[g:U\to V\s\text{so dass}\s \left\{ (g(y),y) \right\}=\left\{ f=0 \right\}\cap V\times U\]
\end{Sat}
\begin{Bem}
  Jacobi Matrix für $\md_xf|_{(\bar x, \bar y)}$
  \[ \begin{pmatrix}
    % TODO markiere x_1 bis x_k
    \Part{f_1}{x_1}&\Part{f_1}{x_2}&\cdots&\Part{f_1}{x_k}&\Part{f_1}{y_1}&\cdots&\Part{f_1}{y_m}\\
    \Part{f_2}{x_1}&\Part{f_2}{x_2}&\cdots&\Part{f_2}{x_k}&\Part{f_2}{y_1}&\cdots&\Part{f_2}{y_m}\\
    \vdots&\vdots&\vdots&\vdots&\vdots&\vdots&\vdots\\
    \Part{f_k}{x_1}&\Part{f_k}{x_2}&\cdots&\Part{f_k}{x_k}&\Part{f_k}{y_1}&\cdots&\Part{f_k}{y_m}\\
  \end{pmatrix}\]
  % TODO Skizze
  $f=f_1$, $(x_1,y_1)$
  \[\left( \Part{f_1}{x_1}, \Part{f_1}{y_1} \right)\]
  \[\md_xf=\left( \Part{f_1}{x_1} \right)\]
  \[\Part{f_1}{x_1}\neq 0\]
\end{Bem}
\begin{Bem}
  Satz nicht benutzen, wenn der Gradient verschwindet!
\end{Bem}
\begin{Bew}
  Sei
  \[\Phi(x,y)=(f(x,y),y)\]
  \[\Phi:\underbrace{\tilde U}_{\subset\mb{R}^{k+m}}\to\mb{R}^{k+m}\]
  $\md\Phi|_{(\bar x, \bar y)}$ Jacobi Matrix
  \[ \begin{pmatrix}
    \Part{f_1}{x_1}&\cdots&\Part{f_1}{x_k}&\Part{f_1}{y_1}\cdots\Part{f_1}{y_m}\\
    \vdots & & \vdots & \vdots & & \vdots \\
    \Part{f_1}{x_1}&\cdots&\Part{f_1}{x_k}&\Part{f_1}{y_1}\cdots\Part{f_1}{y_m}\\
    0 & \cdots & 0 & 1 & \cdots & 0\\
    0 & \cdots & 0 & 0 & 1 \cdots & 0\\
    \vdots & & \vdots &\vdots & & \vdots\\
    0 & \cdots & 0 & 0 & \cdots & 1
  \end{pmatrix} = \begin{pmatrix}
    \md_x f & \md_y f\\
    0 & \id
  \end{pmatrix}\]
\end{Bew}
\begin{Ueb}
  \[\Ker \md \Phi|_{(\bar x,\bar y)}=\left\{ 0 \right\}\]
  sonst $\det \neq 0$
\end{Ueb}

\subsection{Satz über implizite Funktionen}
Wir nutzen die Koordinaten $(x,y) = (x_1,\cdots,x_k,y_1,\cdots,y_n)\in\mb{R}^n\times\mb{R}^k$
\begin{Sat}
  Sei $f:U\to\mb{R}^k$ eine $\mb{C}^1$ Abbildung (wobei $U$ eine offene Menge in $\mb{R}^n\times\mb{R}^k$ ist). Sei $(a,b)\in U$ mit der Eigenschaft dass $\md_y f$ umkehrbar ist und $f(a,b)=0$. Vom letzten Mal:
  \[\md f=(\underbrace{\md_x f}_{J_{fx}}, \underbrace{\md_y f)}_{J_{fy}}\]
  Dann: $\exists$ $U', U''$ Umgebungen von $a$ und $b$ und eine $\mb{C}^1$-Abbildung $g:U'\to U''$ so dass
  \[\left\{ (x,y)\in U'\times U''\s f(x,y)=0 \right\}=\left\{ (x,y(x))\s x\in U' \right\}\]
\end{Sat}
% TODO Skizze (Bsp)
\begin{Bew}
  Sei $\Phi:U\to\mb{R}^n\times\mb{R}^k$
  \[\Phi(x,y)=(\underbrace{x}_{\in\mb{R}^n},\underbrace{f(x,y)}_{\in\mb{R}^k})\in\mb{R}^n\times\mb{R}^k\]
  \[\md\Phi|_{(a,b)}= \begin{pmatrix}
    1 & & 0 & & \\
    & \ddots & 1 & 0 & \\
    0 & & 1 & & \\
    & & & & \\
    & \md_x f & & \md_y f \\
    & & & & \\
  \end{pmatrix} \]
  Letztes Mal: $\md\Phi|_{(a,b)}$ ist umkehrbar. Deswegen $\exists$ $U_0$ offene Umgebung von $(a,b)$, $\exists$ $V$ offene Umgebung von $(a,0)=\Phi(a,b)$ und $\Psi:V\to U_0$ $\mb{C}^1$-Abbildung so dass
  \[\Psi(\Phi(x,y))=\Phi(\Psi(x,y))=(x,y)\]
  d.h. $\Psi=\left( \Phi|_{U_0} \right)^{-1}$
  \begin{eqnarray*}
    \mb{R}^n\times\mb{R}^k\ni\Psi(x,y)=(\underbrace{\xi(x,y)}_{\in\mb{R}^n},\underbrace{\zeta(x,y))}_{\mb{R}^k}\s\forall (x,y)\in V\\
    (x,y)=\Phi(\Psi(x,y))=\Phi(\xi(x,y),\zeta(x,y))\\
    \implies (x,y)=(\xi(x,y), f(x,\zeta(x,y)))\\
    \iff \begin{cases}
      x=\xi(x,y)\\
      y=f(x,\zeta(x,y))
    \end{cases}\implies \Psi(x,y)=(x,\zeta(x,y)) \\
  \end{eqnarray*}
  Aus der zweiten Gleichung $0=f(x,\overbrace{\zeta(x,0)}^{y(x)}$. Deswegen enthält $V$ $(0,0)$. Deswegen $\exists r > 0$ so dass
  \[\underbrace{B^n_r(a)}_{\mb{R}^n}\times \underbrace{B^k_r(0)}_{\mb{R}^k}\subset V\]
  Sei $U'=B_r^n(a)$
  \begin{eqnarray*}
    \Psi(B^n_r(a)\times B_r^k(0))=\text{offene Menge}\\
    \supset B^n_r(a)\times B_p(b)\s \exists p\\
  \end{eqnarray*}
    weil $\Psi(0,0)=(a,\underbrace{\zeta(a,0)}_{\text{soll $b$ sein weil $\Psi$ die Umkehrung von $\Phi$ ist}})$ und $f(a,\zeta(a,0))=0$. $g$ ist eine stetige Funktion
    \begin{eqnarray*}
      g(a)=\zeta(a,0)=b\\
      g^{-1}(B_p(b))=:U'
    \end{eqnarray*}
    $U'$ ist eine offene Menge und enthält $a$.
    \begin{eqnarray*}
      B_p(b)=U''\\
      g:\underbrace{U'}_{\text{Umgebung von $a$}} \to\underbrace{U''}_{\text{Umgebung von $b$}}\\
      f(x,g(x))=0\implies \left\{ (x,y)\in U'\times U'':f(x,y)=0 \right\}\supset \left\{ (x,g(x)): x\in U' \right\}
    \end{eqnarray*}
    Sei $(x,y)\in U'\times U'': f(x,y)=0$. Aber
    \[(x,y)\in U_0\implies \Phi(x,y)=(x,f(x,y))=(x,0)\in V  \implies\Psi(x,0)=(x,y)\]
    weil $\Psi$ die Umkehrung von $\Phi$ ist
    \begin{eqnarray*}
      \Psi(x,0)=(x,\zeta(x,0))=(x,g(x))\implies y=g(x)\\
      \implies \left\{ (x,y=\in U'\times U'':f(x,y)=0 \right\}\subset \left\{ (x,g(x)):x\in U' \right\}\\
    \end{eqnarray*}
\end{Bew}
\begin{Bem}
  Seien $f$ und $g$ wie im letzten Sat. Dann
  \begin{align*}
    \underbrace{f}_{\mb{C}^1}(x,\underbrace{g(x))}_{\mb{C}^1}=0&&\forall x\in U'
  \end{align*}
  \begin{eqnarray*}
    \implies \md (f(x,g(x)))|_{x_0}=0\\
    \md f|_{(x_0,g(x_0))}(\md(x,g(x)))|_{x_0}\\
    \md f|_{(x_0,g(x_0))}=\left( \underbrace{\md_x f|_{(x_0,g(x_0))}}_{n\times k}, \underbrace{\md_y f|_{(x_0,g(x_0))}}_{k\times k} \right)
  \end{eqnarray*}
  Darstellung von $\md(x,g(x))|_{x_0}$
  \[ \begin{pmatrix}
    \id \\ \md g|_{x_0}
  \end{pmatrix} \]
  $\xRightarrow{\text{Matrix-Produkt}}$
  \begin{eqnarray*}
    \md (f(x,g(x)))|_{x_0}=\md_x f|_{(x_0,g(x_0))} + \md_y f|_{(x_0,g(x_0))} \md g|_{x_0}\\
    \md|_{(x_0,g(x_0))}\md_xg|_{x_0}=-\md_x f|_{(x_0,g(x_0))}\\
    \implies \md_x g|_{x_0}= - \left( \md_yf|_{(x_0,g(x_0))} \right)^{-1}\left( \md_xf|_{(x_0,g(x_0))} \right)
  \end{eqnarray*}
\end{Bem}
\begin{Bsp}
  $k=n+1$
  \[f(x_1,y_1)=0\]
  \[\Part{f}{y_1}(a,b)\neq 0\]
  \begin{eqnarray*}
    \exists g:U'\to U''\s\text{mit}\s f(x,g(x))=0\\
    \implies 0=\Part{f}{x_1}(x_1,g(x_1))+\Part{f}{y_1}(x_1,g(x_1))g'(x_1)\\
    \implies g'(x_1)= - \frac{\Part{f}{x_1}(x_1,g(x_1))}{\Part{f}{y_1}(x_1,g(x_1)}
  \end{eqnarray*}
\end{Bsp}
\subsection{Untermannigfaltigkeiten}
\begin{Def}
  Eine Menge $E\subset\mb{R}^N$ ist eine $\mb{C}^1$-Untermannigfaltigkeit mit Dimension $k$ falls $\forall p\in E$ die folgende Eigenschaft gilt: $\exists$ eine Ordnung der Koordinaten $(x_1,\cdots,x_{N-k}, y_1,\cdots,y_n$ so dass
  \begin{itemize}
    \item $p=(\underbrace{a}_{\in\mb{R}^n\times \mb{R}^k},b)$
    \item $\exists$ $U,V$ von $a$ und $b$ (Umgebungen)
    \item $\exists f:U\to V$ $\mb{C}^1$ so dass
      \[(U\times V)\cap E= \left\{ (x,f(x)):x\in U \right\}\]
  \end{itemize}
\end{Def}
% TODO Bsp
\begin{Def}
  Sei $N\geq k+1$ $f:\underbrace{U}_{\mb{R}^N}\to\mb{R}^k$ eine $\mb{C}^1$-Abbildung. $c\in\mb{R}^k$ heisst ein regilärer Wert falls $\forall x_0\in U$ so dass $f(x_0)=c$ hat das Differential $\md f|_{x_0}$ Rang $k$ (maximaler Rang)
\end{Def}
\begin{Sat}
  Falls $c$ ein regulärer Wert ist, dann 
  \[f^{-1}(\left\{ c \right\})=\left\{ z:f(z)=c \right\}\]
  ist eine Untermannigfaltigkeit mit Dimension $N-k$.
\end{Sat}
\begin{Bew}
  Sei $p$ so dass $f(p)=c$
  \[\md f|_p= \left( \underbrace{\Part{f}{z_1}}_{\begin{pmatrix} \Part{f_1}{z_1}\\ \vdots \\ \Part{f_k}{z_1}\end{pmatrix}},\cdots\Part{f}{z_2},\cdots,\Part{f}{z_N} \right)\]
  Neue Ordnung der Koorddinaten:
  \begin{eqnarray*}
    \md f|_p=\left( \Part{f}{x_1},\cdots,\Part{f}{x_{N-k}},\Part{f}{y_1},\cdots,\Part{f}{y_k} \right)\\
    = (\md_x f|_p, \md_y f|_p)
  \end{eqnarray*}
  Rang $k$ $\implies$ $\exists$ $k$ linear unabhängige Spalten. $\xRightarrow{\text{Satz über implizite Funktionen}}$ $\exists$ $\underbrace{U'\times U''}_U$ Umgebung von $p=(a,b)$ so dass
  \[U\cap E=\left\{ q\in U:f(q)-c=0 \right\}=\left\{ (x,g(x)):x\in U' \right\}\]
\end{Bew}
\subsection{Die Multiplikationsregel von Lagrange}
\begin{Sat}
  Sei $U\subset\mb{R}^n$ und seien $(\phi_1,\cdots,\phi_j)$ die Nebenbedingungen. D.h. $j$ verschiedene Funktionen $\phi_i:U\to\mb{R}$. Sei $f:U\to\mb{R}$ eine $\mb{C}^1$-Funktion und $p$ ein Punkt wo die Funktion $f$ das Maximum (bzw. das Minimum) mit Nebenbedingungen erreicht.
  \[(\phi_1(p)=\cdots=\phi_j(p)=0\s\text{und}\s f(p)\geq f(q)\s\forall\s\text{mit}\s\phi_1(q)=\cdots\phi_j(q)=0\]
  Falls $\nabla\phi_1(p),\cdots,\nabla\phi_j(p)$ linear unabhängige Vektoren sind, dann: $\exists \lambda_1,\cdots,\lambda_j$ so dass
  \[\nabla f(p)=\lambda_1\nabla\phi_1(p)+\cdots+\lambda_j\nabla\phi_j(p)\]
  (Multiplikatorregel)
\end{Sat}
\begin{Bem}
  (NB) + (MR) ist eine System von Gleichungen mit Unbekannten $\lambda_1,\cdots,\lambda_j, p$
\end{Bem}
\begin{Bem}
  $ \underbrace{\begin{pmatrix}
    \nabla\phi_1\\ \vdots \\ \nabla\phi_j
  \end{pmatrix}}_{j\times n \s\text{Matrix}} $ sind die Zeilen von $\md \phi$. Lineare Unabhängigkeit $\implies$ $\md \phi|_p$ hat maximaler Rang, d.h. $j$. 
  \[\xRightarrow{\text{Satz über die impliziten Funktionen}} \text{(NB)}\implies \iff \left\{ \phi=0 \right\}\iff \left\{ (x,g(x)), x\in U' \right\}\]
  in einer Umgebung $U'\times U''$.
  Sei $(x_0,g(x_0))=p$. $f$ hat ein Maximum in $p$ unter (NB).
  \[\implies x\mapsto f(x,g(x))=h(x)\]
  hat ein Maximum in $x0$
  \[\implies\md h|_{x_0}=0\implies\text{(MR)}\]
\end{Bem}

\begin{Sat}
  \[\phi=(\phi_1,\cdots,\phi_j):\underbrace{\mb{R}^{j+k}}\to\mb{R}^k\]
  ist eine $\mb{C}^1$-Abbildung.
  \[E=\left\{ q:\phi(q)=0 \right\}\]
  Sei $f:U\to\mb{R}$ ist eine $\mb{C}^1$-Funktion. Sei $p\in E$ so dass
  \begin{align*}
    f(p)\geq f(q)&&\forall q\in E
  \end{align*}
  $f(p)=\max_{q\in E}f$. Bzw. $f(p)\leq f(q)$ für Minimum. Falls der Rang $\md\phi|_p$ maximal ist (d.h. $=k$), dann $\exists \lambda_1,\cdots,\lambda_k\in\mb{R}$ so dass
  \[\nabla f=\lambda_1\nabla \phi_1,\cdots,\lambda_j \nabla\phi_j\]
  $\lambda_1,\cdots,\lambda_n$ sind die Multiplikatoren von Lagrange.
\end{Sat}
\begin{Bew}
  $\md\phi|_p$ eine $j\times n+j$ Matrix. $j$ Zeilen, $n+j$ Spalten $\implies$ die Zeilen sind linear unabhängige Vektoren. $\implies$ $\exists$ $j$ linear unabhängige Spalten. OBdA sind die $j$-Spalten die letzten (rechts). Seien $(x_1,\cdots,x_n,y_1,\cdots,y_j)$ ein System von Koordinaten in $\mb{R}^{n+j}$
  \[\md\phi|_p=(\md_x\phi|_p\underbrace{\md_y\phi|_p}_{j\times j \text{$j$-Matrix}})\]
  Die $j$-Matrix hat Rang $j$ $\implies$ ist umkehrbar. Theorem über die implizite Funktion: $\exists$ $U$ von $a$ und $\exists V$ von $b$ und eine $\mb{C}^1$ Abbildung $g:U\to V$ so dass
  \[E\cap U\times V=\left\{ \phi=0 \right\}\cap U\times V=\left\{ (x,g(x)):x\in U \right\}\]
  Betrachte: $h:U\to\mb{R}$ $h(x)=f(x,g(x))$ in $U$ ist $a$ eine Maximumstelle von $h$
  \begin{eqnarray*}
    h(a)=f(p)\\
    h(x)=f(\underbrace{x,g(x)}_{q\in E})\implies h(a)\geq h(x)\\
    \implies \md_x h|_a=0
  \end{eqnarray*}
  Kettenregel:
  \begin{eqnarray*}
    \md_x h|_a=\md_xf|_{\underbrace{(a,g(a))}_p}+\md_yf|_p\md_xg|_a\\
    \implies 0=\md_xf|_p+\md_yf|_p\md_xg|_a
  \end{eqnarray*}
  Sei $\Psi(x)=(x,g(x))$
  \begin{eqnarray*}
    \md f=(\md_xf\md_yf)\\
    \md\Psi|_a = \begin{pmatrix}
      1&0&\cdots&0\\
      0&1&\cdots&0\\
      \vdots&\vdots&\vdots&\vdots\\
      0&0&\cdots&0\\
      \Part{g_1}{x_1}&&\cdots&\Part{g_1}{x_n}\\
      \vdots&\vdots&\vdots&\vdots\\
      \Part{g_j}{x_n}&&\cdots&\Part{g_j}{x_n}
    \end{pmatrix} = \begin{pmatrix}
      \id\\
      \md_xg|_a
    \end{pmatrix}\\
    \md h|_a=\md f|_{\underbrace{\Psi(a)}_p}\md \psi|_a\\
    = \md_x f|_p+\md_yf|_p\md_xg|_a\\
    D=\begin{pmatrix}
      \md_xf|_p&\md_yf|_p\\
      \md_x\phi|_p&\md_y\phi|_p
    \end{pmatrix}= \begin{pmatrix}
      A&B
    \end{pmatrix}\\
    \md_xf|_a=-\md_yf|_p\md_xg|_a
  \end{eqnarray*}
  Aber 
  \[\md_xg|_a=-(\md_y\phi|_p)^{-1}\md_x\phi|_p\]
  \begin{eqnarray*}
    \phi(x,g(x))=0\\
    \md_x\phi+\md_y\phi\md_xg=0\\
    \implies \md_xg=-(\md_y\phi)^{-1}\md_x\phi
  \end{eqnarray*}
  \[ \cdots \md_xf|_a=-\md_yf|_p\md_xg|_a=\md_yf|p(\md_y\phi|_p)^{-1}\md_x\phi\]
  so:
  \[A:=\begin{pmatrix}
    \md_x f\\
    \md_x\phi
  \end{pmatrix}\]
  \[B:= \begin{pmatrix}
    \md_yf\\
    \md_y\phi
  \end{pmatrix}\]
  \begin{eqnarray*}
    \md_xf&=\md_yf(\md_y\phi)^{-1}\md_x\phi=\md_yf C\\
    \md_x\phi&=\md_y\phi(\md_y\phi)^{-1}\md_x\phi=\md_y\phi C
  \end{eqnarray*}
  $\iff A=BC$ $\implies$ jede Spalte von $A$ ist eine lineare Kombination der Spalten von $B$ $\implies$ Rang $D$ $\leq j$. Allerdings ist Rang $D$ $=j$. $j+1$ Zeilen in $D$.
  \[\begin{pmatrix}
    \Part{f}{x_1}&\cdots&\Part{f}{x_n}&\Part{f}{y_1}&\cdots&\Part{f}{y_j}\\
    \Part{\phi}{x_1}&\cdots&\Part{\phi}{x_n}&\Part{\phi}{y_1}&\cdots&\Part{\phi}{y_j}\\
    \vdots&&\vdots&\vdots&&\vdots\\
    \Part{\phi_j}{x_1}&&&&&\Part{\phi_j}{y_j}
  \end{pmatrix}\]
  $\implies$ $\exists$ $\mu_0,\cdots,\mu_j$ so dass
  \[\mu_0\nabla f(p)+\frac{\mu_1}{\mu_0}\nabla\phi_1(p)+\cdots+\frac{\mu_j}{\mu_0}\nabla\phi(p)=0\]
  $\mu_0\neq 0$ $\implies$ Sei $\lambda=-\frac{\mu_i}{\mu_0}$
  \[\implies\nabla f(p)=\Lambda_1\nabla\phi_1(p)+\cdots+\nabla_j\nabla\phi_j(p)\]
  $\mu_0$ $\implies$ Rang $(\md_x\phi \md_y\phi)$ $\leq j-1$ nicht möglich weil Rang $=j$
\end{Bew}
\begin{Lem}
  Seien $A,B$ und $C$ so dass $A=BC$. Dann sind die Spalten von $A$ Linearkombinationen der Spalten von $B$.
\end{Lem}
\begin{Bew}
  $A$ ist eine $n\times k$ Matrix, $B$ eine $n\times j$ Matrix und $C$ eine $j\times k$ Matrix
  \begin{eqnarray*}
    A=(a_{il})\\
    B=(b_{\alpha\beta}\\
    C=(C_st)\\
    a_{il}=\sum^j_{\alpha=1}b_{i\alpha}c_{\alpha l}
  \end{eqnarray*}
  Die $\lambda$ Spalte von $A$ ist
  \begin{eqnarray*}
    \begin{pmatrix}
      a_{1\lambda}\\
      a_{2\lambda}\\
      \vdots\\
      a_{n\lambda}\\
    \end{pmatrix} = \begin{pmatrix}
      \sum^j_{\alpha=1}b_{1\alpha}c_{\alpha\lambda}\\
      \vdots\\
      \sum^j_{\alpha=1}b_{n\alpha}c_{\alpha\lambda}\\
    \end{pmatrix}\\
    \underbrace{\begin{pmatrix}
      a_{1\lambda}\\
      \vdots\\
      a_{n\lambda}
    \end{pmatrix}}_{\text{Die $\lambda$-Spalte von $A$}}
    = \underbrace{\mu_1\begin{pmatrix}
      b_{11}\\\vdots\\ b_{n1}
    \end{pmatrix}}_{\text{erste Spalte von $B$}}
    + \underbrace{\mu_2\begin{pmatrix}
      b_{12}\\\vdots\\ b_{n2}
    \end{pmatrix}}_{\text{zweite Spalte}}
    + \cdots
    + \mu_j\begin{pmatrix}
      b_{1j}\\\vdots\\ b_{nj}
    \end{pmatrix}
  \end{eqnarray*}
\end{Bew}
\begin{Sat}
  $\left\{ \phi(x,y,z)=0 \right\}=E$ $x_0,y_0,z_0$ eine Extremalstelle von $f$ auf $E$:
  \[\nabla f(x_0,y_0,z_0)=\lambda\nabla\phi(x_0,y_0,z_0)\]
  \[\begin{cases}
    \phi(x_0,y_0,z_0)=0\\
    \Part{f}{x}(x_0,y_0,z_0)=\lambda\Part{\phi}{x}(x_0,y_0,z_0)\\
    \Part{f}{y}(x_0,y_0,z_0)=\lambda\Part{\phi}{y}(x_0,y_0,z_0)\\
    \Part{f}{z}(x_0,y_0,z_0)=\lambda\Part{\phi}{z}(x_0,y_0,z_0)
  \end{cases}\]
\end{Sat}
\section{Integration in $\mb{R}^n$}
\begin{Def}
  Wir betrachten halboffene Würfel:
  \[W:= [a_1,b_1[\times [a_2,b_2[ \times \cdots\times [ a_n,b_n [ \]
  Mass (Volumen) von $R$
  \[\abs{R}:=\prod_{i=1}^n(b_i-a_i)\]
\end{Def}
\begin{Def}
  Eine beschränkte Menge $\Omega\subset\mb{R}^n$ ist Peano-Jordan massbar falls $\forall \varepsilon>0$ $\exists L_1$ und $L_2$ endliche Familien von disjunkten halboffenen Würfel gibt mit:
  \begin{enumerate}
    \item $\bigcup_{R\in L_1}\subset\Omega\subset\bigcup_{R\in L_2}R$
    \item $\sum_{R\in L_2}\abs{R}-\sum_{R\in L_1}\abs{R}<\varepsilon$
  \end{enumerate}
  % TODO Skizze
\end{Def}

\begin{Lem}
  Sei $R$ eine endlische Familie von disjunkten und rechtsoffenen Rechtecken $R$ mit $R\in \Omega$ und $S$ dasselbe mit $\cup_{S\in\mb{C}}\supset\Omega$.
  Falls $\Omega$ Peano-Jordan messbar ist, so gilt:
  \[\sup \left\{ \sum_{R\in\mb{C}^1}\Abs{R} \right\} = \inf \left\{ \sum_{S\in\mb{C}}|S) \right\}\]
  Diese Zahl ist das Mass der Menge $\Omega$. Bezeichnung: $\abs{\Omega}$
\end{Lem}
\begin{Bew}
  Definition und
  \[\cup_{R\in\mb{C}_1}R\subset U_{R\in\mb{C}_2}\implies\sum_{R\in\mb{C}_1}\Abs{R}\leq\sum_{R\in\mb{C}_2}\abs{R}\]
\end{Bew}
\paragraph{Eigenschaften}
\begin{Sat}
  Setze $\abs{\varnothing}:=0$
  Seien $\Omega_1,\Omega_2$ messbar. Es gelten:
  \begin{enumerate}
    \item $\Omega_1\cup\Omega_2$, $\Omega_1\cap\Omega_2$, $\Omega_1\setminus\Omega_2$, $\Omega_2\setminus\Omega_1$ messbar und es gilt
      \[\abs{\Omega_1\cup\Omega_2}=\abs{\Omega_1}+\abs{\Omega_2}-\abs{\Omega_1\cap\Omega_2}\]
    \item Insbesondere gilt:
      \[\abs{\Omega_1\cup\Omega_2}=\abs{\Omega_1}+\abs{\Omega_2}\]
      falls $\Omega_1\cap\Omega_2=\varnothing$
    \item Falls $\Omega_2\subset\Omega_1$
      \[\abs{\Omega_1}=\abs{\Omega_2}+\abs{\Omega_1\setminus\Omega_2}\]
      und
      \[\abs{\Omega_1}\geq\abs{\Omega_2}\]
  \end{enumerate}
\end{Sat}
\begin{Sat}
  \begin{enumerate}
    \item Sei $A$ messbar mit $\abs{A}=0$. Dann ist jede Teilmenge von $A$ messbar.
    \item Sei $\Omega$ messbar, dann ist $\partial\Omega$ messbar und es gilt $\abs{\partial\Omega}=0$
  \end{enumerate}
\end{Sat}
\begin{Kor}
  Sei $\Omega$ messbar, dann sind es auch $\ol{\Omega}$ und $\circ\Omega$
  \[\abs{\Omega}=\abs{\ol\Omega}=\abs{\circ\Omega}\] % TODO: put circ above Omega
\end{Kor}
\begin{Sat}
  Sei $\Omega$ beschränkt und $\partial\Omega$ $\mb{C}^1$-Untermannigfaltigkeit. Dann ist $\Omega$ messbar.
\end{Sat}
\paragraph{Mass und Integral}
\begin{Lem}
  Sei $f[a,1]\to\mb{R}^+$ stetig. Dann ist
  \[\Omega=\left\{ x,y):0\leq x\leq 1, 0\leq y \leq f(x)\right\}\subset\mb{R}^2\]
  messbar und es gilt:
  \[\int_0^1f(t)\md t=\abs{\Omega}\]
\end{Lem}
\begin{Bew}
  Unterteile $[0,1)$ in $n$ rechtsoffene Intervalle der Länge $\frac{1}{n}$ $I_1,\cdots,I_n$. Sei $M_i:=(\max_{I_i}f)+\frac{1}{n}$ und $m_i:=(\min_{I_i}f)$.
  \begin{eqnarray*}
    \mb{C}_1:=\left\{ I_i\times [a,M_i) \right\}\\
    \mb{C}_2:=\left\{ I_i\times [a,m_i) \right\}
  \end{eqnarray*}
  ergibt rechtsoffene Rechtecke, die disjunkt sind.
  \[\implies \cup_{R\in\mb{C}_1}\supset\Omega\supset\cup_{R\in\mb{C}_2}R\]
  Seien
  \begin{eqnarray*}
    A_n:=\sum_{R\in\mb{C}_1}\abs{R}=\sum^n_{i=1}\frac{1}{n}M_i\\
    a_n:=\sum_{R\in\mb{C}_2}\abs{R}=\sum^n_{i=1}\frac{1}{n}m_i
  \end{eqnarray*}
  d.h. $A_n$, $a_n$ sind Integrale von Treppenfunktionen $F_n$, $f_n$ mit $\Norm{F_n-f}\to 0$, $\Norm{f_n-f}\to 0$.
  \[\implies \Limi{n} A_n=\Limi{n}a_n\int_0^1f(t)\md t\]
  d.h.
  \[\abs{\Omega}=\int_0^1f(t)\md t\]
\end{Bew}
\begin{Sat}
  Sei $\Omega\subset\mb{R}^n$ messbar, $f:\ol{\Omega}\to\mb{R}^+$ stetig. Dann ist die Menge
  \[\Gamma:=\left\{ (x_1,\cdots,x_n,x_{n+1}):(x_1,\cdots,x_n)\in\Omega, 0\leq x_{n+1}\leq f(x_1,\cdots,x_n) \right\}\]
  messbar.
\end{Sat}
\begin{Bew}
  Eher eine Skizze. Sei $\varepsilon>0$.
  \begin{enumerate}
    \item Wähle Funktionen $\mb{C}_1$, $\mb{C}_2$ disjunkter, rechtsoffener Rechte, so dass
      \[\cup_{R\in\mb{C}_1}R\subset \Omega \subset \cup_{R\in\mb{C}_2}\]
      und
      \[\left( \sum_{R\in\mb{C}_2}\abs{R}-\sum_{R\in\mb{C}_1}\abs{R} \right)<\varepsilon\]
    \item Wähle eine Verfeinerung $\mb{C}_2'$, so dass
      \[\left\{ R\cap S:R\in\mb{C}_2\s \&\s S\in\mb{C}_1 \right\}\subset\mb{C}_2'\]
      und
      \[\cup_{R\in\mb{C}_1'}R = \cup_{R\in\mb{C}_2'}\]
    \item Wähle eine weitere Verfeinerung (und bezeichne sie immer noch gleich), indem jedes Rechteck in $2^{Nn}$ gleichmässige Rechtecke aufgeteilt wird, so dass
      \[\diam(R)<\frac{1}{k}\s\forall R\in\mb{C}_2'\]
      Sei 
      \[\mb{C}_1':=\left\{ R\in\mb{C}_2':R\subset\Omega \right\}\]
      Es gilt
      \[\sum_{R\in\mb{C}_2'}\abs{R}-\sum_{R\in\mb{C}_1'}\abs{R} < \varepsilon\]
    \item Nehme an, dass $\mb{C}_2'$ keine Rechtecke $R$ enthält mit $R\cap\ol{\Omega}=\varnothing$ (falls ja, werfe sie (oBdA) weg). Def: für jedes Rechteck $R\in\mb{C}_2'$
      \[M_R=\max_{\ol R\cap\ol\Omega}f+\varepsilon, \s m_R:=\min_{\ol R\cap\ol\Omega}\]
      $\ol\Omega$ kompakt $\implies$ $f$ gleichmässig statig $\implies$ $\exists k$ gross genug:
      \[\diam R<\frac{1}{k}\implies M_R-m_R\leq 2\varepsilon\]
    \item Def:
      \begin{eqnarray*}
        \mathscr{S}_1:=\left\{ R\times [a,m_R):R\in\mb{C}_1'\right\}\\
        \mathscr{S}_2:=\left\{ R\times [a,M_R):R\in\mb{C}_2'\right\}\\
        \implies \cup_{\tilde R\in\mathscr{S}_1}\subset\Gamma\subset\cup_{\tilde R\in\mathscr{S}_2}\tilde R
      \end{eqnarray*}
      Es Gilt
      \begin{eqnarray*}
        \sum_{\tilde R\in \mathscr{S}_2}\abs{\tilde R} -\sum_{\tilde R\in \mathscr{S}_1}\abs{\tilde R}\\
        = \sum_{\tilde R\in \mb{C}_2'\setminus\mb{C}_1'}\abs{\tilde R}M_R + \sum_{\tilde R\in \mb{C}_1'}\abs{\tilde R}M_R-\sum_{\tilde R\in \mb{C}_1'}\abs{\tilde R}m_R\\
        \leq (\varepsilon+\max_{\ol\Omega}f)\underbrace{\sum_{\tilde R\in \mb{C}_2'\setminus\mb{C}_1'}\abs{\tilde R}}_{\leq \varepsilon}+\underbrace{(M_R-m_R)}_{\leq 2\varepsilon}\underbrace{\sum_{R\in\mb{C}_1'}\abs{R}}_{\abs\Omega}\\
        \leq \underbrace{(\varepsilon+\max_{\ol\Omega}f+2\abs\Omega)}_{\leq C}\varepsilon\\
        \leq C\varepsilon
      \end{eqnarray*}
      d.h. die Behauptung: $\Omega$ messbar
  \end{enumerate}
\end{Bew}
\begin{Def}
  $f$ und $\Omega$ wie oben.
  \[\int_\Omega f(x)\md x:=\abs{\Gamma}\]
  Falls $f\leq 0$, dann setze
  \[\int_\Omega f:=-\int_\Omega -f\]
  Für allgemeines $f$: $f+:=\max\left\{ f,0 \right\}$, $f^-:=\min\left\{ f,0 \right\}$
  \[\int_\Omega f:=\int_\Omega f^++\int_\Omega f^-\]
\end{Def}
\paragraph{Berechnung von Integralen}
\begin{Sat}
  (Fubini) Sei $\Omega=[a_1,b_1]\times\cdots\times [a_n,b_n]$, $f:\Omega\to\mb{R}^+$ stetig. Dann gilt:
  \[\int_\Omega f=\int_{a_n}^{b_n}\int_{a_{n-1}}^{b_{n-1}}\cdots\int_{a_1}^{b_1}f(x_1,\cdots,x_n)\md x_1\cdots\md x_n\]
\end{Sat}
\begin{Bew}
  (SKizze) OBdA: $a_1=\cdots=a_n=b_1=\cdots=b_n=1$ Unterteile $\Omega$ in $\frac{1}{N^n}$ Würfel $C$
  \[\int_C^\sim f:=\int_{c_n}^{d_n}\cdots\int_{c_1}^{d_1}f(x_1,\cdots,x_n)\md x_1\cdots\md x_n\]
  Es gilt
  \[\int_C^\sim f\sum_C\int_C^\sim f\]
  Setze:
  \begin{eqnarray*}
    M_C:=\max_{\ol C}f\\
    m_C:=\min_{\ol C}f
  \end{eqnarray*}
  Definition
  \begin{eqnarray*}
    A_C:=\sum_C\abs{C}M_C\\
    a_C:=\sum_C\abs{C}m_C\\
  \end{eqnarray*}
  Wie oben erhalten wir
  \[\int_\Omega f:=\Limi{N}A_N=\Limi{N}a_N\]
  andererseits:
  \[\abs{C}m_C\leq\int_C^\sim f\leq\abs{C}M_C\]
  d.h.
  \[a_N\leq \int_\Omega^\sim f\leq A_N\]
  \[\implies \int_\Omega^\sim f=\int_\Omega f\]
\end{Bew}
\begin{Bem}
  In der Tat gilt nicht nur
  \[\Limi{N}A_N=\int_\Omega f=\Limi{N}a_n\]
  Betrachte eine Unterteilung $\mathscr{S}$ von $\Omega$ in Rechtecke der ``Grösse'' $\leq\varepsilon$, d.h. $\diam R\leq \varepsilon$. Für $S\in\mathscr{S}$ wähle $x_s\in S$. Definiere Riemannsche Summe:
  \[R(\varepsilon)=\sum_{S\in\mathscr{S}}f(x_S)\abs{S}\]
  Dann gilt:
  \[\Limo{\varepsilon}R(\varepsilon)=\int_\Omega f\]
\end{Bem}
\begin{Def}
  (Normaler Bereich) Eine Menge $\Omega\subset\mb{R}^n$ heisst \ul{normal}, falls es ein Koordinatensystem und zwei Funktionen $f\geq g$ (stetig) und $a_i\leq b_i$ $i\in \left\{ 1,\cdots,n+1 \right\}$ gibt so dass
  \[\Omega=\left\{ (x_1,\cdots,x_n,x_{n+1}:a_+\leq x_1\leq b_+,\cdots,a_{n-1}\leq n_{n-1}\leq b_{n-1}, g(x_1,\cdots,x_{n-1}\leq x_n\leq f(x_1,\cdots,x_{n-1} \right\}\]
\end{Def}
\begin{Bem}
  Man zeigt: Ist $\Omega$ normal, so gilt:
  \[\int_\Omega F:=\int_{a_1}^{b_1}\cdots\int_{a_{n-1}}^{b_{n-1}}\int_{a_n}^{b_n}F(x_1,\cdots,x_n)\]
  Sei $\Omega\subset\mb{R}^n$ offen mit $\partial\Omega$ $\mb{C}^1$-Untermanigfaltigkeit. Dann gilt $\forall x\in\partial\Omega$ $\exists$ Würfel $C$ so dass $C\cap \Omega$ normal ist. Dan $\ol\Omega$ kompakt ist, lässt sich $\Omega$ mit endlich vielen solchen Würfel überdecken.
\end{Bem}

\begin{Def}
  Eine Teilmenge $\Omega\subset\mb{R}^n$ heisst normal falls $\exists$(ein System von Koordinaten) und zwei Funktionen
  \[g\leq f:\underbrace{\Omega'}_{\subset\mb{R}^{n-1}}\to\mb{R}\]
  so dass
  \[\Omega'=\left[ a_1,b_1 \right]\times\left[ a_2,b_2 \right]\times\cdots\times\left[ a_{n-1},b_{n-1} \right]\]
  und
  \[\Omega=\left\{ x:(x_1,\cdots,x_{n-1})\in\Omega'\s\text{und}\s g(x_1,\cdots,x_{n-1})\leq x_n\leq f(x_1,\cdots,x_{n-1}) \right\}\]
\end{Def}
\begin{Bsp}
  in zwei Dimensionen
  % TODO Skizze
\end{Bsp}
\begin{Bem}
  In diesem Fall
  \[\int_\Omega F=\int_{a_1}^{b_1}\left( \int_{a_2}^{b_2}\left( \cdots \int_{a_{n-1}}^{b_{n-1}}\overbrace{\left( \int^{f(x_1,\ldots,x_{n-1})}_{g(x_1,\ldots,x_{n-1})}F(x_1,\cdots,x_{n-1},x_n)\md x_n \right)}^{G(x_1,\ldots,x_{n-1})}\md x_{n-1} \right)\cdots \right)\md x_1\]
  \[=\int_{a_1}^{b_1}\cdots\int_{a_{n-1}}^{b_{n-1}}\int_{g(x_1,\ldots,x_n)}^{f(x_1,\ldots,x_n)}F(x_1,\cdots,x_{n-1},x_n)\md x_n\cdots\md x_1\]
\end{Bem}
\begin{Bem}
  Wichtig: die Funktion
  \[G(x_1,\cdots,x_{n-1})=\int_{f(x_1,\ldots,x_{n-1})}^{g(x_1,\ldots,x_{n-1})}F(x_1,\cdots,x_{n-1},x_n)\md x_n\]
  ist stetig (ohne Beweis)
\end{Bem}
\begin{Bem}
  in 2d mit $F=1$ ist die Formel
  \[\text{Inhalt von $\Omega$} = \int_{a_1}^{b_1}\left( \int_{g(x_1)}^{f(x_1)}1\md x_2 \right)\md x_1 \left( =\int_{a_1}^{b_1}\left( f(x_1)-g(x_1) \right) \right)\]
\end{Bem}
% TODO Skizze
$\forall$ Intervall $I_i$
\begin{eqnarray*}
  m_i=\min_{I_i}f\\
  m_i=\max_{I_i}f
\end{eqnarray*}
\[\underbrace{\cup_{i=1}^NI_ix[0,m_i[}_{\text{Mass}=\sum^N_{i=1}\frac{b-a}{N}m_i}\subset\Omega\subset \underbrace{\cup_{i=1}^NI_i\times [0,M_i[}_{\text{Mass}=\sum^N_{i=1}\frac{b-a}{N}M_i}\]
Sei $\varepsilon>0$ gegeben aus der gleichmässigen Stetigkeit von $f$ $\exists$ $N_0>0$ so dass
\[M_i-m_i<\varepsilon\s\forall \s\text{Zerlegung mit}\s N\geq N_0\]
So Mass Aussen - Mass Innen
\begin{eqnarray*}
  =\sum^N_{i=1}\frac{b-a}{N}(M_i-m_i)\\
  \leq \varepsilon \sum^N_{i=1}\frac{b-a}{N}\varepsilon(b-a)
\end{eqnarray*}
\[\Limi{N}(\text{Mass Aussen} - \text{Mass Innen})=0\]
% TODO Skizze
\begin{eqnarray*}
  \Omega=\left\{ (x_1,x_2,x_3):(x_1,x_2)\in\Omega - \sqrt{1-(x_1^2+x_2^2)}\leq x_3\leq \sqrt{1-(x_1^2+x_2^2)} \right\}\\
  \Omega'=\left\{ x_1^2+x_2^2\leq 1 \right\}=\left\{ x_1\in[-1,1],-\sqrt{1-x_1^2}\leq x_2\leq\sqrt{1-x_2^2} \right\}\\
  1=\left( \sqrt{x_1^2+x_2^2} \right)^2+f(x_1,x_2)^2\\
  f(x_1,x_2)=\sqrt{1-(x_1^2+x_2^2)}
\end{eqnarray*}
\begin{Def}
  Ultranormale Bereiche $\exists$ ein System von Koordinaten und $a_1\leq b_1$
  \begin{eqnarray*}
    \Omega_1=\left[ a_1,b_1 \right]\\
    f_2,g_2:\Omega_1\to\mb{R}\\
    \Omega_2=\left\{ (x_1,x_2):x_1\in \Omega_1, f_2(x_1)\leq x_2\leq g_2(x_1) \right\}\\
    f_3,g_3:\Omega_2\to\mb{R}\\
    \Omega_3=\left\{ (x_1,x_2,x_3):(x_1,x_2)\in \Omega_2, f_3(x_1,x_2)\leq x_3\leq g_3(x_1,x_2) \right\}\\
    \Omega=\Omega_n=\left\{ (x_1,\cdots,x_{n-1},x_n):(x_1,\cdots,x_{n-1})\in \Omega_{n-1}, f_n(x_1,\cdots,x_{n-1})\leq x_3\leq g_3(x_1,\cdots,x_{n-1}) \right\}\\
  \end{eqnarray*}
\end{Def}
\begin{Bem}
  In diesem Fall
  % TODO overfull
  \[\int_\Omega F=\int_{a_1}^{n_1}\int_{f_2(x_1)}^{g_2(x_1)}\int_{f_3(x_1,x_2)}^{g_3(x_1,x_2)}\cdots\int_{f_n(x_1,\ldots,x_{n-1})}^{g_n(x_1,\ldots,x_{n-1})}F(x_1,\cdots,x_{n-1},x_n)\md x_n\cdots\md x_3\md x_2\md x_1\]
\end{Bem}
\begin{Bsp}
  Inhalt einer 3d Kugel mit Radius 1
  \begin{eqnarray*}
    =\int_{-1}^1\int_{-\sqrt{1-x_1^2}}^{\sqrt{1-x_1^2}}\int_{-\sqrt{1-(x_1^2+x_2^2)}}^{\sqrt{1-(x_1^2+x_2^2)}} 1\md x_3\md x_2\md x_1 =\\
    \int_{-1}^1\int_{-\sqrt{1-x_1^2}}^{\sqrt{1-x_1^2}}(\sqrt{1-(x_1^2+x_2^2)}+\sqrt{1-(x_1^2+x_2^2)}) 1\md x_2\md x_1\\
    = \frac{4\pi}{3}
  \end{eqnarray*}
\end{Bsp}
\begin{Sat}
  Sei $\Omega$ eine Menge, die:
  \begin{enumerate}
    \item beschränkt ist
    \item dessen Rand eine $\mb{C}^1$-Mannigfaltigkeit ist
  \end{enumerate}
  Dann: $\Omega$ ist Peano-Jordan messbar $\exists$ eine Zerlegung von $\Omega$ in endlich viele ultranormale Bereiche.
  % TODO Skizze
  $\implies$ Formel $\int_\Omega F$ (wenn $F$ eine stetig Funktion ist)
\end{Sat}


\newpage

%= Stichwortverzeichnis ======================================================================
\rhead{}
\addcontentsline{toc}{section}{Stichwortverzeichnis}
\printindex

\end{document}

