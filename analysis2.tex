% headers by Alexander Berthold van der Bourg / Pirmin Weigele 

%= Document-Class ==================================================================================
\documentclass[10pt,a4paper]{article}

%= Packages ========================================================================================
\usepackage[utf8x]{inputenc}
\usepackage{ngerman,amsmath,amssymb,amsfonts,mathrsfs}
\usepackage{amsthm}
\usepackage{bbm}
\usepackage{epic,eepic,pstricks,pst-node,pst-plot}
\usepackage{pstricks}
\usepackage{colortbl}
\usepackage{graphicx}
\usepackage{makeidx}
\usepackage{fancyhdr}
\usepackage{latexsym}
\usepackage{psfrag}
\usepackage{enumerate}
\usepackage{float}
\usepackage{dsfont}
\pagestyle{fancy}
\usepackage{multirow, bigdelim, bigstrut}
\usepackage{rotating}
\usepackage{ifthen}
\usepackage{boxedminipage}
\usepackage{mathtools}
\usepackage{trfsigns}
\usepackage{url}
%\usepackage{savetrees}

%= Seiten-Layout =========================================================================
\voffset-22mm \textheight715pt 

%Seitenbreite==============================================================

%\oddsidemargin=-0.2in
%\evensidemargin=-0.4in
%\textwidth=5.2in
%\headwidth=5.2in

%= Index-Befehle ========================================================================
\renewcommand{\indexname}{Stichwortverzeichnis}
\makeindex

%= Befehl-Overwriting =======================================================================
\makeatletter
\makeatother

%= Strings ================================================================
\newcommand{\mainfold}{.}
\newcommand{\prefix}{A1-}

%= Eigene Befehle ==========================================================================
\DeclareMathOperator{\id}{Id}
\DeclareMathOperator{\arccot}{arccot}
\DeclareMathOperator{\arcsinh}{arcsinh}
\DeclareMathOperator{\arccosh}{arccosh}
\DeclareMathOperator{\arctanh}{arctanh}
\DeclareMathOperator{\md}{d}
\DeclareMathOperator{\Grad}{grad}
\DeclareMathOperator{\Spur}{Spur}
\DeclareMathOperator{\Graph}{Graph}
\DeclareMathOperator{\sign}{sign}
\DeclareMathOperator{\Hom}{Hom}
\DeclareMathOperator{\rot}{rot}
\DeclareMathOperator{\Ker}{Ker}
\DeclareMathOperator{\Exp}{Exp}

\newcommand{\Diff}[2]{\displaystyle\frac{\mathrm{d}#1}{\mathrm{d}#2}}
\newcommand{\End}{\hfill{\hbox{$\Box$}}\par\vspace{2mm}}
\newcommand{\eps}{\varepsilon}
\newcommand{\ePic}[1]{\input{\mainfold/graphics/\prefix#1.eepic}}
\newcommand{\pst}[1]{\input{\mainfold/graphics/\prefix#1.pst}}
\newcommand{\pic}[1]{\input{\mainfold/graphics/\prefix#1.pic}}
\newcommand{\Mx}[1]{\begin{pmatrix}#1\end{pmatrix}}
%\newcommand{\im}[1]{\operatorname{Im}(#1)}
%\newcommand{\Include}[4]{\rhead{#2.#3.20#4}\input{\mainfold/lectures/#1-#4-#3-#2.tex}}
\newcommand{\Index}[1]{\emph{#1}\index{#1}}
\newcommand{\Int}[4]{\displaystyle\int\limits_{#1}^{#2}#3\,\mathrm{d}#4}
\newcommand{\diff}[1]{\operatorname{d}\!#1}
\newcommand{\Limi}[1]{\displaystyle\lim_{#1\rightarrow\infty}}
\newcommand{\Limo}[1]{\displaystyle\lim_{#1\rightarrow0}}
\newcommand{\Limu}[2]{\displaystyle\lim_{#1\uparrow #2}}
\newcommand{\Limd}[2]{\displaystyle\lim_{#1\downarrow #2}}
\newcommand{\Lim}[2]{\displaystyle\lim_{#1\rightarrow#2}}
\newcommand{\mb}[1]{\mathbb{#1}}
\newcommand{\ds}{\displaystyle}
\newcommand{\ol}[1]{\overline{#1}}
\newcommand{\Part}[2]{\dfrac{\partial #1}{\partial #2}}
\newcommand{\QED}{\hfill{\hbox{(QED)}}\par\vspace{2mm}}
\newcommand{\re}[1]{\operatorname{Re}(#1)}
\newcommand{\s}{\hspace{2mm}}
\newcommand{\vsa}{\vspace{1mm} \\}
\newcommand{\vsb}{\vspace{2mm} \\}
\newcommand{\vsc}{\vspace{3mm} \\}
% \newcommand{\tr}[1]{\textrm{#1}}
\newcommand{\tr}[1]{\text{#1}}
\newcommand{\ra}{\rightarrow}
\newcommand{\Ra}{\Rightarrow}
\newcommand{\Lra}{\Leftrightarrow}
\newcommand{\La}{\Leftarrow}
\newcommand{\ul}[1]{\underline{#1}}
\newcommand{\rsa}{\rightsquigarrow}
\newcommand{\ara}[2]{\autorightarrow{\ensuremath{#1}}{\ensuremath{#2}}}
\newcommand{\dcp}[2]{\begindc{\commdiag}[#1] #2 \enddc}

%\newcommand{\detmx}{\left| \begin{array} #1 \end{array} \right|}

\newcommand{\grad}[1]{\Grad(#1)}
\newcommand{\fr}[2]{\displaystyle\frac{#1}{#2}} % fertiger bullshit, daf�r gibts \dfrac{}{}
\renewcommand{\Re}{\operatorname{Re}}
\renewcommand{\Im}{\operatorname{Im}}

% ---- DELIMITER PAIRS ----
\def\floor#1{\lfloor #1 \rfloor}
\def\ceil#1{\lceil #1 \rceil}
\def\seq#1{\langle #1 \rangle}
\def\set#1{\{ #1 \}}
\def\abs#1{\mathopen| #1 \mathclose|}	% use instead of $|x|$ 
\def\norm#1{\mathopen\| #1 \mathclose\|}% use instead of $\|x\|$ 

% --- Self-scaling delmiter pairs ---
\def\Floor#1{\left\lfloor #1 \right\rfloor}
\def\Ceil#1{\left\lceil #1 \right\rceil}
\def\Seq#1{\left\langle #1 \right\rangle}
\def\Set#1{\left\{ #1 \right\}}
\def\Abs#1{\left| #1 \right|}
\def\Norm#1{\left\| #1 \right\|}

%Adrians Abbildungs-Environment ==============================================

\newcommand{\Sidein}{\begin{rotate}{90}\small$\in$\end{rotate}}

\newcommand{\Abb}[5][]{\ensuremath{
    \begin{array}{lc}
      \ifthenelse{\equal{#1}{}}{}{#1:}\;\; & 
      \begin{xy}
        \xymatrixrowsep{1em}\xymatrixcolsep{2em}%
        \xymatrix{ #2 \ar[r] \ar@{}[d]^<<<<{\hspace{0.001em} \Sidein}
          & #3  \ar@{}[d]^<<<<{\hspace{0.001em} \Sidein} \\
          #4 \ar@{|->}[r] & #5} \end{xy}
    \end{array}
  }%
}

%= Environments ========================================================================
\def\thechapter{\Roman{chapter}}
\def\thesection{\arabic{section}}
\newtheorem{theorem}{Theorem}[section]
\newenvironment{Bew}{\begin{proof}[Beweis]}{\end{proof}}
\newtheorem{Axi}[theorem]{Axiom}
\newtheorem{Lem}[theorem]{Lemma}
\newtheorem{Kor}[theorem]{Korollar}
\newtheorem{Sat}[theorem]{Satz}
\newtheorem{Prop}[theorem]{Proposition}
\newtheorem{Beh}[theorem]{Behauptung}
\theoremstyle{definition}
\newtheorem{Bsp}[theorem]{Beispiel}
\newtheorem{Def}[theorem]{Definition}
\newtheorem{Ueb}[theorem]{\"Ubung}
\theoremstyle{remark}
\newtheorem{Bem}[theorem]{Bemerkung}
\newtheorem{Eig}[theorem]{Eigenschaften}
\newtheorem{Not}[theorem]{Notation}

\def\pstexInput#1{%
  \begin{center}
    \begin{picture}(0,0)%
      \special{psfile=\mainfold/graphics/A2-#1.pstex}%
    \end{picture}%
    \input{\mainfold/graphics/A2-#1.pstex_t}%
  \end{center}
}

%= Titelseite ===========================================================================
\begin{document}
\headheight15pt
\begin{titlepage}
\hfill
\vspace{20mm}
\pagenumbering{roman}
\begin{center}
{\LARGE Analysis II - Vorlesungs-Script} \vskip 3em {\large Prof. Dr. Camillo De Lellis} \vskip 1.5em
{\large Basisjahr 11 Semester I}\vspace{30mm}\\
{\large {\bf Mitschrift:} \vspace{2mm}\\
Simon Hafner}\vspace{5mm}\\ %30mm
%{\large {\bf Graphics:} \vspace{2mm}\\
%Pirmin Weigele }\vspace{30mm}\\ %30mm
\author{Simon Hafner}

\end{center}
\vfill

\end{titlepage}


%= Inhaltsverzeichnis ==========================================================================
\lhead{}
\rhead{}
\tableofcontents
\newpage
\pagenumbering{arabic}
\setcounter{page}{1}

%= Vorlesung-Skripts ==========================================================================
\cfoot{\thepage}
\fancyhead[L]{\nouppercase{\leftmark}}
\newpage

%Analysis II
\section{Metrik und Topologie des euklidischen Raumes}
$\mb{R}^n=\left\{ \left( x_1,\cdots,x_n \right), x\in\mb{R} \right\}$.
Wir f\"uhren verschiedene neue Begriffe in $\mb{R}^n$ ein:
\begin{itemize}
  \item die Euklidische Norm
  \item der Euklidische Abstand
  \item die entsprechende Topologie.
\end{itemize}
Wir betrachten gleichzeitig die entsprechenden Verallgemeinerungen,
d.h. die ``Abstrakte Theorien'' der
\begin{itemize}
  \item Normierten Vektorr\"aume
  \item Metrischen R\"aume
  \item Topologischen R\"aume.
\end{itemize}
\begin{Def}
  Sei $x\in\mb{R}^n$ ($x=(x_1,\cdots,x_n)$, $x_i\in\mb{R}$). Die Euklidische Norm
von $x$ ist 
  \[\Norm{x}_e=\sqrt{x_1^2+\cdots+x_n^2}=\sqrt{\sum_{i=1}^nx_i^2}\]
(wir schreiben oft $\|x\|$ anstatt $\|x\|_e$).
\end{Def}

Intuitiv: $\Norm{x}=$''der Abstand zwischen $x$ und 0``.  In der Tat, wenn $n=2$,
das Pytaghoras Theorem zeigt dass $\|x\|_e$ die L\"ange des Segments mit Extrema
$x$ and $0$ ist. 

\begin{Lem}\label{l:norm}
  $\Norm{.}$ erf\"ullt die Regeln
  \begin{enumerate}
    \item $\Norm{x}\geq 0$ und $\Norm{x}=0\iff x=0$
    \item $\Norm{\lambda x}=\Abs{\lambda}\Norm{x}$ $\forall \lambda\in\mb{R}$, $\forall x\in\mb{R}$
    \item $\Norm{x+y}\leq\Norm{x}+\Norm{y}$ $\forall x,y\in\mb{R}$
  \end{enumerate}
\end{Lem}
\begin{Bew}
  \begin{enumerate}
    \item $\geq 0$ trivial
      \[x=0\implies \sum x_i^2=0\implies \Norm{x}=0\]
      \[x=0\Leftarrow x_i=0\;\; \forall i\Leftarrow \sum x_i^2=0\Leftarrow \Norm{x}=0\]
    \item \[\Norm{\lambda x}=\sqrt{\sum^n_{i=1}(\lambda x_i)^2} = \sqrt{\lambda^2\sum x^2}=\Abs{\lambda}\sqrt{\sum x^2}=\Abs{\lambda}\Norm{x}\]
%\[\Abs{\lambda}=\frac{\Norm{x}\Abs{\lambda}}{\Norm{x}}\]
    \item Diese Aussage ist \"aquivalent zu
      \[\iff \underbrace{\Norm{x+y}^2}\leq \Norm{x}^2+\Norm{y}^2+2\Norm{x}\Norm{y}\]
Wir rechnen
      \[\sum_{i=1}^n(x_i+y_i)^2=\sum_{i=1}^n\left( x_i^2+y_i^2+2x_iy_i \right)=\Norm{x}^2+\Norm{y}^2\overbrace{2\sum_i x_iy_i}^{Skalarprodukt}\]
Wir definieren 
\[\langle x,y\rangle := \sum_{i=1}^n  x_iy_i\]
Wir brauchen dann die ber\"uhmte Cauchy-Schwartz Ungleichung, d.h.
      \[\langle x,y\rangle\leq \Norm{x}\Norm{y}\, .\]
Diese Ungleichung ist der Inhalt des n\"achsten Satzes.
  \end{enumerate}
\end{Bew}
\begin{Sat}{Cauchy-Schwartzsche Ungleichung}
  \[\sum^n_{i=1}x_iy_i\leq\sqrt{\sum_{i=1}^nx_i^2}\sqrt{\sum_{i=1}^ny_i^2}\]
\end{Sat}
\begin{Bew}
  OBdA $y\neq 0$ ($y=0$ trivial)
  \[t\to g(t)=\sum_{i=1}^n(x_i+ty_i)^2=\left( \sum x_i^2 \right)+2t\sum x_iy_i+t^2\sum y_i^2\]
  \[=\Norm{x}^2+2t\seq{x,y}+\Norm{y}^2t^2\]
  Sei $t_0=\frac{\seq{x,y}}{\Norm{y}^2}$, dann
  \[0\leq g(t_0)=\Norm{x}^2-2\frac{\seq{x,y}^2}{\Norm{y}^2}+\Norm{y}^2\frac{\seq{x,y}^2}{\Norm{y}^4}
 =\Norm{x}^2-\frac{\seq{x,y}^2}{\Norm{y}^2}\]
  \[\implies \seq{x,y}^2\leq\Norm{x}^2\Norm{y}^2\implies \Abs{\seq{x,y}}\leq\Norm{x}\Norm{y}\]
\end{Bew}
\begin{Def}
  Ein normierter Vektorraum ist ein reeller Vektorraum $V$ mit einer Abbildung $\Norm . :V\to\mb R$ so dass:
  \begin{enumerate}
    \item $\Norm x\geq 0$ und $\Norm x=0\iff x=0$ (Nullvektor)
    \item $\Norm{\lambda x}=\abs\lambda\Norm x$ $\forall \lambda\in\mb R$, $\forall x\in V$
    \item $\Norm{x+y}\leq\Norm{x}+\Norm{y}$ $\forall x,y\in V$
  \end{enumerate}
\end{Def}
\begin{Bsp}
  $V=\mb{R}^n$
  \[\Norm{x}_p=\left(\sum\Abs{x_i}^p\right)^{\frac{1}{p}}\s p\geq 1\, .\]
$\|\cdot\|_2$ ist die Euklidische Norm.
\end{Bsp}
\begin{Def}
  Seien $x,y\in\mb{R}^n$. Die Euklidische Metrik ist $d(x,y):=\Norm{x-y}$.
\end{Def}
\begin{Lem}\label{l:euk_met}
  \begin{enumerate}
    \item $d(x,y)\geq 0$ und $d(x,y)=0\iff x=y$
    \item $d(x,y)=d(y,x)$
    \item $d(x,z)\leq d(x,y)+d(y,z)$ (Dreiecksungleichung)
  \end{enumerate}
\end{Lem}
\begin{Bew} Die erste Zwei Aussagen sind trivial. Um die letzte zu beweisen:
  \[\Norm{x-z}\leq\|\underbrace{x-y}_{=:v}\|+\|\underbrace{\Norm{y-z}}_{=:w}\|\, .\]
Aber $x-z=v+w$. Wir wenden die dritte Aussage von Lemma \ref{l:norm} an:
  \[d (x,z) = \Norm{v+w}\leq\Norm v+\Norm w = d (x,y)+d(y,z)\, .\]
\end{Bew}
\begin{Def}
  Ein metrischer Raum ist eine Menge $X$ mit einer Abbildung
  \[d:X\times X\to\mb{R}\s (x,y)\mapsto d(x,y)\in\mb{R}\]so dass
  \begin{enumerate}
    \item $d(x,y)\geq 0$ und $d(x,y)=0\iff x=y$ $\forall x,y\in X$
    \item $d(x,y)=d(y,x)$ $\forall x,y\in X$
    \item $d(x,z)=d(x,y)+d(y,z)$ $\forall x,y,z\in X$
  \end{enumerate}
\end{Def}
\begin{Lem}
  Sei ($V$, $\Norm .$) ein normierter Vektorraum. Dann sind $V$ und $d(x,y)=\Norm{x-y}$ ein metrischer Raum.
\end{Lem}
\begin{Bew} Wir nutzen das gleiche Argument vom Lemma \ref{l:euk_met}.
\end{Bew}
\begin{Def}
  Die offene Kugel mit Radius $r>0$ und Mittelpunkt $x\in\mb{R}^n$ ist die Menge
  \[K_r(x)=\left\{ y\in\mb{R}^n, d(x,y)<r \right\}\]
(Wir werden auch oft $B_r (x)$ statta $K_r (x)$ nutzen.)
\end{Def}
\begin{Def}
  Eine Menge heisst ''Umgebung`` von $x$, wenn $V$ eine offene Kugel mit Mittelpunkt $x$ enth\"alt.
\end{Def}
\begin{Def}
  Eine Menge $U\subset\mb{R}^n$ heisst offen falls $\forall x\in U$ ist $U$ eine Umgebung von $x$, d.h.
  \[\forall x\in U\s\exists \s\text{eine Kugel}\s K_r(x)\subset U\]
\end{Def}
\begin{Bem}
Die Dreiecksungleichung impliziert dass jede offene Kugel eine offene Menge ist. In der Tat,
sei $y\in K_r (x)$. Dann $\rho:=d(x,y) < r$. Sei $\tau:= r-\rho>0$. Falls $z\in K_\tau (y)$,
dann $d(x,z)\leq d(x,y) + d(y,z) = \rho + d (y,z) < \rho +\tau =r$. D.h., $K_\tau (y)\subset K_r (x)$.
Das beweist dass $K_r (x)$ eine Ungebung ihrer ganzen Elementen ist, d.h. $K_r (x)$ ist offen. 
\end{Bem}
\begin{Sat}
  \begin{enumerate}
    \item $\varnothing$ und $\mb{R}^n$ sind offen
    \item Der Schnitt \ul{endlich vieler} offener Mengen ist auch offen.
    \item Die Vereinigung einer \ul{beliebigen} Familie offener Mengen ist auch offen.
  \end{enumerate}
\end{Sat}
\begin{Bew}
  \begin{enumerate}
    \item $\mb{R}^n$ trivialerweise offen, auch $\varnothing$
    \item Sei $x\in U\cap\dots\cap U_N$
      \[\forall i\in\left\{ 1,\dots,N \right\}\s \quad \exists r_i>0 \;\;\mbox{so dass}\;\; K_{r_i}(x)\subset U_i\]
      Sei $r=\min\left\{ r_i,\dots,r_N \right\} > 0$;
      \[\implies K_r(x)\subset U_i\quad\forall i\implies K_r(x)\subset U_1\cap\dots\cap U_N\]
    \item $\left\{ U_\lambda \right\}_{\lambda\in \Lambda}$. Sei $U=\bigcup_{\lambda\in\Lambda}U_\lambda$
      \[x\in U\implies x\in U_\lambda\s\text{f\"ur ein}\s\lambda\in\Lambda\]
      \[\implies \exists K_r(x)\subset U_\lambda\subset U.\]
  \end{enumerate}
\end{Bew}
\begin{Def}
  Ein topologischer Raum ist eine Menge $X$ und eine Menge $O$ von Teilmengen von $X$ so dass:
  \begin{enumerate}
    \item $\varnothing, X\in O$
    \item $U_1\cap\dots\cap_N\in O$ falls $U_i\in O$
    \item $\bigcap_{\lambda\in\Lambda}U_\lambda\in O$ falls $U_i\in O$
  \end{enumerate}
$O$ heisst die {\em Topologie}.
\end{Def}
\begin{Sat}
  Sei $(X,d)$ ein metrischer Raum. Wir definieren die entsprechende offene Kugel mit Mittelpunkt $x\in X$
und Radius $r>0$:
  \[K_r(x)=\left\{ y=X: d(x,y)<r \right\}\]
  Umgebungen und offene Mengen sind wie im Euklidischen Fall definiert. $O=\left\{ \text{offene Menge} \right\}$ definiert eine Topologie.
\end{Sat}


\newpage

%= Stichwortverzeichnis ======================================================================
\rhead{}
\addcontentsline{toc}{section}{Stichwortverzeichnis}
\printindex

\end{document}
