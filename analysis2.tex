% headers by Alexander Berthold van der Bourg / Pirmin Weigele 

%= Document-Class ==================================================================================
\documentclass[10pt,a4paper]{article}

%= Packages ========================================================================================
\usepackage[utf8x]{inputenc}
\usepackage{ngerman,amsmath,amssymb,amsfonts,mathrsfs}
\usepackage{amsthm}
\usepackage{bbm}
\usepackage{epic,eepic,pstricks,pst-node,pst-plot}
\usepackage{pstricks}
\usepackage{colortbl}
\usepackage{graphicx}
\usepackage{makeidx}
\usepackage{fancyhdr}
\usepackage{latexsym}
\usepackage{psfrag}
\usepackage{enumerate}
\usepackage{float}
\usepackage{dsfont}
\pagestyle{fancy}
\usepackage{multirow, bigdelim, bigstrut}
\usepackage{rotating}
\usepackage{ifthen}
\usepackage{boxedminipage}
\usepackage{mathtools}
\usepackage{trfsigns}
\usepackage{url}
%\usepackage{savetrees}

%= Seiten-Layout =========================================================================
\voffset-22mm \textheight715pt 

%Seitenbreite==============================================================

%\oddsidemargin=-0.2in
%\evensidemargin=-0.4in
%\textwidth=5.2in
%\headwidth=5.2in

%= Index-Befehle ========================================================================
\renewcommand{\indexname}{Stichwortverzeichnis}
\makeindex

%= Befehl-Overwriting =======================================================================
\makeatletter
\makeatother

%= Strings ================================================================
\newcommand{\mainfold}{.}
\newcommand{\prefix}{A1-}

%= Eigene Befehle ==========================================================================
\DeclareMathOperator{\id}{Id}
\DeclareMathOperator{\arccot}{arccot}
\DeclareMathOperator{\arcsinh}{arcsinh}
\DeclareMathOperator{\arccosh}{arccosh}
\DeclareMathOperator{\arctanh}{arctanh}
\DeclareMathOperator{\md}{d}
\DeclareMathOperator{\Grad}{grad}
\DeclareMathOperator{\Spur}{Spur}
\DeclareMathOperator{\Graph}{Graph}
\DeclareMathOperator{\sign}{sign}
\DeclareMathOperator{\Hom}{Hom}
\DeclareMathOperator{\rot}{rot}
\DeclareMathOperator{\Ker}{Ker}
\DeclareMathOperator{\Exp}{Exp}
\DeclareMathOperator{\Sym}{Sym}

\newcommand{\Diff}[2]{\displaystyle\frac{\mathrm{d}#1}{\mathrm{d}#2}}
\newcommand{\End}{\hfill{\hbox{$\Box$}}\par\vspace{2mm}}
\newcommand{\eps}{\varepsilon}
\newcommand{\ePic}[1]{\input{\mainfold/graphics/\prefix#1.eepic}}
\newcommand{\pst}[1]{\input{\mainfold/graphics/\prefix#1.pst}}
\newcommand{\pic}[1]{\input{\mainfold/graphics/\prefix#1.pic}}
\newcommand{\Mx}[1]{\begin{pmatrix}#1\end{pmatrix}}
%\newcommand{\im}[1]{\operatorname{Im}(#1)}
%\newcommand{\Include}[4]{\rhead{#2.#3.20#4}\input{\mainfold/lectures/#1-#4-#3-#2.tex}}
\newcommand{\Index}[1]{\emph{#1}\index{#1}}
\newcommand{\Int}[4]{\displaystyle\int\limits_{#1}^{#2}#3\,\mathrm{d}#4}
\newcommand{\diff}[1]{\operatorname{d}\!#1}
\newcommand{\Limi}[1]{\displaystyle\lim_{#1\rightarrow\infty}}
\newcommand{\Limo}[1]{\displaystyle\lim_{#1\rightarrow0}}
\newcommand{\Limu}[2]{\displaystyle\lim_{#1\uparrow #2}}
\newcommand{\Limd}[2]{\displaystyle\lim_{#1\downarrow #2}}
\newcommand{\Lim}[2]{\displaystyle\lim_{#1\rightarrow#2}}
\newcommand{\mb}[1]{\mathbb{#1}}
\newcommand{\ds}{\displaystyle}
\newcommand{\ol}[1]{\overline{#1}}
\newcommand{\Part}[2]{\dfrac{\partial #1}{\partial #2}}
\newcommand{\QED}{\hfill{\hbox{(QED)}}\par\vspace{2mm}}
\newcommand{\re}[1]{\operatorname{Re}(#1)}
\newcommand{\s}{\hspace{2mm}}
\newcommand{\vsa}{\vspace{1mm} \\}
\newcommand{\vsb}{\vspace{2mm} \\}
\newcommand{\vsc}{\vspace{3mm} \\}
% \newcommand{\tr}[1]{\textrm{#1}}
\newcommand{\tr}[1]{\text{#1}}
\newcommand{\ra}{\rightarrow}
\newcommand{\Ra}{\Rightarrow}
\newcommand{\Lra}{\Leftrightarrow}
\newcommand{\La}{\Leftarrow}
\newcommand{\ul}[1]{\underline{#1}}
\newcommand{\rsa}{\rightsquigarrow}
\newcommand{\ara}[2]{\autorightarrow{\ensuremath{#1}}{\ensuremath{#2}}}
\newcommand{\dcp}[2]{\begindc{\commdiag}[#1] #2 \enddc}
\renewcommand{\to}{\rightarrow}

%\newcommand{\detmx}{\left| \begin{array} #1 \end{array} \right|}

\newcommand{\grad}[1]{\Grad(#1)}
\newcommand{\fr}[2]{\displaystyle\frac{#1}{#2}} % fertiger bullshit, daf�r gibts \dfrac{}{}
\renewcommand{\Re}{\operatorname{Re}}
\renewcommand{\Im}{\operatorname{Im}}

% ---- DELIMITER PAIRS ----
\def\floor#1{\lfloor #1 \rfloor}
\def\ceil#1{\lceil #1 \rceil}
\def\seq#1{\langle #1 \rangle}
\def\set#1{\{ #1 \}}
\def\abs#1{\mathopen| #1 \mathclose|}	% use instead of $|x|$ 
\def\norm#1{\mathopen\| #1 \mathclose\|}% use instead of $\|x\|$ 

% --- Self-scaling delmiter pairs ---
\def\Floor#1{\left\lfloor #1 \right\rfloor}
\def\Ceil#1{\left\lceil #1 \right\rceil}
\def\Seq#1{\left\langle #1 \right\rangle}
\def\Set#1{\left\{ #1 \right\}}
\def\Abs#1{\left| #1 \right|}
\def\Norm#1{\left\| #1 \right\|}

%Adrians Abbildungs-Environment ==============================================

\newcommand{\Sidein}{\begin{rotate}{90}\small$\in$\end{rotate}}

\newcommand{\Abb}[5][]{\ensuremath{
    \begin{array}{lc}
      \ifthenelse{\equal{#1}{}}{}{#1:}\;\; & 
      \begin{xy}
        \xymatrixrowsep{1em}\xymatrixcolsep{2em}%
        \xymatrix{ #2 \ar[r] \ar@{}[d]^<<<<{\hspace{0.001em} \Sidein}
          & #3  \ar@{}[d]^<<<<{\hspace{0.001em} \Sidein} \\
          #4 \ar@{|->}[r] & #5} \end{xy}
    \end{array}
  }%
}

%= Environments ========================================================================
\def\thechapter{\Roman{chapter}}
\def\thesection{\arabic{section}}
\newtheorem{theorem}{Theorem}[section]
\newenvironment{Bew}{\begin{proof}[Beweis]}{\end{proof}}
\newtheorem{Axi}[theorem]{Axiom}
\newtheorem{Lem}[theorem]{Lemma}
\newtheorem{Kor}[theorem]{Korollar}
\newtheorem{Sat}[theorem]{Satz}
\newtheorem{Prop}[theorem]{Proposition}
\newtheorem{Beh}[theorem]{Behauptung}
\theoremstyle{definition}
\newtheorem{Bsp}[theorem]{Beispiel}
\newtheorem{Def}[theorem]{Definition}
\newtheorem{Ueb}[theorem]{\"Ubung}
\theoremstyle{remark}
\newtheorem{Bem}[theorem]{Bemerkung}
\newtheorem{Eig}[theorem]{Eigenschaften}
\newtheorem{Not}[theorem]{Notation}

\def\pstexInput#1{%
  \begin{center}
    \begin{picture}(0,0)%
      \special{psfile=\mainfold/graphics/A2-#1.pstex}%
    \end{picture}%
    \input{\mainfold/graphics/A2-#1.pstex_t}%
  \end{center}
}

%= Titelseite ===========================================================================
\begin{document}
\headheight15pt
\begin{titlepage}
\hfill
\vspace{20mm}
\pagenumbering{roman}
\begin{center}
{\LARGE Analysis II - Vorlesungs-Script} \vskip 3em {\large Prof. Dr. Camillo De Lellis} \vskip 1.5em
{\large Basisjahr 11 Semester I}\vspace{30mm}\\
{\large {\bf Mitschrift:} \vspace{2mm}\\
Simon Hafner}\vspace{5mm}\\ %30mm
%{\large {\bf Graphics:} \vspace{2mm}\\
%Pirmin Weigele }\vspace{30mm}\\ %30mm
\author{Simon Hafner}

\end{center}
\vfill

\end{titlepage}


%= Inhaltsverzeichnis ==========================================================================
\lhead{}
\rhead{}
\tableofcontents
\newpage
\pagenumbering{arabic}
\setcounter{page}{1}

%= Vorlesung-Skripts ==========================================================================
\cfoot{\thepage}
\fancyhead[L]{\nouppercase{\leftmark}}
\newpage

%Analysis II
\section{Metrik und Topologie des euklidischen Raumes}
$\mb{R}^n=\left\{ \left( x_1,\cdots,x_n \right), x\in\mb{R} \right\}$.
Wir f\"uhren verschiedene neue Begriffe in $\mb{R}^n$ ein:
\begin{itemize}
  \item die Euklidische Norm
  \item der Euklidische Abstand
  \item die entsprechende Topologie.
\end{itemize}
Wir betrachten gleichzeitig die entsprechenden Verallgemeinerungen,
d.h. die ``Abstrakte Theorien'' der
\begin{itemize}
  \item Normierten Vektorr\"aume
  \item Metrischen R\"aume
  \item Topologischen R\"aume.
\end{itemize}
\begin{Def}
  Sei $x\in\mb{R}^n$ ($x=(x_1,\cdots,x_n)$, $x_i\in\mb{R}$). Die Euklidische Norm
von $x$ ist 
  \[\Norm{x}_e=\sqrt{x_1^2+\cdots+x_n^2}=\sqrt{\sum_{i=1}^nx_i^2}\]
(wir schreiben oft $\|x\|$ anstatt $\|x\|_e$).
\end{Def}

Intuitiv: $\Norm{x}=$''der Abstand zwischen $x$ und 0``.  In der Tat, wenn $n=2$,
das Pytaghoras Theorem zeigt dass $\|x\|_e$ die L\"ange des Segments mit Extrema
$x$ and $0$ ist. 

\begin{Lem}\label{l:norm}
  $\Norm{.}$ erf\"ullt die Regeln
  \begin{enumerate}
    \item $\Norm{x}\geq 0$ und $\Norm{x}=0\iff x=0$
    \item $\Norm{\lambda x}=\Abs{\lambda}\Norm{x}$ $\forall \lambda\in\mb{R}$, $\forall x\in\mb{R}$
    \item $\Norm{x+y}\leq\Norm{x}+\Norm{y}$ $\forall x,y\in\mb{R}$
  \end{enumerate}
\end{Lem}
\begin{Bew}
  \begin{enumerate}
    \item $\geq 0$ trivial
      \[x=0\implies \sum x_i^2=0\implies \Norm{x}=0\]
      \[x=0\Leftarrow x_i=0\;\; \forall i\Leftarrow \sum x_i^2=0\Leftarrow \Norm{x}=0\]
    \item \[\Norm{\lambda x}=\sqrt{\sum^n_{i=1}(\lambda x_i)^2} = \sqrt{\lambda^2\sum x^2}=\Abs{\lambda}\sqrt{\sum x^2}=\Abs{\lambda}\Norm{x}\]
%\[\Abs{\lambda}=\frac{\Norm{x}\Abs{\lambda}}{\Norm{x}}\]
    \item Diese Aussage ist \"aquivalent zu
      \[\iff \underbrace{\Norm{x+y}^2}\leq \Norm{x}^2+\Norm{y}^2+2\Norm{x}\Norm{y}\]
Wir rechnen
      \[\sum_{i=1}^n(x_i+y_i)^2=\sum_{i=1}^n\left( x_i^2+y_i^2+2x_iy_i \right)=\Norm{x}^2+\Norm{y}^2\overbrace{2\sum_i x_iy_i}^{Skalarprodukt}\]
Wir definieren 
\[\langle x,y\rangle := \sum_{i=1}^n  x_iy_i\]
Wir brauchen dann die ber\"uhmte Cauchy-Schwartz Ungleichung, d.h.
      \[\langle x,y\rangle\leq \Norm{x}\Norm{y}\, .\]
Diese Ungleichung ist der Inhalt des n\"achsten Satzes.
  \end{enumerate}
\end{Bew}
\begin{Sat}{Cauchy-Schwartzsche Ungleichung}
  \[\sum^n_{i=1}x_iy_i\leq\sqrt{\sum_{i=1}^nx_i^2}\sqrt{\sum_{i=1}^ny_i^2}\]
\end{Sat}
\begin{Bew}
  OBdA $y\neq 0$ ($y=0$ trivial)
  \[t\to g(t)=\sum_{i=1}^n(x_i+ty_i)^2=\left( \sum x_i^2 \right)+2t\sum x_iy_i+t^2\sum y_i^2\]
  \[=\Norm{x}^2+2t\seq{x,y}+\Norm{y}^2t^2\]
  Sei $t_0=\frac{\seq{x,y}}{\Norm{y}^2}$, dann
  \[0\leq g(t_0)=\Norm{x}^2-2\frac{\seq{x,y}^2}{\Norm{y}^2}+\Norm{y}^2\frac{\seq{x,y}^2}{\Norm{y}^4}
 =\Norm{x}^2-\frac{\seq{x,y}^2}{\Norm{y}^2}\]
  \[\implies \seq{x,y}^2\leq\Norm{x}^2\Norm{y}^2\implies \Abs{\seq{x,y}}\leq\Norm{x}\Norm{y}\]
\end{Bew}
\begin{Def}
  Ein normierter Vektorraum ist ein reeller Vektorraum $V$ mit einer Abbildung $\Norm . :V\to\mb R$ so dass:
  \begin{enumerate}
    \item $\Norm x\geq 0$ und $\Norm x=0\iff x=0$ (Nullvektor)
    \item $\Norm{\lambda x}=\abs\lambda\Norm x$ $\forall \lambda\in\mb R$, $\forall x\in V$
    \item $\Norm{x+y}\leq\Norm{x}+\Norm{y}$ $\forall x,y\in V$
  \end{enumerate}
\end{Def}
\begin{Bsp}
  $V=\mb{R}^n$
  \[\Norm{x}_p=\left(\sum\Abs{x_i}^p\right)^{\frac{1}{p}}\s p\geq 1\, .\]
$\|\cdot\|_2$ ist die Euklidische Norm.
\end{Bsp}
\begin{Def}
  Seien $x,y\in\mb{R}^n$. Die Euklidische Metrik ist $d(x,y):=\Norm{x-y}$.
\end{Def}
\begin{Lem}\label{l:euk_met}
  \begin{enumerate}
    \item $d(x,y)\geq 0$ und $d(x,y)=0\iff x=y$
    \item $d(x,y)=d(y,x)$
    \item $d(x,z)\leq d(x,y)+d(y,z)$ (Dreiecksungleichung)
  \end{enumerate}
\end{Lem}
\begin{Bew} Die erste Zwei Aussagen sind trivial. Um die letzte zu beweisen:
  \[\Norm{x-z}\leq\|\underbrace{x-y}_{=:v}\|+\|\underbrace{\Norm{y-z}}_{=:w}\|\, .\]
Aber $x-z=v+w$. Wir wenden die dritte Aussage von Lemma \ref{l:norm} an:
  \[d (x,z) = \Norm{v+w}\leq\Norm v+\Norm w = d (x,y)+d(y,z)\, .\]
\end{Bew}
\begin{Def}
  Ein metrischer Raum ist eine Menge $X$ mit einer Abbildung
  \[d:X\times X\to\mb{R}\s (x,y)\mapsto d(x,y)\in\mb{R}\]so dass
  \begin{enumerate}
    \item $d(x,y)\geq 0$ und $d(x,y)=0\iff x=y$ $\forall x,y\in X$
    \item $d(x,y)=d(y,x)$ $\forall x,y\in X$
    \item $d(x,z)=d(x,y)+d(y,z)$ $\forall x,y,z\in X$
  \end{enumerate}
\end{Def}
\begin{Lem}
  Sei ($V$, $\Norm .$) ein normierter Vektorraum. Dann sind $V$ und $d(x,y)=\Norm{x-y}$ ein metrischer Raum.
\end{Lem}
\begin{Bew} Wir nutzen das gleiche Argument vom Lemma \ref{l:euk_met}.
\end{Bew}
\begin{Def}
  Die offene Kugel mit Radius $r>0$ und Mittelpunkt $x\in\mb{R}^n$ ist die Menge
  \[K_r(x)=\left\{ y\in\mb{R}^n, d(x,y)<r \right\}\]
(Wir werden auch oft $B_r (x)$ statta $K_r (x)$ nutzen.)
\end{Def}
\begin{Def}
  Eine Menge heisst ''Umgebung`` von $x$, wenn $V$ eine offene Kugel mit Mittelpunkt $x$ enth\"alt.
\end{Def}
\begin{Def}
  Eine Menge $U\subset\mb{R}^n$ heisst offen falls $\forall x\in U$ ist $U$ eine Umgebung von $x$, d.h.
  \[\forall x\in U\s\exists \s\text{eine Kugel}\s K_r(x)\subset U\]
\end{Def}
\begin{Bem}
Die Dreiecksungleichung impliziert dass jede offene Kugel eine offene Menge ist. In der Tat,
sei $y\in K_r (x)$. Dann $\rho:=d(x,y) < r$. Sei $\tau:= r-\rho>0$. Falls $z\in K_\tau (y)$,
dann $d(x,z)\leq d(x,y) + d(y,z) = \rho + d (y,z) < \rho +\tau =r$. D.h., $K_\tau (y)\subset K_r (x)$.
Das beweist dass $K_r (x)$ eine Ungebung ihrer ganzen Elementen ist, d.h. $K_r (x)$ ist offen. 
\end{Bem}
\begin{Sat}
  \begin{enumerate}
    \item $\varnothing$ und $\mb{R}^n$ sind offen
    \item Der Schnitt \ul{endlich vieler} offener Mengen ist auch offen.
    \item Die Vereinigung einer \ul{beliebigen} Familie offener Mengen ist auch offen.
  \end{enumerate}
\end{Sat}
\begin{Bew}
  \begin{enumerate}
    \item $\mb{R}^n$ trivialerweise offen, auch $\varnothing$
    \item Sei $x\in U\cap\dots\cap U_N$
      \[\forall i\in\left\{ 1,\dots,N \right\}\s \quad \exists r_i>0 \;\;\mbox{so dass}\;\; K_{r_i}(x)\subset U_i\]
      Sei $r=\min\left\{ r_i,\dots,r_N \right\} > 0$;
      \[\implies K_r(x)\subset U_i\quad\forall i\implies K_r(x)\subset U_1\cap\dots\cap U_N\]
    \item $\left\{ U_\lambda \right\}_{\lambda\in \Lambda}$. Sei $U=\bigcup_{\lambda\in\Lambda}U_\lambda$
      \[x\in U\implies x\in U_\lambda\s\text{f\"ur ein}\s\lambda\in\Lambda\]
      \[\implies \exists K_r(x)\subset U_\lambda\subset U.\]
  \end{enumerate}
\end{Bew}
\begin{Def}
  Ein topologischer Raum ist eine Menge $X$ und eine Menge $O$ von Teilmengen von $X$ so dass:
  \begin{enumerate}
    \item $\varnothing, X\in O$
    \item $U_1\cap\dots\cap_N\in O$ falls $U_i\in O$
    \item $\bigcap_{\lambda\in\Lambda}U_\lambda\in O$ falls $U_i\in O$
  \end{enumerate}
$O$ heisst die {\em Topologie}.
\end{Def}
\begin{Sat}
  Sei $(X,d)$ ein metrischer Raum. Wir definieren die entsprechende offene Kugel mit Mittelpunkt $x\in X$
und Radius $r>0$:
  \[K_r(x)=\left\{ y=X: d(x,y)<r \right\}\]
  Umgebungen und offene Mengen sind wie im Euklidischen Fall definiert. $O=\left\{ \text{offene Menge} \right\}$ definiert eine Topologie.
\end{Sat}

\subsection{Konvergenz}
Sei $\left\{ x_k \right\}_{k\in\mb{N}}$ $x_k\in\mb{R}$ $x_k=\left( x_{k1}, \cdots, x_{kn} \right)$
\begin{Def}
  Die Folge $\left\{ x_k \right\}$ konvergiert gegen $x_\infty\in\mb{R}^n$ falls
  \[\Limi{k}d(x_k,x_\infty)=0\]
  \[\left( \Limi{k}\Norm{x_k,x_\infty}=0 \right)\]
  Dann schreiben wir
  \[x_\infty=\Limi{k}x_k\]
\end{Def}
\begin{Sat}
  \[x_k\to x_\infty\iff x_{ki}\to x_{\infty_i}\s\forall i\in\left\{ 1,\cdots,n \right\}\]
\end{Sat}
\begin{Bew}
  \[\Norm{x_k - x_\infty}=\sqrt{\sum_{i=1}^n\left( x_{ki}-x_{\infty_i} \right)^2}\geq \abs{x_{ki}-x_{kinfty}}\geq 0\]
  \[\implies 0\leq \Limi{k}\abs{x_{ki}-x_{kinfty}}\leq \lim\Norm{x_k-x_\infty}=0\]
  \[\Norm{x_k-x_\infty}=\underbrace{\sqrt{\sum_{i=1}^n\underbrace{(x_{ki}-x_{\infty_i})^2}_{\to 0}}}_{\to 0}\leq \sum_{i=1}^n\abs{x_{ki}-x_{\infty_i}}\]
  \[\implies \Norm{x_k-x_\infty}\to 0\]
  Eine alternative Formulierung: $\Limi{k}x_k=\left( \Limi{k} x_{k1},\cdots,\Limi{k} x_{kn} \right)$
\end{Bew}
\begin{Bem}
  \[\forall \varepsilon>0 \exists N: \Norm{x_k-x_\infty}<\varepsilon\s\text{falls}\s k\geq N\]
  Für jede Umgebung $U$ von $x_\infty$ fast alle $x_k\in U$.
\end{Bem}
\begin{Def}
  Eine Folge $\left\{ x_k \right\}\subset\mb{R}^n$ heisst Cauchy falls:
  \[\forall \varepsilon>0\s\exists N: m,k\geq N\implies \Norm{x_k-x_m}<\varepsilon\]
\end{Def}
\begin{Lem}
  $\left\{ x_k \right\}\subset\mb{R}^n$ konvergiert genau dann, wenn $\left\{ x_k \right\}$ Cauchy ist.
\end{Lem}
\begin{Bew}
  $\left\{ x_k \right\}$ ist Cauchy $\implies$ $\left\{ x_{k_{\underbrace{i}_{\left\{ \text{fixiert} \right\}}}} \right\}$ Cauchy!
  \[\abs{x_{ki}-x_{m_i}}\leq\Norm{x_k-x_m}\]
  $\implies$ $\left\{ x_k \right\}$ ist eine Cauchyfolge $\stackrel{\text{Erstes Semester}}{\implies}$ $x_{ki}$ konvergiert $\stackrel{\text{Lemma 2}}{\implies}$ $x_k$ konvergiert. $x_k$ konvergiert $\implies$ Cauchyfolge
  \[x_\infty=\Limi{k} x_k\s\forall \varepsilon>0\s\exists N:\Norm{x_k-x_\infty}<\frac{\varepsilon}{2}\s\forall k\geq N\]
  \[k,m\geq N\s \Norm{x_k-x_m}\leq \Norm{x_k-x_\infty}+\Norm{x_\infty-x_m}\leq d(x_k,x_\infty)+(x_\infty,x_m)\]
  \[<\frac{\varepsilon}{2}+\frac{\varepsilon}{2}=\varepsilon\]
\end{Bew}
\begin{Bem}
  In einem metrischen Raum, Cauchy $\Leftarrow$ Konvergenz. Aber allgemein: Cauchy $\not\implies$ Konvergenz. Falls Cauchy $\implies$ Konvergenz, dann ist der metrische Raum vollständig.
\end{Bem}
\begin{Def}
  Eine Folge $\left\{ x_k \right\}\subset\mb{R}^n$ heisst beschränkt falls $\Norm{x_k}$ beschränkt ist.
\end{Def}
\begin{Sat}
  \begin{enumerate}
    \item Eine konvergente Folge ist beschränkt
    \item (Bolzano-Weierstrass) $\left\{ x_k \right\}$ beschränkt $\implies$ $\exists \left\{ x_{k_j} \right\}$ die konvergiert.
  \end{enumerate}
\end{Sat}
\begin{Bew}
  \[\left\{ x_k \right\}\s\text{beschränkt}\implies \left\{ x_{k1} \right\}_{k\in\mb{N}}\s\text{beschränkt}\]
  \[\implies \exists x_{k_j}: x_{k_j1}\to x_1\]
  Ich definiere $y_j=x_{k_j}$ $y_{j1}\to x_1$
  \[y_j\s\text{beschränkt}\implies\exists j_l: y_{j_l2}\to x_2\]
  \[z_l:=y_{j_l}\s\text{und}\s z_{l1}\to x_1, \s x_{l2}\to x_2\]
  \ldots $(n-2)$ Schritte. $w_r$ Teilfolge von $x_k$ mit $w_{ri}\to x_i$
  \[w_r\to(x_1,\cdots,x_n)\]
\end{Bew}
\subsection{Ein bisschen mehr Topologie}
\begin{Def}
  Eine Menge $G\subset\mb{R}^n$ heisst geschlossen falls $G^c(:=\mb{R}^n\setminus G)$ eine offene Menge ist.
\end{Def}
\begin{Bem}
  \[(A\cup B)^c = A^c\cap B^c\]
  \[(A\cap B)^c = A^c\cup B^c\]
\end{Bem}
\begin{Sat}
  \begin{enumerate}
    \item $\varnothing, \mb{R}^n$ sind abgeschlossen
    \item $G_1,\cdots,G_N$ abgeschlossen $\implies$ $G_1\cup G_2\cup \cdots\cup G_N$ abgeschlossen
    \item $\left\{ G_\lambda \right\}_{\lambda\in\Lambda}$ abgeschlossen $\implies$ $\bigcap_{\lambda\in\Lambda} G_\lambda$ abgeschlossen.
  \end{enumerate}
\end{Sat}
\begin{Sat}
  $G\subset\mb{R}^n$ $G$ ist abgeschlossen $\iff$ $\forall$ jede konvergente $\left\{ x_k \right\}\subset G$ gehört der Grenzwert zu $G$ (gilt auch für metrische Räume).
\end{Sat}
\begin{Bew}
  \begin{itemize}
    \item[$\Leftarrow$] Die rechte Eigenschaft gilt. Ziel: $G^c$ ist offen. Sei $x\in G^c$: das Ziel ist eine Kugel $K_r(x)\in G^c$ zu finden. Widerspruchsbeweis: $K_{\frac{1}{j}}(x)\not\subset G^c$, $j\in\mb{N}\setminus\left\{ 0 \right\}$
      \[\implies \exists x_j\in K_{\frac{1}{j}}(x)\cap G\implies \left\{ x_j \right\}\subset G\s\text{und}\s x_j\to x \]
      \[\left\{ x_j \right\}\subset G\s x_j\to x\s x\not\in G\]
      $\implies$ d.h. $G^c$ offen $\implies$ falls $\left\{ x_k \right\}\subset G$ und $x_k\to x$ dann $x\in G$
      Widerspruch: $G^c$ offen, aber $\exists \left\{ x_k \right\}\subset G$ mit Grenzwert $x\not\in G$, d.h. $x\in G^c$. Offenheit von $G^c$.
      \[\implies \exists K_r(x)\subset G^c\implies K_r(x)\cap=\varnothing\]
      d.h. $\exists N$ mit
      \[\Norm{x_N-x}<r\implies x_N\in K_r(x)\cap G\]
  \end{itemize}
\end{Bew}
\begin{Bsp}
  Eine offene Kugel ist nicht geschlossen.
  \[K_r(x)=\left\{ y:\Norm{y-x}<r \right\}\]
  Sei $\left\{ y_k \right\}\in K_r(x)$, (d.h. $\Norm{y_k-x}<r$) mit $y_k\to y$ und $\Norm{y-x}=r$.
\end{Bsp}
\begin{Def}
  Sei $\ol{K_r(x)}:=\left\{ y\in\mb{R}^n:\Norm{y-x}\leq r \right\}$.
\end{Def}
\begin{Ueb}
  $\ol{K_r(x)}$ ist abgeschlossen
\end{Ueb}
\begin{Def}
  $x\in\mb{R}^n$ ist ein Randpunkt von $M$ falls
  \[\forall K_r(x)\s\exists y\in K_r(x)\cap M\s\text{und}\s \exists z\in K_r(x)\cap M^c\]
\end{Def}
\begin{Def}
  Sei $M$ eine Menge in $\mb{R}^n$, dann ist der Rand von $M$
  \[\partial M=\left\{ x\in\mb{R}^n, \s\text{Randpunkt von}\s M \right\}\]
\end{Def}
\begin{Sat}
  $\partial M^c=\partial M$
  \begin{enumerate}
    \item $M\setminus \partial M$ ist die grösste offene Menge die in $M$ enthalten ist.
    \item $M\cup \partial \partial M$ ist die kleinste geschlossene Menge die $M$ enthält.
  \end{enumerate}
\end{Sat}
\begin{Bew}
  $M\setminus \partial M$ ist offen. 
  \[x\in M\setminus \partial M \implies x\in M\s\text{und}\s \exists K_r(x)\s\text{mit}\s K_r(x)\cap M^c=\varnothing\]
  \[\implies K_r(x)\subset M\]
  Sei $y\in K_r(x)$
  \[\implies \abs{y-x}=\rho<r\]
  \[\implies K_{r-\rho}(y)\subset K_r(x)\subset M\implies y\in M,y\not\in \partial M\]
  \[K_r(x)\subset M\setminus \partial M\]
  $x$ ist beliebig $\implies$ $M\setminus \partial M$ ist offen.\\
  Sei $A\subset M$ eine offene Menge. Das Ziel ist $A\subset M\setminus\partial M$. Sei $x\in A$. Ziel:($x\in M\setminus\partial M$) $x\not\in \partial M$.
  \[A\s\text{offen}\implies \exists K_r(x)\subset A\subset M\implies x\not\in \partial M\implies A\subset M\setminus\partial M\]
\end{Bew}

\subsection{Stetigkeit}
\begin{Def}
  Sei $f:\Omega_{\subset\mb{R}^n}\to\mb{R}^k$. $f$ ist stetig an der Stelle $x\in\Omega$ falls $\forall \left\{ x_k \right\}\subset\Omega$ mit $x_k\to x$.
  \[\Limi{k} f(x_k)=f(x)\]
\end{Def}
\begin{Lem}
  \label{l:1102282}
  Eine equivalente Definition:
  \[\forall \varepsilon>0\s\exists \delta>0: f\left( K_\delta(x)\cap \Omega \right)\subset K_\varepsilon(f(x))\]
\end{Lem}
\begin{Bew}
  $\varepsilon$-$\delta$ $\implies$ Folgendefinition. Sei $x_k\to x$. Ziel: $f(x_k)\to f(x)$
  \[\forall \varepsilon>0\s\exists \s\text{mit}\s \underbrace{\Norm{f(x_k)-f(x)}}_{\underbrace{d(f(x_k),f(x))}_{f(x_k)\in K_\varepsilon(f(x))}}<\varepsilon\s\forall k\geq N\]
  \[\exists \delta>0\s \underbrace{f(K_\delta(x))\subset K_\varepsilon(f(x))}\]
  \[\exists\s \Norm{x_k-x}<\delta\s k\geq N\]
  \[x_k \in K_\delta(x)\implies f(x_k)\in K_\varepsilon(f(x))\]
  Folgendefinition $\implies$ ($\varepsilon$-$\delta$)-Defintion. Widerspruchsannahme: 
  \[\exists \varepsilon>0: f(K_\delta(x)\cap \Omega)\not\subset K_\varepsilon(f(x))\s\forall \delta>0\]
  \[\implies\forall \delta>0\s\exists y_\delta\in K_\delta(x)\s\text{und}\s \Norm{f(y_\delta)-f(x)}\geq \varepsilon\]
  Nehmen wir $\delta=\frac{1}{j}$ und $x_j=\frac{y_1}{j}$
  \[\Norm{x_j-x}<\frac{1}{j}\s(\text{weil}\s x_j\in K_{\frac{1}{j}}(x))\]
  \[\Norm{f(x_j)-f(x)}=\Norm{f(y_{\frac{1}{j}}-f(x))}\geq\varepsilon\]
  $x_j\to x$ aber $f(x_j)\not\to f(x)$
\end{Bew}
\begin{Def}
  Die allgemeine Defintion der Stetigkeit für metrische Räume: Seien $(X,d)$ und $(Y,\ol d)$ zwei metrische Räume. Sei $f:X\to Y$. $f$ ist stetig an der Stelle $x$ falls:
  \[\forall \varepsilon>0\s\exists \delta>0\s\text{mit}\s d(y,x)<\delta\implies d(f(y),f(x))<\varepsilon\]
  \[\iff f(K\delta(x))\subset K_\varepsilon(f(x))\]
\end{Def}
\begin{Def}
  Eine $f:X\to Y$ heisst stetig falls $f$ stetig an jeder Stelle $x\in X$ ist.
\end{Def}
\begin{Sat}
  Sei $f:X\to Y$ ($(X,d), (Y\ol d)$ metrische Räume) Dann:
  \begin{enumerate}
    \item Die Stetigkeit in $x$ $\iff$ $\forall$ Umgebung $U$ von $f(x)$ ist $f^{-1}(U)$ eine Umgebung von $x$.
    \item Stetigkeit von $f$ $\iff$ $f^{-1}(U)$ ist offen $\forall U$ offen.
  \end{enumerate}
\end{Sat}
\begin{Bew}
  \begin{enumerate}
    \item
      \begin{itemize}
        \item Stetigkeit $\implies$ Umgebung.
          $U$ Umgebung von $f(x)\implies \exists \delta>0$ mit $K_\delta(f(x))\subset U$
          \[\implies \exists \varepsilon>0:f(K_\varepsilon(x))\subset K_\delta(f(x))\]
          \[\implies f^{-1}(U)\supset f^{-1}(K_\delta(f(x)))\supset K_\varepsilon(x)\implies f^{-1}(U)\s\text{Umgebung von}\s U\]
        \item Umgebung $\implies$ Stetigkeit. Sei $\delta>0$ $U=K_\delta(f(x))$. $U$ Umgebung von $f(x)$. $f^{-1}(U)$ ist eine Umgebung von $x$.
          \[\implies \exists\varepsilon>0:K_\varepsilon(X)\subset f^{-1}(U)\]
          \[\implies f(K_\varepsilon(x))\subset U=K_\delta(f(x))\]
      \end{itemize}
    \item
      \begin{itemize}
        \item 
          Stetigkeit $\implies$ offen. Sei $U$ offen $\iff$ $\forall y\in U$ ist $U$ eine Umgebung von $y$
          \[f^{-1}U\ni x\implies f(x)\in U\stackrel{\text{Stetigkeit in}\s x}{\implies} f^{-1}(U)\s\text{ist eine Umgebung von}\s x\]
          \[\implies f^{-1}(U)\s\text{ist offen}\]
        \item offen $\implies$ Stetigkeit an jedem $x\in X$. Sei $x\in X$, $\delta>0$, $K_\delta(f(x))$ ist offen
          \[f^{-1}(K_\delta(f(x)))\s\text{ist offen}\implies x\in f^{-1}(K_\delta(f(x)))\]
          \[\implies \exists \varepsilon>0: K_\varepsilon(x)\subset f^{-1}(K_\delta(f(x)))\]
          \[f(K_\varepsilon(x))\subset K_\delta(f(x))\]
      \end{itemize}
  \end{enumerate}
\end{Bew}
\subsection{lineare Abbildungen}
\begin{Def}
  Eine Abbildung $L:V\to W$ ($V$, $W$ Vektoren) heisst linear, falls
  \[L(\lambda_1v_1+\lambda_2v_2)=\lambda_1L(v_1)+\lambda_2L(v_2)\s\forall v_1,v_2\in V,\s\forall \lambda_1,\lambda_2\in\mb{R}\]
  \[L:\mb{R}^n\to\mb{R}^k\iff \exists\s\text{eine Matrix}\s L_{ij}:\]
  \[L(x)=\left( \sum^n_{j=1}L_{1j}x_j,\sum^n_{j=1}L_{2j}x_j,\cdots,\sum^n_{j=1}L_{kj}x_j \right)\]
\end{Def}
\begin{Def}
  Sei $L_{ij}$ eine Matrix die die lineare Abbildung $L:\mb{R}^n\to\mb{R}^k$ darstellt. Die Hilbert-Schmidt Norm von $L$ ist
  \[\Norm{L}_{\text{HS}}=\sqrt{\sum^k_{i=1}\sum^n_{j=1}L_{ij}^2}\]
\end{Def}
\begin{Bem}
  $\left\{ L:(L_{ij} n\times k\s\text{Matrixen} \right\}\sim \mb{R}^{nk}$ $\Norm{.}_{\text{HS}}$ ist die euklidische Norm.
\end{Bem}
\begin{Bem}
  Sei $L:\mb{R}^n\to\mb{R}^k$ eine lineare Abbildung und $x\in\mb{R}^n$. Dann $\Norm{L(x)}\leq\Norm{x}\Norm{L}_\text{HS}$.
\end{Bem}
\begin{Kor}
  Sei $L$ wie oben, dann ist $L$ stetig.
\end{Kor}
\begin{Bew}
  Sei $x_k\to x$. Ziel $L(x_k)\to L(x)$
  \[\Norm{L(x_k)-L(x)}=\Norm{L(x_k-x)}\leq\Norm{x_k-x}\Norm{L}_\text{HS}\to 0\]
  \[\implies \Norm{L(x_k)-L(x)}\to 0\]
  \[\implies \text{Stetigkeit}\]
\end{Bew}
\begin{Bew}
  Beweis von \ref{l:1102282}: $L(x)=y$
  \[\Norm{L(x)}^2=\sum^k_{i=1}y_i^2\]
  \[=\sum^k_{i=1}\left( \sum^n_{j=1}L_{ij}x_j \right)^2\stackrel{\text{Cauchy-Schwartz}}{\leq}\sum^k_{i=1}\left( \sum^n_{j=1}L_{ij}^2 \right)\left( \sum_{j=1}x_j \right)^2\]
  \[=\sum^k_{i=1}\sum^n_{j=1}L_{ij}^2\Norm{x}^2=\Norm{x}^2\left( \sum^k_{i=1}\sum^n_{j=1}L_{ij}^2 \right)\]
  \[\Norm{x}^2\Norm{L}^2_\text{HS}\implies \Norm{L(x)}\leq\Norm{x}\Norm{L}_\text{HS}\]
\end{Bew}
\begin{Def}
  Sei $L:V\to W$ eine lineare Abbildung wobei $(V,\Norm{.}_V)$ und $(W,\Norm{.}_W)$ zwei endlich-dimensionierte Vektorräume sind. Die Operatornorm von $L$ ist:
  \[\Norm{L}_{L(V,W)}:=\sup_{\Norm{v}_V\leq 1}\Norm{L(v)}_W\]
\end{Def}
\begin{Sat}
  $\Norm{.}_{L(V,W)}$ ist eine Norm und
  \[\Norm{L(v)}_W\leq \Norm{L}_{L(V,W)}\Norm{v}_V\]
  Deswegen: jede lineare Abbildung $L:V\to W$ ist stetig.
\end{Sat}
\begin{Bew}
  Der Kern ist die folgende Eigenschaft:
  \[\Norm{L}_{L(V,W)}<+\infty\]
  Wenn das gilt dann:
  \begin{enumerate}
    \item 
      \[\underbrace{\Norm{L}_{L(V,W)}}_\text{Kern}\s\text{und}\s\Norm{L}_{L(V,W)}=0\iff L=0\]
      $\Leftarrow$ einfach. Sei $\Norm{L}_{L(V,W)}=0$. Dann sei $v\in V$.
      \[v=0\implies L(v)=0\]
      \[v\neq 0\s z\frac{v}{\Norm{v_V}}\implies \Norm{z}_V=1\]
      \[\Norm{L(z)}_W\leq\sup_{\Norm{y}_V\leq 1}\Norm{L(v)}_W=0\]
      \[\implies L(z)=0\implies L(v)=L\left( \Norm{v}_V z \right)=\Norm{v}_VL(z)=0\]
    \item
      \[\Norm{\lambda L}_{L(V,W)}=\abs{\lambda}\Norm{L}_{L(V,W)}\]
      \[\Norm{\lambda L}_{L(V,W)}=\sup_{\Norm{y}_V\leq 1}\Norm{\lambda L(v)}_W\]
      \[=\sup_{\Norm{y}_V\leq 1}\abs{\lambda}\Norm{L(v)}_W\]
      \[=\abs{\lambda}\sup_{\Norm{y}_V\leq 1}\Norm{L(v)}_W\]
      \[=\abs{\lambda}\Norm{L}_{L(V,W)}\]
    \item
      \[\Norm{L+L'}_{L(V,W)}\]
      \[=\sup_{\Norm{y}_V\leq 1}\Norm{(L+L')(v)}_{L(V,W)}\]
      \[=\sup_{\Norm{y}_V\leq 1}\Norm{L(v)+L'(v)}_{L(V,W)}\]
      \[\leq\sup_{\Norm{y}_V\leq 1}\left( \Norm{L(v)}_W+\Norm{L'(v)}_W\right)\]
      \[\leq\sup_{\Norm{y}_V\leq 1}\Norm{L(v)}_W+\sup_{\Norm{y}_V\leq 1}\Norm{L'(v)}_W\]
      \[=\Norm{L}_{L(V,W)}+\Norm{L'}_{L(V,W)}\]
      Wenn $v_1,\cdots,v_n$ Basis für $V$, $w_1,\cdots,w_k$ Basis für $W$. Die lineare Abbildung $E_{ij}(v_i)=w_j$, $E_{ij}(v_l)=0$ falls $l\neq i$ ist eine Basis für $L(V,W)\implies L=\sum_{i,j}\lambda_{ij}E_{ij}$
  \end{enumerate}
\end{Bew}


\begin{Bem}
Aus der Definition von $\|\cdot\|_{L(V,W)}$ folgt
  \begin{equation}
    \label{e:1103022}
    \Norm{L(v)}_W\leq\Norm{L}_{L(V,W)}\Norm{v}_V \qquad \forall v\in V\, .
  \end{equation}
  Falls $\Norm{v}_V=1$, dann
  \[\Norm{L(v)} \leq \sup_{\Norm{v}_V\leq 1}\Norm{L(v)}_W = \|L\|_{L(V,W)}\]
F\"ur $v=0$ ist $L(v)=0$ und deswegen ist die Ungliechung \eqref{e:1103022} trivial.
Falls $\Norm{v}_V>0$,
  \[\tilde v:=\frac{v}{\Norm{v}_V}\implies\Norm{\tilde v}_V = \frac{\Norm{v}_V}{\Norm{v}_V}=1
  \implies \Norm{L(\tilde v)}_W\leq \Norm{L}_{L(V,W)}\]
  \[\implies \Norm{\frac{1}{\Norm{v}_V}L(v)}_W=\frac{1}{\Norm{v}_V}\Norm{L(v)}_W\]
  \[\implies\frac{\Norm{L(v)}_W}{\Norm{v}_V}\leq \Norm{L}_{L(V,W)}\]
In der Tat,  $\Norm{L}_{L(V,W)}$ ist die {\em optimale Konstante} in \eqref{e:1103022}. D.h.,
f\"ur jede $C<\Norm{L}_{L(V,W)}$ $\exists v\in V$ mit $\|L(v)\|_W > C \|v\|_V$.
\end{Bem}
\begin{Kor}
Seien $V$ und $W$ zwei endlichdimensionierte Vektorr\"aume und $L:V\to W$ eine
lineare Abbildung. Dann $L$ ist stetig.
\end{Kor}

\begin{Bew}
  $\varepsilon-\delta$ Stetigkeit. $v,\varepsilon>0$. Suche $\delta>0$ mit
  \[\Norm{v'-v}_V<\delta\implies\Norm{L(v')-L(v)}_W<\varepsilon\]
  Linearität von $L$
  \[\implies \Norm{L(v')-L(v)}_W=\Norm{L(v'-v)}_W\]
  und aus \eqref{e:1103022}
  \[\Norm{L(v'-v)}\leq\underbrace{\Norm{L}_{L(V,W)}\overbrace{\Norm{v'-v}_V}^{<\delta}}_{<\varepsilon}\]
  \[\implies \delta=\frac{\varepsilon}{\Norm{L}_{L(V,W)}}\]
  $\implies$ Ungleichung erfüllt.
\end{Bew}
\begin{Bem} Seien
  $V=\mb{R}^n$ und $\Norm{.}_V$ die euklidische Norm, $W=\mb{R}^k$ und $\|\cdot\|_W$ die euklidische Norm.
Dann \eqref{e:CS2} ist einfach die folgende Aussage:  
\[\Norm{L}_{L(V,W)}\leq \Norm{L}_{\text{HS}}\]
In Matrixdarstellung:
  \[\Norm{L}_\text{HS}=\sqrt{\sum_{i,j}L_{ij}^2}\]
  \[\Norm{L}_{L(V,W)}:=\sup_{\sum^n_{i=1}v_i^2\leq 1}\sqrt{\sum^k_{j=1}\left( \sum_{i=1}^nL_{ji}v_i \right)^2}\, .\]
In diesem Fall wir nutzen die Notation $\|\cdot\|_O$ f\"ur die Operatornorm.
\end{Bem}
\subsection{Mehr über stetige Funktionen}
\paragraph{Regeln} für stetige Funktionen
\subparagraph{Regel 1}
Seien $f:X\to Y$, $g:X\to Z$ zwei stetige Funktionen ($X$, $Y$ und $Z$ topologische R\"aume).
Dann
  \begin{itemize}
\item falls $Y=Z$ ein normierter Vektorraum ist, $f+g$ ist auch stetig;
\item falls $Y$ ein normierter Vektorraum und $Z=\mb{R}$, $gf$ ist auch stetig;
\item falls $Y=Z=\mb{R}^n$ auch
\[ x\mapsto f(x)\cdot g(x)=\sum_{i=1}^nf_i(x)g_i(x)\]
ist stetig.
  \end{itemize}
\begin{Bew} Wir geben den Beweis
  f\"ur den Fall $X\subset \mb{R}^m$. Der allgemeine Fall lassen wir als eine \"ubung.
In diesem Fall k\"onnen wir die Folgendefinition der Stetigkeit anwenden.
  \[\underbrace{\left\{ x^k \right\}}_{\subset X} x^k\to x\in X\]
  Stetigkeit von $f$ und $g$: $g(x^k)\to g(x)$, $f(x^k)\to f(x)$.
  \[g(x^k)=(g_1(x^k),\cdots,g_m(x^k))\]
  \[g(x)=(g_1(x),\cdots,g_m(x))\]
  \[f(x^k)=(f_1(x^k),\cdots,f_m(x^k))\]
  \[f(x)=(f_1(x),\cdots,f_m(x))\]
  \[(g+f)(x^k)=\left( g_1(x^k)+f_1(x^k),\ldots,g_m(x^k)+f_m(x^k) \right)\] 
\[ \to (g_1 (x)+f_1 (x), \ldots, g_m (x) + f_m (x)) = (g+f) (x)\, .\]
 D.h.
  \[x^k\to x\in X\implies (f+g)(x^k)\to (f+g)(x).\]
DIe anderen Regeln folgen aus \"ahnlichen Argumente.
\end{Bew}
\subparagraph{Regel 2}
Seien $X,Y,Z$ topologische Räume. Seien $f:X\to Y$ und $g:Y\to Z$ stetig. Dann
\[g\circ f:\underbrace{X\to Z}_{x\mapsto g(f(x))}\]
ist stetig.
\begin{Bew}
  Sei $U$ eine offene Menge in $Z$.
  \[(g\circ f)^{-1}(U)=\underbrace{f^{-1}(\underbrace{g^{-1}(U)}_{\text{offen}})}_{\text{offen}}\]
\end{Bew}
\begin{Def}
  Sei $f:X\to \mb{R}$.
  \[\Norm{f}=\sup_{x\in X}\Norm{f(x)}\]
  $f:X\to V$, $V,\Norm{.}_V$ normierter Vektorraum
  \[\Norm{f}=\sup_{x\in X}\Norm{f(x)}_V\]
\end{Def}
\begin{Bem}
  $X$ Menge, $V,\Norm{.}$ ein normierter Vektorraum.
  \[F:=\left\{ f:X\to V\right\} \s\text{mit}\s\Norm{f}\]
  Dann ist $F,\Norm{.}$ ist ein normierter Vektorraum.
\end{Bem}
\begin{Def}
  Eine Folge von Funktionen
  \[f^k:X\to V\]
  konvergiert gleichmässig gegen $f$ falls
  \[\Norm{f^k-f}\to 0\]
\end{Def}
\begin{Bem}
  $x\in X$
  \[\Norm{f^k(x)-f(x)}_V\leq \Norm{f^k-f}\]
  Folgerung $f^k$ konvergiert gleichmässig
  \[\implies f^k(x)\to f(x)\s\forall x\]
\end{Bem}
\begin{Sat}
  Sei $X$ ein metrischer Raum und $f^k:X\to V$ eine Folge die gleichmässig gegen $f$ konvergiert. Dann ist $f$ stetig.
\end{Sat}
\begin{Bew}
  Seien $x\in X$ und $\varepsilon>0$. Wir suchen $\delta>0$ so dass
  \begin{equation}\label{e:ziel}
d(x,y)<\delta\implies\Norm{f(x)-f(y)}<\varepsilon\, .
\end{equation}
Aus der gleichm\"assigen Konvergenz folgt die Existenz von  $N$ so dass
  \[\Norm{f-f^k}<\frac{\varepsilon}{3}\s\text{falls}\s k\geq N\]
  $f^N$ ist stetig: $\exists \delta>0$:
  \[d(x,y)<\delta\implies \Norm{f^N(x)-f^N(y)}<\frac{\varepsilon}{3}\]
 Siene nun $x,y$ s.d. $d(x,y)<\delta$. Dann
  \[\Norm{f(x)-f(y)}=\Norm{(f(x)-f^N(x))+(f^N(x)-f^N(x))+(f^N(y)-f(y))}_V\]
  \[\leq\Norm{f(x)-f^N(x)}_V+\Norm{f^N(x)-f^N(y)}_V+\Norm{f^N(y)-f(y)}_V\]
  \[<\Norm{f^N-f}+\frac{\varepsilon}{3}+\Norm{f^N-f}\]
  \[<\frac{\varepsilon}{3}+\frac{\varepsilon}{3}+\frac{\varepsilon}{3}=\varepsilon\]
\end{Bew}
\subsection{Kompakte Menge}
\begin{Def}
  Eine Menge $K\subset \mb{R}^n$ heisst kompakt falls $K$ abgeschlossen und beschränkt ($\iff \exists B_R(0):K\subset B_R(0)$) ist.
\end{Def}
\begin{Sat}\label{s:KiffFK}
  Sei $K\subset\mb{R}^n$.
 \begin{equation}\label{e:KiffFK}
K \s\text{kompakt}\s \iff\forall \left\{ x^j \right\}\subset K\s\exists \mbox{ Teilfolge} x^{j_l}
\mbox{ die gegen $x\in K$ konvergiert.}
\end{equation}
\end{Sat}
Die Eingeschaft in der rechten Seite von \eqref{e:KiffFK} heisst {\em Folgenkompatkheit}.
Der Satz \ref{s:KiffFK} ist also die folgende Behauptung:
\[
\mbox{falls $K\subset \mb{R}^n$ dann }\quad \mbox{$K$ kompakt} \iff
\mbox{$K$ folgenkompakt.}
\]
\begin{Bew} {\bf Kompaktheit $\implies$ Folgenkompaktheit.}
Sei $K$ kompakt und $\left\{ x^j \right\}\subset K$ eine Folge.
  \[x^j\in K\subset B_R(0)\implies \Norm{x^j}<R\]
Aus der Bolzano-Weiertsrass Eigenschaft
  $\exists x^{j_l}\to x\in\mb{R}^n$. Die abgeschlossenheit von $K$ $\implies$ $x\in K$. 

\medskip

{\bf Folgenkompaktheit $\implies$ Abgeschlossenheit und Beschränktheit.}
  \[\text{$K$ nicht abgeschlossen}\implies \exists x^j\subset K\s\text{mit}\s x^j\to x \not\in K\]
  \[\text{Folgenkompaktheit}\implies \exists x^{j_l}\to y\in K\]
  Widerspruch (weil $x = y$!).

  Sei $K$ nicht beschränkt.
  \[\forall j\in\mb{N}\s B_j(0)\not\supset K\]
  \[\exists x^j\in K\setminus B_j(0)\implies \Norm{x^j}\geq j\]
  Wenn $x^{j_l}\to x$. Aber das impliziert dass $\{\|x^{j_l}\|\}$ eine beschr\"ankte Folge
ist. (Wir wiederlegen das Argumebnt:
  \[\Norm{x^{j_l}}\leq \Norm{x}+\Norm{x^{j_l}-x}\]
  \[\Norm{x}\leq\Norm{x^{j_l}}+\Norm{x-x^{j_l}}\]
  \[\Abs{\Norm{x}-\Norm{x^{j_l}}}\leq\Norm{x-x^{j_l}}\]
  \[\implies\Norm{x^{j_l}}\to\Norm{x})\]
Aber $\Norm{x^{j_l}}=j_l\to+\infty$ $\implies$ Widerspruch.
\end{Bew}

\begin{Sat}
  $E\subset \mb{R}^n$
  \[E\s\text{kompakt}\iff E\s\text{folgenkompakt}\]
  d.h.
  \[\forall \left\{ x_k \right\} \subset E\s\exists\s\text{Teilfolge}\s \left\{ x_{k_l} \right\}\s\text{die gegen $x\in E$ konvergiert}\]
\end{Sat}
\begin{Def}
  (Überdeckungseigenschaft) Eine Teilmenge $E\subset\mb{R}^n$ besitzt die Überdeckungseigenschaft falls:
  \begin{itemize}
    \item $\forall$ Überdeckung $\left\{ U_\lambda \right\}_{\lambda\in\Lambda}$ von $E$ mit offenen Mengen $\exists$ endliche Teilüberdeckung.
      \[\left\{ U_\lambda \right\}_{\lambda\in\Lambda}\s\text{Überdeckung}\iff \bigcup_{\lambda\in\Lambda} U_\lambda\supset E\]
  \end{itemize}
  Teilüberdeckung ist eine Teilfamilie von $\left\{ U_\lambda \right\}$ die noch eine Überdeckung von $E$ ist.
\end{Def}
\begin{Bsp}
  Eine offene Kugel hat diese Eigenschaft nicht.
  \[\forall x\in K_r(0)\s\text{sei}\s K_{\frac{r-\Norm{x}}{2}}(x)=U_x\]
  \begin{enumerate}
    \item $\left\{ U_x \right\}_{x\in K_r(0)}$ ist eine Überdeckung von $K_r(0)$.
  \end{enumerate}
  Einfach weil $x\in U_x$! Sei $U_{x_1},\cdots,U_{x_N}$ eine beliebige endliche Teilfamilie. Sei 
  \[p:=\max_{i\in\left\{ 1,\cdots,N \right\}}\Norm{x_i}<r\]
  $\implies$ falls $\Norm{y}\geq \frac{\Norm{x_i}+r}{2}$ dann $y\not\in U_{x_i}$. So, wenn $\Norm{y}\geq \frac{p+r}{2}$ dann
  \[y\not\in U_{x_1}\cup\cdots\cup U_{x_N}\s\frac{p+r}{2}<r\]
  falls $\Norm{y}=\frac{p+r}{2}$, dann $y\in K_r(0)$. Mit einer geschlossenen Kugel ist das anders.
\end{Bsp}
\begin{Sat}
  Sei $E\subset\mb{R}^n$
  \[E\s\text{kompakt}\iff E\s\text{hat die Überdeckungseigenschaft}\]
\end{Sat}
\begin{Bsp}
  $E=\mb{R}^n$, $U_n=K_{n+1}(0)$.
  \[E\subset \bigcup_{n\in\mb{N}} U_n\]
  Aber $\forall N\in\mb{N}$
  \[\mb{R}^n=E\not\subset \bigcup_{n=0} U_n\]
\end{Bsp}
\begin{Bew}
  $\exists \left\{ x_i \right\}\subset E$ ohne konvergente Teilfolge in $E$ $\implies$ $E$ ist nicht kompakt $\implies$ Überdeckungseigenschaft gilt nicht. Zwei Möglichkeiten:
  \begin{enumerate}
    \item $\exists$ eine Teilfolge $\left\{ y_i \right\}\subset E$ $y_i\to y$ $y\not\in E$
    \item $\exists$ eine Teilfolge $\left\{ y_i \right\}\subset E$ $y_i\to +\infty$
  \end{enumerate}
  Beim ersten ist die Menge offen. 
  \[U_0:=\mb{R}^n\setminus \underbrace{\left( \left\{ y_1 \right\}\cup\left\{ y \right\} \right)}_{E\s\text{ist abgeschlossen}}\]
  Beim zweiten gilt:
  \[U_0=\mb{R}^n\setminus\underbrace{\left\{ x_i \right\}}_{F}\s\text{ist offen}\]
  \[U_n=U_0\cup \left\{ y_1,\cdots,y_{n-1} \right\} \s n\geq 0\]
  $U_n$ ist auch offen.
  \[\bigcup_{n=0}^{\infty}U_n= \begin{cases}
    \mb{R}^n\setminus \left\{ y \right\}& \text{im Fall 1}\\
    \mb{R}^n\setminus & \text{im Fall 2}
  \end{cases}\]
  Aber jede endliche Familie
  \[U_1\cup \cdots\cup U_n\not\supset E\]
  in beiden Fällen lassen wir unendlich viele Punkte weg. $E$ kompakt $\implies$ Überdeckungseigenschaft. $E$ ist beschränkt und abgeschlossen und sei $\left\{ U_\lambda \right\}_{\lambda\in\Lambda}$ eine Familie von offenen Mengen mit $E\subset\left\{ U_\lambda \right\}_{\lambda\in\Lambda}$. Wir decken die Menge $U$ mit Würfel:
  \[\left[k_1,k_1+1\right]\times \left[ k_2,k_2+1 \right]\times \cdots\times \left[ k_n,k_n+1 \right]\]
  \[W_1\cup\cdots\cup W_M\]
  Falls jedes $E\cap W_i$ mit einer endlischen Familie von $\left\{ U_\lambda \right\}$ überdeckt wird, dann finde ich eine endliche Überdeckung von $E$ wenn $N$ gross genug ist. So, angenommen dass die Überdeckungseigenschaft nicht gilt.
  \[\exists E_i:= E\cap W_i:\]
  \begin{enumerate}
    \item $\left\{ U_\lambda \right\}_{\lambda\in\Lambda}$ eine Überdeckung von $E_1$
    \item keine endliche Teilfamilie deckt $E_1$
  \end{enumerate}
  Teilen wir $W_i$ in $2^n$ Würfel mit Seite $\frac{1}{2}$
  \[\tilde W_1,\cdots,\tilde W_2\]
  \[\exists E_2:= E\cap \tilde W_i:\s\text{so dass die beiden Eigenschaften noch gelten}\]
  Induktiv
  \[E\supset E_1\supset E_2\supset\cdots\]
  jede $E_i\subset W^i$ Würfel mit Seite $2^{-i+1}$ und die beiden Eigenschaften gelten mit $E_j$ statt $E_i$.\\
  $\left\{ x_k \right\}\subset E$. $\left\{ x_k \right\}$ ist eine Cauchy-Folge. $j,k>i$, $x_k,x_j\in W$ mit Seite $w^{-i+1}$ $\Norm{x_j-x_k}\leq \sqrt{n}2^{-i+1}$
  \[\implies x_j\to x\in E\to x\in U\in \left\{ U_\lambda \right\}_{\lambda\in\Lambda}\implies K_r(x)\supset U\]
  \[x\in E, x\in E^i\s\forall i\implies x\in W^i\]
  \[\implies W^i\subset B_r(x)\subset U\]
  für $i$ gross genug
  \[\implies E_i\subset U\]
  $\implies$ wir haben eine endliche Teilüberdeckung $\left\{ U \right\}\subset\left\{ U_\lambda \right\}$ gefunden $\implies$ Widerspruch mit den beiden Eigenschaften.
\end{Bew}
\begin{Bem}
  $f$ stetig $\implies$ $f^{-1}(U)$ offen falls $U$ offen.
\end{Bem}
\begin{Bew}
  Sei $\left\{ U_\lambda \right\}$ eine Überdeckung (mit offenen Mengen) von $f(E)$, dann ist $\left\{ f^{-1}\left( U_\lambda \right) \right\}$ ein Überdeckung von $E$.
  \[\exists f^{-1}(U_{\lambda_1}),\cdots,f^{-1}(U_{\lambda_N}\s\text{Teilüberdeckung von $E$}\]
  $U_{\lambda_i},\cdots,U_{\lambda_N}$ ist eine Überdeckung von $f(E)$ $\implies$ $f(E)$ ist kompakt
\end{Bew}
\begin{Kor}
  Wenn $F:E\to \mb{R}$ stetig ist und $E\subset\mb{R}^n$ kompakt ist, besitzt $f$ ein Maximum und ein Minimum.
\end{Kor}
\begin{Bew}
  $f(E)\subset\mb{R}$ ist kompakt.
  \[s=\sup f(E)<+\infty\]
  \[\exists \left\{ x_k \right\}\subset f(E)\s\text{mit}\s x_k\to s\xRightarrow{\text{abgeschlossen}}s\in s\in f(E)\]
  \[\left( s-\frac{1}{k}\implies \exists x_k\in f(E)\s\text{mit}\s x_k>s-\frac{1}{k},x_k\leq s \right)\]
  $\implies$ $s$ ist ein Maximum.
\end{Bew}
\begin{Def}
  Das Intervallschachtelungsprinzip in $\mb{R}$. Sei $I_j$ eine Intervallschachtelung:
  \begin{enumerate}
    \item \[I_j=\left[ a_j,b_j \right]\]
    \item \[I_0\supset I_1\supset \cdots \supset I_j\supset_{j+1}\]
    \item \[b_j-a_j\to 0\]
  \end{enumerate}
  \[\implies \bigcap^\infty_{j=0}E_j\neq\varnothing\]
\end{Def}
\begin{Sat}
  Sei $E_j$ eine Folge von kompakten Mengen mit $E_j\supset E_{j+1}$ $\forall j$ ($E_0\subset\mb{R}^n$)
  \[\bigcap_{j=1}^\infty E_j\neq\varnothing\s\text{falls}\s E_j\neq\varnothing \s\forall j\]
\end{Sat}
\begin{Bew}
  Sei $E_j$ wie im Satz mit $E_j\neq\varnothing$, aber $\bigcap_{j=0}^\infty E_j=\varnothing$. Sei $U_j:=\mb{R}^n\setminus E_j\implies U_j$ ist offen. $\bigcup_{j=1}^\infty U_j=\mb{R}^n$ $\left\{ U_j \right\}$ ist eine Überdeckung von $E_0$. Aber $U_1\cup\cdots\cup U_N=U_N$ (weil $U_{j+1}\supset U_j$)
  \[U_N\not\supset E_N\neq \varnothing\s E_N\subset E_0\]
  Keine endliche Teilfamilie von $\left\{ U_j \right\}$ ist eine Überdeckung von $E_0$. Widerspruch wegen Kompaktheit von $E_0$.
\end{Bew}

\subsection{Differenzierbare Funktionen}
\paragraph{Erinnerung} $f:\mb{R}\to \mb{R}$ heisst differenzierbar in $a\in \mb{R}$ falls
\[f'(a)=\Limo{h}\frac{f(a+h)-f(a)}{h}\]
existiert. Was geschieht mit Funktionen von mehrere Variablen? Die ``Tangentensteigung'' hängt auch von der Richtung ab. D.h. Es gibt eine lineare Abbildung $L:\mb{R}^2\to\mb{R}$
\begin{Def}
  $f:U\to\mb{R}$, $U\subset\mb{R}^n$ offen, heisst differenzierbar in $a\in U$, falls
  \begin{equation}
    \label{e:1103091}
    \Limo{h}\frac{f(a+h)-f(a)-Lh}{\Norm{h}}=0
  \end{equation}
  wobei $L:\mb{R}^n\to\mb{R}$ eine lineare Abbildung ist.
\end{Def}
\begin{Bem}
  $n=1$: Es gilt $Lh = f'(a)h$
\end{Bem}
\begin{Bem}
  Die lineare Abbilung $L$ in \ref{e:1103091} ist eindeutig definiert. Annahme $L'\neq L$ erfüllt die Bedungung. Sei $v\in\mb{R}^n$ mit $\Norm{v}=1$. Es gilt:
  \[(L-L')(v)\stackrel{\text{linear und}\s \Norm{v}=1}{=}\lim_{t\downarrow 0}\frac{(L-L')(tv)}{\Norm{tv}}\stackrel{\ref{e:1103091}\s h=tv}{=}\implies L=L'\]
\end{Bem}
\begin{Bem}
  Wir können \ref{e:1103091} auch anders beschreiben:
  \[f(a+h)-f(a)=Lh+\underbrace{R(h)}_{\text{Restglied}}\]
  Dann gilt
  \begin{equation}
    \label{e:1103092}
    \ref{e:1103091} \iff \Limo{h}\frac{R(h)}{\Norm{h}}=0
  \end{equation}
\end{Bem}
\begin{Def}
  $L$ heisst Differential von $f$ in $a$. Man schreibt $\md f(a)$. Sei nun $\left\{ e_1,\cdots,e_n \right\}$ die Standardbasis $\mb{R}^n$, $h=(h_1,\cdots,h_n)\in\mb{R}^n$
  \[\implies \md f(a)h=\md f(a)\left( \sum_{i=1}^kh_i-e_i \right)=\sum_{i=1}^nh_i\md f(a)e_i\]
\end{Def}
\begin{Def}
  \[f'(a)=(\md f(a)e_1,\cdots,\md f(a)e_n)\]
  heisst Ableitung
\end{Def}
\begin{Def}
  \[Tf(x,a)=f(a)+f'(a)(x-a)\s\text{(Ebene (tangential))}\]
  lineare Approximation
\end{Def}
\begin{Sat}
  $f$ differentierbar in $a$ $\implies$ $f$ ist stetig in $a$
\end{Sat}
\begin{Bew}
  \[\Abs{f(a+b)-f(a)}=\Abs{\md f(a)h+R(h)}\leq\Abs{\md f(a)}+\underbrace{\Abs{R(h)}}_{\to 0}\]
\end{Bew}
\begin{Bsp}
  $f(x)=Ax+b$, $A\in M_a(1,n,\mb{R})$, $b\in\mb{R}$
  \begin{Beh}
    $Lh:=ah$ ist linear
    \[\md f(a)h=Ah, \s f'(a)=A\]
  \end{Beh}
  \begin{Bew}
    \[f(a+h)-f(a)-Lh=\not{R(h)}=0\]
  \end{Bew}
\end{Bsp}
\begin{Bsp}
  $f(x):=x^TAx$, $A=(a_{ij})\in\Sym(n,\mb{R})$
  \[f(a+h)-f(a)-\underbrace{2a^TAh}_{\md f(a)h}+\underbrace{h^TAh}_{R(h)}\]
  $Lh:=2a^TAh$ ist linear (in $h$), $R(h)=h^TAh$ ($=\sum h_ia_{ik}h_l$)
  z.z.: $\Abs{Rh}\leq\sum^h_{i,j=1}\Abs{a_{ij}}\Norm{h}_{\infty}^2$, d.h. $\frac{R(h)}{\Norm{h}}\to 0$ (falls $\Norm{h}\to 0$)
\end{Bsp}
\paragraph{Ziel}
Wir wollen $\md f(a)h$ berechnen. sei $t\in\mb{R}$
\[f(a+th)=f(a)+\md f(a)th+R(th)\]
\begin{equation}
  \label{e:1103094}
  \implies \md f(a)h=\Limo{t}\frac{f(a+th)-f(a)}{t}
\end{equation}
\begin{Def}
  $f:U\to\mb{R}$, $a\in U$. Die Richtungsableitung von $f$ in Richtung $h\in\mb{R}^n$ ist der Grenzwert (falls er existiert)
  \[\partial_nf(a):=\Limo{t}\frac{f(a+th)-f(a)}{t}\]
  Die Ableitungen in Richtung $e_1,\cdots,e_n$ heissen partielle Ableitungen in $a$. Wir schreiben
  \[\partial_{ei}f(a)=\partial_if(a)=\frac{\partial f}{\partial x_i}(a)=f_{xi}(a)\]
\end{Def}
\begin{Bem}
  Wir haben \ul{nicht} vorausgesetzt, dass $f$ differenzierbar ist in $a$!
\end{Bem}
\begin{Sat}
  Sei $f$ in $a$ differenzierbar. Dann existieren die Richtungsableitungen in jede Richtung. Insbesondere existieren die aprtiellen Ableitungen. Es gelten:
  \begin{equation}
    \label{e:1103095}
    \md f(a)h=f'(a)h=\partial_nf(a)=\sum_{i=1}^n\partial_if(a)h_i
  \end{equation}
  und
  \[f'(a)=\left( \partial_1f(a),\cdots,\partial_nf(a) \right)\]
\end{Sat}
\begin{Bew}
  Existenz der Richtungsableitung oke (Herleitung von \ref{e:1103094})
\end{Bew}
\paragraph{Frage}
Wie berechnet man die partielle Ableitung effizient? Es gilt:
\[\partial_if(a)=\Limo{t}\frac{f(a+t_{ei})-f(a)}{t},\s a=(a_1,\cdots,a_n)\]
\[g_i(x):=f(a_1,\cdots,a_{i-1},x,a_{i+1},\cdots,a_n)\]
\[\partial_if(a)=\Limo{t}\frac{g(a_i+t)-f(a_i)}{t}=g'(a_i)\]
\begin{Bsp}
  \[f(x,y):=\sin(2x)e^{3y}\]
  \[\partial_xf=2e^{3y}\cos(2x)\]
  \[\partial_yf=\sin(2x)e^{3y}3\]
\end{Bsp}
\paragraph{Frage}
Wann folgt aus der Existenz der partiellen Ableitung (Richtungsableitung) die Differenzierbarkeit?
\begin{Bsp}
  \[f(x,y)= \begin{cases}
    \frac{x^2y}{x^2+y^2}&(x,y)\neq(0,0)\\
    0&(x,y)=(0,0)
  \end{cases}\]
  Es gilt: $f(tx,ty)=tf(x,y)$, d.h. der Graph von $f$ besteht aus Geraden durch $0$, für $h=(h_1,h_2)\in\mb{R}^2$
  \[\implies \partial_hf(0,0)=\Limo{t}\frac{f(th_1,th_2)-f(0,0)}{k}=\Limo{t}\frac{t}{t}f(h_1,h_2)=f(h_1,h_2)\]
  \[\implies \partial f(0,0)=f(h_1,h_2)\]
  \[\partial_{e_1}f(0,0)=f(1,0)=0\]
  \[\partial_{e_2}f(0,0)=f(0,1)=0\]
  \paragraph{Annahme}
  $f$ ist in $(0,0)$ differenzierbar
  \[\xRightarrow{\text{aus}\s\ref{e:1103095}}\underbrace{\partial_nf(0,0)}_{=\md f(a)h=0}=\underbrace{\partial_1f(a)}_{0}(h_1)+\underbrace{\partial_2f(a)}_0(h_2)=0\]
  \[\implies \md f(a)=0\]
  \paragraph{Test}
  $L=0$
  \[\frac{f(h_1,h_1)-\overbrace{f(a_0)-L(h_1,h_1)}}{\Norm{(h_1,h_1)}_\infty}=\frac{h_1^3}{2h_1^2\Abs{h_1}}\to\pm\frac{1}{2}\]
  $\implies$ $f$ ist in $(0,0)$ nicht differenzierbar.
\end{Bsp}

\subsubsection{Das Differenzial}
$f:\Omega\to\mb{R}$, $\Omega\subset\mb{R}^n$, Umgebung von $x$.
\[f\s\text{diff in $x$}\iff \exists L:\mb{R}^n\to\mb{R}\s\text{linear s.d.}\]
\begin{equation}
  \label{e:1103141}
  \lim_{h\downarrow 0}\frac{f(x+h)-f(x)-L(h)}{\Norm{h}}=0
\end{equation}
\[\lim_{h\downarrow 0}G(h)=0\iff \forall \varepsilon>0\exists \delta>0\s\Norm{h}<\delta\implies\abs{G(h)}<\varepsilon\]
\[\iff \forall h_k=0\s G(h_k)\to 0\]
Wenn $f$ differenzierbar ist und \ref{e:1103141} erfüllt, heisst $L$ das Differential von $f$.
\[L=\md f\]
\[\md f_x \s \text{das Differential an der Stelle $x$}\]
\subsubsection{Richtungsableitung}
$x\in\Omega$, $h\in\mb{R}^m$, $g(t)=f(x+th)$ (wohldefiniert für $\abs{t}$ klein)
\[\partial_n f(x)=g'(0)=\Limo{t}\frac{f(x+th)-f(x)}{t}\]
\subsubsection{Partielle Ableitung}
$(x_1,\cdots,x_n)$ Kond. in $\mb{R}^n$ $y\in \Omega$ so dass $\Omega$ eine Umgebung von $y$ ist
\[\Part{f}{x_i}(y)\left( =\partial_{x_i}f(y) \right)=\Limo{t}\frac{y_1,\dots,y_i+t,\dots,y_n-f(y)}{t}\]
Falls $e_i=(0,\dots,0,\underbrace{1}_i,0,\dots,0)$
\[=\Limo{t}\frac{f(y+te_i)-f(y)}{t}=\partial_{e_i}f(y)\]
\begin{Sat}
  (Hauptkriterium der Differenzierbarkeit) Sei $f:U\to \mb{R}$ und $U$ eine Umgebung von $y$. Falls $\Part{f}{x_1},\dots,\Part{f}{x_n}$ \ul{in $U$} existieren und stetig in \ul{in $y$} sind, dann ist $f$ in $y$ differenzierbar.
\end{Sat}
\begin{Bew}
  $h=(h_1,\dots,h_n)\in\mb{R}^n$
  \[L(h)=\sum^n_{i=1}\Part{f}{x_i}(y)h_i\]
  \paragraph{Ziel} $L$ ist das Differential von $f$
  \[\Limo{h}\frac{f(x+h)-f(x)-L(h)}{\Norm{h}}=0\]
  \[f(x+h)-f(x)=f(x+(h_1,\dots,h_n))-f(y+(h_1,\dots,h_{n-1},0)+f(y+(h_1,\dots,h_{n-1}, 0)-\dots\]
  \[+\dots\s(\text{$i$te Zeile})\]
  \begin{equation}
    \label{e:1103143}
    +f(y+(k,0,\dots,0))-f(y)
  \end{equation}
  $i\in\left\{ 1,\dots,n \right\}$
  \[g(t))=f(y+(h_1,\dots,h_{i-1},th_i,0,\dots,0)\]
  \[\text{$i$te Zeile}=g_i(1)-g_i(0)=g_i'(\xi_i)\s\xi\in \left[ 0,1 \right]\]
  \[g_i'(t)=\Limo{\varepsilon}\frac{g_i(t+\varepsilon)-g_i(t)}{\varepsilon}\]
  \[=h_i\Limo{\varepsilon}\frac{f(y_1+h_1,\dots,y_{i-1},y_i+(t+\varepsilon)h_i,y_{i+1}\dots,y_n)-f(y_1+h_1,\dots,y_i+th_i,\dots,y_n}{\varepsilon h_i}\]
  \[=h_i\Part{f}{x_i}\left( y_1+h_i,\dots,y_i+th_1,y_{i+1},\dots,y_n \right)\]
  \[\text{$i$te Zeile}=h_i\Part{f}{x_i}(y_1+h_1,\dots,y_{i-1}h_{i-1},y_i+\xi_ih_i,y_{i+1},\dots,y_n)\]
  \[\zeta_i=\left( h_1,\dots,h_{i-1},\xi h_i,0,\dots,0 \right)\]
  \begin{equation}
    \label{e:1103144}
    =h_i\Part{f}{x_i}(y+\zeta_i)
  \end{equation}
  \ref{e:1103144} in \ref{e:1103143}:
  \begin{equation}
    \label{e:1103145}
    f(y+h)-f(y)=\sum_{i=1}^nh_i\Part{f}{x_i}(y+\zeta_i)
  \end{equation}
  \[f(x+h)-f(x)-L(h)\]
  \begin{equation}
    \label{e:1103146}
    =\sum_{i=1}^nh_i\left( \Part{f}{x_i}(y+\zeta_i)-\Part{f}{x_i}(y) \right)
  \end{equation}
  \[\frac{\abs{f(x+h)-f(x)-L(h)}}{\Norm{h}}\]
  \begin{equation}
    \label{e:1103147}
    \stackrel{\ref{e:1103146}}{\leq}\sum_{i=1}^n\frac{\abs{h_i}\abs{\Part{f}{x_i}(y+\zeta_i)-\Part{f}{x_i}(y)}}{\Norm{h}}
  \end{equation}
  Wenn $\Norm{h}\to 0$, $\Norm{\zeta}\to 0$. Die Stetigkeit von $\Part{f}{x_i}$ in $y$ impliziert
  \[\Part{f}{x_i}(y+\zeta_i)\to\Part{f}{x_i}\]
  Die rechte Seite von \ref{e:1103147} $\to 0$ wenn $h\to 0$ $\implies$ \ref{e:1103142}.
\end{Bew}
\begin{Def}
  Der Gradient an der Stelle $x_0$ist der Vektor
  \[\left( \Part{f}{x_1}(x_0),\dots,\Part{f}{x_i}(x_0) \right)=\nabla f(x_0)\]
\end{Def}
\begin{Bem}
  \[df|_{x_0}(h)\left( \partial_nf(x_0) \right)=\sum_{i=1}^nh_i\Part{f}{x_i}(x_0)\]
  \[\left(\seq{\nabla f(x_0), h}\right)=\nabla f(x_0)h\]
  \[\abs{\partial_nf(x_0)}\stackrel{\text{Cauchy-Schwartz}}{\leq}\Norm{\nabla f(x_0)}\Norm{h}\]
  Falls $\Norm{h}=1$, dann
  \[\abs{\partial_nf(x_0)}\leq\Norm{\nabla f(x_0)}\]
  Fall $\Norm{\nabla f(x_0)}\neq 0$, wenn wir 
  \[K=\frac{\nabla f(x_0)}{\Norm{\nabla f(x_0)}}\]
  bekommen wir $\Norm{K}=1$ und
  \[\partial_Kf(x_0)=\Norm{\nabla f(x_0)}\]
  Deswegen:
  \[K=\frac{\nabla f(x_0)}{\Norm{\nabla f(x_0)}}\]
  ist die Richtung der maximalen Steigung und
  \[\Norm{\nabla f(x_0)}\]
  ist die maximale Steigung.
\end{Bem}
\subsection{Rechenregeln}
\begin{Sat}
  Sei $U$ eine Umgebung von $x\in\mb{R}^n$ und $f,g:U\to\mb{R}$ in $x$ differenzierbar. Dann sind $f+g$ und $fg$ auch differenzierbar in $x$ und
  \[\md (f+g)|_x=\md f|_x+\md g|x\]
  \[\md(fg)=f(x)\md g|x+g(x)\md f|_x\]
  Falls $f(x)\neq 0$ ist auch $\frac{1}{f}$ in $x$ differenzierbar
  \[\md \left( \frac{1}{f} \right)|_x=-\frac{1}{(f(x))^2}\md f|_x\]
\end{Sat}
\begin{Kor}
  $g(x)\neq 0$, dann
  \[\md \left( \frac{f}{g} \right)|_x=\frac{1}{g(x)}\md f|_x-\frac{f(x)}{g(x)^2}\md g|_x\]
  \[=\frac{g(x)\md f|_x-f(x)\md g|_x}{g(x)^2}\]
\end{Kor}
\begin{Bew}
  Das Ziel ist eine lineare Abbildung $L$ zu finden so dass
  \[\Limo{h}\frac{\frac{1}{f(x+h)-\frac{1}{f(x)}-L(h)}}{\Norm{h}}\]
  \[L=-\frac{1}{f(x)^2}\md f|_x\]
  \[\Limo{h}\frac{\overbrace{\frac{1}{f(x+h)-\frac{1}{f(x)}-\frac{1}{f(x)^2}(h)\md f|_x(h)}}^A}{\Norm{h}}=\frac{B+C}{\Norm{h}}\]
  \[\frac{1}{f(x+h)}-\frac{1}{f(x)}=\frac{f(x)-f(x+h)}{f(x)f(x+h)}\]
  $f(x+h)\neq 0$ falls $\Norm{h}$ klein genug
  \[\frac{f(x+h)-f(x)-\md f|_x(h)}{\Norm{h}}\to 0\]
  \[A=\left[ \frac{-(-f(x)+f(x+h))}{f(x)f(x+h)} \frac{\md f|_x(h)}{f(x)f(x+h)}\right]=C\]
  \[+\frac{-\md f|_x(h)}{f(x)f(x+h)} + \frac{\md f|_x(h)}{f(x)^2}=B\]
  \[\frac{B}{\Norm{h}}=-\frac{1}{f(x)f(x+h)}\underbrace{\frac{f(x+h)-f(x)-\md f|_x(h)}{\Norm{h}}}_{\to 0}\]
  \[\Limo{h} f(x+h)=f(x)\neq 0\]
  \[\Limo{h}\frac{B}{\Norm{h}}=0\]
  Diff von $f$ für $\Norm{h}\to 0$
  \[\frac{C}{\Norm{h}}=\underbrace{\frac{\md f|_x(h)}{\Norm{h}}}_{\text{ist beschränkt}}\frac{1}{f(x)}\underbrace{\left( \frac{1}{f(x)}-\frac{1}{f(x+h)} \right)}_{\to 0}\]
  Sei $L=\md f|_x$ und $\Norm{L}_O$ ihre Operatornorm
  \[\abs{\md f|_x(h)}=\abs{L(h)}\leq\Norm{K}_O\Norm{h}\]
  \[\implies \frac{\abs{\md f|_x(h)}}{\Norm{h}}\leq\Norm{L}\]
\end{Bew}

\begin{Def}
  Eine Kurve ist eine Abbildung $\gamma:[a,b]\to\mb{R}^n$ (d.h. $\forall t\s\gamma(t)\in\mb{R}^n$ 
  \[\gamma(t)=(\gamma_1(t),\cdots,\gamma_n(t))\]
  deswegen $t\to\gamma_i(t)\in\mb{R}$. Die Kurve $\gamma$ heisst differenzierbar wenn jede $\gamma_i$ differenzierbar ist.
  \[\gamma'=(\gamma'(t),\cdots,\gamma_n'(t))\]
\end{Def}
\begin{Sat}(Kettenregel 1. Version) Sei $f:U\to\mb{R}$ mit $U$ Umgebung von $x$ und $f$ differenzierbar in $x$. Sei $\gamma:[a,b]\to U$ eine differenzierbare Kurve mit $\gamma(t_0)=x$. Sei $g=f\circ \gamma$
  \[g(t)=f(\gamma(t))\]
  Sei $g$ in $t_0$ differenzierbar. Dann
  \[g'(t_0)=\md|_\gamma(t_0)(\dot\gamma(t_0))=\seq{\nabla f(\gamma(t_0)),\dot\gamma(t_0)}\]
  % TODO
\end{Sat}
\begin{Bew}
  Das Ziel:
  \[\Limo{h} \frac{g(t_0+h)-g(t_0)-h\left[ \md f|_{\gamma(t_0)}(\dot\gamma(t_0)) \right]}{h}=0\]
  \begin{equation}
    \label{e:1103161}
    R(h)=g(t_0+h)-g(t_0)-g(t_0)-h\left[ \md f|_{\gamma(t_0)}(\dot\gamma(t_0)) \right]
  \end{equation}
  \begin{equation}
    \label{e:1103162}
    \Limo{h}\frac{R(h)}{h}=0
  \end{equation}
  Neue Notation
  \[\ref{e:1103162}\iff R(h)=o(h)\]
  \[x_0=\gamma(t_0)\]
  Annahmen: Differenzierbarkeit von $f$
  \[\Limo{k}\frac{f(x_0+k)-f(x_0)-\md f|_{x_0}(k)}{\Norm{k}}\left( =\frac{r(k)}{\Norm{k}} \right)=0\]
  \[\left( r(k)=o(\Norm{k}) \right)\]
  Differenzierbarkeit von $\gamma$:
  \[\Limo{k}\frac{\gamma(x_0+k)-\gamma(x_0)-\md h\gamma'|_{x_0}(k)}{h}\left( =\frac{p(k)}{\Norm{k}} \right)=0\]
  \[p(h)=o(h)\]
  \[\gamma(t_0+h)=\gamma(t_0)+k\left( =\gamma(t_0+h)-\underbrace{\gamma(t_0)}_{x_0} \right)\]
  \[g(t_0+h)-g(t_0)=f(\gamma(t_0+h))-g(\overbrace{\gamma(t_0)}^{x_0})\]
  \[=f(\gamma(t_0)-k)-f(\gamma(t_0))=\md f|_{\gamma(t_0)}(k)+r(k)\]
  \[=\md f|_{\gamma(t_0)}(\gamma(t_0+h)-\gamma(t_0))+r(k)\]
  \[=\md f|_{\gamma(t_0)}(h\dot\gamma(t_0)+p(h))+r(k)\]
  \[\stackrel{\text{Linearität von $\md f$}}{=}h\md f|_{\gamma(t_0)}(\dot\gamma(t_0))+\md f|_{\gamma(t_0)}(p(h))+r(k)\]
  \[g(t_0+h)-g(t_0)-h\md f|_{\gamma(t_0)}(\dot\gamma(t_0))\]
  \[=f|_{\gamma(t_0)}(p(h))+r(\gamma(t_0+h)-\gamma(t_0))=R(h)\]
  \[\abs{R(h)}\leq \frac{\underbrace{\overbrace{\abs{f|_{\gamma(t_0)}(p(h))}}^{L}+r(\gamma(t_0+h)-\gamma(t_0))}}{\Norm{h}}\]
  \[\leq\Norm{L}\frac{p(h)}{\Norm{h}}+\frac{r(\gamma(t_0+h)-\gamma(t_0)}{\Norm{h}}\]
  Ziel
  \[\Limo{h}\frac{r(\gamma(t_0+h)-\gamma(t_0)}{\abs{h}}\]
  Falls 
  \[r(\gamma(t_0+h)-\gamma(t_0)=0\]
  dann $r(0)=0$. Wenn 
  \[r(\gamma(t_0+h)-\gamma(t_0)\neq 0\]
  \[=\frac{r(\gamma(t_0+h)-\gamma(t_0)}{\Norm{\gamma(t_0+h)-\gamma(t_0)}}\frac{\Norm{t_0+h)-\gamma(t_0)}}{\abs{h}}\]
  \[\frac{r(\gamma(t_0+h)-\gamma(t_0)}{\Norm{\gamma(t_0+h)-\gamma(t_0)}}=\frac{r(k)}{\Norm{k}}\to 0\]
  \ldots wenn $\Norm{k}\to 0$ und $h\to 0$. Es fehlt die Beschränktheit von
  \[\frac{\Norm{t_0+h)-\gamma(t_0)}}{\abs{h}}\]
  \[\frac{t_0+h)-\gamma(t_0)}{h}-\frac{\not h\dot \gamma(t_0)}{\not h}=\frac{p(h)}{h}\]
  \[\frac{\gamma(t_0+h)-\gamma(t_0)}{h}=\underbrace{\dot\gamma(t_0)}_{\text{konstant}}+\underbrace{\frac{p(h)}{h}}_{\to 0}\]
  Deswegen
  \[\Limo{h}\frac{\Norm{\gamma(t_0+h)-\gamma(t_0)}}{\abs{h}}=\Norm{\dot\gamma(t_0)}\]
  \[\implies \frac{\abs{R(h)}}{\Norm{h}}\to 0\]
  $\implies$ Differenzierbarkeit und Kettenregel!
\end{Bew}
\begin{Bem}
  Der Gradient ist orthogonal zur Niveaumenge (Höhenlinien).
\end{Bem}
\begin{Def}
  Sei $\gamma:[a,b]\to U$ eine differenzierbare Kurve, $U$ offen. Sei $f:U\to\mb{R}$ differenzierbar. Wenn $f(\gamma(t))=c_0$ ($c_0$ hängt nicht von $t$ ab). Dann
  \[\nabla f(\gamma(t))\bot \dot\gamma(t)\]
  d.h.
  \[\seq{\nabla f(\gamma(t)), \dot\gamma(t) }=0\]
  \[0=g'(t)=(f(\gamma(t)))'\stackrel{\text{Kettenregel}}{=}\seq{\nabla f(\gamma(t)), \dot\gamma(t)}\]
\end{Def}
\subsection{Mittelwertsatz und Schrankensatz}
  $f:[a,b]\to\mb{R}$, $\xi\in ]a,b[$
  \[f(b)-f(a)=f'(\xi)(b-a)\]
  Sei nun:
  \[f:U\mapsto\mb{R}\s\text{differenzierbar auf $U$}\]
  \[x,y\in U\s\text{so dass das Segment}\s[x,y]\subset U\]
  Was ist ein Segment? Gerade durch $x$ und $y$
  \[\left\{ x+t(y-x)|t\in \mb{R} \right\}\]
  \[\left[ [x,y] \right]=\left\{ x+t(y-x)|t\in \left[ 0,1 \right] \right\}\]
  \[\gamma(t):= x+t(y-x)\]
  \[f(y)-f(X)=\left( x_1+t(y_1-x_1),\cdots,x_n+t(y_n-x_n) \right)\]
  $\gamma$ ist differenzierbar.
  \[g=f\circ \gamma g(t)=f(\gamma(t))\]
  \[g(1)-g(0)=g'(\tau)\s\text{für}\s \tau\in ]0,1[\]
  \[f(y)-f(x)=\md f|_{\gamma(\tau)}(\dot\gamma(\tau))\]
  \[\dot\gamma(\tau)=(\gamma_1'(\tau),\cdots,\gamma_n'(\tau))\]
  \[=(y_1-x_1,\cdots,y_n-x_n)=y-x\]
  \[\gamma(\tau)=\xi\]
  \begin{equation}
    \label{e:1103163}
    f(y)-f(x)=\md f|_\xi(y-x)=\partial_{y-x}f(\xi)
  \end{equation}
\begin{Sat}
  (Mittelwertsatz) $U$ offen, $[x,y]\subset U$ und $f:U\to\mb{R}$ differenzierbar. Dann $\exists \xi\in ]x,y[$ so das \ref{e:1103163} gilt.
\end{Sat}
\begin{Def}
  $U$ sternförmig: wenn $0\in U$ und $[x,0]\subset U$ $\forall x\in U$. Sternförmig mit Zentrum $x_0$ wenn $x_0\in U$ $[x,x_0]\subset U$ $\forall x\in U$
\end{Def}
\begin{Sat}
  (Schrankensatz) Sei $U$ eine offene Menge, die sternförmig ist und $f:U\to\mb{R}$ eine differenzierbare Funktion mit
  \[\sup_{x\in U}\Norm{\md f|_x}_{O}=S<\infty \left( =\sup_{x\in U}\Norm{\nabla f(x)} \right)\]
  Dann
  \[\abs{f(x)-f(0)}\leq S\Norm{x}\]
  Wenn $U$ konvex ist, d.h. das Segment $[x,y]\subset U$ $\forall x,y\in U$, dann
  \[\abs{f(x)-f(y)}\leq S\Norm{y-x}\]
\end{Sat}
\begin{Def}
  $f:\underbrace{K}_{\in \mb{R}^n}\to\mb{R}$ heisst Lipschitz wenn $\exists L[0,+\infty[$ so dass
  \[\abs{f(y)-f(x)}\leq L\Norm{y-x}\s\forall x,y\in K\]
  Wenn $f:(X,d)\to\mb{R}$ Lipschitz bedeutet die Existenz eines $L$ so dass
  \[\abs{f(y)-f(x)}\leq L d(y-x)\s\forall x,y\in K\]
\end{Def}


\newpage

%= Stichwortverzeichnis ======================================================================
\rhead{}
\addcontentsline{toc}{section}{Stichwortverzeichnis}
\printindex

\end{document}

