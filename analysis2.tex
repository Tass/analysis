% headers by Alexander Berthold van der Bourg / Pirmin Weigele 

%= Document-Class ==================================================================================
\documentclass[10pt,a4paper]{article}

%= Packages ========================================================================================
\usepackage[utf8x]{inputenc}
\usepackage{ngerman,amsmath,amssymb,amsfonts,mathrsfs}
\usepackage{amsthm}
\usepackage{bbm}
\usepackage{epic,eepic,pstricks,pst-node,pst-plot}
\usepackage{pstricks}
\usepackage{colortbl}
\usepackage{graphicx}
\usepackage{makeidx}
\usepackage{fancyhdr}
\usepackage{latexsym}
\usepackage{psfrag}
\usepackage{enumerate}
\usepackage{float}
\usepackage{dsfont}
\pagestyle{fancy}
\usepackage{multirow, bigdelim, bigstrut}
\usepackage{rotating}
\usepackage{ifthen}
\usepackage{boxedminipage}
\usepackage{mathtools}
\usepackage{trfsigns}
\usepackage{url}
%\usepackage{savetrees}

%= Seiten-Layout =========================================================================
\voffset-22mm \textheight715pt 

%Seitenbreite==============================================================

%\oddsidemargin=-0.2in
%\evensidemargin=-0.4in
%\textwidth=5.2in
%\headwidth=5.2in

%= Index-Befehle ========================================================================
\renewcommand{\indexname}{Stichwortverzeichnis}
\makeindex

%= Befehl-Overwriting =======================================================================
\makeatletter
\makeatother

%= Strings ================================================================
\newcommand{\mainfold}{.}
\newcommand{\prefix}{A1-}

%= Eigene Befehle ==========================================================================
\DeclareMathOperator{\id}{Id}
\DeclareMathOperator{\arccot}{arccot}
\DeclareMathOperator{\arcsinh}{arcsinh}
\DeclareMathOperator{\arccosh}{arccosh}
\DeclareMathOperator{\arctanh}{arctanh}
\DeclareMathOperator{\md}{d}
\DeclareMathOperator{\Grad}{grad}
\DeclareMathOperator{\Spur}{Spur}
\DeclareMathOperator{\Graph}{Graph}
\DeclareMathOperator{\sign}{sign}
\DeclareMathOperator{\Hom}{Hom}
\DeclareMathOperator{\rot}{rot}
\DeclareMathOperator{\Ker}{Ker}
\DeclareMathOperator{\Exp}{Exp}

\newcommand{\Diff}[2]{\displaystyle\frac{\mathrm{d}#1}{\mathrm{d}#2}}
\newcommand{\End}{\hfill{\hbox{$\Box$}}\par\vspace{2mm}}
\newcommand{\eps}{\varepsilon}
\newcommand{\ePic}[1]{\input{\mainfold/graphics/\prefix#1.eepic}}
\newcommand{\pst}[1]{\input{\mainfold/graphics/\prefix#1.pst}}
\newcommand{\pic}[1]{\input{\mainfold/graphics/\prefix#1.pic}}
\newcommand{\Mx}[1]{\begin{pmatrix}#1\end{pmatrix}}
%\newcommand{\im}[1]{\operatorname{Im}(#1)}
%\newcommand{\Include}[4]{\rhead{#2.#3.20#4}\input{\mainfold/lectures/#1-#4-#3-#2.tex}}
\newcommand{\Index}[1]{\emph{#1}\index{#1}}
\newcommand{\Int}[4]{\displaystyle\int\limits_{#1}^{#2}#3\,\mathrm{d}#4}
\newcommand{\diff}[1]{\operatorname{d}\!#1}
\newcommand{\Limi}[1]{\displaystyle\lim_{#1\rightarrow\infty}}
\newcommand{\Limo}[1]{\displaystyle\lim_{#1\rightarrow0}}
\newcommand{\Limu}[2]{\displaystyle\lim_{#1\uparrow #2}}
\newcommand{\Limd}[2]{\displaystyle\lim_{#1\downarrow #2}}
\newcommand{\Lim}[2]{\displaystyle\lim_{#1\rightarrow#2}}
\newcommand{\mb}[1]{\mathbb{#1}}
\newcommand{\ds}{\displaystyle}
\newcommand{\ol}[1]{\overline{#1}}
\newcommand{\Part}[2]{\dfrac{\partial #1}{\partial #2}}
\newcommand{\QED}{\hfill{\hbox{(QED)}}\par\vspace{2mm}}
\newcommand{\re}[1]{\operatorname{Re}(#1)}
\newcommand{\s}{\hspace{2mm}}
\newcommand{\vsa}{\vspace{1mm} \\}
\newcommand{\vsb}{\vspace{2mm} \\}
\newcommand{\vsc}{\vspace{3mm} \\}
% \newcommand{\tr}[1]{\textrm{#1}}
\newcommand{\tr}[1]{\text{#1}}
\newcommand{\ra}{\rightarrow}
\newcommand{\Ra}{\Rightarrow}
\newcommand{\Lra}{\Leftrightarrow}
\newcommand{\La}{\Leftarrow}
\newcommand{\ul}[1]{\underline{#1}}
\newcommand{\rsa}{\rightsquigarrow}
\newcommand{\ara}[2]{\autorightarrow{\ensuremath{#1}}{\ensuremath{#2}}}
\newcommand{\dcp}[2]{\begindc{\commdiag}[#1] #2 \enddc}

%\newcommand{\detmx}{\left| \begin{array} #1 \end{array} \right|}

\newcommand{\grad}[1]{\Grad(#1)}
\newcommand{\fr}[2]{\displaystyle\frac{#1}{#2}} % fertiger bullshit, daf�r gibts \dfrac{}{}
\renewcommand{\Re}{\operatorname{Re}}
\renewcommand{\Im}{\operatorname{Im}}

% ---- DELIMITER PAIRS ----
\def\floor#1{\lfloor #1 \rfloor}
\def\ceil#1{\lceil #1 \rceil}
\def\seq#1{\langle #1 \rangle}
\def\set#1{\{ #1 \}}
\def\abs#1{\mathopen| #1 \mathclose|}	% use instead of $|x|$ 
\def\norm#1{\mathopen\| #1 \mathclose\|}% use instead of $\|x\|$ 

% --- Self-scaling delmiter pairs ---
\def\Floor#1{\left\lfloor #1 \right\rfloor}
\def\Ceil#1{\left\lceil #1 \right\rceil}
\def\Seq#1{\left\langle #1 \right\rangle}
\def\Set#1{\left\{ #1 \right\}}
\def\Abs#1{\left| #1 \right|}
\def\Norm#1{\left\| #1 \right\|}

%Adrians Abbildungs-Environment ==============================================

\newcommand{\Sidein}{\begin{rotate}{90}\small$\in$\end{rotate}}

\newcommand{\Abb}[5][]{\ensuremath{
    \begin{array}{lc}
      \ifthenelse{\equal{#1}{}}{}{#1:}\;\; & 
      \begin{xy}
        \xymatrixrowsep{1em}\xymatrixcolsep{2em}%
        \xymatrix{ #2 \ar[r] \ar@{}[d]^<<<<{\hspace{0.001em} \Sidein}
          & #3  \ar@{}[d]^<<<<{\hspace{0.001em} \Sidein} \\
          #4 \ar@{|->}[r] & #5} \end{xy}
    \end{array}
  }%
}

%= Environments ========================================================================
\def\thechapter{\Roman{chapter}}
\def\thesection{\arabic{section}}
\newtheorem{theorem}{Theorem}[section]
\newenvironment{Bew}{\begin{proof}[Beweis]}{\end{proof}}
\newtheorem{Axi}[theorem]{Axiom}
\newtheorem{Lem}[theorem]{Lemma}
\newtheorem{Kor}[theorem]{Korollar}
\newtheorem{Sat}[theorem]{Satz}
\newtheorem{Prop}[theorem]{Proposition}
\newtheorem{Beh}[theorem]{Behauptung}
\theoremstyle{definition}
\newtheorem{Bsp}[theorem]{Beispiel}
\newtheorem{Def}[theorem]{Definition}
\newtheorem{Ueb}[theorem]{\"Ubung}
\theoremstyle{remark}
\newtheorem{Bem}[theorem]{Bemerkung}
\newtheorem{Eig}[theorem]{Eigenschaften}
\newtheorem{Not}[theorem]{Notation}

\def\pstexInput#1{%
  \begin{center}
    \begin{picture}(0,0)%
      \special{psfile=\mainfold/graphics/A2-#1.pstex}%
    \end{picture}%
    \input{\mainfold/graphics/A2-#1.pstex_t}%
  \end{center}
}

%= Titelseite ===========================================================================
\begin{document}
\headheight15pt
\begin{titlepage}
\hfill
\vspace{20mm}
\pagenumbering{roman}
\begin{center}
{\LARGE Analysis II - Vorlesungs-Script} \vskip 3em {\large Prof. Dr. Camillo De Lellis} \vskip 1.5em
{\large Basisjahr 11 Semester I}\vspace{30mm}\\
{\large {\bf Mitschrift:} \vspace{2mm}\\
Simon Hafner}\vspace{5mm}\\ %30mm
%{\large {\bf Graphics:} \vspace{2mm}\\
%Pirmin Weigele }\vspace{30mm}\\ %30mm
\author{Simon Hafner}

\end{center}
\vfill

\end{titlepage}


%= Inhaltsverzeichnis ==========================================================================
\lhead{}
\rhead{}
\tableofcontents
\newpage
\pagenumbering{arabic}
\setcounter{page}{1}

%= Vorlesung-Skripts ==========================================================================
\cfoot{\thepage}
\fancyhead[L]{\nouppercase{\leftmark}}
\newpage

%Analysis II
\section{Metrik und Topologie des euklidischen Raumes}
$\mb{R}^n=\left\{ \left( x_1,\cdots,x_n \right), x\in\mb{R} \right\}$.
Wir f\"uhren verschiedene neue Begriffe in $\mb{R}^n$ ein:
\begin{itemize}
  \item die Euklidische Norm
  \item der Euklidische Abstand
  \item die entsprechende Topologie.
\end{itemize}
Wir betrachten gleichzeitig die entsprechenden Verallgemeinerungen,
d.h. die ``Abstrakte Theorien'' der
\begin{itemize}
  \item Normierten Vektorr\"aume
  \item Metrischen R\"aume
  \item Topologischen R\"aume.
\end{itemize}
\begin{Def}
  Sei $x\in\mb{R}^n$ ($x=(x_1,\cdots,x_n)$, $x_i\in\mb{R}$). Die Euklidische Norm
von $x$ ist 
  \[\Norm{x}_e=\sqrt{x_1^2+\cdots+x_n^2}=\sqrt{\sum_{i=1}^nx_i^2}\]
(wir schreiben oft $\|x\|$ anstatt $\|x\|_e$).
\end{Def}

Intuitiv: $\Norm{x}=$''der Abstand zwischen $x$ und 0``.  In der Tat, wenn $n=2$,
das Pytaghoras Theorem zeigt dass $\|x\|_e$ die L\"ange des Segments mit Extrema
$x$ and $0$ ist. 

\begin{Lem}\label{l:norm}
  $\Norm{.}$ erf\"ullt die Regeln
  \begin{enumerate}
    \item $\Norm{x}\geq 0$ und $\Norm{x}=0\iff x=0$
    \item $\Norm{\lambda x}=\Abs{\lambda}\Norm{x}$ $\forall \lambda\in\mb{R}$, $\forall x\in\mb{R}$
    \item $\Norm{x+y}\leq\Norm{x}+\Norm{y}$ $\forall x,y\in\mb{R}$
  \end{enumerate}
\end{Lem}
\begin{Bew}
  \begin{enumerate}
    \item $\geq 0$ trivial
      \[x=0\implies \sum x_i^2=0\implies \Norm{x}=0\]
      \[x=0\Leftarrow x_i=0\;\; \forall i\Leftarrow \sum x_i^2=0\Leftarrow \Norm{x}=0\]
    \item \[\Norm{\lambda x}=\sqrt{\sum^n_{i=1}(\lambda x_i)^2} = \sqrt{\lambda^2\sum x^2}=\Abs{\lambda}\sqrt{\sum x^2}=\Abs{\lambda}\Norm{x}\]
%\[\Abs{\lambda}=\frac{\Norm{x}\Abs{\lambda}}{\Norm{x}}\]
    \item Diese Aussage ist \"aquivalent zu
      \[\iff \underbrace{\Norm{x+y}^2}\leq \Norm{x}^2+\Norm{y}^2+2\Norm{x}\Norm{y}\]
Wir rechnen
      \[\sum_{i=1}^n(x_i+y_i)^2=\sum_{i=1}^n\left( x_i^2+y_i^2+2x_iy_i \right)=\Norm{x}^2+\Norm{y}^2\overbrace{2\sum_i x_iy_i}^{Skalarprodukt}\]
Wir definieren 
\[\langle x,y\rangle := \sum_{i=1}^n  x_iy_i\]
Wir brauchen dann die ber\"uhmte Cauchy-Schwartz Ungleichung, d.h.
      \[\langle x,y\rangle\leq \Norm{x}\Norm{y}\, .\]
Diese Ungleichung ist der Inhalt des n\"achsten Satzes.
  \end{enumerate}
\end{Bew}
\begin{Sat}{Cauchy-Schwartzsche Ungleichung}
  \[\sum^n_{i=1}x_iy_i\leq\sqrt{\sum_{i=1}^nx_i^2}\sqrt{\sum_{i=1}^ny_i^2}\]
\end{Sat}
\begin{Bew}
  OBdA $y\neq 0$ ($y=0$ trivial)
  \[t\to g(t)=\sum_{i=1}^n(x_i+ty_i)^2=\left( \sum x_i^2 \right)+2t\sum x_iy_i+t^2\sum y_i^2\]
  \[=\Norm{x}^2+2t\seq{x,y}+\Norm{y}^2t^2\]
  Sei $t_0=\frac{\seq{x,y}}{\Norm{y}^2}$, dann
  \[0\leq g(t_0)=\Norm{x}^2-2\frac{\seq{x,y}^2}{\Norm{y}^2}+\Norm{y}^2\frac{\seq{x,y}^2}{\Norm{y}^4}
 =\Norm{x}^2-\frac{\seq{x,y}^2}{\Norm{y}^2}\]
  \[\implies \seq{x,y}^2\leq\Norm{x}^2\Norm{y}^2\implies \Abs{\seq{x,y}}\leq\Norm{x}\Norm{y}\]
\end{Bew}
\begin{Def}
  Ein normierter Vektorraum ist ein reeller Vektorraum $V$ mit einer Abbildung $\Norm . :V\to\mb R$ so dass:
  \begin{enumerate}
    \item $\Norm x\geq 0$ und $\Norm x=0\iff x=0$ (Nullvektor)
    \item $\Norm{\lambda x}=\abs\lambda\Norm x$ $\forall \lambda\in\mb R$, $\forall x\in V$
    \item $\Norm{x+y}\leq\Norm{x}+\Norm{y}$ $\forall x,y\in V$
  \end{enumerate}
\end{Def}
\begin{Bsp}
  $V=\mb{R}^n$
  \[\Norm{x}_p=\left(\sum\Abs{x_i}^p\right)^{\frac{1}{p}}\s p\geq 1\, .\]
$\|\cdot\|_2$ ist die Euklidische Norm.
\end{Bsp}
\begin{Def}
  Seien $x,y\in\mb{R}^n$. Die Euklidische Metrik ist $d(x,y):=\Norm{x-y}$.
\end{Def}
\begin{Lem}\label{l:euk_met}
  \begin{enumerate}
    \item $d(x,y)\geq 0$ und $d(x,y)=0\iff x=y$
    \item $d(x,y)=d(y,x)$
    \item $d(x,z)\leq d(x,y)+d(y,z)$ (Dreiecksungleichung)
  \end{enumerate}
\end{Lem}
\begin{Bew} Die erste Zwei Aussagen sind trivial. Um die letzte zu beweisen:
  \[\Norm{x-z}\leq\|\underbrace{x-y}_{=:v}\|+\|\underbrace{\Norm{y-z}}_{=:w}\|\, .\]
Aber $x-z=v+w$. Wir wenden die dritte Aussage von Lemma \ref{l:norm} an:
  \[d (x,z) = \Norm{v+w}\leq\Norm v+\Norm w = d (x,y)+d(y,z)\, .\]
\end{Bew}
\begin{Def}
  Ein metrischer Raum ist eine Menge $X$ mit einer Abbildung
  \[d:X\times X\to\mb{R}\s (x,y)\mapsto d(x,y)\in\mb{R}\]so dass
  \begin{enumerate}
    \item $d(x,y)\geq 0$ und $d(x,y)=0\iff x=y$ $\forall x,y\in X$
    \item $d(x,y)=d(y,x)$ $\forall x,y\in X$
    \item $d(x,z)=d(x,y)+d(y,z)$ $\forall x,y,z\in X$
  \end{enumerate}
\end{Def}
\begin{Lem}
  Sei ($V$, $\Norm .$) ein normierter Vektorraum. Dann sind $V$ und $d(x,y)=\Norm{x-y}$ ein metrischer Raum.
\end{Lem}
\begin{Bew} Wir nutzen das gleiche Argument vom Lemma \ref{l:euk_met}.
\end{Bew}
\begin{Def}
  Die offene Kugel mit Radius $r>0$ und Mittelpunkt $x\in\mb{R}^n$ ist die Menge
  \[K_r(x)=\left\{ y\in\mb{R}^n, d(x,y)<r \right\}\]
(Wir werden auch oft $B_r (x)$ statta $K_r (x)$ nutzen.)
\end{Def}
\begin{Def}
  Eine Menge heisst ''Umgebung`` von $x$, wenn $V$ eine offene Kugel mit Mittelpunkt $x$ enth\"alt.
\end{Def}
\begin{Def}
  Eine Menge $U\subset\mb{R}^n$ heisst offen falls $\forall x\in U$ ist $U$ eine Umgebung von $x$, d.h.
  \[\forall x\in U\s\exists \s\text{eine Kugel}\s K_r(x)\subset U\]
\end{Def}
\begin{Bem}
Die Dreiecksungleichung impliziert dass jede offene Kugel eine offene Menge ist. In der Tat,
sei $y\in K_r (x)$. Dann $\rho:=d(x,y) < r$. Sei $\tau:= r-\rho>0$. Falls $z\in K_\tau (y)$,
dann $d(x,z)\leq d(x,y) + d(y,z) = \rho + d (y,z) < \rho +\tau =r$. D.h., $K_\tau (y)\subset K_r (x)$.
Das beweist dass $K_r (x)$ eine Ungebung ihrer ganzen Elementen ist, d.h. $K_r (x)$ ist offen. 
\end{Bem}
\begin{Sat}
  \begin{enumerate}
    \item $\varnothing$ und $\mb{R}^n$ sind offen
    \item Der Schnitt \ul{endlich vieler} offener Mengen ist auch offen.
    \item Die Vereinigung einer \ul{beliebigen} Familie offener Mengen ist auch offen.
  \end{enumerate}
\end{Sat}
\begin{Bew}
  \begin{enumerate}
    \item $\mb{R}^n$ trivialerweise offen, auch $\varnothing$
    \item Sei $x\in U\cap\dots\cap U_N$
      \[\forall i\in\left\{ 1,\dots,N \right\}\s \quad \exists r_i>0 \;\;\mbox{so dass}\;\; K_{r_i}(x)\subset U_i\]
      Sei $r=\min\left\{ r_i,\dots,r_N \right\} > 0$;
      \[\implies K_r(x)\subset U_i\quad\forall i\implies K_r(x)\subset U_1\cap\dots\cap U_N\]
    \item $\left\{ U_\lambda \right\}_{\lambda\in \Lambda}$. Sei $U=\bigcup_{\lambda\in\Lambda}U_\lambda$
      \[x\in U\implies x\in U_\lambda\s\text{f\"ur ein}\s\lambda\in\Lambda\]
      \[\implies \exists K_r(x)\subset U_\lambda\subset U.\]
  \end{enumerate}
\end{Bew}
\begin{Def}
  Ein topologischer Raum ist eine Menge $X$ und eine Menge $O$ von Teilmengen von $X$ so dass:
  \begin{enumerate}
    \item $\varnothing, X\in O$
    \item $U_1\cap\dots\cap_N\in O$ falls $U_i\in O$
    \item $\bigcap_{\lambda\in\Lambda}U_\lambda\in O$ falls $U_i\in O$
  \end{enumerate}
$O$ heisst die {\em Topologie}.
\end{Def}
\begin{Sat}
  Sei $(X,d)$ ein metrischer Raum. Wir definieren die entsprechende offene Kugel mit Mittelpunkt $x\in X$
und Radius $r>0$:
  \[K_r(x)=\left\{ y=X: d(x,y)<r \right\}\]
  Umgebungen und offene Mengen sind wie im Euklidischen Fall definiert. $O=\left\{ \text{offene Menge} \right\}$ definiert eine Topologie.
\end{Sat}

\subsection{Konvergenz}
Sei $\left\{ x_k \right\}_{k\in\mb{N}}$ $x_k\in\mb{R}$ $x_k=\left( x_{k1}, \cdots, x_{kn} \right)$
\begin{Def}
  Die Folge $\left\{ x_k \right\}$ konvergiert gegen $x_\infty\in\mb{R}^n$ falls
  \[\Limi{k}d(x_k,x_\infty)=0\]
  \[\left( \Limi{k}\Norm{x_k,x_\infty}=0 \right)\]
  Dann schreiben wir
  \[x_\infty=\Limi{k}x_k\]
\end{Def}
\begin{Sat}
  \[x_k\to x_\infty\iff x_{ki}\to x_{\infty_i}\s\forall i\in\left\{ 1,\cdots,n \right\}\]
\end{Sat}
\begin{Bew}
  \[\Norm{x_k - x_\infty}=\sqrt{\sum_{i=1}^n\left( x_{ki}-x_{\infty_i} \right)^2}\geq \abs{x_{ki}-x_{kinfty}}\geq 0\]
  \[\implies 0\leq \Limi{k}\abs{x_{ki}-x_{kinfty}}\leq \lim\Norm{x_k-x_\infty}=0\]
  \[\Norm{x_k-x_\infty}=\underbrace{\sqrt{\sum_{i=1}^n\underbrace{(x_{ki}-x_{\infty_i})^2}_{\to 0}}}_{\to 0}\leq \sum_{i=1}^n\abs{x_{ki}-x_{\infty_i}}\]
  \[\implies \Norm{x_k-x_\infty}\to 0\]
  Eine alternative Formulierung: $\Limi{k}x_k=\left( \Limi{k} x_{k1},\cdots,\Limi{k} x_{kn} \right)$
\end{Bew}
\begin{Bem}
  \[\forall \varepsilon>0 \exists N: \Norm{x_k-x_\infty}<\varepsilon\s\text{falls}\s k\geq N\]
  Für jede Umgebung $U$ von $x_\infty$ fast alle $x_k\in U$.
\end{Bem}
\begin{Def}
  Eine Folge $\left\{ x_k \right\}\subset\mb{R}^n$ heisst Cauchy falls:
  \[\forall \varepsilon>0\s\exists N: m,k\geq N\implies \Norm{x_k-x_m}<\varepsilon\]
\end{Def}
\begin{Lem}
  $\left\{ x_k \right\}\subset\mb{R}^n$ konvergiert genau dann, wenn $\left\{ x_k \right\}$ Cauchy ist.
\end{Lem}
\begin{Bew}
  $\left\{ x_k \right\}$ ist Cauchy $\implies$ $\left\{ x_{k_{\underbrace{i}_{\left\{ \text{fixiert} \right\}}}} \right\}$ Cauchy!
  \[\abs{x_{ki}-x_{m_i}}\leq\Norm{x_k-x_m}\]
  $\implies$ $\left\{ x_k \right\}$ ist eine Cauchyfolge $\stackrel{\text{Erstes Semester}}{\implies}$ $x_{ki}$ konvergiert $\stackrel{\text{Lemma 2}}{\implies}$ $x_k$ konvergiert. $x_k$ konvergiert $\implies$ Cauchyfolge
  \[x_\infty=\Limi{k} x_k\s\forall \varepsilon>0\s\exists N:\Norm{x_k-x_\infty}<\frac{\varepsilon}{2}\s\forall k\geq N\]
  \[k,m\geq N\s \Norm{x_k-x_m}\leq \Norm{x_k-x_\infty}+\Norm{x_\infty-x_m}\leq d(x_k,x_\infty)+(x_\infty,x_m)\]
  \[<\frac{\varepsilon}{2}+\frac{\varepsilon}{2}=\varepsilon\]
\end{Bew}
\begin{Bem}
  In einem metrischen Raum, Cauchy $\Leftarrow$ Konvergenz. Aber allgemein: Cauchy $\not\implies$ Konvergenz. Falls Cauchy $\implies$ Konvergenz, dann ist der metrische Raum vollständig.
\end{Bem}
\begin{Def}
  Eine Folge $\left\{ x_k \right\}\subset\mb{R}^n$ heisst beschränkt falls $\Norm{x_k}$ beschränkt ist.
\end{Def}
\begin{Sat}
  \begin{enumerate}
    \item Eine konvergente Folge ist beschränkt
    \item (Bolzano-Weierstrass) $\left\{ x_k \right\}$ beschränkt $\implies$ $\exists \left\{ x_{k_j} \right\}$ die konvergiert.
  \end{enumerate}
\end{Sat}
\begin{Bew}
  \[\left\{ x_k \right\}\s\text{beschränkt}\implies \left\{ x_{k1} \right\}_{k\in\mb{N}}\s\text{beschränkt}\]
  \[\implies \exists x_{k_j}: x_{k_j1}\to x_1\]
  Ich definiere $y_j=x_{k_j}$ $y_{j1}\to x_1$
  \[y_j\s\text{beschränkt}\implies\exists j_l: y_{j_l2}\to x_2\]
  \[z_l:=y_{j_l}\s\text{und}\s z_{l1}\to x_1, \s x_{l2}\to x_2\]
  \ldots $(n-2)$ Schritte. $w_r$ Teilfolge von $x_k$ mit $w_{ri}\to x_i$
  \[w_r\to(x_1,\cdots,x_n)\]
\end{Bew}
\subsection{Ein bisschen mehr Topologie}
\begin{Def}
  Eine Menge $G\subset\mb{R}^n$ heisst geschlossen falls $G^c(:=\mb{R}^n\setminus G)$ eine offene Menge ist.
\end{Def}
\begin{Bem}
  \[(A\cup B)^c = A^c\cap B^c\]
  \[(A\cap B)^c = A^c\cup B^c\]
\end{Bem}
\begin{Sat}
  \begin{enumerate}
    \item $\varnothing, \mb{R}^n$ sind abgeschlossen
    \item $G_1,\cdots,G_N$ abgeschlossen $\implies$ $G_1\cup G_2\cup \cdots\cup G_N$ abgeschlossen
    \item $\left\{ G_\lambda \right\}_{\lambda\in\Lambda}$ abgeschlossen $\implies$ $\bigcap_{\lambda\in\Lambda} G_\lambda$ abgeschlossen.
  \end{enumerate}
\end{Sat}
\begin{Sat}
  $G\subset\mb{R}^n$ $G$ ist abgeschlossen $\iff$ $\forall$ jede konvergente $\left\{ x_k \right\}\subset G$ gehört der Grenzwert zu $G$ (gilt auch für metrische Räume).
\end{Sat}
\begin{Bew}
  \begin{itemize}
    \item[$\Leftarrow$] Die rechte Eigenschaft gilt. Ziel: $G^c$ ist offen. Sei $x\in G^c$: das Ziel ist eine Kugel $K_r(x)\in G^c$ zu finden. Widerspruchsbeweis: $K_{\frac{1}{j}}(x)\not\subset G^c$, $j\in\mb{N}\setminus\left\{ 0 \right\}$
      \[\implies \exists x_j\in K_{\frac{1}{j}}(x)\cap G\implies \left\{ x_j \right\}\subset G\s\text{und}\s x_j\to x \]
      \[\left\{ x_j \right\}\subset G\s x_j\to x\s x\not\in G\]
      $\implies$ d.h. $G^c$ offen $\implies$ falls $\left\{ x_k \right\}\subset G$ und $x_k\to x$ dann $x\in G$
      Widerspruch: $G^c$ offen, aber $\exists \left\{ x_k \right\}\subset G$ mit Grenzwert $x\not\in G$, d.h. $x\in G^c$. Offenheit von $G^c$.
      \[\implies \exists K_r(x)\subset G^c\implies K_r(x)\cap=\varnothing\]
      d.h. $\exists N$ mit
      \[\Norm{x_N-x}<r\implies x_N\in K_r(x)\cap G\]
  \end{itemize}
\end{Bew}
\begin{Bsp}
  Eine offene Kugel ist nicht geschlossen.
  \[K_r(x)=\left\{ y:\Norm{y-x}<r \right\}\]
  Sei $\left\{ y_k \right\}\in K_r(x)$, (d.h. $\Norm{y_k-x}<r$) mit $y_k\to y$ und $\Norm{y-x}=r$.
\end{Bsp}
\begin{Def}
  Sei $\ol{K_r(x)}:=\left\{ y\in\mb{R}^n:\Norm{y-x}\leq r \right\}$.
\end{Def}
\begin{Ueb}
  $\ol{K_r(x)}$ ist abgeschlossen
\end{Ueb}
\begin{Def}
  $x\in\mb{R}^n$ ist ein Randpunkt von $M$ falls
  \[\forall K_r(x)\s\exists y\in K_r(x)\cap M\s\text{und}\s \exists z\in K_r(x)\cap M^c\]
\end{Def}
\begin{Def}
  Sei $M$ eine Menge in $\mb{R}^n$, dann ist der Rand von $M$
  \[\partial M=\left\{ x\in\mb{R}^n, \s\text{Randpunkt von}\s M \right\}\]
\end{Def}
\begin{Sat}
  $\partial M^c=\partial M$
  \begin{enumerate}
    \item $M\setminus \partial M$ ist die grösste offene Menge die in $M$ enthalten ist.
    \item $M\cup \partial \partial M$ ist die kleinste geschlossene Menge die $M$ enthält.
  \end{enumerate}
\end{Sat}
\begin{Bew}
  $M\setminus \partial M$ ist offen. 
  \[x\in M\setminus \partial M \implies x\in M\s\text{und}\s \exists K_r(x)\s\text{mit}\s K_r(x)\cap M^c=\varnothing\]
  \[\implies K_r(x)\subset M\]
  Sei $y\in K_r(x)$
  \[\implies \abs{y-x}=\rho<r\]
  \[\implies K_{r-\rho}(y)\subset K_r(x)\subset M\implies y\in M,y\not\in \partial M\]
  \[K_r(x)\subset M\setminus \partial M\]
  $x$ ist beliebig $\implies$ $M\setminus \partial M$ ist offen.\\
  Sei $A\subset M$ eine offene Menge. Das Ziel ist $A\subset M\setminus\partial M$. Sei $x\in A$. Ziel:($x\in M\setminus\partial M$) $x\not\in \partial M$.
  \[A\s\text{offen}\implies \exists K_r(x)\subset A\subset M\implies x\not\in \partial M\implies A\subset M\setminus\partial M\]
\end{Bew}


\newpage

%= Stichwortverzeichnis ======================================================================
\rhead{}
\addcontentsline{toc}{section}{Stichwortverzeichnis}
\printindex

\end{document}
