\subsection{Höhere partielle Ableitungen}
Sei
\[f:\mb{R}^n \supset \Omega\to\mb{R}\]
Die partiellen Ableitungen von $f$:
\[\Part{f}{x_i}(x)=\Limo{\varepsilon}\frac{f(x+\varepsilon e_i)-f(x)}{\varepsilon}
\quad \mbox{wobei } e_i=(0,\cdots,1,\cdots,0)\]
Falls die partielle Ableitung $\Part{f}{x_i}$ \"uberall existiert dann bekommen wir eine
neue Funktion
\[\Omega \ni x \mapsto \Part{f}{x_i}\in\mb{R}\, .
\]
Wir k\"onnen diese neune Funktion noch ableiten. Wir definieren
\[\frac{\partial^2f}{\partial x_j\partial x_i}  (x)
:= \Part{\left( \Part{f}{x_i} \right)}{x_j}(x)
=\lim_{\varepsilon\downarrow 0}\frac{\Part{f}{x_i}(x+\varepsilon e_j)-\Part{f}{x_i}(x)}{\varepsilon}\, .\]
Wenn auch diese \"uberall existiert, k\"onnen wir noch ableiten:
\[\frac{\partial^3f}{\partial x_k\partial x_j \partial x_i}(x)
:=\lim_{\varepsilon\downarrow 0}\frac{\frac{\partial^2f}{\partial x_i}(x+\varepsilon e_j)-\frac{\partial ^2f}{\partial x_j \partial x_i}(x)}{\varepsilon}\, .\]
Und so weiter. Die Anzahl Ableitungen dir wir nehmen ist die {\em Ordnung} der
h\"oheren Partiellen Ableitung. D.h.
\[
\frac{\partial^k f}{\partial x_{i_1} \ldots \partial x_{i_k}}
\]
ist eine partielle Ableitung mit Ordnung $k$.

Ausserdem wir nutzen die Notation
\[\frac{\partial^2 f}{\partial x_i^2} = \frac{\partial^2 f}{\partial x_i \partial x_i}\quad
\frac{\partial^3 f}{\partial x_i^3}= \frac{\partial^3 f}{\partial x_i \partial x_i \partial x_i}\]
und so weiter. 
\begin{Sat}[Lemma von Schwarz]
Sei $f:\Omega\to\mb{R}$ eine Funktion die in einer Umgebung von $p\in\Omega$ die partielle Ableitungen $\Part{f}{x_i}$, $\Part{f}{x_j}$ und $\frac{\partial^2 f}{\partial x_i \partial x_j}$ besitzt. Falls $\frac{\partial^2 f}{\partial x_i \partial x_j}$ stetig in $p$ ist, dann existiert $\frac{\partial^2 f}{\partial x_j\partial x_i}(p)$ und
  \[\frac{\partial^2 f}{\partial x_i\partial x_j}(p)=\frac{\partial^2 f}{\partial x_j\partial x_i}(p)\, .\]
\end{Sat}
\begin{Bsp} Wir kontrollieren die Plausibilit\"at dieses Satzes mit einer ziemlichen grossen Famile
von Funktionen: Die Polynome. Sei
  \[f(x_1,x_2)=\sum_{i=1}^{N_1}\sum_{j=1}^{N_2}a_{ij}x_1^ix_2^j\]
Dann wir k\"onnen explizit die folgenden partiellen Ableitungen rechnen:
  \[\Part{f}{x_1}=\sum_{i=1}^{N_1}\sum_{j=1}^{N_2}ia_{ij}x_1^{i-1}x_2^j\]
  \[\frac{\partial^2 f}{\partial x_2\partial x_1}=\sum_{i=1}^{N_1}\sum_{j=1}^{N_2}ija_{ij}x_1^{i-1}x_2^{j-1}\]
  \[\Part{f}{x_2}=\sum_{i=1}^{N_1}\sum_{j=1}^{N_2}ja_{ij}x_1^ix_2^{j-1}\]
  \[\frac{\partial^2 f}{\partial x_1\partial x_2}=\sum_{i=1}^{N_1}\sum_{j=1}^{N_2}ija_{ij}x_1^{i-1}x_2^{j-1}\, .\]
\end{Bsp}
\begin{Bsp} Aber, ohne gewisse Annahmen, ist der Satz falsch. 
  Sei zum Beispiel $V:\mb{R}\to\mb{R}$ eine Funktion die nicht differenzierbar ist und definieren wir
  \[v:\mb{R}^2\to\mb{R} \qquad v(x_1,x_2)=V(x_2)\]
Dann,
  \[\Part{f}{x_1}=0 \qquad\mbox{und}\qquad \frac{\partial^2f}{\partial x_2\partial x_1}=0\, .\]
Aber $\Part{f}{x_2}$ existiert nicht und deswegen auch $\frac{\partial^2 f}{\partial x_1 \partial x_2}$
nocht existiert.
\end{Bsp}
\begin{Bew}[Beweis des Lemmas von Schwarz]
  Die Idee ist ein "Art von Mittelwertsatz" zu benutzen.
  \paragraph{Schritt 1} Von Dimension $n\to 2$
  \[f(x_1,\cdots,x_i,\cdots,x_j,\cdots,x_n)\]
  \[p=(p_1,\cdots,p_i,\cdots,p_j,\cdots,p_n)\]
Wir definieren $g:\mb{R}^2 \supset U\to\mb{R}$ als
  \[g(y,z)=g(p_1,\cdots,p_{i-1},y,p_{i+1},\cdots,p_{j-1},z,p_{j+1},\cdots,p_n)\, .\]
Dann,
  \[\Part{f}{x_i}(p)=\Part{g}{y}(p_i,p_j)\qquad \Part{f}{x_j}(p)=\Part{g}{z}(p_i,p_j)\]
  \[\frac{\partial f}{\partial x_j\partial x_i}=\frac{\partial^2 g}{\partial z\partial y}(p_i,p_j)
\qquad \frac{\partial f}{\partial x_i\partial x_j}(p)=\frac{\partial g}{\partial y \partial z}(p_i,p_j)\]
(Wir rechnen zum Beispiel 
  \begin{eqnarray*}
\Part{f}{x_i}(p)& =& \lim_{\varepsilon to 0} \frac{f(p_1, \ldots, p_i+\varepsilon, \ldots, p_j, \ldots p_n)
- f (p)}{\varepsilon}\\ 
&=& \lim_{\varepsilon to 0} \frac{g(p_1+\varepsilon, p_2) - g (p_1, p_2)}{\varepsilon}
= \frac{\partial g}{\partial y} (p_i, p_j). \quad\Big)\, .
\end{eqnarray*}
Deswegen, oBdA beweisen wir nun den Fall $n=2$ des Satzes.

\medskip

{\bf Schritt 2}
Sei  $f:\mb{R}^2 \supset\Omega\to\mb{R}$ und $(a,b)\in \Omega$.
Wir wissen dass  $\Part{f}{x_1}$, $\Part{f}{x_2}$ und $\frac{\partial^2 f}{\partial x_2\partial x_1}$ in einer Umgebunv von $p=(a,b)$ existieren  und $\frac{\partial^2 f}{\partial x_2\partial x_1}$ stetig auf $p$ ist. Zu beweisen: $\frac{\partial^2 f}{\partial x_2\partial x_1}(p)$ existiert und
  \[\frac{\partial^2 f}{\partial x_2\partial x_1}(p)=\frac{\partial^2 f}{\partial x_1\partial x_2}(p)\, .\]
 F\"ur jede $h,k\in\mb{R}\setminus\left\{ 0 \right\}$ wir definieren den Rechteck $Q$ mit
Ecken $(a,b)$, $(a+h, b)$, $(a, b+k)$, $(a+h, b+k$. D.h. $Q= [a, a_h]\times [b, b+k]$. Wir definieren 
  \[D_Qf:=f(a+h,b+k)-f(a+h, b)-f(a,b+k)+f(a,b)\] und bemerken dass
  \begin{eqnarray}
\Limo{k} \Limo{h}\frac{D_Qf}{hk}
&=&\Limo{k} \Limo{h}\frac{f(a+h,b+k)-f(a,b+k)}{hk}-\frac{f(a+h,b)-f(a,b)}{hk}\nonumber\\
 &=&\Limo{k}\frac{\Part{f}{x_1}(a,b+k)-\Part{f}{x_1}(a,b)}{k}=\frac{\partial^2f}{\partial x_2\partial x_1}(a,b)
\label{e:kh}
\end{eqnarray}
und
\begin{equation}\label{e:hk}  
\Limo{h}\left( \Limo{k}\frac{D_Qf}{hk} \right)
=\Limo{h}\frac{\Part{f}{x_2}(a+h,b)-\Part{f}{x_2}(a,b)}{h}\, .
\end{equation}
Die Existenz des Grenzwerts in \eqref{e:hk} impliziert die Existenz der partiellen Ableitung
$\frac{\partial^2 f}{\partial x_1\partial _2} (a,b)$. In diesem Fall ist es auch
 \[\Limo{h}\Limo{k}\frac{D_Qf}{hk} =\frac{\partial^2f}{\partial x_1\partial x_2}(a,b)\]
Wir werden nun die existenz dieses zweiten Grenzwerts beweisen. Gleichzeitig erhalten wir dass die
Grenzwerte in \eqref{e:kh} und \eqref{e:hk} gleich sind (i.e. wir k\"onnen ``$h$ und $k$ 
im Grenzwert vertauschen''). 
 
Wir behaupten ($\forall h,k$ klein genug) die Existenz von einer Stelle $(\xi,\zeta)\in Q$ so dass
  \begin{equation}
    \label{e:1103231}
    \frac{D_Qf}{hk}=\frac{\partial^2f}{\partial x_2\partial x_1}(\xi, \zeta)
  \end{equation}
Das folgt wenn wir zwei Mal den Mittelwertsatz anwenden. OBdA
nehmem wir $h,k>0$ an. Dann 
  \[\frac{D_Qf}{hk}= \frac{1}{h}\left\{ \frac{f(a+h,b+k)-f(a+h,b)}{k}-\frac{f(a,b+k)-f(a,b)}{k} \right\}\]
  \[=\frac{1}{h}\left\{ g(a+h)-g(a) \right\}\stackrel{\text{Mittelwertsatz}}{=}g'(\xi)\]
wobei 
  \[g(z):=\frac{f(z,b+k)-f(z,b)}{k}\, \]
und $\xi$ eine Stelle in $]X, X+h[$ ist. $g$ ist in der Tat  differenzierbar und
\[g'(z)=\frac{1}{k}\left( \Part{f}{x_1}(z,b+k)-\Part{f}{x_1}(z,b) \right)\, .\]
Deswegen, wenn wir einen zweiten Mal den Mittelwertsatz anwenden,
\begin{eqnarray*}
\frac{D_Qf}{hk}
&=&\frac{1}{k}\left( \Part{f}{x_1}(\xi,b+k)-\Part{f}{x_1}(\xi,b) \right)\\
 &=&\Part{f}{x_2}\left( \Part{f}{x_1} \right)(\xi,\zeta) =
\frac{\partial^2 f}{\partial x_2\partial x_1} (\xi, \zeta)\, .
\end{eqnarray*}
Nun nutzen wir die Stetigkeit der Funktion $\frac{\partial^2 f}{\partial x_2\partial x_1}$:
\begin{eqnarray*} 
\lim_{k\to 0} \left(\lim_{h\to 0} \frac{D_Qf}{hk}\right)
&=& \lim_{\zeta\to b} \left(\lim_{\xi\to a} \frac{\partial^2 f}{\partial x_2\partial x_1} (\xi, \zeta)\right)\\
&=& \frac{\partial^2 f}{\partial x_2\partial x_1} (a,b) =  \lim_{\xi\to a} \left(\lim_{\zeta\to b} \frac{\partial^2 f}{\partial x_2\partial x_1} (\xi, \zeta)\right)\\
&=& \lim_{h\to 0} \left(\lim_{k\to 0} \frac{D_Qf}{hk}\right)
\end{eqnarray*}
\end{Bew}
