%TODO
\section{Das Taylorpolynom}
\begin{Def}
Sei nun $a\in\Omega\subset \mb{R}^n$, $f:\Omega\to \mb{R}$ und $w\in\mb{R}^n$. 
Falls die ganzen Ableitungen mit Ordnung $k$ in $a$ existieren, dann definieren wir 
\[:=\sum^n_{i_1=1}\cdots\sum^n_{i_k=1}\frac{\partial^kf(a)}{\partial x_{i_1}\cdots\partial x_{i_k}}w_{i_1}\cdots w_{i_k}\]
und das Taylor Polynom 
\[T^k_xf(z)=f(x)+\md f|_x(z-x)+\cdots+\frac{1}{k!}\md f^{(k)}|_x(z-x)^k\]
\end{Def}

\begin{Def}
Eine Funktion $f:\Omega\to\mb{R}$ heisst $C^k$ falls die ganzen 
partiellen Ableitungen mit Ordnung $\leq k$ \"uberall existieren und stetig  
sind.\end{Def}

\begin{Sat}[Verallgemeinerte Lagrange Fehlerabsch\"atzung]\label{s:Lag}
Sei $f\in C^{k+1}$ und $K_r (a)\in \Omega$. Dann, $\forall x\in K_r (a)$ $\exists \xi\in [x,k]$ s.d.
\begin{equation}\label{e:Lagrange}
R^k_a f (x) := f(x) - T^k_x f (x) = \frac{1}{(k+1)!}\md f^{(k+1)}|_\xi (x-a)^{k+1}\, .
\end{equation}
Falls $f\in C^k$, dann $f(x) - T^k_x f(x) = o (\|x\|^k)$ .
\end{Sat}
\begin{Bew}
{\bf Teil 1: Beweis von \eqref{e:Lagrange}} Sei $g (t) := f(tx + (1-t)a)$. Wir wenden die Kettenregel
$k+1$ Mal und rechnen:
\begin{eqnarray}
g' (t) &=& df|_{tx + (1-t)a} (x-a)\nonumber\\
g''(t) &=& d^2f|_{tx + (1-t)a} (x-a)^2\nonumber\\
&\ldots&\label{e:formel}\\
g^{(k+1)} (t) &=& d^{(k+1)} f|_{tx + (1-t)a} (x-a)^{k+1}\nonumber
\end{eqnarray}
Die Lagrange Fehlerabsch\"atzung f\"ur Funktionen einer Variable gibt
die existenz einer Stelle $\tau\in ]0,1[$ s.d.
\begin{equation}\label{e:Lag_1}
g(1) = \sum_{i=0}^k \frac{1}{i!} g^{(i)} (0) + \frac{1}{(k+1)!} g^{(k+1)} (\tau)\, .
\end{equation}
(Zur Erinnerung: wir nutzen die Konvention Konvention $g^{(0)} (0) = g(0)$ und deswegen
\[\frac{1}{0!}\md f^{(0)}|_x(z-x)^0=f(x)\, .\quad \Big)\]
Die Stelle $\xi:= \tau x + (1-\tau) a$ liegt auf dem Segment $[a,x]$. Mit den Formeln
\eqref{e:formel} schreiben wir \eqref{e:Lag_1} als
\begin{eqnarray*}
f(x) &=& g(1)=  \sum_{i=0}^i \frac{1}{i!} df^{(i)}|_a (x-a)^i + \frac{1}{(k+1)!} df^{(k+1)}|_\xi (a-x)^{k+1}\\
&=& T^k_a f (x) +  \frac{1}{(k+1)!} df^{(k+1)}|_\xi (a-x)^{k+1}
\end{eqnarray*}

\medskip

{\bf Teil 2} Sei nun $f\in C^k$. Wir nuzten \eqref{e:Lagrange} und (f\"ur $x\in K_r (a)$) schreiben
\begin{equation}\label{e:Lag_2}
f (x) = T^{k_1}_a f (x) + \frac{1}{k!} df^{(k)} |_\xi (a-x)^k\, .
\end{equation}
Deswegen
\begin{eqnarray}
&&\left|f(x) - T^k _a f (x)\right| = \left|\frac{1}{k!} df^{(k)} |_a (a-x)^k - \frac{1}{k!} df^{(k)} |_\xi (a-x)^k\right|\nonumber\\
&=& \frac{1}{k!} \left| \sum_{i_1=1}^n \ldots \sum_{i_k=1}^n 
\left(\frac{\partial^k f}{\partial x_{i_1} \ldots \partial x_{i_k}} (a) -
\frac{\partial^k f}{\partial x_{i_1} \ldots \partial x_{i_k}} (\xi)\right) (x_{i_1}-a_{i_1}) \ldots (x_{i_k}-a_{i_k})\right|\nonumber\\
&\leq& \frac{\|x-a\|^k}{k!}  \sum_{i_1=1}^n \ldots \sum_{i_k=1}^n 
\left|\frac{\partial^k f}{\partial x_{i_1} \ldots \partial x_{i_k}} (a) -
\frac{\partial^k f}{\partial x_{i_1} \ldots \partial x_{i_k}} (\xi)\right|\, .\label{e:Lag_3}
\end{eqnarray}
Die Stetigkeit der partiellen Ableitungen impliziert 
\[
\lim_{\xi\to 0} \left(\frac{\partial^k f}{\partial x_{i_1} \ldots \partial x_{i_k}} (a) -
\frac{\partial^k f}{\partial x_{i_1} \ldots \partial x_{i_k}} (\xi)\right) = 0\, .
\]
$x\to a$ impliziert $\xi\to 0$ und aus \eqref{e:Lag_3} schliessen wir 
\[
\lim_{x\to a}\frac{\left|f(x) - T^k _a f (x)\right|}{\|x-a\|^k} = 0\, .
\]
\end{Bew}

Falls $f$ beliebig mal differenzierbar ist (in diesem Fall schreiben wir $f\in C^{\infty}(\Omega)$; d.h. die ganzen partiellen Ableitungen existieren und sind stetig), können wir die Taylorreihe schreiben:
\[\sum_{k=0}^\infty\frac{1}{k!}\md f^{(k)}|_x(z-x)^k\]

\begin{Def}
  Eine Funktion $f\in C^\infty(\Omega)$ heisst analytisch wenn $\forall x\in \Omega$ $\exists B_r(x)\subset\Omega$ mit der Eigenschaft dass:
  \[T_x(z)=f(z)\s\forall z\in B_r(x)\, .\]
In diesem Fall schreiben wir $f\in C^\omega (\Omega)$.
\end{Def}

\subsection{Das Taylorpolynom zweiter Ordnung}
Wir schreiben noch ein Mal die Approximation mit dem Taylonrpolynom zweiter Ordnung
f\"ur eine $C^2$ Funktion:
\[f(z)=f(x)+\underbrace{\sum^n_{i=1}\Part{f}{x_i}(x)(z_i-x_i)}_{\seq{\nabla f(x), z-x}}\]
\begin{equation}
  \label{e:110328gelb}
  +\frac{1}{2}\sum^n_{i=1}\sum^n_{j=1}\frac{\partial^2f}{\partial x_i\partial x_j}(x)(z_i-x_i)(z_j-x_j)
+ R (z)\, .
\end{equation}
Aus dem Satz \ref{s:Lag} wissen wir dass $R (z) = o (\|z-x\|^2)$. Falls $f\in C^3$ dann wissen wir noch
mehr: $R(z) = O (\|z-x\|^3)$ (wir f\"uhren hier eine neue Notation ein: wenn $g$ eine nichtnegative Funktion
ist, die Schreibung $R(z) = O(g(z))$ bedeutet die Existenz einer Umgebung $U$ von $x$ und einer
Konstant $C$ s.d. $|R(z)|\leq Cg (z)$ $\forall z\in U$).

Wir definieren die Hessche Matrix
\[Hf(x)=\left( \frac{\partial f}{\partial x_i\partial x_j}(x) \right)\]
\[ \begin{pmatrix}
  \frac{\partial^2 f}{\partial x_1^2} & \frac{\partial^2 f}{\partial x_1\partial x_2} & \frac{\partial^2 f}{\partial x_1\partial x_3} \\
  \frac{\partial^2 f}{\partial x_2\partial x_1} & \frac{\partial^2 f}{\partial x_2^2} & \frac{\partial^2 f}{\partial x_2\partial x_3} \\
  \frac{\partial^2 f}{\partial x_3\partial x_1} & \frac{\partial^2 f}{\partial x_3\partial x_2}& \frac{\partial^2 f}{\partial x_3^2}
\end{pmatrix}\]
\begin{Bem}
  Schwarz $\implies$ $Hf(x)$ ist symmetrisch wenn alle Ableitungen zweiter Ordnung stetig sind.
\end{Bem}
Wir rechnen
 \[\underbrace{\sum_i\frac{\partial^2 f}{\partial x_1\partial x_j}(x)(z_i-x_i),\cdots,\sum_i\frac{\partial^2 f}{\partial x_n\partial x_i}(x)(z_i-x_i)}_{=Hf(x)(z-x)}\]
und deswegen
  \[\sum^n_{j=1}(z_j-x_j)\sum^n_{i_1}\frac{\partial^2 f}{\partial x_j\partial x_i}(x)(z_j-x_j)\]
  \[=\seq{z-x, Hf(x)(z-x)}=(z-x)^THf(x)(z-x)\]
  Wenn $A$ eine $n\times n$ Matrix, die Abbildung
  \[w\mapsto w^T A w\qquad  (=\seq{w,Aw})\]
  ist eine ``quadratische Form'' auf $\mb{R}^n$. $w^TAw$ ist das Matrix Produkt der:
$1\times n$ Matrix $w^T$ (``eine Zeile''), $n\times n$ Matrix $A$ und $n\times 1$ Matrix
$w$ (``eine Spalte'').

Das Taylorpolynom zweiter Ordnung ist dann
\[T^2_xf(z)=f(x)+\seq{\nabla f(x), z-x}+\frac{1}{2}(z-x)^T Hf(x)(z-x)\]
\begin{Kor}
  Falls $f\in C^3(\Omega)$ und $B_r(x)\subset\Omega$
  \[f(z)=T^2_x+O(\Norm{x-z}^3)\]
  d.h.
  \[\Abs{f(z)-T_x^2f(z)}\leq C\Norm{z-x}^3\]
\end{Kor}
\begin{Kor}
  Falls $f\in C^2(\Omega)$ und $B_r(x)\subset\Omega$, dann
  \[f(z)=T^2_xf(z)+o(\Norm{z-x}^2)\]
  d.h.
  \[\lim_{z\to x}\frac{f(z)-T_x^2f(z)}{\Norm{z-x}^2}=0\]
\end{Kor}
\begin{Bew}
Die Taylorapproximation mit Ordnung 1:
  \[f(z)=T^1_xf(z)+\frac{1}{2}(z-x)^THf(\zeta)(z-x)\]
Dann,
  \[f(z)-T_x^2f(z)=\frac{1}{2}(z-x)^THf(\zeta)(z-x)-\frac{1}{2}(z-x)^THf(x)(z-x)\]
  \[=\frac{1}{2}(z-x)^T(Hf(\zeta)-Hf(x))(z-x)\]
  \[\leq \frac{1}{2}\Norm{z-x}\Norm{Hf(\zeta)-Hf(x)(z-x)}\]
  \[\leq \frac{1}{2}\Norm{z-x}\Norm{Hf(\zeta)-Hf(x)}_O\Norm{z-x}\]
  \[= \frac{1}{2}\Norm{z-x}^2\Norm{Hf(\zeta)-Hf(x)}_O\]
  \[\frac{\Abs{f(z-)-T_x^2f(z)}}{\Norm{z-x}^2}\leq\frac{1}{2}\Norm{Hf(\zeta)-Hf(x)}_O\]
  \[\Norm{\zeta-x}\leq\Norm{z-x}^x\]
  Stetigkeit der Ableitungen 2. Ordnung
  \[\implies \lim_{\zeta\to x}\Norm{Hf(\zeta)-Hf(x)}_O=0\]
\end{Bew}
