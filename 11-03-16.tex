\subsection{Kettenregel}
\begin{Def}
  Eine Kurve ist eine Abbildung $\gamma:[a,b]\to\mb{R}^n$.  
\end{Def}
Diese Definition bedeutet dass $\gamma(t)\in\mb{R}^n$ $\forall t$.
Seien nun $\gamma_i (t)$ die Koordinaten des Vektors $\gamma (t)$:
  \[\gamma(t)=(\gamma_1(t),\cdots,\gamma_n(t))\, .\]
Jede $t\to\gamma_i(t)\in\mb{R}$ ist eine reelvertige Funktion einer Variabel.

\begin{Def}
Die Kurve $\gamma$ heisst differenzierbar wenn jede $\gamma_i$ differenzierbar ist.
In diesem Fall definieren wir
  \[\dot{\gamma} (t):=(\gamma'(t),\cdots,\gamma_n'(t))\]
\end{Def}

\begin{Sat}(Kettenregel 1. Version) Sei $f:U\to\mb{R}$ mit $U$ Umgebung von $x_0$ und $f$ differenzierbar in $x_0$. Sei $\gamma:[a,b]\to U$ eine differenzierbare Kurve mit $\gamma(t_0)=x_0$. Sei $g=f\circ \gamma$
(i.e. $g(t)=f(\gamma(t))$). Dann ist $g$ in $t_0$ differenzierbar und
  \[g'(t_0)=\md f|_{\gamma(t_0)}(\dot\gamma(t_0))=\seq{\nabla f(\gamma(t_0)),\dot\gamma(t_0)}\, .\]
\end{Sat}
\begin{Bew} Das Ziel:
  \[\Limo{h} \frac{g(t_0+h)-g(t_0)-h\left[ \md f|_{\gamma(t_0)}(\dot\gamma(t_0)) \right]}{h}=0\, .\]
Wir definieren
  \begin{equation}
    \label{e:1103161}
    R(h)=g(t_0+h)-g(t_0)-g(t_0)-h\left[ \md f|_{\gamma(t_0)}(\dot\gamma(t_0))\, . \right]
  \end{equation}
Dann wollen wir die folgende Behauptung zeigen:
  \begin{equation}
    \label{e:1103162}
    \Limo{h}\frac{R(h)}{|h|}=0
  \end{equation}
 Wir f\"uhren eine neue Notation ein:
wir sagen dass $R(h) = o (|h|)$ falls \eqref{e:1103161} gilt.

Aus der Differenzierbarkeit von $f$
  \[\Limo{k}\frac{f(x_0+k)-f(x_0)-\md f|_{x_0}(k)}{\Norm{k}}\left( =: \frac{r(k)}{\Norm{k}} \right)=0\, ,\]
d.h.
  \[r(k)=o(\Norm{k})\]
Die  Differenzierbarkeit von $\gamma$ impliziert
  \[\Limo{k}\frac{\gamma(t_0+h)-\gamma(t_0) - \dot{\gamma} (t_0)}{h}
\left( =:\frac{p(h)}{h} \right)=0\, ,\]
d.h.
\[p(h)=o(|h|)\]
Wir setzten
\[k = \gamma(t_0+h) - \gamma (t_0)
\]
und schreiben
 \begin{eqnarray*}
g(t_0+h)-g(t_0) &=& f(\gamma(t_0+h))-g(\overbrace{\gamma(t_0)}^{x_0})=
f(\gamma(t_0)+k)-f(\gamma(t_0))\nonumber\\
&=&\md f|_{\gamma(t_0)}(k)+r(k)\nonumber\\
&=&\md f|_{\gamma(t_0)}(\gamma(t_0+h)-\gamma(t_0))+r(k)\nonumber\\
&=&\md f|_{\gamma(t_0)}(h\dot\gamma(t_0)+p(h))+r(k)\nonumber\\
&\stackrel{\text{Linearität von $\md f$}}{=}& h\md f|_{\gamma(t_0)}(\dot\gamma(t_0))+\md f|_{\gamma(t_0)}(p(h))+r(k)\, .
\end{eqnarray*}
Deswegen
\[
R(h) = g(t_0+h)-g(t_0)-h\md f|_{\gamma(t_0)}(\dot\gamma(t_0))
=df|_{\gamma(t_0)}(p(h))+r(\gamma(t_0+h)-\gamma(t_0))\, .
\]
\begin{eqnarray*}
\Abs{R(h)} &\leq& 
|\underbrace{df|_{\gamma(t_0)}}_L (p(h))|+|r(\gamma(t_0+h)-\gamma(t_0))|\nonumber\\
&\leq&\Norm{L}_O \|p(h)\|+\frac{r(\gamma(t_0+h)-\gamma(t_0)}{\Norm{h}}
\end{eqnarray*}
Aber $p(h)= o (|h|) \implies \Norm{L}_O \|p(h)\| = o(|h|)$. Nun beweisen wir auch
  \[\Limo{h}\frac{r(\gamma(t_0+h)-\gamma(t_0)}{\Abs{h}} = 0\]
Wir unterscheiden zwei F\"alle. Falls 
  \[\gamma(t_0+h)-\gamma(t_0)=0,\]
  dann $r (\gamma(t_0+h)-\gamma(t_0))= r(0)=0$. Wenn 
  \[\gamma(t_0+h)-\gamma(t_0)\neq 0\]
dan schreiben wir
\[
\frac{r (\gamma(t_0+h)-\gamma(t_0))}{|h|} =\frac{r(\gamma(t_0+h)-\gamma(t_0)}{\Norm{\gamma(t_0+h)-\gamma(t_0)}}\frac{\Norm{t_0+h)-\gamma(t_0)}}{\Abs{h}}
\]
Nun  \[\frac{r(\gamma(t_0+h)-\gamma(t_0)}{\Norm{\gamma(t_0+h)-\gamma(t_0)}}=\frac{r(k)}{\Norm{k}}\to 0\]
(weil $k= \gamma(t_0+h)-\gamma(t_0)\to 0$ wenn $h\to 0$). 
Ausserdem
\[
\frac{\gamma (t_0+h)-\gamma(t_0)}{h} = =\underbrace{\dot\gamma(t_0)}_{\text{konstant}}+\underbrace{\frac{p(h)}{h}}_{\to 0}
\]
  Deswegen
  \[\Limo{h}\frac{\Norm{\gamma(t_0+h)-\gamma(t_0)}}{\Abs{h}}=\Norm{\dot\gamma(t_0)}\]
  \[\implies \frac{\Abs{R(h)}}{\Norm{h}}\to 0\]
  $\implies$ Differenzierbarkeit und Kettenregel!
\end{Bew}

\begin{Bem}
Als Korollar der Kettenregel erhalten wir das folgene geometrische Korollar:
der Gradient ist orthogonal zur Niveaumenge der Funktion (Höhenlinien, wenn der Definitionsbreich
der Funktion $2$-dimensioniert ist).
In der Tat, sei $\gamma:[a,b]\to U$ eine differenzierbare Kurve, $U$ offen. Sei $f:U\to\mb{R}$ differenzierbar. Wenn $f(\gamma(t))=c_0$ ($c_0$ hängt nicht von $t$ ab), dann
  \[\nabla f(\gamma(t))\bot \dot\gamma(t)\]
  d.h.
  \[\seq{\nabla f(\gamma(t)), \dot\gamma(t) }=0,\]
weil
  \[0=g'(t)=(f(\gamma(t)))'\stackrel{\text{Kettenregel}}{=}\seq{\nabla f(\gamma(t)), \dot\gamma(t)}\]
\end{Bem}
\subsection{Mittelwertsatz und Schrankensatz}
 Sei $f:[a,b]\to\mb{R}$ eine differenzierbare Funktion. Dann $\exists \xi\in ]a,b[$ s.d.
  \[f(b)-f(a)=f'(\xi)(b-a)\]
  Sei nun:
  \[f:U\mapsto\mb{R}\s\text{differenzierbar auf $U$}\]
  \[x,y\in U\s\text{so dass das Segment}\s[x,y]\subset U\]
 Das Segment $[x,y]$ ist die Menge
 \[\left[ [x,y] \right]=\left\{ x+t(y-x)|t\in \left[ 0,1 \right] \right\}\, .\]
Wir definieren
  \[\gamma(t):= x+t(y-x)  \qquad \mbox{und } \qquad g=f\circ \gamma \quad
(\mbox{d.h. }g(t)=f(\gamma(t)))\, .\]
Dann
  \[f(y)-f(x)=g(1)-g(0) \, .\]
Ausserdem, $\gamma$ ist differenzierbar und
\[\dot\gamma(\tau)=(\gamma_1'(\tau),\cdots,\gamma_n'(\tau))
=(y_1-x_1,\cdots,y_n-x_n)=y-x\]
Aus dem Mittelwertsatz f\"ur reelwertige Funktionen einer Variabel $\exists \tau\in ]0,1[$ s.d.
  \[f(y) - f(x) = g(1)-g(0)=g'(\tau)\stackrel{\text{Kettenregel}}{=} \md f|_{\gamma(\tau)}(\dot\gamma(\tau))
= \md f|_{\gamma(\tau)} (y-x)\]
  D.h. $\exists \xi\in [x,y]$ s.d.
  \begin{equation}
    \label{e:1103163}
    f(y)-f(x)=\md f|_\xi(y-x)=\partial_{y-x}f(\xi)
  \end{equation}
\begin{Sat}
  (Mittelwertsatz) $U$ offen, $[x,y]\subset U$ und $f:U\to\mb{R}$ differenzierbar. Dann $\exists \xi\in ]x,y[$ so das \eqref{e:1103163} gilt.
\end{Sat}
\begin{Def}
Sei $U\subset \mb{R}^n$ eine Menge. $U$ heisst sternförmig mit Zentrum $x_0\in U$: wenn $[x_0, x]\subset U
\forall x\in U$
\end{Def}
\begin{Sat}
  (Schrankensatz) Sei $U$ eine offene Menge, die sternförmig ist und $f:U\to\mb{R}$ eine differenzierbare Funktion mit
  \[\sup_{x\in U}\Norm{\md f|_x}_{O}=S<\infty \left( =\sup_{x\in U}\Norm{\nabla f(x)} \right)\]
  Dann
  \[\Abs{f(x)-f(0)}\leq S\Norm{x}\]
  Wenn $U$ konvex ist, d.h. das Segment $[x,y]\subset U$ $\forall x,y\in U$, dann
  \[\Abs{f(x)-f(y)}\leq S\Norm{y-x}\]
\end{Sat}
\begin{Def}
  $f:\underbrace{K}_{\in \mb{R}^n}\to\mb{R}$ heisst Lipschitz wenn $\exists L\in [0,+\infty[$ so dass
  \[\Abs{f(y)-f(x)}\leq L\Norm{y-x}\s\forall x,y\in K\]
Sei $(X,d)$ ein metrischer Raum. $f:(X,d)\to\mb{R}$ heisst Lipschitz falls $\exists L<\infty$ so dass
  \[\Abs{f(y)-f(x)}\leq L d(y, x)\s\forall x,y\in K\]
\end{Def}

\begin{Kor}
Sei $U$ offen und konvex und $f:U\to \mb{R}$ eine differenzierbare Funktion mit beschr\"ankten
partiellen Ableitungen. Dann ist $f$ Lipschitz.
\end{Kor}
