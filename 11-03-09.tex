\subsection{Differenzierbare Funktionen}
\paragraph{Erinnerung} $f:\mb{R}\to \mb{R}$ heisst differenzierbar in $a\in \mb{R}$ falls
\[f'(a)=\Limo{h}\frac{f(a+h)-f(a)}{h}\]
existiert. Was geschieht mit Funktionen von mehrere Variablen? Die ``Tangentensteigung'' hängt auch von der Richtung ab. D.h. Es gibt eine lineare Abbildung $L:\mb{R}^2\to\mb{R}$
\begin{Def}
  $f:U\to\mb{R}$, $U\subset\mb{R}^n$ offen, heisst differenzierbar in $a\in U$, falls
  \begin{equation}
    \label{e:1103091}
    \Limo{h}\frac{f(a+h)-f(a)-Lh}{\Norm{h}}=0
  \end{equation}
  wobei $L:\mb{R}^n\to\mb{R}$ eine lineare Abbildung ist.
\end{Def}
\begin{Bem}
  $n=1$: Es gilt $Lh = f'(a)h$
\end{Bem}
\begin{Bem}
  Die lineare Abbilung $L$ in \ref{e:1103091} ist eindeutig definiert. Annahme $L'\neq L$ erfüllt die Bedungung. Sei $v\in\mb{R}^n$ mit $\Norm{v}=1$. Es gilt:
  \[(L-L')(v)\stackrel{\text{linear und}\s \Norm{v}=1}{=}\lim_{t\downarrow 0}\frac{(L-L')(tv)}{\Norm{tv}}\stackrel{\ref{e:1103091}\s h=tv}{=}\implies L=L'\]
\end{Bem}
\begin{Bem}
  Wir können \ref{e:1103091} auch anders beschreiben:
  \[f(a+h)-f(a)=Lh+\underbrace{R(h)}_{\text{Restglied}}\]
  Dann gilt
  \begin{equation}
    \label{e:1103092}
    \ref{e:1103091} \iff \Limo{h}\frac{R(h)}{\Norm{h}}=0
  \end{equation}
\end{Bem}
\begin{Def}
  $L$ heisst Differential von $f$ in $a$. Man schreibt $\md f(a)$. Sei nun $\left\{ e_1,\cdots,e_n \right\}$ die Standardbasis $\mb{R}^n$, $h=(h_1,\cdots,h_n)\in\mb{R}^n$
  \[\implies \md f(a)h=\md f(a)\left( \sum_{i=1}^kh_i-e_i \right)=\sum_{i=1}^nh_i\md f(a)e_i\]
\end{Def}
\begin{Def}
  \[f'(a)=(\md f(a)e_1,\cdots,\md f(a)e_n)\]
  heisst Ableitung
\end{Def}
\begin{Def}
  \[Tf(x,a)=f(a)+f'(a)(x-a)\s\text{(Ebene (tangential))}\]
  lineare Approximation
\end{Def}
\begin{Sat}
  $f$ differentierbar in $a$ $\implies$ $f$ ist stetig in $a$
\end{Sat}
\begin{Bew}
  \[\Abs{f(a+b)-f(a)}=\Abs{\md f(a)h+R(h)}\leq\Abs{\md f(a)}+\underbrace{\Abs{R(h)}}_{\to 0}\]
\end{Bew}
\begin{Bsp}
  $f(x)=Ax+b$, $A\in M_a(1,n,\mb{R})$, $b\in\mb{R}$
  \begin{Beh}
    $Lh:=ah$ ist linear
    \[\md f(a)h=Ah, \s f'(a)=A\]
  \end{Beh}
  \begin{Bew}
    \[f(a+h)-f(a)-Lh=\not{R(h)}=0\]
  \end{Bew}
\end{Bsp}
\begin{Bsp}
  $f(x):=x^TAx$, $A=(a_{ij})\in\Sym(n,\mb{R})$
  \[f(a+h)-f(a)-\underbrace{2a^TAh}_{\md f(a)h}+\underbrace{h^TAh}_{R(h)}\]
  $Lh:=2a^TAh$ ist linear (in $h$), $R(h)=h^TAh$ ($=\sum h_ia_{ik}h_l$)
  z.z.: $\Abs{Rh}\leq\sum^h_{i,j=1}\Abs{a_{ij}}\Norm{h}_{\infty}^2$, d.h. $\frac{R(h)}{\Norm{h}}\to 0$ (falls $\Norm{h}\to 0$)
\end{Bsp}
\paragraph{Ziel}
Wir wollen $\md f(a)h$ berechnen. sei $t\in\mb{R}$
\[f(a+th)=f(a)+\md f(a)th+R(th)\]
\begin{equation}
  \label{e:1103094}
  \implies \md f(a)h=\Limo{t}\frac{f(a+th)-f(a)}{t}
\end{equation}
\begin{Def}
  $f:U\to\mb{R}$, $a\in U$. Die Richtungsableitung von $f$ in Richtung $h\in\mb{R}^n$ ist der Grenzwert (falls er existiert)
  \[\partial_nf(a):=\Limo{t}\frac{f(a+th)-f(a)}{t}\]
  Die Ableitungen in Richtung $e_1,\cdots,e_n$ heissen partielle Ableitungen in $a$. Wir schreiben
  \[\partial_{ei}f(a)=\partial_if(a)=\frac{\partial f}{\partial x_i}(a)=f_{xi}(a)\]
\end{Def}
\begin{Bem}
  Wir haben \ul{nicht} vorausgesetzt, dass $f$ differenzierbar ist in $a$!
\end{Bem}
\begin{Sat}
  Sei $f$ in $a$ differenzierbar. Dann existieren die Richtungsableitungen in jede Richtung. Insbesondere existieren die aprtiellen Ableitungen. Es gelten:
  \begin{equation}
    \label{e:1103095}
    \md f(a)h=f'(a)h=\partial_nf(a)=\sum_{i=1}^n\partial_if(a)h_i
  \end{equation}
  und
  \[f'(a)=\left( \partial_1f(a),\cdots,\partial_nf(a) \right)\]
\end{Sat}
\begin{Bew}
  Existenz der Richtungsableitung oke (Herleitung von \ref{e:1103094})
\end{Bew}
\paragraph{Frage}
Wie berechnet man die partielle Ableitung effizient? Es gilt:
\[\partial_if(a)=\Limo{t}\frac{f(a+t_{ei})-f(a)}{t},\s a=(a_1,\cdots,a_n)\]
\[g_i(x):=f(a_1,\cdots,a_{i-1},x,a_{i+1},\cdots,a_n)\]
\[\partial_if(a)=\Limo{t}\frac{g(a_i+t)-f(a_i)}{t}=g'(a_i)\]
\begin{Bsp}
  \[f(x,y):=\sin(2x)e^{3y}\]
  \[\partial_xf=2e^{3y}\cos(2x)\]
  \[\partial_yf=\sin(2x)e^{3y}3\]
\end{Bsp}
\paragraph{Frage}
Wann folgt aus der Existenz der partiellen Ableitung (Richtungsableitung) die Differenzierbarkeit?
\begin{Bsp}
  \[f(x,y)= \begin{cases}
    \frac{x^2y}{x^2+y^2}&(x,y)\neq(0,0)\\
    0&(x,y)=(0,0)
  \end{cases}\]
  Es gilt: $f(tx,ty)=tf(x,y)$, d.h. der Graph von $f$ besteht aus Geraden durch $0$, für $h=(h_1,h_2)\in\mb{R}^2$
  \[\implies \partial_hf(0,0)=\Limo{t}\frac{f(th_1,th_2)-f(0,0)}{k}=\Limo{t}\frac{t}{t}f(h_1,h_2)=f(h_1,h_2)\]
  \[\implies \partial f(0,0)=f(h_1,h_2)\]
  \[\partial_{e_1}f(0,0)=f(1,0)=0\]
  \[\partial_{e_2}f(0,0)=f(0,1)=0\]
  \paragraph{Annahme}
  $f$ ist in $(0,0)$ differenzierbar
  \[\xRightarrow{\text{aus}\s\ref{e:1103095}}\underbrace{\partial_nf(0,0)}_{=\md f(a)h=0}=\underbrace{\partial_1f(a)}_{0}(h_1)+\underbrace{\partial_2f(a)}_0(h_2)=0\]
  \[\implies \md f(a)=0\]
  \paragraph{Test}
  $L=0$
  \[\frac{f(h_1,h_1)-\overbrace{f(a_0)-L(h_1,h_1)}}{\Norm{(h_1,h_1)}_\infty}=\frac{h_1^3}{2h_1^2\Abs{h_1}}\to\pm\frac{1}{2}\]
  $\implies$ $f$ ist in $(0,0)$ nicht differenzierbar.
\end{Bsp}
